\documentclass[12pt]{article}
\usepackage{homework}
\pagestyle{fancy}

% assignment information
\def\course{Symmetry and Relativity}
\def\assignmentno{Problem Set 4}
\def\assignmentname{Dynamics and Electromagnetism in Special Relativity}
\def\name{Xin, Wenkang}
\def\time{\today}

\begin{document}

\begin{titlepage}
    \begin{center}
        \large
        \textbf{\course}

        \vfill

        \Huge
        \textbf{\assignmentno}

        \vspace{1.5cm}

        \large{\assignmentname}

        \vfill

        \large
        \name

        \time
    \end{center}
\end{titlepage}


%==========
\pagebreak
\section*{}
%==========


\problem{1}{Sticky collision}
Initially, we have two 4-momenta $P_{1}^{\mu} = (4mc, p_{0})$ and $P_{2}^{\mu} = (2mc, 0)$, where $p_{0} = \sqrt{3}mc$. Suppose that after the collision, the combined system has a 4-momentum $P^{\mu} = (E_{f}/c, p_{f}) =(6mc, p_{0})$, since we demand conservation of 4-momentum. We have:
\begin{equation}
    \frac{E_{f}^{2}}{c^{2}} - p_{f}^{2} = m_{f}^{2}c^{2}
\end{equation}

On the other hand, $p_{0}$ satisfies $16m^{2}c^{2} - p_{0}^{2} = m^{2}c^{2}$, so $p_{0} = \sqrt{15}mc$. We can then solve for $m_{f}$:
\begin{equation}
    m_{f} = \sqrt{21} m
\end{equation}

To find the velocity of the combined system, consider the relation $E_{f} = \gamma m_{f}c^{2}$, leading to $\gamma = 6/\sqrt{21}$. The velocity is then:
\begin{equation}
    v = \beta c = \sqrt{\frac{5}{12}} c
\end{equation}
\qed


\problem{2}{Pair production}
In the centre of mass frame, the two photons have the 4-momenta $P_{1}^{\mu} = (E'/c, E'/c)$ and $P_{2}^{\mu} = (E'/c, -E'/c)$. After production of the electron-positron pair, the 4-momenta are $P_{\pm}^{\mu} = (mc, 0)$. Thus, we require $2E'/c = 2mc$ so the threshold energy is $E' = mc^{2}$.

On the other hand, the velocity of the centre of mass frame is:
\begin{equation}
    v_{\text{cm}} = \frac{(E_{0}/c - E/c) c^{2}}{E_{0}/c + E/c} = \frac{E_{0} - E}{E_{0} + E} c
\end{equation}

Thus, we find the relation between $E'$ and $E_{0}$:
\begin{equation}
    E' = \gamma \frac{E_{0}}{c} - \gamma \beta \frac{E_{0}}{c} = \frac{E_{0}}{c} \sqrt{\frac{1 - \beta}{1 + \beta}} = \sqrt{E_{0} E}
\end{equation}
where $\beta = (E_{0} - E)/(E_{0} + E)$.

Since $E' = mc^{2}$, we have:
\begin{equation}
    E = \frac{mc^{4}}{E_{0}}
\end{equation}
\qed


\problem{3}{Two-body decay}
In the centre of mass frame, the initial 4-momenta $(Mc, 0)$ splits into $P_{1}^{\mu} = (E_{1}'/c, p')$ and $P_{2}^{\mu} = (E_{2}'/c, -p')$. We have the equations:
\begin{equation}
    \begin{split}
        E_{1}' + E_{2}' &= M c^{2} \\
        E_{1}'^{2}/c^{2} - p'^{2} &= m_{1}^{2}c^{2} \\
        E_{2}'^{2}/c^{2} - p'^{2} &= m_{2}^{2}c^{2}
    \end{split}
\end{equation}

Solving for $E_{1}'$ and $E_{2}'$:
\begin{equation}
    \begin{split}
        E_{1}' &= \frac{c^{2}}{2} \left( M + \frac{m_{1}^{2} - m_{2}^{2}}{M} \right) \\
        E_{2}' &= \frac{c^{2}}{2} \left( M + \frac{m_{2}^{2} - m_{1}^{2}}{M} \right)
    \end{split}
\end{equation}
which leads to an expression for $p'$:
\begin{equation}
    p' = \frac{c}{2M} \left[ (m_{1}^{2} + m_{2}^{2} - M^{2})^{2} - 4m_{1}^{2}m_{2}^{2} \right]^{1/2}
\end{equation}

To find the Lorentz factor, first note that in the lab frame, $\gamma = E/Mc^{2}$ so $\beta = \sqrt{1 - M^{2}c^{4}/E^{2}}$. $E_{1}$ and $E_{1}'$ are related via Lorentz transformations. If the products are emitted along the line of motion, we have:
\begin{equation}
    E_{1} = \gamma (E_{1}' + v p')
\end{equation}

If the products are emitted perpendicular to the line of motion, we have:
\begin{equation}
    E_{1}= \gamma E_{1}'
\end{equation}
\qed


\problem{4}{Motion in an electromagnetic field}

\subproblem{a}
Following transformation rules for the electromagnetic field, we have:
\begin{equation}
    \begin{split}
        \mathbf{E}' = \gamma (E \hat{\mathbf{x}} - vB \hat{\mathbf{x}}) \\
        \mathbf{B}' = \gamma (B \hat{\mathbf{y}} - \frac{1}{c^{2}} vE \hat{\mathbf{y}})
    \end{split}
\end{equation}

For $\mathbf{E}' = 0$, we need $v = E/B$, which leads to $\mathbf{B}' = \gamma B \hat{\mathbf{y}}$. We must have $v = E/B < c$ for this to be possible.

\subproblem{b}
In frame $S'$, the particle has velocity:
\begin{equation}
    \mathbf{u}' = \frac{1}{1 + vu_{z}/c^{2}} \left[ (u_{z} - v) \hat{\mathbf{z}} + \sqrt{1 - \beta^{2}} u_{x} \hat{\mathbf{x}} \right]
\end{equation}

As in $S'$, the magnetic field is along $\hat{\mathbf{y}}$, the particle undergoes a circular motion in the $xz$-plane. The radius of the circle is:
\qed


\problem{5}{Interactions between two charged beams in a magnetic field}

\subproblem{a}
In the lab frame, the two particle beams can be represented by the 4-currents $J_{1}^{\mu} = J_{2}^{\mu} = (c\lambda/A, v\lambda/A)$. The electric field produced by one of the beam at the other is $E = \lambda/(2\pi\epsilon_{0}r)$. The balance of forces is:
\begin{equation}
    E\lambda = Bv\lambda
\end{equation}
which leads to:
\begin{equation}
    B = \frac{E}{v} = \frac{\lambda}{2\pi\epsilon_{0}vd}
\end{equation}

\subproblem{b}
In the rest frame of the beams, we transform the 4-currents to:
\begin{equation}
    (J_{1}')^{\mu} =
    \begin{pmatrix}
        \gamma        & -\gamma \beta \\
        -\gamma \beta & \gamma
    \end{pmatrix}
    \begin{pmatrix}
        c\lambda/A \\
        v\lambda/A
    \end{pmatrix}
    =
    \gamma \frac{\lambda}{A}
    \begin{pmatrix}
        c - \beta v \\
        -\beta c + v
    \end{pmatrix}
    =
    \begin{pmatrix}
        c \rho' \\
        j'
    \end{pmatrix}
\end{equation}

This gives the linear charge density $\lambda' = A \rho' = \sqrt{1 - \beta^{2}} \lambda$ and the current density $j' = 0$. The force by one beam on the other is:
\begin{equation}
    f' = E' \lambda' = \frac{\lambda'^{2}}{2\pi\epsilon_{0}d} = (1 - \beta^{2}) \frac{\lambda^{2}}{2\pi\epsilon_{0}d}
\end{equation}

Now transform this force back to the lab frame:
\begin{equation}
    f = \gamma f' = \sqrt{1 - \beta^{2}} \frac{\lambda^{2}}{2\pi\epsilon_{0}d}
\end{equation}

This must be equal to the magnetic force $Bv\lambda$ in the lab frame, so we have:

\begin{equation}
    B = \frac{1}{\gamma} \frac{\lambda}{2\pi\epsilon_{0}vd}
\end{equation}
\qed


\problem{6}{Covariant generalisation of Ohm's law}

\subproblem{a}
In the rest frame of the conductor, the product $U_{\nu} J^{\nu}$ evaluates to $-\rho_{0} c^{2}$. Thus:
\begin{equation}
    \begin{split}
        (J_{0})^{\mu} &= -\frac{1}{c^{2}} (U_{\nu} J^{\nu}) (U_{0})^{\mu} + \sigma_{0} (F_{0})^{\mu \nu} (U_{0})_{\nu} \\
        &= \rho_{0} (U_{0})^{\mu} + \sigma_{0} (F_{0})^{\mu \nu} (U_{0})_{\nu}
    \end{split}
\end{equation}

Noting that the only non-zero component of $U_{0}$ is its time component, we have:
\begin{equation}
    \mathbf{j} = (J_{0})^{i} = \sigma_{0} (F_{0})^{i0} (-c) = \sigma_{0} \mathbf{E}_{0}
\end{equation}

\subproblem{b}
In an arbitrary frame, we have:
\begin{equation}
    J^{\mu} = \rho_{0} U^{\mu} + \sigma_{0} F^{\mu \nu} U_{\nu}
\end{equation}
since the product $U_{\nu} J^{\nu}$ is a scalar invariant.

Isolating the spatial components of $J^{\mu}$, we have:
\begin{equation}
    \begin{split}
        J^{i} &= \rho_{0} \gamma v_{i} + \gamma \sigma_{0} (-F^{i0}c + F^{ij} v_{j}) \\
        &= \gamma \rho_{0} v_{i} + \gamma \sigma_{0} (E_{i} + \epsilon_{ijk} v_{j} B_{k}) \\
        &= \gamma \rho_{0} \mathbf{v} + \gamma \sigma_{0} (\mathbf{E} + \mathbf{v} \times \mathbf{B})
    \end{split}
\end{equation}

On the other hand, note the temporal component of $J^{\mu}$:
\begin{equation}
    \begin{split}
        J^{0} &= \rho_{0} \gamma c + \sigma_{0} F^{0\nu} U_{\nu} \\
        &= \gamma \rho_{0} c + \gamma_{v} \sigma_{0} \frac{E_{i} v_{i}}{c} \\
        &= \rho c
    \end{split}
\end{equation}
which means:
\begin{equation}
    \gamma \rho_{0} = \rho - \gamma \sigma_{0} \frac{\mathbf{E} \cdot \mathbf{v}}{c^{2}}
\end{equation}

Combining the two results, we have:
\begin{equation}
    \mathbf{j} = \rho \mathbf{v} + \gamma \sigma_{0} \left( \mathbf{E} + \mathbf{v} \times \mathbf{B} - \frac{\mathbf{E} \cdot \mathbf{v}}{c^{2}} \mathbf{v} \right)
\end{equation}

\subproblem{c}
If $\rho_{0} = 0$, from the above results, we have:
\begin{equation}
    \rho = \gamma \sigma_{0} \frac{\mathbf{E} \cdot \mathbf{v}}{c^{2}}
\end{equation}
and:
\begin{equation}
    \mathbf{j} = \gamma \sigma_{0} \left( \mathbf{E} + \mathbf{v} \times \mathbf{B} \right)
\end{equation}
\qed


\problem{7}{Angular momentum of the electromagnetic field}

\subproblem{a}
We have:
\begin{equation}
    \begin{split}
        \partial_{\alpha} M^{\alpha \beta \gamma} &= \partial_{\alpha} (X^{\gamma} T^{\alpha \beta}) - \partial_{\alpha} (X^{\beta} T^{\alpha \gamma}) \\
        &= (\partial_{\alpha} X^{\gamma}) T^{\alpha \beta} + X^{\gamma} (\partial_{\alpha} T^{\alpha \beta}) - (\partial_{\alpha} X^{\beta}) T^{\alpha \gamma} - X^{\beta} (\partial_{\alpha} T^{\alpha \gamma}) \\
        &= \delta^{\gamma}_{\alpha} T^{\alpha \beta} - \delta^{\beta}_{\alpha} T^{\alpha \gamma} \\
        &= T^{\alpha \beta} - T^{\beta \gamma} \\
        &= 0
    \end{split}
\end{equation}
where the third equality follows because we are relabelling the indices and the final equality follows from the symmetry of $T^{\alpha \beta}$.

\subproblem{b}
\begin{equation}
    \begin{split}
        \partial_{\alpha} M^{\alpha i j} &= \partial_{\alpha} (X^{j} T^{\alpha i}) - \partial_{\alpha} (X^{i} T^{\alpha j}) \\
        &= T^{ji} + X^{j} (0 - \partial_{\alpha} T^{\alpha 0}) - T^{ij} - X^{i} (0 - \partial_{\alpha} T^{\alpha 0}) \\
        &= X^{i} \partial_{\alpha} T^{\alpha 0} - X^{j} \partial_{\alpha} T^{\alpha 0} \\
        &= cX^{i} \nabla \cdot \mathbf{g} - cX^{j} \nabla \cdot \mathbf{g} \\
    \end{split}
\end{equation}


\end{document}
\documentclass[12pt]{article}
\usepackage{homework}
\pagestyle{fancy}

% assignment information
\def\course{Symmetry and Relativity}
\def\assignmentno{Problem Set 3}
\def\assignmentname{Kinematics and Dynamics in Special Relativity}
\def\name{Xin, Wenkang}
\def\time{\today}

\begin{document}

\begin{titlepage}
    \begin{center}
        \large
        \textbf{\course}

        \vfill

        \Huge
        \textbf{\assignmentno}

        \vspace{1.5cm}

        \large{\assignmentname}

        \vfill

        \large
        \name

        \time
    \end{center}
\end{titlepage}


%==========
\pagebreak
\section*{}
%==========


\problem{1}{Evaluation of derivatives for a four-vector field}
We define the covariant derivative:

\begin{equation}
    \partial_{\mu} = \left( \frac{1}{c} \frac{\partial}{\partial t}, \nabla \right)
\end{equation}

and the contravariant derivative:

\begin{equation}
    \partial^{\mu} = \left( -\frac{1}{c} \frac{\partial}{\partial t}, \nabla \right)
\end{equation}

We have the following calculations:

\begin{equation}
    \partial_{\lambda} X^{\lambda} = 4
\end{equation}

\begin{equation}
    \partial^{\mu} (X_{\lambda} X^{\lambda}) = 0
\end{equation}

as $X_{\lambda} X^{\lambda}$ is an invariant scalar.

\begin{equation}
    \partial^{\mu} \partial_{\mu} X_{\nu} X^{\nu} = 0
\end{equation}

as $X_{\nu} X^{\nu}$ is an invariant scalar.

\begin{equation}
    \partial^{\mu} X^{\nu} =
    \begin{pmatrix}
        \partial^{0} X^{0} & \partial^{0} X^{1} & \partial^{0} X^{2} & \partial^{0} X^{3} \\
        \partial^{1} X^{0} & \partial^{1} X^{1} & \partial^{1} X^{2} & \partial^{1} X^{3} \\
        \partial^{2} X^{0} & \partial^{2} X^{1} & \partial^{2} X^{2} & \partial^{2} X^{3} \\
        \partial^{3} X^{0} & \partial^{3} X^{1} & \partial^{3} X^{2} & \partial^{3} X^{3} \\
    \end{pmatrix}
    =
    \begin{pmatrix}
        -1 & 0 & 0 & 0 \\
        0  & 1 & 0 & 0 \\
        0  & 0 & 1 & 0 \\
        0  & 0 & 0 & 1 \\
    \end{pmatrix}
    =
    \eta^{\mu \nu}
\end{equation}

\begin{equation}
    \begin{split}
        \partial_{\lambda} F^{\lambda} &= 2\partial_{\lambda} X^{\lambda} + \partial_{\lambda} \left[ K^{\lambda} (X_{\nu} X^{\nu}) \right] \\
        &= 8
    \end{split}
\end{equation}

since the second term is a constant scalar.

\begin{equation}
    \partial^{\mu} (\partial_{\lambda} F^{\lambda}) = \partial^{\mu} (8) = 0
\end{equation}

\begin{equation}
    \partial^{\mu} \partial_{\mu} \sin{(K_{\lambda} X^{\lambda})} = 0
\end{equation}

as $\sin{(K_{\lambda} X^{\lambda})}$ is an invariant scalar.
\qed


\problem{2}{Properties of spacetime intervals}

\subproblem{a}
A time-like 4-vector satisfies $A^{\mu} A_{\mu} < 0$. We attempt to find a Lorentz transformation such that $A'^{i} = 0$ for $i = 1, 2, 3$. It is customary to align our axes such that the only non-zero components of $A^{\mu}$ are $A^{0}$ and $A^{1}$. Then we have:

\begin{equation}
    A^{\mu} A_{\mu} = -A^{0} A^{0} + A^{1} A^{1} < 0
\end{equation}

Consider the Lorentz transformation along the $x$-axis:

\begin{equation}
    \begin{split}
        A'^{0} &= \gamma (A^{0} - \beta A^{1}) \\
        A'^{1} &= \gamma (A^{1} - \beta A^{0}) \\
        A'^{2} &= A^{2} = 0 \\
        A'^{3} &= A^{3} = 0
    \end{split}
\end{equation}

$A'^{1}$ can be made zero by choosing $\beta = A^{1}/A^{0} < 1$. This can always be done since $A^{0} > A^{1}$. This shows that there always exists a frame in which a time-like 4-vector has zero spatial components.

\subproblem{b}
Suppose that in a reference frame $S$, the two events $A^{\mu}$ and $B^{\mu}$ are simultaneous. Then we have $A^{0} = B^{0}$ or their separation satisfies:

\begin{equation}
    X^{\mu} = A^{\mu} - B^{\mu} = (0, X^{1}, X^{2}, X^{3})
\end{equation}

That is, the two events are separated by a space-like interval.

Suppose on the contrary that in a reference frame $S$, the two events $A^{\mu}$ and $B^{\mu}$ are separated by a time-like interval. Then their separation satisfies $X^{\mu} X_{\mu} > 0$. We attempt to find a Lorentz transformation such that $X'^{0} = 0$. Again, we align our axes such that the only non-zero components of $X^{\mu}$ are $X^{0}$ and $X^{1}$. Then we have:

\begin{equation}
    X^{\mu} X_{\mu} = -X^{0} X^{0} + X^{1} X^{1} > 0
\end{equation}

and a Lorentz transformation along the $x$-axis:

\begin{equation}
    X'^{0} = \gamma (X^{0} - \beta X^{1})
\end{equation}

which is an equality satisfied by $\beta = X^{0}/X^{1} < 1$.

This shows that two events are simultaneous if and only if they are separated by a space-like interval in some reference frame.

\subproblem{c}
Consider two events separated by a time-like interval $X^{\mu} X_{\mu} < 0$. We define the temporal order of the events as the sign of $X^{0}$. For this sign to be conserved under Lorentz transformations, we consider a general Lorentz transformation:

\begin{equation}
    X'^{0} = \gamma (X^{0} - \beta_{x} X^{1} - \beta_{y} X^{2} - \beta_{z} X^{3})
\end{equation}

Again, we can align our axes such that the only non-zero components of $X^{\mu}$ are $X^{0}$ and $X^{1}$. Then we have $X'^{0} = \gamma (X^{0} - \beta_{x} X^{1})$, where $\beta_{x} < 1$. Let us demand $\gamma = \Lambda^{0}_{0} > 0$.

We have demanded $X^{\mu}$ to be time-like, that is $X^{0}X^{0} > X^{1}X^{1}$. Say $X^{0} > 0$. It is clear that there is no way to choose $\beta_{x}$ or $X^{1}$ such that $X'^{0} < 0$. The same argument can be made for $X^{0} < 0$. This shows that the temporal order of two events separated by a time-like interval is always conserved under Lorentz transformations satisfying $\gamma > 0$. The complete opposite happens if $\gamma < 0$, which yields an improper Lorentz transformation.

\subproblem{d}
If two 4-vectors are orthogonal, they satisfy:

\begin{equation}
    A^{\mu} B_{\mu} = A^{0} B^{0} - A^{1} B^{1} - A^{2} B^{2} - A^{3} B^{3} = 0
\end{equation}

Suppose that $A^{\mu}$ is time-like so that $A^{0} A_{0} > A^{1} A_{1} + A^{2} A_{2} + A^{3} A_{3}$. We can orient our axes such that only $A^{0}$ and $A^{1}$ are non-zero. Then we have:

\begin{equation}
    A^{\mu} B_{\mu} = A^{0} B^{0} - A^{1} B^{1} = 0
\end{equation}

This implies:

\begin{equation}
    (B^{0})^{2} = \left( \frac{A^{1}}{A^{0}} \right)^{2} (B^{1})^{2} < (B^{1})^{2}
\end{equation}

which implies that $B^{\mu}$ must be space-like.

\subproblem{e}
Suppose instead that $A^{\mu}$ is light-like so that $A^{0} A_{0} = A^{1} A_{1}$. We have:

\begin{equation}
    (B^{0})^{2} = \left( \frac{A^{1}}{A^{0}} \right)^{2} (B^{1})^{2} = (B^{1})^{2}
\end{equation}

Unless $B^{2} = B^{3} = 0$, $B^{\mu}$ will be time-like. If $B^{2} = B^{3} = 0$, then $B^{\mu}$ is light-like.

\subproblem{f}
The world line of an observer is necessarily time-like. Thus, if some displacement vector is orthogonal to the world line, it must be space-like. By previous results, there must exist a frame in which the displacement vector has zero temporal component, i.e. the events are simultaneous. The position of the observer can be described by a time-like 4-vector $A^{\mu}$ and the displacement vector by a space-like 4-vector $X^{\mu}$. Since they are orthogonal, they satisfy $A^{\mu} X_{\mu} = 0$. Let us align our axes such that only $X^{0}$ and $X^{1}$ are non-zero. Then we have:

\begin{equation}
    A^{\mu} X_{\mu} = A^{0} X^{0} - A^{1} X^{1} = 0
\end{equation}

Then we choose a Lorentz transformation along the $x$-axis such that:

\begin{equation}
    X'^{0} = \gamma (X^{0} - \beta X^{1}) = 0
\end{equation}

This is a condition on $\beta = X^{0}/X^{1}$. But from the orthogonality condition, we have $A^{0} X^{0} = A^{1} X^{1}$ so that $\beta = A^{1}/A^{0}$. Let us consider transforming the observer's position vector:

\begin{equation}
    \begin{split}
        A'^{0} &= \gamma (A^{0} - \beta A^{1}) \\
        A'^{1} &= \gamma (A^{1} - \beta A^{0}) = 0 \\
        A'^{2} &= A^{2} \\
        A'^{3} &= A^{3}
    \end{split}
\end{equation}
\qed


\problem{3}{Motion under a constant force}

\subproblem{a}
The particle in the constant electric field has the momentum:

\begin{equation}
    P^{\mu} = (E/c, qE_{x}t)
\end{equation}

where $E = mc^{2} + qE_{x}x$ is the total energy of the particle.
Consider the invariant length of the momentum:

\begin{equation}
    P^{\mu} P_{\mu} = -\frac{E^{2}}{c^{2}} + q^{2}E_{x}^{2}t^{2} = -m^{2}c^{2}
\end{equation}

Expanding the expression for $E$ and simplifying, we have the equation of motion:

\begin{equation}
    x^{2} + \frac{2mc^{2}}{qE_{x}} x - c^{2}t^{2} = 0
\end{equation}

which is equivalent to:

\begin{equation}
    \left( x + \frac{c^{2}}{\alpha} \right)^{2} - c^{2}t^{2} = \frac{c^{4}}{\alpha^{2}}
\end{equation}

where we claim that $\alpha = qE_{x}/m$ is the proper acceleration of the particle.

\subproblem{b}
Consider the acceleration of the particle in frame $S$.

\begin{equation}
    \begin{split}
        A^{\mu} &= \frac{1}{m} \gamma_{v} \frac{dP^{\mu}}{dt} \\
        &= \frac{1}{m} \gamma_{v} \left( \frac{qE_{x}}{c} \frac{\mathrm{d}x}{\mathrm{d}t}, qE_{x} \right)
    \end{split}
\end{equation}

Now we have the proper acceleration:

\begin{equation}
    a_{0}^{2} = A^{\mu} A_{\mu} = \gamma_{v}^{2} \left[ -\frac{q^{2}E_{x}^{2}}{m^{2}c^{2}} \left( \frac{\mathrm{d}x}{\mathrm{d}t} \right)^{2} + \frac{q^{2}E_{x}^{2}}{m^{2}} \right] = \frac{q^{2}E_{x}^{2}}{m^{2}}
\end{equation}

which confirms our previous claim.

\subproblem{c}
At point $P(ct, x)$, we construct an instantaneous inertial frame $S'$ with its origin coinciding with that of $S$. As observed in $S'$, the the point $P$ has the coordinates:

\begin{equation}
    P' = \gamma_{v} (ct - \beta x, x - \beta ct)
\end{equation}

so $ct_{p}' = \gamma_{v} (ct - \beta x)$.

As observed in $S'$, the point $A(0, -c^{2}/\alpha)$ has the coordinates:

\begin{equation}
    A' = \gamma_{v} \left( \beta \frac{c^{2}}{\alpha}, -\frac{c^{2}}{\alpha} \right)
\end{equation}

so $ct_{A}' = \gamma_{v} \beta (c^{2}/\alpha)$.

But we know from the equation of motion that $c^{2}/\alpha + x = (c^{4}/\alpha^{2} + c^{2}t^{2})^{1/2}$. Therefore, $ct_{A}' = -\gamma_{v} \beta (c^{4}/\alpha^{2} + c^{2}t^{2})^{1/2}$.

Let us compute the `slope' of the line $P'A'$ in the $S'$ frame:

\begin{equation}
    \frac{ct_{P}' - ct_{A}'}{x_{P}' - x_{A}'} = \frac{ct - \beta x - \beta c^{2}/\alpha}{x - \beta ct + c^{2}/\alpha}
\end{equation}
\qed


\problem{4}{Circular motion in a magnetic field}

\subproblem{a}
For a pure magnetic field, the Lagrangian of a charged particle is:

\begin{equation}
    \mathcal{L} = -mc^{2} \sqrt{1 - \frac{v^{2}}{c^{2}}} + q \mathbf{v} \cdot \mathbf{A}
\end{equation}

where $\mathbf{A}$ only has spatial components.

We identify the 4-momentum:

\begin{equation}
    P^{\mu} = \frac{\partial \mathcal{L}}{\partial \mathbf{v}} = \gamma m \mathbf{v} + q \mathbf{A}
\end{equation}

The Hamiltonian (energy) of the particle is:

\begin{equation}
    \begin{split}
        \mathcal{H} &= \mathbf{v} \cdot \frac{\partial \mathcal{L}}{\partial \mathbf{v}} - \mathcal{L} \\
        &= \gamma m v^{2} + q \mathbf{v} \cdot \mathbf{A} + \frac{1}{\gamma} mc^{2} - q \mathbf{v} \cdot \mathbf{A} \\
        &= \gamma m v^{2} + \frac{1}{\gamma} mc^{2} \\
        &= \gamma m c^{2}
    \end{split}
\end{equation}

The equation of motion is given by the Euler-Lagrange equation:

\begin{equation}
    \frac{\mathrm{d}}{\mathrm{d}t} \left( \gamma m \mathbf{v} + q \mathbf{A} \right) = \frac{\partial }{\partial \mathbf{r}} \left( q \mathbf{A} \cdot \mathbf{v} \right)
\end{equation}

Consider the vector identity:

\begin{equation}
    \nabla \left( \mathbf{A} \cdot \mathbf{v} \right) = (\mathbf{A} \cdot \nabla) \mathbf{v} + (\mathbf{v} \cdot \nabla) \mathbf{A} + \mathbf{A} \times (\nabla \times \mathbf{v}) + \mathbf{v} \times (\nabla \times \mathbf{A}) = (\mathbf{v} \cdot \nabla) \mathbf{A} + \mathbf{v} \times (\nabla \times \mathbf{A})
\end{equation}

since the derivatives are taken with $\mathbf{v}$ held constant.

On the other hand:

\begin{equation}
    \frac{\mathrm{d}A^{i}}{\mathrm{d}t} = \frac{\partial A^{i}}{\partial t} + \frac{\partial A^{i}}{\partial x^{j}} v^{j} = \frac{\partial \mathbf{A}}{\partial t} + (\mathbf{v} \cdot \nabla) \mathbf{A}
\end{equation}

Combining, we have the equation of motion:

\begin{equation}
    \frac{\mathrm{d}\mathbf{p}}{\mathrm{d}t} = -q \frac{\partial \mathbf{A}}{\partial t} + q \mathbf{v} \times (\nabla \times \mathbf{A})
\end{equation}

For a constant magnetic field $\mathbf{H}$, we can write $\mathbf{A} = \frac{1}{2} \mathbf{H} \times \mathbf{r}$ and $\mathbf{H} = \nabla \times \mathbf{A}$. The equation of motion becomes:

\begin{equation}
    \frac{\mathrm{d}\mathbf{p}}{\mathrm{d}t} = q \mathbf{v} \times \mathbf{H}
\end{equation}

Finally, consider the time derivative of $p^{2}$:

\begin{equation}
    \frac{\mathrm{d}p^{2}}{\mathrm{d}t} = 2 \mathbf{p} \cdot \frac{\mathrm{d}\mathbf{p}}{\mathrm{d}t} = 0
\end{equation}

This is because in the expression for $\dot{\mathbf{p}}$, the right-hand-side is orthogonal to $\mathbf{v}$ and thus $\mathbf{p}$. Thus, we have proven that the particle has constant momentum and velocity. Since the energy is a function of $\gamma$, the energy is also constant.

\subproblem{b}
We rewrite the equation of motion in terms of the velocity:

\begin{equation}
    \frac{E}{c^{2}} \frac{\mathrm{d}\mathbf{v}}{\mathrm{d}t} = q \mathbf{v} \times \mathbf{H}
\end{equation}

This is solved by:

\begin{equation}
    \begin{split}
        v_{x} = \frac{v_{0}}{\sqrt{2}} \cos{(\omega t)} \\
        v_{y} = \frac{v_{0}}{\sqrt{2}} \sin{(\omega t)} \\
    \end{split}
\end{equation}

where $\omega = qc^{2}H/E$ and $v_{0} = pc^{2}/E$.

Integrating the equations of motion, we have a condition on the radius of the circular motion:

\begin{equation}
    r = \frac{v_{0}}{\omega} = \frac{pc^{2}}{qc^{2}H} = \frac{p}{qH}
\end{equation}

\subproblem{c}
In $S$, the 4-velocity of the particle is $U^{\mu} = \gamma_{v} (c, v_{x}, v_{y}, 0)$. In some other frame $S'$ moving at velocity $u$ in the $x$-direction, the 4-velocity becomes $U'^{\mu} = \gamma_{v'} (c, v'_{x}, v'_{y}, 0)$. But by the Lorentz transformation:

\begin{equation}
    U'^{0} = \gamma_{u} \gamma_{v} (c - \beta_{u} v_{x}) = \gamma_{v'} c
\end{equation}

Since $v_{x}$ is varying, $\gamma_{v'}$ is also varying, implying that $S'$ does not observe the particle to be moving at constant velocity. This is in a moving frame, the field is not purely magnetic so the particle may accelerate.

\subproblem{d}
For synchronisation, we need $\pi/\omega = (1/f)/2$, or:

\begin{equation}
    2\pi f = \omega = \frac{qc^{2}H}{E}
\end{equation}

Since $E$ is increasing after every period, we demand $E$ to be increasing in tandem. This is achieved by having a non-uniform magnetic field $H(r)$ as a function of the energy $E$. The initial energy is $\Delta E$ so the magnetic field at the centre is:

\begin{equation}
    H_{i} = \frac{2\pi f \Delta E}{qc^{2}}
\end{equation}

The final energy is $E_{f}$, leading to:

\begin{equation}
    H_{f} = \frac{2\pi f E_{f}}{qc^{2}}
\end{equation}

Upon exiting the cyclotron, the momentum of the particle satisfies $p = rqH_{f}$ or:

\begin{equation}
    p = rqH_{f} = rq \left( \frac{2\pi f E_{f}}{qc^{2}} \right) = 2\pi c^{-2} f r E_{f}
\end{equation}

so that the exit velocity is:

\begin{equation}
    v_{f} = c^{2}\frac{p}{E_{f}} = 2\pi f r = \qty{1.2e12}{ms^{-1}}
\end{equation}

Each revolution leads to $2\Delta E$ increase in energy, so the total number of revolutions is:

\begin{equation}
    N = \frac{E_{f}}{2\Delta E} = 684
\end{equation}

The total time taken is:

\begin{equation}
    T = \frac{N}{f} = \qty{1.14e-6}{s}
\end{equation}
\qed


\problem{5}{Motion in a magnetic dipole}

We have the equation of motion:

\begin{equation}
    \frac{E}{c^{2}} \frac{\mathrm{d}\mathbf{v}}{\mathrm{d}t} = q \mathbf{v} \times (\nabla \times \mathbf{A})
\end{equation}

Consider dotting the equation with $\hat{z}$:

\begin{equation}
    \begin{split}
        \hat{z} \cdot \frac{\mathrm{d}\mathbf{v}}{\mathrm{d}t} &\propto \hat{z} \cdot [\mathbf{v} \times (\nabla \times \mathbf{A})] \\
        &= \mathbf{v} \cdot [(\nabla \times \mathbf{A}) \times \hat{z}] \\
    \end{split}
\end{equation}

Consider the following vector identity:

\begin{equation}
    \nabla (\hat{z} \cdot \mathbf{A}) = (\hat{z} \cdot \nabla) \mathbf{A} + (\mathbf{A} \cdot \nabla) \hat{z} + \hat{z} \times (\nabla \times \mathbf{A}) + \mathbf{A} \times (\nabla \times \hat{z})
\end{equation}

Now, $\mathbf{A}$ does not have a $z$ component, so the left-hand-side is zero. On the right-hand-side, the first term is zero since $\mathbf{A}$ does not have a $z$ component. The second and last terms are zero since $\hat{z}$ is a constant vector. Thus, we must have $\mathbf{v} \cdot (\nabla \times \mathbf{A}) = 0$. This means that $\hat{z} \cdot \dot{\mathbf{v}} = 0$ and the motion is confined to the $xy$-plane.

We may now use cylindrical coordinates $(r, \phi, z)$ to describe the motion. Consider the vector potential:

\begin{equation}
    \mathbf{A} = \frac{\mu_{0}}{4\pi} \frac{M \hat{z} \times \mathbf{r}}{r^{3}} = \frac{\mu_{0} M}{4\pi r^{2}} \hat{\phi}
\end{equation}

We can write the Lagrangian:

\begin{equation}
    \mathcal{L} = -\frac{mc^{2}}{\gamma} + q \mathbf{v} \cdot \mathbf{A} = -mc^{2} \left( 1 - \frac{\dot{r}^{2} + r^{2} \dot{\phi}^{2}}{c^{2}} \right)^{1/2} + \frac{q\mu_{0} M}{4\pi} \frac{1}{r} \dot{\phi}
\end{equation}

Apparently $\phi$ is a cyclic coordinate so the angular momentum is conserved:

\begin{equation}
    p_{\phi} = \frac{\partial \mathcal{L}}{\partial \dot{\phi}} = \gamma m r^{2} \dot{\phi} + \frac{q\mu_{0} M}{4\pi} \frac{1}{r} = \text{constant}
\end{equation}

Apparently the energy is also conserved, so we may write $E = \gamma m c^{2}$ as a constant. Now consider setting $\dot{r} = 0$ so that $v^{2} = r^{2} \dot{\phi}^{2}$. We have from the energy conservation:

\begin{equation}
    \begin{split}
        \frac{m^{2}c^{4}}{E^{2}} &= 1 - \frac{r^{2} \dot{\phi}^{2}}{c^{2}} \\
        \frac{m^{2}c^{4}}{E^{2}} &= 1 - \frac{r^{2}}{c^{2}} \left[ \frac{p_{\phi} - \frac{q\mu_{0} M}{4\pi} \frac{1}{r}}{(E/c^{2}) r} \right]^{2} \\
    \end{split}
\end{equation}

From the initial condition, we know $p_{\phi} = q\mu_{0} M/(4\pi r_{0})$ and $E = mc^{2}(1 - v_{0}^{2}/c^{2})^{-1/2}$. We can solve for $r$:

\begin{equation}
    r = r_{0} \left( 1 \pm \frac{4\pi}{q\mu_{0} M} \frac{E}{c} \sqrt{1 - \frac{m^{2}c^{4}}{E^{2}}} \right)^{-1}
\end{equation}
\qed


\problem{6}{Relativistic rocket}


We know that rapidity is additive under Lorentz transformations. Consider the rocket in its instantaneous rest frame $S'$ which moves at velocity $v$ relative to the Earth $S$. In a short time interval $\mathrm{d}\tau$, the rocket acquires a velocity $\mathrm{d}v$ relative to $S'$. Consider its 4-momentum, which was initially $\mathbf{P} = (m(\tau)c, 0)$. Since the fuel is turned into photons, the 4-momentum becomes:

\begin{equation}
    \mathbf{P} + \mathrm{d}\mathbf{P} = (m c - \mathrm{d}m c, \mathrm{d}m c)
\end{equation}

Therefore, we have the increment in velocity:

\begin{equation}
    \begin{split}
        \gamma_{v} (m - \mathrm{d}m) \mathrm{d}v = \mathrm{d}m c \\
        \mathrm{d}v \approx \frac{c}{m} \mathrm{d}m
    \end{split}
\end{equation}

On the other hand, the $\mathrm{d}\rho = \tanh^{-1}(\mathrm{d}v/c) \approx \mathrm{d}v/c$ so that:

\begin{equation}
    \mathrm{d}\rho \approx \frac{\mathrm{d}m}{m}
\end{equation}

where $\mathrm{d}m$ is a negative quantity.

This equation is also satisfied in the Earth frame as rapidity is additive. Changing to velocity in the Earth frame, we have:

\begin{equation}
    \frac{\mathrm{d}M}{M} = -\frac{\mathrm{d}v/c}{1 - v^{2}/c^{2}}
\end{equation}

Integrating, we have:

\begin{equation}
    \rho = \tanh^{-1} \left( \frac{v}{c} \right) = -\ln{\left( \frac{M_{f}}{M_{i}} \right)}
\end{equation}

The energy of the rocket is:

\begin{equation}
    E = \frac{mc^{2}}{1 - \rho^{2}}
\end{equation}

When $M_{f} = 0$, we have $\rho \to \infty$ so the energy of the rocket tends to zero.
\qed


\end{document}
\documentclass[12pt]{article}
\usepackage{homework}
\pagestyle{fancy}

% assignment information
\def\course{Symmetry and Relativity}
\def\assignmentno{Problem Set 2}
\def\assignmentname{Symmetries}
\def\name{Xin, Wenkang}
\def\time{\today}

\begin{document}

\begin{titlepage}
    \begin{center}
        \large
        \textbf{\course}

        \vfill

        \Huge
        \textbf{\assignmentno}

        \vspace{1.5cm}

        \large{\assignmentname}

        \vfill

        \large
        \name

        \time
    \end{center}
\end{titlepage}


%==========
\pagebreak
\section*{}
%==========


\problem{1}{Derivation of the rotation formula}
Given that:

\begin{equation}
    \hat{\mathbf{u}} \cdot \mathbf{J} =
    \begin{bmatrix}
        0      & -u_{z} & u_{y}  \\
        u_{z}  & 0      & -u_{x} \\
        -u_{y} & u_{x}  & 0
    \end{bmatrix}
\end{equation}

we have its square:

\begin{equation}
    \begin{split}
        (\hat{\mathbf{u}} \cdot \mathbf{J})^{2} &=
        \begin{bmatrix}
            0      & -u_{z} & u_{y}  \\
            u_{z}  & 0      & -u_{x} \\
            -u_{y} & u_{x}  & 0
        \end{bmatrix}
        \begin{bmatrix}
            0      & -u_{z} & u_{y}  \\
            u_{z}  & 0      & -u_{x} \\
            -u_{y} & u_{x}  & 0
        \end{bmatrix} \\
        &=
        \begin{bmatrix}
            -u_{y}^{2} - u_{z}^{2} & u_{x}u_{y}             & u_{x}u_{z}             \\
            u_{x}u_{y}             & -u_{x}^{2} - u_{z}^{2} & u_{y}u_{z}             \\
            u_{x}u_{z}             & u_{y}u_{z}             & -u_{x}^{2} - u_{y}^{2}
        \end{bmatrix} \\
    \end{split}
\end{equation}

Let us explicitly check:

\begin{equation}
    (\hat{\mathbf{u}} \cdot \mathbf{J}) \mathbf{v} =
    \begin{bmatrix}
        u_{y}v_{z} - u_{z}v_{y} \\
        u_{z}v_{x} - u_{x}v_{z} \\
        u_{x}v_{y} - u_{y}v_{x}
    \end{bmatrix}
    =
    \mathbf{u} \times \mathbf{v}
\end{equation}

and thus:

\begin{equation}
    (\hat{\mathbf{u}} \cdot \mathbf{J})^{2} \mathbf{v} = \mathbf{u} \times (\mathbf{u} \times \mathbf{v}) = \mathbf{u} (\mathbf{u} \cdot \mathbf{v}) - \mathbf{u}^{2} \mathbf{v} = \mathbf{u} (\mathbf{u} \cdot \mathbf{v}) - \mathbf{v}
\end{equation}

This means:

\begin{equation}
    \begin{split}
        R \mathbf{v} &= \mathbf{v} + (1 - \cos{\theta}) (\hat{\mathbf{u}} \cdot \mathbf{J})^{2} \mathbf{v} + \sin{\theta} \hat{\mathbf{u}} \cdot \mathbf{J} \mathbf{v} \\
        &= \mathbf{v} + (1 - \cos{\theta}) (\mathbf{u} (\mathbf{u} \cdot \mathbf{v}) - \mathbf{v}) + \sin{\theta} (\mathbf{u} \times \mathbf{v}) \\
        &= \cos{\theta} \mathbf{v} + \sin{\theta} (\mathbf{u} \times \mathbf{v}) + (1 - \cos{\theta}) \mathbf{u} (\mathbf{u} \cdot \mathbf{v})
    \end{split}
\end{equation}
\qed


\problem{2}{Lorentz transformation for a boost in an arbitrary direction}


\subproblem{a}
Consider the form of $\hat{\mathbf{u}} \cdot \mathbf{K}$:

\begin{equation}
    \hat{\mathbf{u}} \cdot \mathbf{K} =
    \begin{bmatrix}
        0     & n_{x} & n_{y} & n_{z} \\
        n_{x} & 0     & 0     & 0     \\
        n_{y} & 0     & 0     & 0     \\
        n_{z} & 0     & 0     & 0
    \end{bmatrix}
\end{equation}

Then:

\begin{equation}
    \begin{split}
        (\hat{\mathbf{u}} \cdot \mathbf{K})^{3} &=
        \begin{bmatrix}
            0     & n_{x} & n_{y} & n_{z} \\
            n_{x} & 0     & 0     & 0     \\
            n_{y} & 0     & 0     & 0     \\
            n_{z} & 0     & 0     & 0
        \end{bmatrix}
        \begin{bmatrix}
            1 & 0          & 0          & 0          \\
            0 & n_{x}^{2}  & n_{x}n_{y} & n_{x}n_{z} \\
            0 & n_{x}n_{y} & n_{y}^{2}  & n_{y}n_{z} \\
            0 & n_{x}n_{z} & n_{y}n_{z} & n_{z}^{2}
        \end{bmatrix} \\
        &=
        \begin{bmatrix}
            0     & n_{x} & n_{y} & n_{z} \\
            n_{x} & 0     & 0     & 0     \\
            n_{y} & 0     & 0     & 0     \\
            n_{z} & 0     & 0     & 0
        \end{bmatrix} \\
        &= \hat{\mathbf{u}} \cdot \mathbf{K}
    \end{split}
\end{equation}

Thus all odd powers of $\hat{\mathbf{u}} \cdot \mathbf{K}$ are identical to the first power, and all even powers are identical to the second power. This means that:

\begin{equation}
    \begin{split}
        \Lambda &= \exp{-\zeta \hat{\mathbf{u}} \cdot \mathbf{K}} \\
        &= \sum_{n = 0}^{\infty} \frac{(-\zeta)^{n}}{n!} (\hat{\mathbf{u}} \cdot \mathbf{K})^{n} \\
        &= \mathbb{I} - \sum_{\text{odd } n} \frac{\zeta^{n}}{n!} (\hat{\mathbf{u}} \cdot \mathbf{K}) + \sum_{\text{even } n} \frac{\zeta^{n}}{n!} (\hat{\mathbf{u}} \cdot \mathbf{K})^{2} \\
        &= \mathbb{I} - (\sinh{\zeta}) \hat{\mathbf{u}} \cdot \mathbf{K} + (\cosh{\zeta} - 1) (\hat{\mathbf{u}} \cdot \mathbf{K})^{2}
    \end{split}
\end{equation}

\subproblem{b}
Suppose $\mathbf{r} = \mathbf{v} t$ and $\mathbf{r}' = \mathbf{0}$. We demand that $\Lambda \mathbf{r} = \mathbf{r}'$ so that:

\begin{equation}
    \begin{bmatrix}
        ct' \\
        0   \\
        0   \\
        0
    \end{bmatrix}
    =
    \begin{bmatrix}
        ct \\
        x  \\
        y  \\
        z
    \end{bmatrix}
    - (\sinh{\zeta})
    \begin{bmatrix}
        0     & n_{x} & n_{y} & n_{z} \\
        n_{x} & 0     & 0     & 0     \\
        n_{y} & 0     & 0     & 0     \\
        n_{z} & 0     & 0     & 0
    \end{bmatrix}
    \begin{bmatrix}
        ct \\
        x  \\
        y  \\
        z
    \end{bmatrix}
    + (\cosh{\zeta} - 1)
    \begin{bmatrix}
        1 & 0          & 0          & 0          \\
        0 & n_{x}^{2}  & n_{x}n_{y} & n_{x}n_{z} \\
        0 & n_{x}n_{y} & n_{y}^{2}  & n_{y}n_{z} \\
        0 & n_{x}n_{z} & n_{y}n_{z} & n_{z}^{2}
    \end{bmatrix} \\
    \begin{bmatrix}
        ct \\
        x  \\
        y  \\
        z
    \end{bmatrix}
\end{equation}

Now focus on the $x$ component of the equation:

\begin{equation}
    \begin{split}
        0 &= x - \sinh{\zeta} n_{x} ct + (\cosh{\zeta} - 1) n_{x} (n_{x} x + n_{y} y + n_{z} z) \\
        &= x - \sinh{\zeta} n_{x} ct + (\cosh{\zeta} - 1) n_{x} (\mathbf{n} \cdot \mathbf{v}) t
    \end{split}
\end{equation}

Now note that $\mathbf{n} = \mathbf{v}/\left\lvert \mathbf{v} \right\rvert$ so that $v_{x} = v n_{x}$. Thus:

\begin{equation}
    0 = v n_{x} t - \sinh{\zeta} n_{x} ct + (\cosh{\zeta} - 1) n_{x} v t
\end{equation}

solving which gives $\tanh{\zeta} = v/c = \beta$.

Then:

\begin{equation}
    \begin{split}
        \Lambda &= \mathbb{I} - \frac{\beta}{\sqrt{1 - \beta^{2}}} \hat{\mathbf{u}} \cdot \mathbf{K} + \left( \frac{1}{\sqrt{1 - \beta^{2}}} - 1 \right) (\hat{\mathbf{u}} \cdot \mathbf{K})^{2} \\
        &=
        \begin{bmatrix}
            \gamma              & -\gamma \beta n_{x}        & -\gamma \beta n_{y}        & -\gamma \beta n_{z}        \\
            -\gamma \beta n_{x} & 1 + (\gamma - 1) n_{x}^{2} & (\gamma - 1) n_{x} n_{y}   & (\gamma - 1) n_{x} n_{z}   \\
            -\gamma \beta n_{y} & (\gamma - 1) n_{x} n_{y}   & 1 + (\gamma - 1) n_{y}^{2} & (\gamma - 1) n_{y} n_{z}   \\
            -\gamma \beta n_{z} & (\gamma - 1) n_{x} n_{z}   & (\gamma - 1) n_{y} n_{z}   & 1 + (\gamma - 1) n_{z}^{2}
        \end{bmatrix}
    \end{split}
\end{equation}


\problem{3}{Decomposing a Lorentz transformation into boost and rotation}


Define a matrix $L$ that satisfies $L^{\intercal}gL$ where $g = \operatorname{diag}{(-1, 1, 1, 1)}$. It being in $SO(1, 3)$, we need $L^{\intercal} = L$.


\subproblem{a}
Writing the relation in index notation, the inverse of $L$ is defined as:

\begin{equation}
    (L^{-1})_{ij} = L^{ij} = g^{ik}L_{k}^{j} = g^{ik}g^{jl}L_{kl}
\end{equation}

From the relation, we have:

\begin{equation}
    \begin{split}
        L_{ik}g^{kl}(L^{\intercal})_{lj} &= g_{ij} \\
        L_{ik}g_{kl}L_{lj} &= g_{ij}
    \end{split}
\end{equation}

First consider $i = j = 0$ so that the RHS is $-1$. The non-zero components of the LHS are obtained by setting $k = l$ so that:

\begin{equation}
    -L_{00}L_{00} + \sum_{\alpha = 1, 2, 3} L_{0\alpha}L_{0\alpha} = -1
\end{equation}

Consider $i \ne j$ so that the RHS is zero. The non-zero components of the LHS are obtained by setting $k = l$ so that:

\begin{equation}
    -L_{i0}L_{0j} + \sum_{\alpha = 1, 2, 3} L_{i\alpha}L_{\alpha j} = 0
\end{equation}

Now take $j = 0$ so that $i$ takes $1, 2, 3$:

\begin{equation}
    L_{00}L_{i0} - \sum_{\alpha = 1, 2, 3} L_{i\alpha}L_{\alpha 0} = 0
\end{equation}

\subproblem{b}
We demand that transformations of coordinates between two frames is facilitated by $L$, i.e. $X'^{\mu} = L^{\mu}_{\nu} X^{\nu} = g^{\mu \rho} L_{\rho \nu} X^{\nu}$. Consider the origin of frame $S'$ so that $X'^{\mu} = (ct', 0, 0, 0)$. Setting $\mu = 0$ demands $\rho = 0$, leading to:

\begin{equation}
    ct' = -L_{0 \nu} X^{\nu}
\end{equation}

For $i = 1, 2, 3$:

\begin{equation}
    0 = L_{i \nu} X^{\nu} = L_{i 0} X^{0} + L_{i j} X^{j}
\end{equation}

Multiply the second equation by $L_{00}$:

\begin{equation}
    \begin{split}
        0 &= L_{00}L_{i 0}X^{0} + L_{00}L_{i j} X^{j} \\
        &= L_{i\alpha}L_{\alpha 0} X^{0} + L_{00}L_{i j} X^{j} \\
        &= L_{ij} (L_{j0} X^{0} + L_{00} X^{j})
    \end{split}
\end{equation}

where at the third equality we relabelled $\alpha \to j$.

But we know that $X^{\nu}$ has to be $(ct, \mathbf{v}t)$ so $X^{0} = ct$ and $X^{j} = v^{j}t$. This implies:

\begin{equation}
    \beta^{i} \equiv \frac{v_{i}}{c} = -\frac{L_{i0}}{L_{00}}
\end{equation}

which determines $L$ up to $L_{00}$.

Using the first equation we derived, we have:

\begin{equation}
    \begin{split}
        -L_{00}L_{00} + \sum_{\alpha = 1, 2, 3} L_{0\alpha}L_{0\alpha} &= -1 \\
        -1 + \beta^{2} &= -L_{00}^{-2}
    \end{split}
\end{equation}

which immediately gives $L_{00} = \gamma \equiv (1 - \beta^{2})^{-1/2}$.

\subproblem{c}
To demonstrate $R = L\Lambda^{-1}$ is in $SO(1, 3)$, it suffices to check the inner product condition:

\begin{equation}
    R^{\intercal} g R = (\Lambda^{-1})^{\intercal} L^{\intercal} g L \Lambda^{-1} = (\Lambda^{-1})^{\intercal} g \Lambda^{-1} = g
\end{equation}

since $\Lambda$ preserves inner product.

From the previous problem we have:

\begin{equation}
    \Lambda^{-1}(\beta) = \Lambda(-\beta) = \mathbb{I} + \gamma \beta \mathbf{n} \cdot \mathbf{K} + (\gamma - 1) (\mathbf{n} \cdot \mathbf{K})^{2}
\end{equation}

Then:

\begin{equation}
    \begin{split}
        R_{00} &= L_{0\mu} [\Lambda^{-1}(\beta)]_{\mu 0} \\
        &= L_{0\mu} \left( \delta_{\mu 0} + \gamma \beta n_{\mu} + (\gamma - 1) \delta_{0 \mu} \right) \\
        &= \gamma L_{00} + \gamma \beta L_{0i} n_{i} \\
        &= \gamma^{2} + \gamma \beta (-\gamma \beta^{i}) n_{i} \\
        &= \gamma^{2} (1 - \beta^{2}) \\
        &= 1
    \end{split}
\end{equation}

since $n_{i} = v_{i}/\left\lvert \mathbf{v} \right\rvert = \beta_{i}/\beta$
\qed


\problem{4}{Lorentz transformations and their expressions through SU(2)}

\subproblem{a}
Suppose $M^{\dagger}M = MM^{\dagger} = \mathbb{I}$. Consider $M = \mathbb{I} + X$. To first order in $X$, we need:

\begin{equation}
    X^{\dagger} = -X
\end{equation}

Assume a $X$ in the form:

\begin{equation}
    X =
    \begin{pmatrix}
        a & b \\
        c & d
    \end{pmatrix}
\end{equation}

We need the equations:

\begin{equation}
    \begin{split}
        a + a^{*} &= d + d^{*} = 0 \\
        b + c^{*} &= 0 \\
    \end{split}
\end{equation}

The first line implies that $a$ and $d$ are purely imaginary but otherwise independent. For the second line, we write $c = -b^{*}$ so that $c$ is determined by $b$. This gives three independent parameters, which means that there must be three independent generators.

\subproblem{b}
Since $J_{i}$ are generators of $SU(2)$, they satisfy $J_{i}^{\dagger} = -J_{i}$. On the other hand, if we write:

\begin{equation}
    M = \mathbb{I} + i z_{1} J_{1} + i z_{2} J_{2} + i z_{3} J_{3}
\end{equation}

we require:

\begin{equation}
    \begin{split}
        M^{\dagger}M &= \mathbb{I} \\
        i(z_{1}J_{1} - z_{1}^{*}J_{1}\dagger) + i(z_{2}J_{2} - z_{2}^{*}J_{2}\dagger) + i(z_{3}J_{3} - z_{3}^{*}J_{3}\dagger) &= 0
    \end{split}
\end{equation}

Since $J_{i}^{\dagger} = -J_{i}$, we have $z_{i} = -z_{i}^{*}$, which means that $z_{i}$ are purely imaginary. Thus:

\begin{equation}
    K_{i}^{\dagger} = (z_{i}J_{i})^{\dagger} = -z_{i}^{*}J_{i}^{\dagger} = z_{i}J_{i} = K_{i}
\end{equation}

Suppose we write $K_{i}$ in the form:

\begin{equation}
    K_{i} =
    \begin{pmatrix}
        \alpha & \beta  \\
        \gamma & \delta
    \end{pmatrix}
\end{equation}

For it to be Hermitian, we need $\alpha = \alpha^{*}$ and $\delta = \delta^{*}$, which means they are real. We also need $\beta = \gamma^{*}$, which means that $\beta$ and $\gamma$ are complex conjugates of each other. In a more symmetric form, we can write:

\begin{equation}
    K_{i} =
    \begin{pmatrix}
        a + d  & b - ic \\
        b + ic & a - d
    \end{pmatrix}
    =
    a\mathbb{I} + b\sigma_{1} + c\sigma_{2} + d\sigma_{3}
\end{equation}

which gives:

\begin{equation}
    M = \mathbb{I} - i(a_{1} \sigma_{1} + a_{2} \sigma_{2} + a_{3} \sigma_{3})
\end{equation}

\subproblem{c}
We have demonstrated via $K_{i}$ that any $2 \times 2$ Hermitian matrix $S$ can be written as a linear combination of $\mathbb{I}$ and $\sigma_{i}$. Let us rename the coefficients of $\sigma_{i}$:

\begin{equation}
    S = ct\mathbb{I} + x\sigma_{1} + y\sigma_{2} + z\sigma_{3}
\end{equation}

The determinant of $S$ is:

\begin{equation}
    \det(S) = c^{2}t^{2} - x^{2} - y^{2} - z^{2} \\
\end{equation}

Consider the transformation $S' = LSL^{\dagger}$. We can show that $S'$ is Hermitian:

\begin{equation}
    (S')^{\dagger} = (LSL^{\dagger})^{\dagger} = (L^{\dagger})^{\dagger} S^{\dagger} L^{\dagger} = LSL^{\dagger} = S'
\end{equation}

Thus $L$ can be viewed as a Lorentz transformation and $S$ and $S'$ represent spacetimes events.

\subproblem{d}
Consider $L = \operatorname{diag}{(a, b)}$ with $\det(L) = ab = 1$. Then:

\begin{equation}
    LSL^{\dagger} =
    \begin{pmatrix}
        a & 0 \\
        0 & b
    \end{pmatrix}
    \begin{pmatrix}
        ct & x \\
        y  & z
    \end{pmatrix}
    \begin{pmatrix}
        a & 0 \\
        0 & b
    \end{pmatrix}
    =
    \begin{pmatrix}
        a^{2} ct & ab x    \\
        ab y     & b^{2} z
    \end{pmatrix}
    =
    \begin{pmatrix}
        ct' & x' \\
        y'  & z'
    \end{pmatrix}
\end{equation}

This means that $t' = a^{2} t$, $x' = x$, $y' = y$, and $z' = z/a^{2}$. This is a boost in the $z$ direction with a factor of $a^{2}$.

Consider another form of $L$:

\begin{equation}
    L = (\cosh{q}) \mathbb{I} - (\sinh{q}) \sigma_{3}
    =
    \begin{pmatrix}
        \cosh{q}-\sinh{q} & 0                 \\
        0                 & \cosh{q}+\sinh{q}
    \end{pmatrix}
\end{equation}

which apparently satisfies $\det(L) = 1$.

\subproblem{e}
Consider $S = x\sigma_{1} + y\sigma_{2} + z\sigma_{3}$. Then:

\begin{equation}
    \tr(S) = x\tr(\sigma_{1}) + y\tr(\sigma_{2}) + z\tr(\sigma_{3}) = 0
\end{equation}

Let us demand $\tr(S') = 0$ as well. Then:

\begin{equation}
    \tr(S') = \tr(LSL^{\dagger}) = \tr(LL^{\dagger}S) = 0 = \tr(S)
\end{equation}

which demands $L$ to be unitary, i.e. $L$ is in $SU(2)$.

We can thus write $L$ in the form:

\begin{equation}
    L = \exp[-i(a_{1} \sigma_{1} + a_{2} \sigma_{2} + a_{3} \sigma_{3})]
\end{equation}

Let $a_{1} = a_{2} = 0$ and $a_{3} = \theta$, then:

\begin{equation}
    \begin{split}
        L &= \exp[-i\theta \sigma_{3}] \\
        &= \sum_{n = 0}^{\infty} \frac{(-i\theta)^{n}}{n!} \sigma_{3}^{n} \\
    \end{split}
\end{equation}

However, it is easy to check that $\sigma_{3}^{n}$ is $\mathbb{I}$ for even $n$ and $\sigma_{3}$ for odd $n$. Thus:

\begin{equation}
    \begin{split}
        L &= \sum_{\text{even } n} \frac{(-i\theta)^{n}}{n!} \mathbb{I} + \sum_{\text{odd } n} \frac{(-i\theta)^{n}}{n!} \sigma_{3} \\
        &= \cos{\theta} \mathbb{I} - i\sin{\theta} \sigma_{3} \\
        &=
        \begin{pmatrix}
            e^{-i\theta} & 0           \\
            0            & e^{i\theta}
        \end{pmatrix}
    \end{split}
\end{equation}
\qed


\end{document}
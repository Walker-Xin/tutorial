\documentclass[12pt]{article}
\usepackage{homework}
\pagestyle{fancy}

% assignment information
\def\course{Condensed Matter Physics}
\def\assignmentno{Problem Set 1}
\def\assignmentname{Einstein, Debye, Drude, and Free Electron Models}
\def\name{Xin, Wenkang}
\def\time{\today}

\begin{document}

\begin{titlepage}
    \begin{center}
        \large
        \textbf{\course}

        \vfill

        \Huge
        \textbf{\assignmentno}

        \vspace{1.5cm}

        \large{\assignmentname}

        \vfill

        \large
        \name

        \time
    \end{center}
\end{titlepage}


%==========
\pagebreak
\section*{}
%==========


\problem{1.1}{Einstein solid}


\subproblem{a}
Given $H = p^{2}/2m + kx^{2}/2$, we have the classical partition function:
\begin{equation}
    \begin{split}
        Z &= \frac{1}{(2\pi \hbar)^{3}} \int e^{-\beta H} \, \mathrm{d}^{3}x \, \mathrm{d}^{3}p \\
        &= \frac{1}{(2\pi \hbar)^{3}} \int e^{-\beta p^{2}/2m} \, \mathrm{d}^{3}p \int e^{-\beta kx^{2}/2} \, \mathrm{d}^{3}x \\
        &= \frac{1}{(2\pi \hbar)^{3}} (4\pi)^{2} \int_{0}^{\infty} e^{-\beta p^{2}/2m} p^{2} \, \mathrm{d}p \int_{0}^{\infty} e^{-\beta kx^{2}/2} x^{2} \, \mathrm{d}x \\
        &= \frac{2}{\pi \hbar^{3}} \frac{\pi}{16} \left( \frac{\beta}{2m} \right)^{-3/2} \left( \frac{k\beta}{2} \right)^{-3/2} \\
        &= \frac{1}{\hbar^{3}} \left( \frac{m}{k} \right)^{3/2} \beta^{-3}
    \end{split}
\end{equation}

The internal energy is given by:
\begin{equation}
    \begin{split}
        U &= -\frac{\partial}{\partial \beta} \ln Z \\
        &= -\frac{\partial}{\partial \beta} (\ln{\beta^{-3}}) \\
        &= 3\beta^{-1} \\
        &= 3k_{B}T
    \end{split}
\end{equation}
which gives a heat capacity of $C_{V} = 3k_{B}$.

If we consider a solid of $N$ atoms, the partition function becomes:
\begin{equation}
    Z_{N} = Z^{N}
\end{equation}
since in a solid, the atoms are distinguishable (by their positions).

The internal energy is then:
\begin{equation}
    \begin{split}
        U_{N} &= -\frac{\partial}{\partial \beta} \ln Z_{N} \\
        &= 3Nk_{B}T
    \end{split}
\end{equation}
leading to a heat capacity of $C_{V} = 3Nk_{B} = 3R$.

\subproblem{b}
For a quantum mechanical harmonic oscillator, the energy levels are quantized as:

\begin{equation}
    E_{j} = \hbar \omega \left( j + \frac{1}{2} \right)
\end{equation}

This gives the partition function:
\begin{equation}
    Z = \sum_{j=0}^{\infty} e^{-\beta \hbar \omega (j + 1/2)} = \frac{e^{-\beta \hbar \omega / 2}}{1 - e^{-\beta \hbar \omega}} = \frac{e^{\beta \hbar \omega / 2}}{e^{\beta \hbar \omega} - 1}
\end{equation}
which can be related to the Bose-Einstein statistics:
\begin{equation}
    Z = e^{-\beta \hbar \omega / 2} n_{B}(\beta \hbar \omega)
\end{equation}

This implies that a quantum harmonic oscillator can be treated as a Bose-Einstein gas with a single quantum state.

The internal energy is given by:

\begin{equation}
    \begin{split}
        U &= -\frac{\partial}{\partial \beta} \ln Z \\
        &= \frac{\hbar \omega}{2} + \hbar \omega \frac{1}{e^{\beta \hbar \omega} - 1}
    \end{split}
\end{equation}
where we should discard the zero-point energy $\hbar \omega / 2$.

The heat capacity is then:

\begin{equation}
    \begin{split}
        C_{V} &= \frac{\partial U}{\partial T} \\
        &= \frac{\partial U}{\partial \beta} \frac{\partial \beta}{\partial T} \\
        &= \frac{\hbar^{2} \omega^{2}}{k_{B}T^{2}} \frac{e^{\beta \hbar \omega}}{(e^{\beta \hbar \omega} - 1)^{2}}
    \end{split}
\end{equation}

At high temperatures, $\beta \hbar \omega \ll 1$, we have $e^{\beta \hbar \omega} \approx 1 + \beta \hbar \omega$, and the internal energy becomes:

\begin{equation}
    U = \frac{\hbar \omega}{2} + \frac{\hbar \omega}{e^{\beta \hbar \omega} - 1} \approx \frac{\hbar \omega}{2} + k_{B}T
\end{equation}

leading to a constant heat capacity of $C_{V} = k_{B}$ as expected from the equipartition theorem.
\qed


\problem{1.2}{Debye Theory}


\subproblem{a}
In Debye theory, we assume the following:

\begin{itemize}
    \item The solid is isotropic and homogeneous.
    \item Atoms behave as quantum harmonic oscillators that do not interact with each other.
    \item Size of the solid is large enough for continuum approximation to be valid.
\end{itemize}

We know the density of states in $k$-space $g(k) \, \mathrm{d}k = 3 \times Vk^{2}/(2\pi^{2})$ where the factor $3$ accounts for the three modes of vibration. Transforming to $\omega$-space using the relation $\omega = vk$:

\begin{equation}
    \tilde{g}(\omega) = g(k) \frac{\mathrm{d}k}{\mathrm{d}\omega} = \frac{3V}{2\pi^{2}} \frac{\omega^{2}}{v^{3}}
\end{equation}

We can then write the internal energy as:

\begin{equation}
    \begin{split}
        U &= \int_{0}^{\omega^{*}} \hbar \omega \tilde{g}(\omega) \left( \frac{1}{e^{\beta \hbar \omega} - 1} \right) \, \mathrm{d}\omega \\
        &= \frac{3V\hbar}{2\pi^{2}v^{3}} \int_{0}^{\omega^{*}} \frac{\omega^{3}}{e^{\beta \hbar \omega} - 1} \, \mathrm{d}\omega \\
        &= \frac{9n\hbar}{\omega_{d}^{3}} \int_{0}^{\omega^{*}} \frac{\omega^{3}}{e^{\beta \hbar \omega} - 1} \, \mathrm{d}\omega
    \end{split}
\end{equation}

where we define $\omega_{d} \equiv 6\pi^{2}nv^{3}$ as the Debye frequency and the upper limit of the integral is given by the condition:

\begin{equation}
    \int_{0}^{\omega^{*}} \tilde{g}(\omega) \, \mathrm{d}\omega = 3N
\end{equation}

At low temperatures, $\beta \hbar \omega \gg 1$, we can approximate the integrand as:

\begin{equation}
    \begin{split}
        U &\approx \frac{9n\hbar}{\omega_{d}^{3}} \int_{0}^{\omega^{*}} \omega^{3} e^{-\beta \hbar \omega} \, \mathrm{d}\omega \\
        &= \frac{9n\hbar}{\omega_{d}^{3}} e^{-\beta \hbar \omega} \frac{sx   }{}
    \end{split}
\end{equation}


At high temperatures, $\beta \hbar \omega \ll 1$, we can expand the integrand as:

\begin{equation}
    U \approx \frac{9n\hbar}{\omega_{d}^{3}} \int_{0}^{\omega^{*}} \omega^{2} \frac{1}{\beta \hbar \omega} \, \mathrm{d}\omega = \frac{9n\hbar}{\omega_{d}^{3}} \frac{(\omega^{*})^{3} k_{B}}{3\hbar} T
\end{equation}

which leads to a constant heat capacity.

\subproblem{b}


\qed


\problem{1.3}{Drude Theory of Transport in Metals}

\subproblem{a}
Given a scattering time $\tau$, we have the differential equation:

\begin{equation}
    \mathrm{d}p = F \, \mathrm{d}t -\frac{p}{\tau} \, \mathrm{d}t
\end{equation}

This means that in equilibrium, the average momentum is $F\tau$. Now let $F = -eE$ and $p = mv$, we have the drift velocity:

\begin{equation}
    \mathbf{v} = -\frac{e\tau}{m} \mathbf{E}
\end{equation}

The current density is then:

\begin{equation}
    \mathbf{J} = -ne\mathbf{v} = \frac{ne^{2}\tau}{m} \mathbf{E}
\end{equation}

so that the conductivity is $\sigma_{0} = ne^{2}\tau/m$.

\subproblem{b}
Now we include magnetic effects in the Lorentz force, leading to:

\begin{equation}
    \mathbf{v} = -\frac{\tau e}{m} \left( \mathbf{E} + \mathbf{v} \times \mathbf{B} \right)
\end{equation}

Working in index notation and solving for $E_{i}$:

\begin{equation}
    \begin{split}
        E_{i} &= -\frac{m}{\tau e} v_{i} - \epsilon_{ijk} v_{j} B_{k} \\
        &= \frac{m}{\tau n e^{2}} J_{i} + \frac{1}{ne} \epsilon_{ijk} J_{j} B_{k} \\
        &= \left( \frac{m}{\tau n e^{2}} \delta_{ij} + \frac{1}{ne} \epsilon_{ijk} B_{k} \right) J_{j}
    \end{split}
\end{equation}

Defining the resistivity matrix via $\mathbf{E} = \mathbf{\rho} \mathbf{J}$, we have $\rho_{ij}$:

\begin{equation}
    \rho_{ij} = \frac{m}{\tau n e^{2}} \delta_{ij} + \frac{1}{ne} \epsilon_{ijk} B_{k}
\end{equation}

Consider the simple case where $B$ only has a $z$-component, then $\rho_{ij}$ becomes:

\begin{equation}
    \mathbf{\rho} =
    \begin{pmatrix}
        \rho_{0}  & \rho_{H} & 0        \\
        -\rho_{H} & \rho_{0} & 0        \\
        0         & 0        & \rho_{0}
    \end{pmatrix}
\end{equation}

where $\rho_{0} = m/\tau ne^{2}$ and $\rho_{H} = B/ne$.

The inverse of this matrix is:

\begin{equation}
    \mathbf{\sigma} =
    \begin{pmatrix}
        \frac{\rho_{0}}{\rho_{0}^{2} + \rho_{H}^{2}} & -\frac{\rho_{H}}{\rho_{0}^{2} + \rho_{H}^{2}} & 0                  \\
        \frac{\rho_{H}}{\rho_{0}^{2} + \rho_{H}^{2}} & \frac{\rho_{0}}{\rho_{0}^{2} + \rho_{H}^{2}}  & 0                  \\
        0                                            & 0                                             & \frac{1}{\rho_{0}}
    \end{pmatrix}
\end{equation}

\subproblem{c}
The Hall effect is defined by the second term in the resistivity tensor:

\begin{equation}
    R_{H} \equiv -\frac{1}{ne}
\end{equation}

We have the estimate of the Hall coefficient:

\begin{equation}
    R_{H} = -\frac{1}{ne} = -\frac{M}{\rho e}
\end{equation}

and a current density $J = I/d^{2}$, leading to a Hall voltage:

\begin{equation}
    V_{H} = \frac{JB}{ned}
\end{equation}

\subproblem{d}

\subproblem{e}
Now consider AC field where $E_{i} = E_{0, i} e^{i\omega t}$ and $B_{i} = B_{0, i} e^{i\omega t}$ where $B_{0, i}$ satisfies:

\begin{equation}
    B_{0, i} = \frac{\epsilon_{ijk} \kappa_{j} E_{0, k}}{\omega}
\end{equation}

where $\kappa_{j}$ is the wave vector.

Consider the conductivity equation:

\begin{equation}
    E_{i} = \left( \frac{m}{\tau n e^{2}} \delta_{ij} + \frac{1}{ne} \epsilon_{ijk} B_{k} \right) J_{j}
\end{equation}

In the limit of $\tau \to \infty$, we have the $x$-component of the equation:

\begin{equation}
    E_{0, x} e^{i\omega t} = \frac{1}{ne} B_{z} J_{y} = \frac{1}{ne} c E_{0, x} e^{i\omega t} J_{y}
\end{equation}

which gives $J_{y} = \frac{ne}{c}$ as a constant.

This implies that even if $E_{0, x} \to 0$, the current $J_{y}$ is still finite, indicating a diverging conductivity.
\qed


\problem{1.4}{Fermi surface in the free electron (Sommerfeld) theory of metals}


\subproblem{a}
The Fermi energy is defined as the energy of the highest occupied state at zero temperature in a system of fermions. It is the chemical potential at zero temperature. Given the Fermi energy $E_{F}$, we define Fermi temperature $T_{F} = E_{F}/k_{B}$. The Fermi surface is the boundary in the phase space that separates the filled states from the empty states at zero temperature.

\subproblem{b}
Given the density of states $g(k) = 2 \times Vk^{2}/(2\pi^{2})$, where the factor $2$ accounts for the two spin states, we can calculate the Fermi wave vector $k_{F}$:

\begin{equation}
    N = \int_{0}^{k_{F}} g(k) \, \mathrm{d}k = \frac{Vk_{F}^{3}}{3\pi^{2}}
\end{equation}

which gives $k_{F} = (3\pi^{2}n)^{1/3}$.

The Fermi energy is then:

\begin{equation}
    E_{F} = \frac{\hbar^{2}k_{F}^{2}}{2m} = \frac{\hbar^{2}(3\pi^{2}n)^{2/3}}{2m}
\end{equation}

Consider the derivative $\mathrm{d}E_{F}/\mathrm{d}N$:

\begin{equation}
    \frac{\mathrm{d}E_{F}}{\mathrm{d}N} = \frac{2}{3} \frac{\hbar^{2}(3\pi^{2}V^{-1})^{2/3}}{2m} N^{-1/3} = \frac{2}{3} \frac{E_{F}}{N}
\end{equation}

so that $\mathrm{d}N/\mathrm{d}E_{F} = 3N/(2E_{F})$.

\subproblem{c}
For sodium with $\rho = 10^{3}$ kg/m$^{3}$ and $M = 23u$, we have $n = \rho/M$ so that the Fermi energy is:

\begin{equation}
    E_{F} = 5.11 \times 10^{-19} \, \mathrm{J}
\end{equation}

\subproblem{d}
For a 2D gas, we have $g(k) = Ak/2\pi$ so that:

\begin{equation}
    N = \int_{0}^{k_{F}} g(k) \, \mathrm{d}k = \frac{Ak_{F}^{2}}{4\pi}
\end{equation}

and:

\begin{equation}
    E_{F} = \frac{\hbar^{2}k_{F}^{2}}{2m} = \frac{2\hbar^{2}\pi n}{m}
\end{equation}
\qed


\problem{1.5}{Velocities in the free electron theory}


\subproblem{a}
Given the Fermi wave vector $k_{F}$, we have the Fermi velocity:

\begin{equation}
    v_{F} = \frac{\hbar k_{F}}{m} = \frac{\hbar}{m} (3\pi^{2}n)^{1/3}
\end{equation}

\subproblem{b}
We know that $J = \sigma E$ and $J = \left\lvert nev_{F} \right\rvert$, so that:

\begin{equation}
    v_{d} = \left\lvert \frac{\sigma E}{ne} \right\rvert
\end{equation}

From the scattering time analysis, we have $mv_{d} = eE\tau$ so that:

\begin{equation}
    \begin{split}
        m \frac{\sigma E}{ne} &= eE\tau \\
        \sigma &= \frac{ne^{2}\tau}{m} \\
        &= \frac{ne^{2}\lambda}{mv_{F}}
    \end{split}
\end{equation}

\subproblem{c}
In the given scenario, we have $v_{d} = 4.36 \times 10^{-3} \, \mathrm{m/s}$ and $v_{F} = 2.92 \times 10^{-18} \, \mathrm{m/s}$. The drift velocity is much larger than the Fermi velocity, indicating that the electrical effects overwhelm the quantum mechanical effects. The mean free path is then $\lambda = 8.4 \times 10^{-28} \, \mathrm{m}$, which is too small to be true. In fact, from $n = 1/d^3$, we have $d = 2.3 \times 10^{-10} \, \mathrm{m}$ which is the lattice spacing of copper.
\qed


\problem{1.6}{Physical properties of the free electron gas}


\subproblem{a}
We attempt to give an estimate of the integral:

\begin{equation}
    U = \frac{V}{2\pi^{2}} \left( \frac{2m}{\hbar^{2}} \right)^{3/2} \int_{0}^{\infty} \frac{\epsilon^{3/2}}{e^{\beta(\epsilon - \mu)} + 1} \, \mathrm{d}\epsilon
\end{equation}

We argue that this can be expanded into a series in $T$:

\begin{equation}
    U = U(T = 0) + C_{V} T + \mathcal{O}(T^{2})
\end{equation}

To determine $C_{V}$, note the fact that the deviation of the Fermi statistics from a step function is approximately symmetric about $\mu$ and the smeared region is of width $k_{B}T$. Then the excited electrons contribute to the energy by the amount:

\begin{equation}
    C_{V} T = \gamma V \tilde{g}(E_{F}) k_{B}T T
\end{equation}

which suggests tha the heat capacity is proportional to $T$.

\subproblem{b}
The electron in a magnetic field has energy:

\begin{equation}
    \epsilon = \frac{\hbar^{2}k^{2}}{2m} \pm \mu_{B} B
\end{equation}

and we have the magnetisation:

\begin{equation}
    M = -\frac{1}{V} \frac{\partial U}{\partial B}
\end{equation}

When magnetic field is applied, spin-up electrons are are more energetically expensive by $\mu_{B} B$ so that the number decrease by $\tilde{g}(E_{F}) \mu_{B} B/2$, which is compensated by the increase in the number of spin-down electrons. The magnetisation is then:

\begin{equation}
    M = \tilde{g}(E_{F}) \mu_{B}^{2} B
\end{equation}

which leads to the magnetic susceptibility:

\begin{equation}
    \chi = \mu_{0} \mu_{B}^{2} \tilde{g}(E_{F}) = \mu_{0} \mu_{B}^{2} \frac{3n}{2E_{F}}
\end{equation}

\subproblem{c}
Classically, the heat capacity of a gas is constant at high temperatures and the susceptibility is inverse proportional to temperature (Curie's law). The above results differ from the classical results.

\subproblem{d}
For the heat capacity of the form $C = \gamma T + \alpha T^{3}$, the first term is explained by our previous analysis. The second term is the next order $T^{2}$ in the energy expansion.
\qed




\end{document}
\documentclass[12pt]{article}
\usepackage{homework}
\pagestyle{fancy}

% assignment information
\def\course{Symmetry and Relativity}
\def\assignmentno{Problem Set 1}
\def\assignmentname{Kinematics of fluids}
\def\name{Xin, Wenkang}
\def\time{\today}

\begin{document}

\begin{titlepage}
    \begin{center}
        \large
        \textbf{\course}

        \vfill

        \Huge
        \textbf{\assignmentno}

        \vspace{1.5cm}

        \large{\assignmentname}

        \vfill

        \large
        \name

        \time
    \end{center}
\end{titlepage}


%==========
\pagebreak
\section*{}
%==========


\problem{1}{Eulerian and Lagrangian descriptions}

\subproblem{a}
Since the fluid is incompressible and steady, the density $\rho$ is constant and the velocity field $\mathbf{v}$ is independent of time. By conservation of mass, the input from the circular area above equals the output from the side:
\begin{equation}
    \rho v_{0} \pi r^{2} = \rho v_{r} (2\pi r) h
\end{equation}
which gives $v_{r} = (r/2h) v_{0}$.

For incomprehensibility, we have $\nabla \cdot \mathbf{v} = 0$, which gives:
\begin{equation}
    \frac{1}{r} \frac{\partial}{\partial r} (r v_{r}) + \frac{\partial v_{z}}{\partial z} = 0
\end{equation}
where we have ignored any azimuthal component due to symmetry.

Substituting $v_{r}$ into the equation and solving, we find $v_{z} = -(z/h) v_{0} + f(r)$, where $f(r)$ is a function of $r$ only. Since the fluid is at rest at $z = 0$, we have $f(r) = 0$. Thus, $v_{z} = -(z/h) v_{0}$.

To find the velocity potential, we demand:
\begin{equation}
    \frac{v_{0}}{2h} r \hat{\mathbf{r}} - \frac{v_{0}}{h} z \hat{\mathbf{z}} = \nabla \phi
\end{equation}
which gives:
\begin{equation}
    \phi = \frac{v_{0}}{4h} r^{2} - \frac{v_{0}}{2h} z^{2}
\end{equation}

\subproblem{b}
Consider $\mathbf{v} = \nabla \times \mathbf{A}$, where $\mathbf{A} = A_{\theta} \hat{\mathbf{\theta}}$. We have the equations:
\begin{equation}
    \begin{split}
        -\frac{\partial A_{\theta}}{\partial z} = \frac{v_{0}}{2h} r \\
        \frac{1}{r} \frac{\partial}{\partial r} (r A_{\theta}) = -\frac{v_{0}}{h} z
    \end{split}
\end{equation}
which gives $A_{\theta} = -(v_{0}/2h) r z$.

The streamlines are given by curves of constant $A_{\theta}$. They are hyperbolae of the form $r z = \text{constant}$. Since the fluid is steady, the streamlines are also the trajectories of fluid particles.

The equipotential surfaces are given by curves of constant $\phi$. They are paraboloids of the form $r^{2} - 2z^{2} = \text{constant}$.

\subproblem{c}
Consider the initial condition $\mathbf{r}(0) = (r_{0}, \theta_{0}, z_{0})$. Due to symmetry, the $\theta$ component will stay constant. Focusing on the radial component, we have:
\begin{equation}
    \frac{\mathrm{d}r}{\mathrm{d}t} = \frac{v_{0}}{2h} r
\end{equation}
which can be solved by separation of variables to give:
\begin{equation}
    r(t) = r_{0} e^{v_{0} t/(2h)}
\end{equation}

Similarly, the $z$ component satisfies:
\begin{equation}
    \frac{\mathrm{d}z}{\mathrm{d}t} = -\frac{v_{0}}{h} z
\end{equation}
which gives:
\begin{equation}
    z(t) = z_{0} e^{-v_{0} t/h}
\end{equation}

It is easy to see that $r(t) z(t) = r_{0} z_{0} = \text{constant}$, which is the equation of the streamlines.
\qed


\problem{2}{Streamlines and trajectories}

\subproblem{a}
The stream function $A_{z}$ satisfies:
\begin{equation}
    \begin{split}
        \frac{\partial A_{z}}{\partial y} &= v_{0} \\
        -\frac{\partial A_{x}}{\partial z} &= kt
    \end{split}
\end{equation}
which gives $A_{z} = v_{0}y - ktx$.

The streamlines are given by curves of constant $A_{z}$, which are straight lines of the form $y = (kt/v_{0}) x + \text{constant}$. The time component changes the gradient of the lines.

\subproblem{b}
The equations of motion can be trivially solved to give:
\begin{equation}
    \begin{split}
        x(t) &= v_{0} t \\
        y(t) &= \frac{1}{2} kt^{2}
    \end{split}
\end{equation}
which gives $y(x) = (k/2v_{0}) x^{2}$, which is a parabola.
\qed


\problem{3}{Acceleration in a rotating frame}
\subproblem{a}
First note the following relation:
\begin{equation}
    \mathbf{v}_{r} = \mathbf{v}_{f} - \mathbf{\Omega} \times \mathbf{r}
\end{equation}
which follows from simple velocity addition.

Consider:
\begin{equation}
    \begin{split}
        \left( \frac{\mathrm{D} \mathbf{v}_{f}}{\mathrm{D} t} \right)_{f} &= \left( \frac{\mathrm{D} \mathbf{v}_{f}}{\mathrm{D} t} \right)_{r} + \mathbf{\Omega} \times \mathbf{v}_{f} \\
        &= \left( \frac{\mathrm{D} \mathbf{v}_{r}}{\mathrm{D} t} \right)_{r} + \left[ \frac{\mathrm{D} (\mathbf{\Omega} \times \mathbf{r})}{\mathrm{D} t} \right]_{r} + \mathbf{\Omega} \times \mathbf{v}_{r} \\
    \end{split}
\end{equation}
where we have used the above relation.

Now focus on the second term:
\begin{equation}
    \begin{split}
        \left[ \frac{\mathrm{D} (\mathbf{\Omega} \times \mathbf{r})}{\mathrm{D} t} \right]_{r} &= \left[ \frac{\mathrm{D} (\mathbf{\Omega} \times \mathbf{r})}{\mathrm{D} t} \right]_{f} - \mathbf{\Omega} \times (\mathbf{\Omega} \times \mathbf{r}) \\
        &= \mathbf{\Omega} \times \mathbf{v}_{f} - \mathbf{\Omega} \times (\mathbf{\Omega} \times \mathbf{f})
    \end{split}
\end{equation}

On the other hand, the third term can be written as:
\begin{equation}
    \mathbf{\Omega} \times \mathbf{v}_{f} = \mathbf{\Omega} \times \mathbf{v}_{r} + \mathbf{\Omega} \times (\mathbf{\Omega} \times \mathbf{r})
\end{equation}
where we use the fact that $\mathbf{\Omega}$ is constant.

Putting everything together, we find:
\begin{equation}
    \left( \frac{\mathrm{D} \mathbf{v}_{f}}{\mathrm{D} t} \right)_{f} = \left( \frac{\mathrm{D} \mathbf{v}_{r}}{\mathrm{D} t} \right)_{r} + 2 \mathbf{\Omega} \times \mathbf{v}_{r} + \mathbf{\Omega} \times (\mathbf{\Omega} \times \mathbf{r})
\end{equation}

\subproblem{b}
For a fluid element moving along a circle at constant speed, we have $\mathbf{v}_{r} = \mathbf{0}$ so $\mathbf{f}_{\text{cor}} = \mathbf{0}$ and $\mathbf{f}_{\text{cen}} = -\omega^{2} r \hat{\mathbf{r}}$.
\qed


\problem{4}{Deformation of a fluid element}

\subproblem{a}
We have the deformation tensor and its components:
\begin{equation}
    \begin{split}
        D_{ij} &= \frac{\partial v_{i}}{\partial x_{j}} =
        \begin{pmatrix}
            \alpha & 2\beta  \\
            0      & -\alpha
        \end{pmatrix} \\
        e_{ij} &= \frac{1}{2} (D_{ij} + D_{ji}) =
        \begin{pmatrix}
            \alpha & \beta   \\
            \beta  & -\alpha
        \end{pmatrix} \\
        \omega_{ij} &= \frac{1}{2} (D_{ij} - D_{ji}) =
        \begin{pmatrix}
            0      & \beta \\
            -\beta & 0
        \end{pmatrix}
    \end{split}
\end{equation}

The diagonal components of $e_{ij}$ is an expansion, the off-diagonal components is deformation, and $\omega_{ij}$ is a rotation.

\subproblem{b}
Labelling the points of the rectangle as $A, B, C, D$, starting from the top left and going clockwise, we have:
\begin{equation}
    \begin{split}
        \mathbf{v}_{A} &= D_{xx} (-\delta x) \hat{\mathbf{x}} + D_{yy} \delta y \hat{\mathbf{y}} = \alpha (-\delta x \hat{\mathbf{x}} - \delta y \hat{\mathbf{y}}) \\
        \mathbf{v}_{B} &= D_{xx} \delta x \hat{\mathbf{x}} + D_{yy} \delta y \hat{\mathbf{y}} = \alpha (\delta x \hat{\mathbf{x}} - \delta y \hat{\mathbf{y}}) \\
        \mathbf{v}_{C} &= D_{xx} \delta x \hat{\mathbf{x}} + D_{yy} (-\delta y) \hat{\mathbf{y}} = \alpha (\delta x \hat{\mathbf{x}} + \delta y \hat{\mathbf{y}}) \\
        \mathbf{v}_{D} &= D_{xx} (-\delta x) \hat{\mathbf{x}} + D_{yy} (-\delta y) \hat{\mathbf{y}} = \alpha (-\delta x \hat{\mathbf{x}} + \delta y \hat{\mathbf{y}})
    \end{split}
\end{equation}
\qed


\problem{5}{Vorticity}

\subproblem{a}
The vorticity is given by:
\begin{equation}
    \begin{split}
        \mathbf{\omega} &= \nabla \times (\mathbf{\Omega} \times \mathbf{r}) \\
        &= -(\mathbf{\Omega} \cdot \nabla) \mathbf{r} + \mathbf{\Omega} (\nabla \cdot \mathbf{r}) \\
        &= 2 \mathbf{\Omega}
    \end{split}
\end{equation}
since:
\begin{equation}
    (\mathbf{\Omega} \cdot \nabla) \mathbf{r}= \Omega (\nabla \mathbf{r})_{3j} = \mathbf{\Omega}
\end{equation}

The streamlines are concentric circles with the center at the origin.

\subproblem{b}
The vorticity is given by:
\begin{equation}
    \begin{split}
        \mathbf{\omega} &= \nabla \times \mathbf{v} \\
        &= -k \hat{\mathbf{z}}
    \end{split}
\end{equation}

The streamlines are lines parallel to the $x$ with increasing length as $y$ increases.

\subproblem{c}
The vorticity is given by:
\begin{equation}
    \begin{split}
        \mathbf{\omega} &= \nabla \times \mathbf{v} \\
        &= \frac{1}{r} \frac{\partial}{\partial r} \left( r \frac{k}{r} \right) \hat{\mathbf{\theta}} \\
        &= \mathbf{0}
    \end{split}
\end{equation}
\qed


\problem{6}{Stream function and flow rate}

\subproblem{a}
First note that we may write the normal vector as:
\begin{equation}
    \hat{n} = \frac{1}{\mathrm{d}l} (\mathrm{d}y \hat{\mathbf{x}} - \mathrm{d}x \hat{\mathbf{y}})
\end{equation}

Thus:
\begin{equation}
    \begin{split}
        Q &= \int_{A}^{B} \mathbf{v} \cdot \hat{n} \, \mathrm{d}l \\
        &= \int_{A}^{B} v_{x} \, \mathrm{d}y - v_{y} \, \mathrm{d}x \\
        &= \int_{A}^{B} \frac{\partial \psi}{\partial y} \, \mathrm{d}y + \frac{\partial \psi}{\partial x} \, \mathrm{d}x \\
        &= \int_{A}^{B} \mathrm{d}\psi \\
        &= \psi(B) - \psi(A)
    \end{split}
\end{equation}

$Q$ quantifies how the velocity of the flow changes according to the distance between two streamlines.

\subproblem{b}
Note:
\begin{equation}
    \begin{split}
        \nabla \cdot \mathbf{v} &= -\frac{1}{r} \frac{\partial}{\partial r} \left( r v_{\theta} \right) + \frac{\partial v_{z}}{\partial z} \\
        &= -\frac{1}{r} \frac{\partial}{\partial r} \left( \frac{\partial \psi}{\partial z} \right) + \frac{\partial}{\partial z} \left( \frac{1}{r} \frac{\partial \psi}{\partial r} \right) \\
        &= 0
    \end{split}
\end{equation}
as required by incompressibility.

On streamlines, we have a change in $\psi$:
\begin{equation}
    \begin{split}
        \delta \psi &= \frac{\partial \psi}{\partial r} \delta r + \frac{\partial \psi}{\partial z} \delta z \\
        &= r v_{z} \delta r - r v_{r} \delta z
    \end{split}
\end{equation}

But since on streamlines, $\delta r/v_{r} = \delta z/v_{z}$, we have $\delta \psi = 0$.

It can be easily checked that $\mathbf{v} = \nabla \times (\psi/r) \hat{\mathbf{\theta}}$.

We can compute the required flow rate as the difference between two surface integrals:
\begin{equation}
    \begin{split}
        Q &= \int_{S_{+}} \mathbf{v} \cdot \mathrm{d}\mathbf{S} - \int_{S_{-}} \mathbf{v} \cdot \mathrm{d}\mathbf{S} \\
        &= \int_{S_{+}} \nabla \times \left( \frac{\psi}{r} \hat{\mathbf{\theta}} \right) \cdot \mathrm{d}\mathbf{S} - \int_{S_{-}} \nabla \times \left( \frac{\psi}{r} \hat{\mathbf{\theta}} \right) \cdot \mathrm{d}\mathbf{S} \\
        &= \oint_{C_{+}} \frac{\psi}{r} \, \mathrm{d}l - \oint_{C_{-}} \frac{\psi}{r} \, \mathrm{d}l \\
        &= 2\pi (\psi_{+} - \psi_{-})
    \end{split}
\end{equation}
where the third equality follows from Stokes' theorem.
\qed


\problem{7}{Mass conservation}
In a time interval $\delta t$, the mass of the fluid element changes according to:
\begin{equation}
    m \to m - \rho(t) Q \delta t + \rho_{0} Q \delta t
\end{equation}

This means that the total density changes according to:
\begin{equation}
    \rho \to \rho - \frac{Q}{V} (\rho - \rho_{0}) \delta t
\end{equation}
where the second term is the change $\delta \rho$ in the density.

We then have the differential equation:
\begin{equation}
    \frac{\partial \rho}{\partial t} = -\frac{Q}{V} (\rho - \rho_{0})
\end{equation}
which has the solution:
\begin{equation}
    \rho(t) = \rho_{0} \left( 1 + 0.025 e^{-Qt/V} \right)
\end{equation}

Demanding $\rho(t) = 0.99 \rho(0)$, we find $t = 0.527 V/Q$.
\qed


\problem{8}{Hydrostatic pressure}

\subproblem{a}
The total force acting on a fluid element is given by the integral of the pressure over its surface:
\begin{equation}
    \begin{split}
        \mathbf{F} &= -\int_{S} P \, \mathrm{d}\mathbf{S} \\
        &= -\int_{S} \nabla P \, \mathrm{d}V
    \end{split}
\end{equation}
which suggests that the force per unit volume is $-\nabla P$.

In a gravitational field, $\nabla P = -\rho g \hat{\mathbf{z}}$ so that $P = P_{0} - \rho g z$.

\subproblem{b}
The force exerted on the cube immersed in the fluid is given by:
\begin{equation}
    F = \int \rho g \mathbf{z} \, \mathrm{d}V = \rho g V \hat{\mathbf{z}} = mg \hat{\mathbf{z}}
\end{equation}
where the last equality follows by balancing the forces.

Then we deduce the mass of the displaced fluid equals the mass of the cube, which gives the Archimedes' principle.
\qed

\end{document}
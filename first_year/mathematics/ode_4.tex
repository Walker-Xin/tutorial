\documentclass[12pt]{article}
\usepackage{homework}
\pagestyle{fancy}

% assignment information
\def\course{Ordinary Differential Equations}
\def\assignmentno{Problem Set 4}
\def\assignmentname{Systems of Linear ODEs}
\def\name{Xin, Wenkang}
\def\time{\today}
\begin{document}

\begin{titlepage}
    \begin{center}
        \large
        \textbf{\course}

        \vfill

        \Huge
        \textbf{\assignmentno}

        \vspace{1.5cm}

        \large{\assignmentname}

        \vfill

        \large
        \name

        \time
    \end{center}
\end{titlepage}


%==========
\pagebreak
\section*{Minimal Set}
%==========


All $C_{i}$ appearing in the following solutions are arbitrary constants unless otherwise stated.


\problem{4.1}{}
Adding the equations yield first order equation in terms of $(x + y)$:

\begin{equation}
    \frac{\mathrm{d}}{\mathrm{d}t} (x + y) + (a - b)(x + y) = f
\end{equation}

Solving this equation yields:

\begin{equation}
    x + y = C_{1} e^{(b - a)t} - \frac{f}{b - a}
\end{equation}

Subtracting the two equations yields:

\begin{equation}
    \frac{\mathrm{d}}{\mathrm{d}t} (x - y) + (a + b)(x - y) = f
\end{equation}

and:

\begin{equation}
    x - y = C_{2} e^{-(b + a)t} + \frac{f}{b + a}
\end{equation}

Hence the solutions are:

\begin{equation}
    \begin{split}
        x(t) = C_{1} e^{(b - a)t} + C_{2} e^{-(b + a)t} + \frac{af}{a^{2} - b^{2}} \\
        y(t) = C_{1} e^{(b - a)t} - C_{2} e^{-(b + a)t} + \frac{bf}{a^{2} - b^{2}}
    \end{split}
\end{equation}
\qed


\problem{4.2}{}
The system can be written in the following matrix form:

\begin{equation}
    \begin{pmatrix}
        1 & 2 \\
        1 & 1
    \end{pmatrix}
    \begin{pmatrix}
        y' \\
        z'
    \end{pmatrix}
    =
    \begin{pmatrix}
        -4 & -10 \\
        -1 & 1
    \end{pmatrix}
    \begin{pmatrix}
        y \\
        z
    \end{pmatrix}
    +
    \begin{pmatrix}
        2 \\
        -3
    \end{pmatrix}
\end{equation}

Uncoupling by left multiplying both sides with the inverse of the first matrix:

\begin{equation}
    \begin{pmatrix}
        y' \\
        z'
    \end{pmatrix}
    =
    \begin{pmatrix}
        2  & 12  \\
        -3 & -11
    \end{pmatrix}
    \begin{pmatrix}
        y \\
        z
    \end{pmatrix}
    +
    \begin{pmatrix}
        -8 \\
        5
    \end{pmatrix}
\end{equation}

Solving for the eigenvalues and eigenvectors of the coefficient matrix gives $\lambda_{1} = -2$ corresponding to eigenvector $\mathbf{v}_{1} = (-3, 1)^{\intercal}$ and $\lambda_{2} = -7$ corresponding to eigenvector $\mathbf{v}_{2} = (-4, 3)^{\intercal}$. By observation, a particular solutions is $(y, z) = (-2, 1)$. Thus the general solution has the form:

\begin{equation}
    \begin{pmatrix}
        y \\
        z
    \end{pmatrix}
    =
    C_{1} \begin{pmatrix}
        -3 \\
        1
    \end{pmatrix}
    e^{-2x}
    +
    C_{2} \begin{pmatrix}
        -4 \\
        3
    \end{pmatrix}
    e^{-7x}
    +
    \begin{pmatrix}
        -2 \\
        1
    \end{pmatrix}
\end{equation}

Substituting the initial conditions $C_{1} = 12/5$ and $C_{2} = 1/5$. Thus the solutions are:

\begin{equation}
    \begin{split}
        y(x) &= -\frac{36}{5} e^{-2x} - \frac{4}{5} e^{-7x} - 2 \\
        z(x) &= \frac{12}{5} e^{-2x} + \frac{3}{5} e^{-7x} + 1
    \end{split}
\end{equation}
\qed


\problem{4.3}{}

\subproblem{i}{}
The system can be written in the following matrix form:

\begin{equation}
    \begin{pmatrix}
        \dot{x} \\
        \dot{y} \\
        \dot{z}
    \end{pmatrix}
    =
    \begin{pmatrix}
        0 & -2 & 2  \\
        1 & -1 & 1  \\
        0 & 1  & -1
    \end{pmatrix}
    \begin{pmatrix}
        x \\
        y \\
        z
    \end{pmatrix}
\end{equation}

Solving for the eigenvalues and eigenvectors of the coefficient matrix gives $\lambda_{1} = 0$ corresponding to eigenvector $\mathbf{v}_{1} = (0, 1, 1)^{\intercal}$, $\lambda_{2} = -1 + i$ corresponding to $\mathbf{v}_{2} = (-2, i, 1)^{\intercal}$ and $\lambda_{3} = -1 - i$ corresponding to $\mathbf{v}_{3} = (-2, -i, 1)^{\intercal}$. Taking the Wronskian of the fundamental matrix at $t = 0$ yields a value of $4i$, verifying the linear independence of the solutions. Thus the solution has the form:

\begin{equation}
    \begin{pmatrix}
        x \\
        y \\
        z
    \end{pmatrix}
    =
    C_{1} \begin{pmatrix}
        0 \\
        1 \\
        1
    \end{pmatrix}
    +
    C_{2} \begin{pmatrix}
        -2 \\
        i  \\
        1
    \end{pmatrix}
    e^{-(1 - i)t}
    +
    C_{3} \begin{pmatrix}
        -2 \\
        -i \\
        1
    \end{pmatrix}
    e^{-(1 + i)t}
\end{equation}

Taking only the reals part yields the solutions:

\begin{equation}
    \begin{split}
        x(t) &= -2 (C_{2} + C_{3}) e^{-t} \cos{t} \\
        y(t) &= C_{1} + (C_{3} - C_{2}) e^{-t} \sin{t} \\
        z(t) &= C_{1} + (C_{2} + C_{3}) e^{-t} \cos{t}
    \end{split}
\end{equation}

\subproblem{ii}{}
The system can be written in the following matrix form:

\begin{equation}
    \begin{pmatrix}
        \dot{x} \\
        \dot{y} \\
        \dot{z}
    \end{pmatrix}
    =
    \begin{pmatrix}
        4 & -1 & -1 \\
        1 & 2  & -1 \\
        1 & -1 & 2
    \end{pmatrix}
    \begin{pmatrix}
        x \\
        y \\
        z
    \end{pmatrix}
\end{equation}

Solving for the eigenvalues and eigenvectors of the coefficient matrix gives a two-fold degenerate $\lambda_{1} = 3$ corresponding to eigenvectors $\mathbf{v}_{11} = (1, 0, 1)^{\intercal}$ and $\mathbf{v}_{12} = (1, 1, 0)^{\intercal}$ and $\lambda_{2} = 2$ corresponding to $\mathbf{v}_{2} = (1, 1, 1)^{\intercal}$. Thus, the solutions are:

\begin{equation}
    \begin{split}
        x(t) &= (C_{1} + C_{2}) e^{3t} + C_{3} e^{2t} \\
        y(t) &= C_{2} e^{3t} + C_{3} e^{2t} \\
        z(t) &= C_{1} e^{3t} + C_{3} e^{2t}
    \end{split}
\end{equation}
\qed


\problem{4.4}{}

\subproblem{i}{}
The system can be written in the following matrix form:

\begin{equation}
    \begin{pmatrix}
        \dot{x} \\
        \dot{y} \\
        \dot{z}
    \end{pmatrix}
    =
    \begin{pmatrix}
        4  & 3  & -3 \\
        -3 & -2 & 3  \\
        3  & 3  & -2
    \end{pmatrix}
    \begin{pmatrix}
        x \\
        y \\
        z
    \end{pmatrix}
    +
    \begin{pmatrix}
        0 \\
        0 \\
        2
    \end{pmatrix}
    e^{-t}
\end{equation}

Solving for the eigenvalues and eigenvectors of the coefficient matrix gives a two-fold degenerate $\lambda_{1} = 1$ corresponding to eigenvectors $\mathbf{v}_{11} = (1, 0, 1)^{\intercal}$ and $\mathbf{v}_{12} = (-1, 1, 0)^{\intercal}$ and $\lambda_{2} = -2$ corresponding to $\mathbf{v}_{2} = (1, -1, 1)^{\intercal}$. Suppose that a particular solution has the form $(x, y, z) = (A, B, C)e^{-t}$. We have:

\begin{equation}
    -\begin{pmatrix}
        A \\
        B \\
        C
    \end{pmatrix}
    =
    \begin{pmatrix}
        4  & 3  & -3 \\
        -3 & -2 & 3  \\
        3  & 3  & -2
    \end{pmatrix}
    \begin{pmatrix}
        A \\
        B \\
        C
    \end{pmatrix}
    +
    \begin{pmatrix}
        0 \\
        0 \\
        2
    \end{pmatrix}
\end{equation}

Solving this system of equations gives $A = 3$, $B = -3$ and $C = 2$. Therefore, the solutions are:

\begin{equation}
    \begin{split}
        x(t) &= (C_{1} - C_{2}) e^{t} + C_{3} e^{-2t} + 3e^{-t} \\
        y(t) &= C_{2} e^{t} - C_{3} e^{-2t} - 3e^{-t} \\
        z(t) &= C_{1} e^{t} + C_{3} e^{-2t} + 2e^{-t}
    \end{split}
\end{equation}

\subproblem{ii}{}
The system can be written in the following matrix form:

\begin{equation}
    \begin{pmatrix}
        \dot{x} \\
        \dot{y} \\
        \dot{z}
    \end{pmatrix}
    =
    \begin{pmatrix}
        -5 & 1  & -2 \\
        -1 & -1 & 0  \\
        6  & -2 & 2
    \end{pmatrix}
    \begin{pmatrix}
        x \\
        y \\
        z
    \end{pmatrix}
    +
    \begin{pmatrix}
        0 \\
        2 \\
        0
    \end{pmatrix}
    \sinh{t}
    +
    \begin{pmatrix}
        1 \\
        1 \\
        -2
    \end{pmatrix}
    \cosh{t}
\end{equation}

or writing the hyperbolic functions in terms of exponentials:

\begin{equation}
    \begin{pmatrix}
        \dot{x} \\
        \dot{y} \\
        \dot{z}
    \end{pmatrix}
    =
    \begin{pmatrix}
        -5 & 1  & -2 \\
        -1 & -1 & 0  \\
        6  & -2 & 2
    \end{pmatrix}
    \begin{pmatrix}
        x \\
        y \\
        z
    \end{pmatrix}
    +
    \begin{pmatrix}
        1/2 \\
        3/2 \\
        -1
    \end{pmatrix}
    e^{t}
    +
    \begin{pmatrix}
        1/2  \\
        -1/2 \\
        -1
    \end{pmatrix}
    e^{-t}
\end{equation}

Solving for the eigenvalues and eigenvectors of the coefficient matrix gives $\lambda_{1} = -2$ corresponding to eigenvector $\mathbf{v}_{1} = (-1, -1, 1)^{\intercal}$, $\lambda_{2} = -1 + i$ corresponding to $\mathbf{v}_{2} = (-1, -i, 2)^{\intercal}$ and $\lambda_{3} = -1 - i$ corresponding to $\mathbf{v}_{3} = (-1, i, 2)^{\intercal}$. Suppose that a particular solution has the form $(x, y, z) = (A, B, C)e^{t} + (D, E, F)e^{-t}$. We have:

\begin{equation}
    \begin{pmatrix}
        A \\
        B \\
        C
    \end{pmatrix}
    =
    \begin{pmatrix}
        -5 & 1  & -2 \\
        -1 & -1 & 0  \\
        6  & -2 & 2
    \end{pmatrix}
    \begin{pmatrix}
        A \\
        B \\
        C
    \end{pmatrix}
    +
    \begin{pmatrix}
        1/2 \\
        3/2 \\
        -1
    \end{pmatrix}
\end{equation}

and

\begin{equation}
    -\begin{pmatrix}
        D \\
        E \\
        F
    \end{pmatrix}
    =
    \begin{pmatrix}
        -5 & 1  & -2 \\
        -1 & -1 & 0  \\
        6  & -2 & 2
    \end{pmatrix}
    \begin{pmatrix}
        D \\
        E \\
        F
    \end{pmatrix}
    +
    \begin{pmatrix}
        1/2  \\
        -1/2 \\
        -1
    \end{pmatrix}
\end{equation}

Solving these two systems of equations gives $(A, B, C) = (-D, -E, -F) = (1/2, 1/2, -1)$. This implies that the particular solution is $(x, y, z) = (1, 1, -2)\sinh{t}$. Taking the real parts only, the solutions are:

\begin{equation}
    \begin{split}
        x(t) &= -C_{1} e^{-2t} - (C_{2} + C_{3}) e^{-t} \cos{t} + \sinh{t} \\
        y(t) &= -C_{1} e^{-2t} + (C_{2} + C_{3}) e^{-t} \cos{t} + \sinh{t} \\
        z(t) &= C_{1} e^{-2t} - 2(C_{2} - C_{3}) e^{-t} \cos{t} - 2 \sinh{t}
    \end{split}
\end{equation}
\qed

%==========
\pagebreak
\section*{}
%==========


\problem{}{}


\end{document}
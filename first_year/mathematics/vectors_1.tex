\documentclass[12pt]{article}
\usepackage{homework}
\pagestyle{fancy}

% assignment information
\def\course{Vectors \& Matrices}
\def\assignmentno{Problem Set 1}
\def\assignmentname{Vectors, vector spaces and geometry}
\def\name{Xin, Wenkang}
\def\time{\today}

\begin{document}

\begin{titlepage}
    \begin{center}
        \large
        \textbf{\course}

        \vfill

        \Huge
        \textbf{\assignmentno}

        \vspace{1.5cm}

        \large{\assignmentname}

        \vfill

        \large
        \name

        \time
    \end{center}
\end{titlepage}


%==========
\pagebreak
\section*{Vectors, vector spaces and geometry}
%==========


\problem{1}{}
(a) is obviously a linearly independent set. To check whether the remaining sets are linearly independent, compute their determinant:

\begin{equation}
    \det{\begin{pmatrix}
            0 & 1 & 0 \\
            1 & 1 & 0 \\
            1 & 1 & 1
        \end{pmatrix}} = -1 \ne 0
\end{equation}

so (b) is linearly independent.

\begin{equation}
    \det{\begin{pmatrix}
            1 & 2 & 1  \\
            0 & 3 & 6  \\
            1 & 1 & -1
        \end{pmatrix}} = 0
\end{equation}

so (c) is linearly dependent. The subset $\left\{ \mathbf{v}_{1}, \mathbf{v}_{2} \right\}$ is linearly independent.

For (d), suppose $\mathbf{v}_{3} = \alpha \mathbf{v}_{1} + \beta \mathbf{v}_{2}$. We have the equation system:

\begin{equation}
    \begin{split}
        \alpha + 2\beta &= -3 \\
        2\alpha + \beta &= 6 \\
        \beta &= -4 \\
        -3\alpha - 4\beta &= 1 \\
    \end{split}
\end{equation}

which has a solution $(\alpha, \beta) = (5, -4)$.

Thus (d) is linearly dependent
\qed


\problem{2}{}

\subproblem{a}
The set is spanned by a single vector $(2, 2, 1)^{\intercal}$ so it is closed under addition and multiplication and has a zero vector. Thus it is a sub space.

\subproblem{b}
There is no zero vector in this set so it is not a sub space.

\subproblem{c}
For $f:R \to R$ that satisfies $f(x) = f(-x)$, define the sum of two functions as:

\begin{equation}
    \begin{split}
        (f_{1} + f_{2})(x) \equiv f_{1}(x) + f_{2}(x)
    \end{split}
\end{equation}

$f(x) = 0$ apparently belongs to the subset. Note that:

\begin{equation}
    \begin{split}
        (\alpha f_{1} + \beta f_{2})(x) = \alpha f_{1}(x) + \beta f_{2}(x) = \alpha f_{1}(-x) + \beta f_{2}(-x) = (\alpha f_{1} + \beta f_{2})(-x)
    \end{split}
\end{equation}

Thus the subset is a vector space.
\qed


\problem{3}{}
The dimension of $\mathbf{M}_{3 \times 3}$ is nine.

\subproblem{a}
Let $A = \sum A_{ij}, B = \sum B_{ij} \in V$, where $V$ is the subset of symmetric $3 \times 3$ matrices. Apparently the $3 \times 3$ zero matrix belongs to $V$. Consider a linear combination of $A$ and $B$:

\begin{equation}
    (\alpha A + \beta B)_{ij} = \alpha A_{ij} + \beta B_{ij} = \alpha A_{ji} + \beta B_{ji} = (\alpha A + \beta B)_{ji}
\end{equation}

Thus $(\alpha A + \beta B) \in V$, and $V$ thus forms a sub vector space. It has a dimension of \mistake{four} with its basis is given by:

\begin{equation}
    \left\{
    I_{3},
    \begin{pmatrix}
        0 & 1 & 0 \\
        1 & 0 & 0 \\
        0 & 0 & 0
    \end{pmatrix},
    \begin{pmatrix}
        0 & 0 & 1 \\
        0 & 0 & 0 \\
        1 & 0 & 0
    \end{pmatrix},
    \begin{pmatrix}
        0 & 0 & 0 \\
        0 & 0 & 1 \\
        0 & 1 & 0
    \end{pmatrix}
    \right\}
\end{equation}

\begin{correction}
    It has a dimension of six with its basis is given by:

    \begin{equation}
        \left\{
        \begin{pmatrix}
            1 & 0 & 0 \\
            0 & 0 & 0 \\
            0 & 0 & 0
        \end{pmatrix},
        \begin{pmatrix}
            0 & 0 & 0 \\
            0 & 1 & 0 \\
            0 & 0 & 0
        \end{pmatrix},
        \begin{pmatrix}
            0 & 0 & 0 \\
            0 & 0 & 0 \\
            0 & 0 & 1
        \end{pmatrix},
        \begin{pmatrix}
            0 & 1 & 0 \\
            1 & 0 & 0 \\
            0 & 0 & 0
        \end{pmatrix},
        \begin{pmatrix}
            0 & 0 & 1 \\
            0 & 0 & 0 \\
            1 & 0 & 0
        \end{pmatrix},
        \begin{pmatrix}
            0 & 0 & 0 \\
            0 & 0 & 1 \\
            0 & 1 & 0
        \end{pmatrix}
        \right\}
    \end{equation}
\end{correction}

\subproblem{b}
Let $A = \sum A_{ij}, B = \sum B_{ij} \in W$, where $W$ is the subset of antisymmetric $3 \times 3$ matrices. Apparently the $3 \times 3$ zero matrix belongs to $W$. Consider a linear combination of $A$ and $B$:

\begin{equation}
    (\alpha A + \beta B)_{ij} = \alpha A_{ij} + \beta B_{ij} = -\alpha A_{ji} - \beta B_{ji} = -(\alpha A + \beta B)_{ji}
\end{equation}

Thus $(\alpha A + \beta B) \in W$, and $W$ thus forms a sub vector space. It has a dimension of three with its basis is given by:

\begin{equation}
    \left\{
    \begin{pmatrix}
        0  & 1 & 0 \\
        -1 & 0 & 0 \\
        0  & 0 & 0
    \end{pmatrix},
    \begin{pmatrix}
        0  & 0 & 1 \\
        0  & 0 & 0 \\
        -1 & 0 & 0
    \end{pmatrix},
    \begin{pmatrix}
        0 & 0  & 0 \\
        0 & 0  & 1 \\
        0 & -1 & 0
    \end{pmatrix}
    \right\}
\end{equation}

\subproblem{c}
For symmetric matrices, the dimension is \mistake{$(n^{2} - n)/2 + 1$}. For antisymmetric matrices, the dimension is $(n^{2} - n)/2$. Their basis are given in a similar fashion as above.

\begin{correction}
    For symmetric matrices, the dimension is \mistake{$(n^{2} - n)/2 + n$}.
\end{correction}
\qed


\problem{4}{}

\begin{equation}
    \det{\begin{pmatrix}
            1  & 0  & 2 \\
            -1 & 1  & 0 \\
            0  & -1 & 1
        \end{pmatrix}} = 3 \ne 0
\end{equation}

so the set is linearly independent and thus is a basis of $\mathbb{R}^{3}$. For a general $\mathbf{v} \in \mathbb{R}^{3}$:

\begin{equation}
    \mathbf{v} = a \mathbf{v}_{1} + b \mathbf{v}_{2} + c \mathbf{v}_{3}
\end{equation}

where $a, b, c$ are the coordinates of $\mathbf{v}$ relative to the basis determined by the equation system:

\begin{equation}
    \begin{pmatrix}
        1  & 0  & 2 \\
        -1 & 1  & 0 \\
        0  & -1 & 1
    \end{pmatrix}
    \begin{pmatrix}
        a \\
        b \\
        c
    \end{pmatrix} =
    \begin{pmatrix}
        x \\
        y \\
        z
    \end{pmatrix}
\end{equation}
\qed


\problem{5}{}

\subproblem{a}
Let $\mathbf{u}, \mathbf{v} \in U \cap W$. Apparently $\mathbf{0} \in U \cap W$ as $\mathbf{0} \in U$ and $\mathbf{0} \in W$. As $U \cap W$ is a sub set of $V$, $\mathbf{u}, \mathbf{v} \in V$. Consider a linear combination of $\mathbf{u}$ and $\mathbf{v}$:

\begin{equation}
    \alpha \mathbf{u} + \beta \mathbf{v}
\end{equation}

As $\mathbf{u}, \mathbf{v} \in U$ and $U$ is a sub vector space, $\alpha \mathbf{u} + \beta \mathbf{v} \in U$. A similar argument can be made for $W$. Hence:

\begin{equation}
    (\alpha \mathbf{u} + \beta \mathbf{v}) \in U \cap W
\end{equation}

and thus $U \cap W$ is a sub vector space.

\subproblem{b}
Let $\mathbf{a}, \mathbf{b} \in U + W$, such that $\mathbf{a} = \mathbf{u}_{1} + \mathbf{w}_{1}$ and $\mathbf{b} = \mathbf{u}_{2} + \mathbf{w}_{2}$. Apparently $\mathbf{0} \in U + W$. Consider a linear combination of $\mathbf{a}$ and $\mathbf{b}$:

\begin{equation}
    \alpha \mathbf{a} + \beta \mathbf{b} = (\alpha \mathbf{u}_{1} + \beta \mathbf{u}_{2}) + (\alpha \mathbf{w}_{1} + \beta \mathbf{w}_{2})
\end{equation}

The first term on the right-hand side belongs to $U$ while the second term belongs to $W$ due to them being vector spaces. Thus $(\alpha \mathbf{a} + \beta \mathbf{b}) \in U + W$ and $U + W$ is a sub vector space.

\subproblem{c}
This is a very loosely argued proof.

Let $U \cap W$ have a basis $\{\mathbf{v}_{i}\}$ of $n$ elements, so that it has a dimension of $n$. Suppose that in order for $\{\mathbf{v}_{i}\}$ to be a basis of $U$, there needs to be an additional set $\{\mathbf{u}_{j}'\}$ of $m$ vectors, where $0 \le m \le \dim{U}$. Also suppose that in order for $\{\mathbf{v}_{i}\}$ to be a basis of $W$, there needs to be an additional set $\{\mathbf{w}_{k}'\}$ of $l$ vectors, where $0 \le l \le \dim{W}$. Then we can assert that the dimension of $U$ is $n + m$ and the dimension of $W$ is $n + l$.

An element in $U + W$ can be thus expressed as $\sum \alpha_{i} \mathbf{v}_{i} + \sum \beta_{j} \mathbf{u}_{j}' + \sum \gamma_{k} \mathbf{w}_{k}'$ where $i, j, k$ take the values from $1$ to $n, l, m$ respectively. Since it has been assumed that $\{\mathbf{v}_{i}, \mathbf{u}_{j}'\}$ and $\{\mathbf{v}_{i}, \mathbf{w}_{k}'\}$ are bases, they must be linearly independent. Then it follows that:

\begin{equation}
    \dim{(U + W)} = n + l + m = (n + m) + (n + l) - n = \dim{U} + \dim{W} - \dim{(U \cap W)}
\end{equation}
\qed


\problem{6}{}

\subproblem{a}
This follows from the anticommutative property of the cross product.

\subproblem{b}

\begin{equation}
    \begin{split}
        \mathbf{a} \times (\mathbf{b} \times \mathbf{c}) &= \mathbf{a} \times \left( \sum_{i} \epsilon_{ijk} b_{j} c_{k} \right) \\
        &= \epsilon_{lmi} a_{m} \epsilon_{ijk} b_{j} c_{k} \\
        &= \epsilon_{ilm} \epsilon_{ijk} a_{m} b_{j} c_{k} \\
        &= (\delta_{lj} \delta_{mk} - \delta_{lk} \delta_{mj}) a_{m} b_{j} c_{k} \\
        &= a_{m} c_{m} b_{l} - a_{m} b_{m} c_{l} \\
        &= (\mathbf{a} \cdot \mathbf{c})\mathbf{b} - (\mathbf{a} \cdot \mathbf{b})\mathbf{c}
    \end{split}
\end{equation}

\subproblem{c}

\begin{equation}
    \begin{split}
        (\mathbf{a} \times \mathbf{c}) \cdot (\mathbf{c} \times \mathbf{d}) &= \epsilon_{ijk} a_{j} b_{k} \epsilon_{ilm} c_{l} d_{m} \\
        &= (\delta_{lj} \delta_{mk} - \delta_{lk} \delta_{mj}) a_{j} b_{k} c_{l} d_{m} \\
        &= a_{j} b_{k} c_{j} d_{k} - a_{j} b_{k} c_{k} d_{j} \\
        &= (\mathbf{a} \cdot \mathbf{c}) (\mathbf{b} \cdot \mathbf{d}) - (\mathbf{a} \cdot \mathbf{d}) (\mathbf{b} \cdot \mathbf{c})
    \end{split}
\end{equation}
\qed


\problem{7}{}

\subproblem{a}

\begin{equation}
    \begin{split}
        (\mathbf{a} \times \mathbf{b}) &= \mathbf{a} - \mathbf{b} \\
        \mathbf{a} \cdot (\mathbf{a} \times \mathbf{b}) &= \mathbf{a} \cdot (\mathbf{a} - \mathbf{b}) \\
        0 &= \mathbf{c} \cdot (\mathbf{a} - \mathbf{b}) \\
        \mathbf{a} &= \mathbf{b}
    \end{split}
\end{equation}

provided that $\mathbf{a} \ne \mathbf{0}$.

\subproblem{b}

\begin{equation}
    \begin{split}
        (\mathbf{a} \times \mathbf{b}) \cdot \mathbf{c} = \mathbf{a} \cdot (\mathbf{b} \times \mathbf{c}) = \mathbf{a} \cdot (\mathbf{b} \times \mathbf{a}) = 0
    \end{split}
\end{equation}

\subproblem{c}
$\mathbf{a} \times \mathbf{c} = \mathbf{b} \times \mathbf{c}$ implies $(\mathbf{a} -\mathbf{b}) \times \mathbf{c} = 0$ and thus $(\mathbf{a} - \mathbf{b}) \perp \mathbf{c}$. Thus:

\begin{equation}
    (\mathbf{a} - \mathbf{b}) \cdot \mathbf{c} = \pm \left\lvert \mathbf{a} - \mathbf{b} \right\rvert \left\lvert \mathbf{c} \right\rvert
\end{equation}

\subproblem{d}

\begin{equation}
    (\mathbf{a} \times \mathbf{b}) \times (\mathbf{c} \times \mathbf{b}) = [(\mathbf{a} \times \mathbf{b}) \cdot \mathbf{b}] \mathbf{c} - [(\mathbf{a} \times \mathbf{b}) \cdot \mathbf{c}] \mathbf{b} = [(\mathbf{b} \times \mathbf{a}) \cdot \mathbf{c}] \mathbf{b} = \mathbf{b} [\mathbf{b} \cdot (\mathbf{a} \times \mathbf{c})]
\end{equation}
\qed


\problem{8}{}

\subproblem{a}
The magnitude of the cross product is twice the area of the triangle, so they are equal.

\subproblem{b}

\begin{equation}
    A = \frac{1}{2} \left\lvert PQ \times PR \right\rvert = \frac{1}{2} \left\lvert (1, -4, -1)^{\intercal} \times (-2, -1, 1)^{\intercal} \right\rvert = \frac{\sqrt{107}}{2} \text{ units}^{2}
\end{equation}
\qed


\problem{9}{}

\subproblem{a}
$\mathbf{a}$ and $\mathbf{b}$ form a plane with the normal vector $\mathbf{n} = \mathbf{a} \times \mathbf{b} = -7 \mathbf{i} + 5 \mathbf{j} + \mathbf{k}$, and the plane has the scalar equation:

\begin{equation}
    \mathbf{r} \cdot \mathbf{n} = 0
\end{equation}

$\mathbf{c} \cdot \mathbf{n} = -7 + 10 - 3 = 0$, so $\mathbf{c}$ is on the plane, and thus they are coplanar.

\subproblem{b}
The normal vector is $\mathbf{n} = \mathbf{a} \times \mathbf{b} = \mathbf{i} + 2 \mathbf{j} - 3 \mathbf{k}$. For a vector $\mathbf{r}$ to satisfy the given conditions, have $\mathbf{r} \cdot \mathbf{n} = 0$ and $\mathbf{r} \cdot \mathbf{a} = 0$. This leads to two linear equations:

\begin{equation}
    \begin{split}
        x + 2y - 3z = 0 \\
        x + y + z = 0
    \end{split}
\end{equation}

Solving this yields:

\begin{equation}
    \mathbf{r} = \lambda \begin{pmatrix}
        -5 \\
        4  \\
        1
    \end{pmatrix}
\end{equation}

for $\lambda \in \mathbb{R}$.
\qed


\problem{10}{}

\subproblem{a}

\begin{equation}
    \begin{split}
        \mathbf{v}_{i} \cdot \mathbf{v}_{j}' = \frac{1}{V} [\mathbf{v}_{i} \cdot (\mathbf{v}_{k} \times \mathbf{v}_{i})] = 0 \\
        \mathbf{v}_{i} \cdot \mathbf{v}_{i}' = \frac{1}{V} [\mathbf{v}_{i} \cdot (\mathbf{v}_{j} \times \mathbf{v}_{k})] = 1
    \end{split}
\end{equation}

Let $\mathbf{w} = \sum_{i} \alpha_{i} \mathbf{v}_{i}$. Then:

\begin{equation}
    \mathbf{w} \cdot \mathbf{v}_{j}' = \sum_{i} \alpha_{i} \mathbf{v}_{i} \cdot \mathbf{v}_{j}' = \sum_{i} \delta_{ij} \alpha_{i} = \alpha_{j}
\end{equation}

\subproblem{b}

\begin{equation}
    \begin{split}
        V' &= \frac{1}{V} (\mathbf{v}_{2} \times \mathbf{v}_{3}) \cdot (\mathbf{v}_{2}' \times \mathbf{v}_{3}') \\
        &= - \frac{1}{V^{3}} (\mathbf{v}_{2} \times \mathbf{v}_{3}) \cdot \left[ (\mathbf{v}_{3} \times \mathbf{v}_{1}) \cdot (\mathbf{v}_{2} \times \mathbf{v}_{1}) \right] \\
        &= - \frac{1}{V^{3}} (\mathbf{v}_{2} \times \mathbf{v}_{3}) \cdot \left[\mathbf{v}_{1} (\mathbf{v}_{1} \cdot (\mathbf{v}_{3} \times \mathbf{v}_{2})) \right] \\
        &= \frac{\left[ \mathbf{v}_{1} \cdot (\mathbf{v}_{2} \times \mathbf{v}_{3}) \right]^{2}}{V^{3}} \\
        &= \frac{1}{V}
    \end{split}
\end{equation}

\subproblem{c}

\begin{equation}
    \begin{split}
        \frac{1}{V'} \mathbf{v}_{i}' \times \mathbf{v}_{j}' &= V \frac{1}{V^{2}} \left[ (\mathbf{v}_{j} \times \mathbf{v}_{k}) \times (\mathbf{v}_{k} \times \mathbf{v}_{i}) \right] \\
        &= -\frac{1}{V} \left[ (\mathbf{v}_{j} \times \mathbf{v}_{k}) \times (\mathbf{v}_{i} \times \mathbf{v}_{k}) \right] \\
        &= -\frac{1}{V} \mathbf{v}_{k} \left[ \mathbf{v}_{k} \cdot (\mathbf{v}_{j} \times \mathbf{v}_{i}) \right] \\
        &= \mathbf{v}_{k}
    \end{split}
\end{equation}
\qed


\problem{11}{}
The x-axis has the vector equation:

\begin{equation}
    \mathbf{r} = \lambda \begin{pmatrix}
        1 \\
        0 \\
        0
    \end{pmatrix}
\end{equation}

where $\lambda \in \mathbb{R}$.

The distance is given by:

\begin{equation}
    \begin{split}
        d^{2} &= (\mathbf{p} - \mathbf{r})^{2} \\
        &= \mathbf{p}^{2} - 2 \mathbf{p} \cdot \mathbf{r} + \mathbf{r}^{2} \\
        &= 29 - 4\lambda + \lambda^{2} \\
        &= (\lambda - 2)^{2} + 25
    \end{split}
\end{equation}

Thus the minimum distance is $5 \text{ units}$.
\qed


\problem{12}{}
By inspection, the vector equation of the line is:

\begin{equation}
    \mathbf{r} =
    \begin{pmatrix}
        2 \\
        1 \\
        5
    \end{pmatrix} +
    \lambda \begin{pmatrix}
        4 \\
        3 \\
        2
    \end{pmatrix}
\end{equation}

where $\lambda \in \mathbb{R}$.

The distance is given by:

\begin{equation}
    d^{2} = \mathbf{r}^{2} = 30 + 42\lambda + 29\lambda^{2}
\end{equation}

The minimum distance is thus $\sqrt{429/29} \text{ units}$.
\qed

\problem{13}{}
Following the usual procedure:

\begin{equation}
    \begin{split}
        d^{2} &= (\mathbf{r}_{1} - \mathbf{r}_{2})^{2} \\
        &= \mathbf{r}_{1}^{2} - 2 \mathbf{r}_{1} \cdot r_{2} + \mathbf{r}_{2}^{2} \\
        &= \mathbf{q}_{1}^{2} + 2\lambda_{1} (\mathbf{q}_{1} \cdot \mathbf{m}_{1}) + \mathbf{m}_{1}^{2} \lambda_{1}^{2} - 2 \left[ \mathbf{q}_{1} \mathbf{q}_{2} + \lambda_{2} (\mathbf{q}_{1} \cdot \mathbf{m}_{2}) +\lambda_{1} (\mathbf{q}_{2} \cdot \mathbf{m}_{1}) + \lambda_{1} \lambda_{2} \mathbf{m}_{1} \mathbf{m}_{2} \right] + \\
        &  \mathbf{q}_{2}^{2} + 2\lambda_{1} (\mathbf{q}_{2} \cdot \mathbf{m}_{2}) + \mathbf{m}_{2}^{2} \lambda_{2}^{2}
    \end{split}
\end{equation}

To achieve minimum of $d^{2}(\lambda_{1}, \lambda_{2})$:

\begin{equation}
    \mathrm{d}d^{2} = \frac{\partial d^{2}}{\partial \lambda_{1}} \mathrm{d}\lambda_{1} + \frac{\partial d^{2}}{\partial \lambda_{2}} \mathrm{d}\lambda_{2} = 0
\end{equation}

Each term yields an equation. Putting them together:

\begin{equation}
    \begin{split}
        \mathbf{m}_{1} \cdot (\mathbf{q}_{1} + \mathbf{m}_{1} \lambda_{1} - \mathbf{q}_{2} - \mathbf{m}_{2} \lambda_{2}) = 0 \\
        \mathbf{m}_{2} \cdot (\mathbf{q}_{2} + \mathbf{m}_{2} \lambda_{2} - \mathbf{q}_{1} - \mathbf{m}_{1} \lambda_{1}) = 0
    \end{split}
\end{equation}

This implies either $\mathbf{r}_{1} = \mathbf{r}_{2}$, which is trivial, or $(\mathbf{r}_{1} - \mathbf{r}_{2})$ is perpendicular to both $\mathbf{m}_{1}$ and $\mathbf{m}_{2}$, i.e., $(\mathbf{r}_{1} - \mathbf{r}_{2}) = k(\mathbf{m}_{1} \times \mathbf{m}_{2})$ for some non-zero $k \in \mathbb{R}$.

For the given two lines, $\mathbf{m}_{1} \times \mathbf{m}_{2} = (-3, 0, 3)^{\intercal}$. Then:

\begin{equation}
    \begin{pmatrix}
        4 + 2\lambda_{1} - \lambda_{2} \\
        4 + \lambda_{1} - 2\lambda_{2} \\
        2\lambda_{1} - \lambda_{2}
    \end{pmatrix}
    =
    k\begin{pmatrix}
        -3 \\
        0  \\
        3
    \end{pmatrix}
\end{equation}

This can be transformed into the system of linear equations:

\begin{equation}
    \begin{pmatrix}
        -2 & 1 & -3 \\
        -1 & 2 & 0  \\
        -2 & 1 & 3
    \end{pmatrix}
    \begin{pmatrix}
        \lambda_{1} \\
        \lambda_{2} \\
        k
    \end{pmatrix}
    =
    \begin{pmatrix}
        4 \\
        4 \\
        0
    \end{pmatrix}
\end{equation}

the solution of which is $(\lambda_{1}, \lambda_{2}, k) = (0, 2, -2/3)$.

Thus the minimum distance is $\mistake{2\sqrt{2}} \text{ units}$.

\begin{correction}
    Thus the minimum distance is $3\sqrt{2} \text{ units}$.
\end{correction}
\qed


\problem{14}{}
For Cartesian description, note that:

\begin{equation}
    \mathbf{n} = P_{1} P_{2} \times P_{1} P_{3} =
    \det{
        \begin{pmatrix}
            \mathbf{i} & \mathbf{j} & \mathbf{k} \\
            1          & 3          & -2         \\
            -3         & 4          & 1
        \end{pmatrix}}
    =
    \begin{pmatrix}
        11 \\
        5  \\
        13
    \end{pmatrix}
\end{equation}

Thus:

\begin{equation}
    \mathbf{r} \cdot \mathbf{n} = OP_{1} \cdot \mathbf{n} = 30
\end{equation}

or

\begin{equation}
    11x + 5y + 13z = 30
\end{equation}
\qed


\problem{15}{}
For vector description, note that $\mathbf{b} - \mathbf{a} = (-4, 1, -3)^{\intercal}$ and $\mathbf{c} - \mathbf{a} = (-1, 0, 1)^{\intercal}$. Thus:

\begin{equation}
    \mathbf{r} = \mathbf{a} +
    \lambda \begin{pmatrix}
        -4 \\
        1  \\
        -3
    \end{pmatrix}
    +
    \mu \begin{pmatrix}
        -1 \\
        0  \\
        1
    \end{pmatrix}
\end{equation}

where $\lambda, \mu \in \mathbb{R}$.

For Cartesian description, note that:

\begin{equation}
    \mathbf{n} = (\mathbf{b} - \mathbf{a}) \times (\mathbf{c} - \mathbf{a}) =
    \det{
        \begin{pmatrix}
            \mathbf{i} & \mathbf{j} & \mathbf{k} \\
            -4         & 1          & -3         \\
            -1         & 0          & 1
        \end{pmatrix}}
    =
    \begin{pmatrix}
        1 \\
        7 \\
        1
    \end{pmatrix}
\end{equation}

Thus:

\begin{equation}
    \mathbf{r} \cdot \mathbf{n} = \mathbf{a} \cdot \mathbf{n} = 18
\end{equation}

or

\begin{equation}
    x + 7y + z = 18
\end{equation}
\qed


\problem{16}{}
For the line:

\begin{equation}
    \mathbf{r} =
    \lambda \begin{pmatrix}
        1 \\
        1 \\
        1
    \end{pmatrix}
\end{equation}

For the plane:

\begin{equation}
    \mathbf{r} =
    \begin{pmatrix}
        -1 \\
        1  \\
        -2
    \end{pmatrix}
    +
    \mu \begin{pmatrix}
        2 \\
        4 \\
        -3
    \end{pmatrix}
    +
    \kappa \begin{pmatrix}
        1 \\
        1 \\
        -1
    \end{pmatrix}
\end{equation}

For an intersection, we have the system of linear equations:

\begin{equation}
    \begin{pmatrix}
        1 & -2 & -1 \\
        1 & -4 & -1 \\
        1 & 3  & 1
    \end{pmatrix}
    \begin{pmatrix}
        \lambda \\
        \mu     \\
        \kappa
    \end{pmatrix}
    =
    \begin{pmatrix}
        -1 \\
        1  \\
        -2
    \end{pmatrix}
\end{equation}

the solution of which is $(\lambda, \mu, \kappa) = (-1, -1, 2)$.

Thus the intersection is the point $(-1, -1, -1)^{\intercal}$.
\qed


\end{document}
\documentclass[12pt]{article}
\usepackage{homework}
\pagestyle{fancy}

% assignment information
\def\course{Vectors \& Matrices}
\def\assignmentno{Problem Set 2}
\def\assignmentname{Matrices, linear maps and linear equations}
\def\name{Xin, Wenkang}
\def\time{November 8, 2022}

\begin{document}

\begin{titlepage}
    \begin{center}
        \large
        \textbf{\course}

        \vfill

        \Huge
        \textbf{\assignmentno}

        \vspace{1.5cm}

        \large{\assignmentname}

        \vfill

        \large
        \name

        \time
    \end{center}
\end{titlepage}


%==========
\pagebreak
\section*{Matrices, linear maps and linear equations}
%==========


\problem{1}{}

\subproblem{a}

\begin{equation}
    [A(BC)]_{ik} = A_{ij} (BC)_{jk} = A_{ij} B_{jl} C_{lk}
\end{equation}

\begin{equation}
    [(AB)C]_{ik} = (AB)_{ij} C_{jk} = A_{il} B_{lj} C_{jk} = A_{ij} B_{jl} C_{lk} = [A(BC)]_{ik}
\end{equation}

with a change of dummy variable $l \to j$ and $j \to l$.

\subproblem{b}

\begin{equation}
    (A + B)_{ij}^{\intercal} = (A + B)_{ji} = A_{ji} + B_{ji} = A_{ij}^{\intercal} + B_{ij}^{\intercal}
\end{equation}

\begin{equation}
    (\alpha A)_{ij}^{\intercal} = (\alpha A)_{ji} = \alpha A_{ji} = \alpha (A_{ij}^{\intercal})
\end{equation}

\begin{equation}
    (AB)_{ij}^{\intercal} = (AB)_{ji} = A_{jk} B_{ki} = (B^{\intercal})_{ik} (A^{\intercal})_{kj} = (B^{\intercal} A^{\intercal})_{ij}
\end{equation}

\subproblem{c}

\begin{equation}
    (A + B)_{ij}^{\dagger} = (A + B)_{ji}^{*} = A_{ji}^{*} + B_{ji}^{*} = A_{ij}^{\dagger} + B_{ij}^{\dagger}
\end{equation}

\begin{equation}
    (\alpha A)_{ij}^{\dagger} = (\alpha A)_{ji}^{*} = \alpha^{*} A_{ji}^{*} = \alpha^{*} (A_{ij}^{\dagger})
\end{equation}

\begin{equation}
    (AB)_{ij}^{\intercal} = (AB)_{ji}^{*} = A_{jk}^{*} B_{ki}^{*} = B^{\dagger}_{ik} A^{\dagger}_{kj} = (B^{\dagger} A^{\dagger})_{ij}
\end{equation}

\subproblem{d}

\begin{equation}
    \begin{split}
        (AB)(AB)^{-1} &= I \\
        B^{-1} A^{-1} (AB)(AB)^{-1} &= B^{-1} A^{-1} \\
        (AB)^{-1} &= B^{-1} A^{-1}
    \end{split}
\end{equation}

We have:

\begin{equation}
    A^{\intercal} (A^{\intercal})^{-1} = I
\end{equation}

Taking the transpose:

\begin{equation}
    \left[ (A^{\intercal})^{-1} \right]^{\intercal} A = I
\end{equation}

Since the inverse of a matrix is unique:

\begin{equation}
    \left[ (A^{\intercal})^{-1} \right]^{\intercal} = A^{-1}
\end{equation}

Taking the transpose again:

\begin{equation}
    (A^{\intercal})^{-1} = (A^{-1})^{\intercal}
\end{equation}


\qed


\problem{2}{}

\begin{equation}
    \begin{split}
        A \to
        \begin{pmatrix}
            1 & 1  & 0  & -1 \\
            0 & 2  & -2 & 1  \\
            3 & 2  & 0  & -4 \\
            1 & -2 & a  & 0
        \end{pmatrix}
        \to
        \begin{pmatrix}
            1 & 1 & 0   & -1   \\
            0 & 2 & -2  & 1    \\
            0 & 0 & -1  & -1/2 \\
            0 & 0 & a-3 & 5/2
        \end{pmatrix}
        \to
        \begin{pmatrix}
            1 & 1 & 0  & -1 \\
            0 & 2 & -2 & 1  \\
            0 & 0 & -2 & -1 \\
            0 & 0 & a  & 4
        \end{pmatrix}
    \end{split}
\end{equation}

\mistake{Thus, if $a = 8$, $\rank(A) = 4$. If $a \ne 8$, $\rank(A) = 3$.}

\begin{correction}
    Thus, if $a = 8$, $\rank(A) = 3$. If $a \ne 8$, $\rank(A) = 4$.
\end{correction}
\qed


\problem{3}{}

\begin{equation}
    \begin{split}
        \left( \begin{array}{ccc|ccc}
            1 & 0  & -1 & 1 & 0 & 0 \\
            2 & 1  & -2 & 0 & 1 & 0 \\
            1 & -3 & 0  & 0 & 0 & 1
        \end{array} \right)
        \to
        \left( \begin{array}{ccc|ccc}
            1 & 0  & -1 & 1  & 0 & 0 \\
            0 & 1  & 0  & -2 & 1 & 0 \\
            0 & -3 & 1  & -1 & 0 & 1
        \end{array} \right)
        \to
        \left( \begin{array}{ccc|ccc}
            1 & 0 & 0 & -6 & 3 & 1 \\
            0 & 1 & 0 & -2 & 1 & 0 \\
            0 & 0 & 1 & -7 & 3 & 1
        \end{array} \right)
    \end{split}
\end{equation}

Thus:

\begin{equation}
    A^{-1} =
    \begin{pmatrix}
        -6 & 3 & 1 \\
        -2 & 1 & 0 \\
        -7 & 3 & 1
    \end{pmatrix}
\end{equation}
\qed


\problem{4}{}
We have \mistake{$A' = \psi' f \psi = \psi' \phi \phi' f \phi \phi' \psi = PAP^{-1}$}, where $A = \phi' f \phi$, $\phi = I$ and $P = \psi' \phi = \psi'$.

Thus:

\begin{equation}
    P = \psi^{-1} =
    \begin{pmatrix}
        1 & 0 \\
        1 & 1
    \end{pmatrix}^{-1}
    =
    \begin{pmatrix}
        1  & 0 \\
        -1 & 1
    \end{pmatrix}
\end{equation}

\begin{equation}
    A' =
    \begin{pmatrix}
        1  & 0 \\
        -1 & 1
    \end{pmatrix}
    A
    \begin{pmatrix}
        1 & 0 \\
        1 & 1
    \end{pmatrix}
    =
    \begin{pmatrix}
        3  & 2  \\
        -2 & -1
    \end{pmatrix}
\end{equation}

\begin{correction}
    We have $A' = (\phi')^{-1} f \phi' = (\phi')^{-1} \phi \phi^{-1} f \phi \phi^{-1} \phi' = PAP^{-1}$, where $A = \phi^{-1} f \phi$, $\phi = I$ and $P = (\phi')^{-1} \phi = (\phi')^{-1}$.

    Thus:

    \begin{equation}
        P = (\phi')^{-1} =
        \begin{pmatrix}
            1 & 0 \\
            1 & 1
        \end{pmatrix}^{-1}
        =
        \begin{pmatrix}
            1  & 0 \\
            -1 & 1
        \end{pmatrix}
    \end{equation}

    \begin{equation}
        A' =
        \begin{pmatrix}
            1  & 0 \\
            -1 & 1
        \end{pmatrix}
        A
        \begin{pmatrix}
            1 & 0 \\
            1 & 1
        \end{pmatrix}
        =
        \begin{pmatrix}
            3  & 2  \\
            -2 & -1
        \end{pmatrix}
    \end{equation}

    P is the matrix that transforms the standard coordinate vector to the coordinate vector in the new basis.
\end{correction}
\qed


\problem{5}{}

\subproblem{a}

\begin{equation}
    \begin{split}
        f(\alpha \mathbf{a} + \beta \mathbf{u}) &= (\alpha \mathbf{a} + \beta \mathbf{u}) - 2 \left[ \mathbf{n} \cdot (\alpha \mathbf{a} + \beta \mathbf{u}) \right] \mathbf{n} \\
        &= \alpha \mathbf{v} - 2 \left( \mathbf{n} \cdot \alpha \mathbf{a} \right) \mathbf{n} + \beta \mathbf{u} - 2 \left( \mathbf{n} \cdot \beta \mathbf{u} \right) \mathbf{n} \\
        &= \alpha f(\mathbf{a}) + \beta f(\mathbf{u})
    \end{split}
\end{equation}

Thus $f$ is linear.

\begin{equation}
    \begin{split}
        f[f(\mathbf{v})] &= f[\mathbf{v} - 2(\mathbf{n} \cdot \mathbf{v}) \mathbf{n}] \\
        &= \mathbf{v} - 2(\mathbf{n} \cdot \mathbf{v}) \mathbf{n} - 2 \left\{ \mathbf{n} \cdot [\mathbf{v} - 2(\mathbf{n} \cdot \mathbf{v}) \mathbf{n}] \right\} \\
        &= \mathbf{v} - 2(\mathbf{n} \cdot \mathbf{v}) \mathbf{n} - 2 \left\{ \mathbf{n} \cdot \mathbf{v} - 2(\mathbf{n} \cdot \mathbf{v}) (\mathbf{n} \cdot \mathbf{n})] \right\} \\
        &= \mathbf{v}
    \end{split}
\end{equation}

\subproblem{b}
Let $\mathbf{v} = (a, b, c)^{\intercal}$ and $\mathbf{n} = (\alpha, \beta, \gamma)$ relative to the standard basis. Then:

\begin{equation}
    A \mathbf{v} =
    \begin{pmatrix}
        a \\
        b \\
        c
    \end{pmatrix}
    -
    2 (a\alpha + b\beta + c\gamma)
    \begin{pmatrix}
        \alpha \\
        \beta  \\
        \gamma
    \end{pmatrix}
    =
    \begin{pmatrix}
        (1 - 2\alpha^{2}) a - 2\alpha\beta b - 2\alpha\gamma c \\
        -2\alpha\beta a + (1 - 2\beta^{2}) b - 2\beta\gamma c  \\
        -2\alpha\gamma a - 2\beta\gamma b + (1 - 2\gamma^{2}) c
    \end{pmatrix}
\end{equation}

Thus:

\begin{equation}
    A =
    I - 2 \begin{pmatrix}
        \alpha^{2}   & \alpha\beta & \alpha\gamma \\
        \alpha\beta  & \beta^{2}   & \beta\gamma  \\
        \alpha\gamma & \beta\gamma & \gamma^{2}
    \end{pmatrix}
    =
    I - 2 \begin{pmatrix}
        \alpha & \alpha & \alpha \\
        \beta  & \beta  & \beta  \\
        \gamma & \gamma & \gamma
    \end{pmatrix}
    \begin{pmatrix}
        \alpha & 0     & 0      \\
        0      & \beta & 0      \\
        0      & 0     & \gamma
    \end{pmatrix}
\end{equation}

\subproblem{c}
We have $\hat{A} = \psi' f \psi = \psi' \phi \phi' f \phi \phi' \psi = PAP^{-1}$, where $A = \phi' f \phi$, $\phi = I$ and $P = \psi' \phi = \psi' = \left( \mathbf{u}_{1}, \mathbf{u}_{2}, \mathbf{n} \right)^{-1}$. Thus:

\begin{equation}
    \begin{split}
        \hat{A} &=
        I - 2\begin{pmatrix}
            \mathbf{u}_{1} & \mathbf{u}_{2} & \mathbf{n}
        \end{pmatrix}^{-1}
        \begin{pmatrix}
            \alpha & \alpha & \alpha \\
            \beta  & \beta  & \beta  \\
            \gamma & \gamma & \gamma
        \end{pmatrix}
        \begin{pmatrix}
            \alpha & 0     & 0      \\
            0      & \beta & 0      \\
            0      & 0     & \gamma
        \end{pmatrix}
        \begin{pmatrix}
            \mathbf{u}_{1} & \mathbf{u}_{2} & \mathbf{n}
        \end{pmatrix} \\
        &=
        I - 2\begin{pmatrix}
            u_{1x} & u_{1y} & u_{1z} \\
            u_{2x} & u_{2y} & u_{2z} \\
            \alpha & \beta  & \gamma
        \end{pmatrix}
        \begin{pmatrix}
            \alpha & \alpha & \alpha \\
            \beta  & \beta  & \beta  \\
            \gamma & \gamma & \gamma
        \end{pmatrix}
        \begin{pmatrix}
            \alpha & 0     & 0      \\
            0      & \beta & 0      \\
            0      & 0     & \gamma
        \end{pmatrix}
        \begin{pmatrix}
            u_{1x} & u_{2x} & \alpha \\
            u_{1y} & u_{2y} & \beta  \\
            u_{1z} & u_{2z} & \gamma
        \end{pmatrix} \\
        &=
        I - 2\begin{pmatrix}
            u_{1x} & u_{1y} & u_{1z} \\
            u_{2x} & u_{2y} & u_{2z} \\
            \alpha & \beta  & \gamma
        \end{pmatrix}
        \begin{pmatrix}
            0 & 0 & \alpha \\
            0 & 0 & \beta  \\
            0 & 0 & \gamma
        \end{pmatrix} \\
        &=
        I - 2\begin{pmatrix}
            0 & 0 & 0 \\
            0 & 0 & 0 \\
            0 & 0 & 1
        \end{pmatrix} \\
        &=
        \begin{pmatrix}
            1 & 0 & 0  \\
            0 & 1 & 0  \\
            0 & 0 & -1
        \end{pmatrix}
    \end{split}
\end{equation}

where the identity $\mathbf{u}_{\alpha} \cdot \mathbf{n} = 0$ has been used.
\qed


\problem{6}{}

\subproblem{a}
We have that $\tr(A) = \delta_{ij} A_{ij}$. Thus

\begin{equation}
    \tr(AB) = \delta_{ij} (AB)_{ij} = \delta_{ij} A_{ik} B_{kj} = \delta_{ij} A_{jk} B_{ki} = \delta_{ij} B_{ik} A_{kj} = \tr(BA)
\end{equation}

\subproblem{b}

\begin{equation}
    \tr(PAP^{-1}) = \tr(AP^{-1}P) = \tr(A)
\end{equation}

A basis change of matrices should not affect the trace of a matrix.

\subproblem{c}
The trace is unity.
\qed


\problem{7}{}

\subproblem{a}
Because the equation is linear (with respect to the derivatives) and differentiation is a linear operation.

\subproblem{b}
The differentiation operation can be represented by the matrix:

\begin{equation}
    \begin{pmatrix}
        0 & 0 & 0 \\
        2 & 0 & 0 \\
        0 & 1 & 0
    \end{pmatrix}
\end{equation}

Thus:

\begin{equation}
    A =
    \begin{pmatrix}
        0 & 1 & 0 \\
        0 & 0 & 1 \\
        0 & 0 & 0
    \end{pmatrix}
    \begin{pmatrix}
        0 & 0 & 0 \\
        2 & 0 & 0 \\
        0 & 1 & 0
    \end{pmatrix}^{2}
    +
    \begin{pmatrix}
        0 & 0 & 0 \\
        2 & 0 & 0 \\
        0 & 1 & 0
    \end{pmatrix}
    -
    \begin{pmatrix}
        0 & 1 & 0 \\
        0 & 0 & 1 \\
        0 & 0 & 0
    \end{pmatrix}
    \begin{pmatrix}
        0 & 0 & 0 \\
        2 & 0 & 0 \\
        0 & 1 & 0
    \end{pmatrix}
    + 2I
    =
    \begin{pmatrix}
        0 & 0 & 0 \\
        4 & 1 & 0 \\
        0 & 1 & 2
    \end{pmatrix}
\end{equation}

\subproblem{c}
Consider the augmented matrix:

\begin{equation}
    \left( \begin{array}{ccc|c}
        0 & 0 & 0 & 0 \\
        4 & 1 & 0 & 0 \\
        0 & 1 & 2 & 0
    \end{array} \right)
    \to
    \left( \begin{array}{ccc|ccc}
        0 & 0 & 0  & 0 \\
        4 & 0 & -2 & 0 \\
        0 & 1 & 2  & 0
    \end{array} \right)
\end{equation}

Thus:

\begin{equation}
    \ker{A} = \left\{ p(x) \vert p(x) = \lambda(x^{2} - 4x + 2), \lambda \in \mathbb{R} \right\}
\end{equation}

\subproblem{d}

Note that:

\begin{equation}
    D \left( x^{2} - 4x + 2 \right) = x(2) + (1 - x)(2x - 4) + 2x^{2} - 8x + 4 = 0
\end{equation}
\qed


\problem{8}{}

\subproblem{a}

\begin{equation}
    \left( \begin{array}{ccc|c}
        2 - \lambda & 1           & 2           & 0 \\
        1           & 4 - \lambda & -1          & 0 \\
        2           & -1          & 2 - \lambda & 0
    \end{array} \right)
    \to
    \left( \begin{array}{ccc|ccc}
        1 & 4 - \lambda                & -1          & 0 \\
        0 & 2\lambda - 9               & 4 - \lambda & 0 \\
        0 & \lambda^{2} - 4\lambda - 2 & 0           & 0
    \end{array} \right)
\end{equation}

Thus:
\begin{equation}
    \rank(A) =
    \begin{cases}
        2 & \text{if } \lambda = 4 \\
        3 & \text{else}
    \end{cases}
\end{equation}

\subproblem{b}
The solution is either unique, corresponding to $\rank(A) = 3$ or infinite (a line), corresponding to $\rank(A) = 2$.
\qed


\problem{9}{}

\begin{equation}
    \left( \begin{array}{ccc|c}
        1 & 1 & 1  & 1        \\
        1 & 2 & 4  & \eta     \\
        1 & 4 & 10 & \eta^{2}
    \end{array} \right)
    \to
    \left( \begin{array}{ccc|c}
        1 & 1 & 1 & 1                    \\
        0 & 1 & 3 & \eta - 1             \\
        0 & 0 & 0 & (\eta - 1)(\eta - 2)
    \end{array} \right)
\end{equation}

Thus $\rank(A) = 2$ and the solution should be a line. A solution only exists if $\eta = 1$ or $\eta = 2$. If $\eta = 1$, then the solution is $(1, 0, 0)^{\intercal} + \lambda \mathbf{P}$. If $\eta = 2$, then the solution is $(0, 1, 0)^{\intercal} + \lambda \mathbf{P}$, where $\mathbf{P} = (2, -3, 1)^{\intercal}$.
\qed


\problem{10}{}

\begin{equation}
    \left( \begin{array}{ccc|c}
        3 & 2  & -1     & 10    \\
        5 & -1 & -4     & 17    \\
        1 & 5  & \alpha & \beta
    \end{array} \right)
    \to
    \left( \begin{array}{ccc|c}
        1 & 5   & \alpha       & \beta       \\
        0 & -13 & -1 - 3\alpha & 10 - 3\beta \\
        0 & -26 & -4 - 5\alpha & 17 - 5\beta
    \end{array} \right)
    \to
    \left( \begin{array}{ccc|c}
        1 & 5   & \alpha       & \beta       \\
        0 & -13 & -1 - 3\alpha & 10 - 3\beta \\
        0 & 0   & \alpha - 2   & \beta - 3
    \end{array} \right)
\end{equation}

If $\alpha \ne 2$, the solution is unique. If $\alpha = 2$, $\beta = 3$ then the solution is infinite (a line). If $\alpha = 2$ and $\beta \ne 3$, then there is no solution.
\qed


\problem{11}{}

\subproblem{a}
The essential reason is that addition is a linear operation.

\subproblem{b}
We only need to show that the dimension of the semi-magic squares space is five and that the matrices $M_{i}$ are linearly independent. To prove the dimension, note that a $3 \times 3$ matrix can be represented by a vector of length nine:

\begin{equation}
    \mathbf{v} =
    \begin{pmatrix}
        a & b & c \\
        d & e & f \\
        g & h & i
    \end{pmatrix}
    \equiv
    \begin{pmatrix}
        a & b & c & d & e & f & g & h & i
    \end{pmatrix}^{\intercal}
\end{equation}

For a semi-magic square, the elements must satisfy:

\begin{equation}
    a + b + c = d + e + f = g + h + i = a + d + g = b + e + h = c + f + i
\end{equation}

This gives us the following system of equations:

\begin{equation}
    A \mathbf{v} =
    \begin{pmatrix}
        1 & 1  & 1  & -1 & -1 & -1 & 0  & 0  & 0  \\
        1 & 1  & 1  & 0  & 0  & 0  & -1 & -1 & -1 \\
        1 & -1 & 0  & 1  & -1 & 0  & 1  & -1 & 1  \\
        1 & 0  & -1 & 1  & 0  & -1 & 1  & 0  & -1 \\
        0 & 1  & 1  & 0  & -1 & -1 & 0  & 0  & 0  \\
        1 & 0  & 1  & 0  & -1 & 0  & 0  & -1 & 0  \\
    \end{pmatrix}
    \mathbf{v}
    =
    \begin{pmatrix}
        0 \\
        0 \\
        0 \\
        0 \\
        0 \\
        0
    \end{pmatrix}
\end{equation}

Performing Gaussian elimination on $A$ gives us:

\begin{equation}
    \begin{pmatrix}
        1 & 0 & 0 & 0 & 0 & 0  & 0  & -1/2 & -1/2 \\
        0 & 1 & 0 & 0 & 0 & -1 & 0  & 1/2  & -1/2 \\
        0 & 0 & 1 & 0 & 0 & 1  & -1 & -1   & 0    \\
        0 & 0 & 0 & 1 & 0 & 0  & 0  & -1/2 & -1/2 \\
        0 & 0 & 0 & 0 & 1 & 1  & -1 & -1/2 & -1/2 \\
        0 & 0 & 0 & 0 & 0 & 0  & 0  & 0    & 0    \\
    \end{pmatrix}
\end{equation}

Thus, the nullity of $A$ is four and $\rank(A) = 5$. It is also trivial to verify that the matrices $M_{i}$ are linearly independent. Thus, $M_{i}$ form a basis of the semi-magic squares space.
\qed


\problem{12}{}

\subproblem{a}

The equations can be rearranged into a system of $n$ equations with the form:

\begin{equation}
    x_{i} - \sum_{j \in L_{i}} \frac{x_{j}}{n_{j}} = 0
\end{equation}

Thus, the elements of the coefficient matrix $A$ are:

\begin{equation}
    A_{ij} = \delta_{ij} - \Delta_{jL_{i}} \frac{1}{n_{j}}
\end{equation}

where:

\begin{equation}
    \Delta_{jL_{i}} =
    \begin{cases}
        1 \text{ if } j \in L_{i} \\
        1 \text{ if } j \notin L_{i}
    \end{cases}
\end{equation}

Implicit in the definition is the fact that $i \notin L_{i}$ (a site does not have a link to itself).


\end{document}
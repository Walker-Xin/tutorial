\documentclass[12pt]{article}
\usepackage{homework}
\pagestyle{fancy}

% assignment information
\def\course{Ordinary Differential Equations}
\def\assignmentno{Problem Set 6}
\def\assignmentname{Masses on Springs and Other Things}
\def\name{Xin, Wenkang}
\def\time{\today}

\begin{document}

\begin{titlepage}
    \begin{center}
        \large
        \textbf{\course}

        \vfill

        \Huge
        \textbf{\assignmentno}

        \vspace{1.5cm}

        \large{\assignmentname}

        \vfill

        \large
        \name

        \time
    \end{center}
\end{titlepage}


%==========
\pagebreak
\section*{Minimal Set}
%==========


\problem{6.1}{Coupled Pendula}

\subproblem{a}{}
Consider the vector $Y(t) = (y_{1}, y_{2})^{\intercal}$, with the following system of equations:

\begin{equation}
    \ddot{Y} =
    -\begin{pmatrix}
        \omega_{0}^{2} + k/m & -k/m                 \\
        -k/m                 & \omega_{0}^{2} + k/m
    \end{pmatrix}
    Y
\end{equation}

Solving for the eigenvalues yields $\lambda_{1} = \omega_{0}^{2}$, $\Omega_{1} = \omega_{0}$ and $\mathbf{e}_{1} = (1, 1)^{\intercal}$; $\lambda_{2} = \omega_{0}^{2} + 2k/m$, $\Omega_{2} = \sqrt{\omega_{0}^{2} + 2k/m}$ and $\mathbf{e}_{2} = (1, -1)^{\intercal}$.

The solutions are then given by:

\begin{equation}
    \begin{pmatrix}
        y_{1} \\
        y_{2}
    \end{pmatrix} =
    A \sin{(\Omega_{1}t + \phi_{1})}
    \begin{pmatrix}
        1 \\
        1
    \end{pmatrix}
    +
    B \sin{(\Omega_{2}t + \phi_{2})}
    \begin{pmatrix}
        1 \\
        -1
    \end{pmatrix}
\end{equation}

Substituting the initial conditions yields $A = v/(2\Omega_{1})$, $A = v/(2\Omega_{2})$, $\phi_{1} = \phi_{2} = 0$. Therefore:

\begin{equation}
    \begin{split}
        y_{1}(t) &= \frac{v}{2} \left( \frac{1}{\Omega_{1}} \sin{\Omega_{1}t} + \frac{1}{\Omega_{2}} \sin{\Omega_{2}t} \right) \\
        y_{2}(t) &= \frac{v}{2} \left( \frac{1}{\Omega_{1}} \sin{\Omega_{1}t} - \frac{1}{\Omega_{2}} \sin{\Omega_{2}t} \right)
    \end{split}
\end{equation}

\subproblem{b}{}
With $k/m = 2\epsilon g/l$, we have $\Omega_{2} = \sqrt{1 + 4\epsilon} \omega_{0} \approx (1 + 2\epsilon) \Omega_{1}$, so that $\Delta = (\Omega_{2} - \Omega_{1})/2 = \epsilon \omega_{0}$

\subproblem{c}{}
Consider the $\sin{\Omega_{2}t}$ term:

\begin{equation}
    \begin{split}
        \sin{\Omega_{2}t} &= \sin{(\Omega_{1}t + 2\epsilon \Omega_{1}t)} \\
        &= \sin{(\Omega_{1}t)} \cos{(2\epsilon \Omega_{1}t)} + \cos{(\Omega_{1}t)} \sin{(2\epsilon \Omega_{1}t)} \\
        &\approx \sin{(\Omega_{1}t)} + 2\epsilon \Omega_{1}t \cos{(\Omega_{1}t)}
    \end{split}
\end{equation}

Thus the minimal amplitude is at first order of $\epsilon$.
\qed


\problem{6.2}{Hanging Masses}
Let the point of contact with the wall be the origin and take downwards as positive. Let $y_{1}$ be the coordinate of the upper mass and $y_{2}$ be the coordinate of the lower mass. Let the original length of the spring be $L$. Consider the vector $Y(t) = (y_{1}, y_{2})^{\intercal}$, with the following system of equations:

\begin{equation}
    \ddot{Y} =
    -\frac{k}{m}
    \begin{pmatrix}
        2  & -1 \\
        -1 & 1
    \end{pmatrix}
    Y
    +
    \frac{kL}{m}
    \begin{pmatrix}
        0 \\
        1
    \end{pmatrix}
\end{equation}

We may disregard the constant term as it manifests as a constant offset in the solution. Define $\omega_{0} = \sqrt{k/m}$. Solving for the eigenvalues yields $\lambda_{1} = (3 + \sqrt{5})\omega_{0}^{2}/2$ and $\mathbf{e}_{1} = ((-1-\sqrt{5})/2, 1)^{\intercal}$; $\lambda_{2} = (3 - \sqrt{5})\omega_{0}^{2}/2$ and $\mathbf{e}_{2} = ((-1+\sqrt{5})/2, 1)^{\intercal}$. Thus, the normal modes are:

\begin{equation}
    q_{1} = \frac{-1-\sqrt{5}}{2} y_{1} + y_{2}
\end{equation}

corresponding to the eigenfrequency $\Omega_{1} = \sqrt{\frac{3 + \sqrt{5}}{2}} \omega_{0}$, and:

\begin{equation}
    q_{2} = \frac{-1+\sqrt{5}}{2} y_{1} + y_{2}
\end{equation}

corresponding to the eigenfrequency $\Omega_{2} = \sqrt{\frac{3 - \sqrt{5}}{2}} \omega_{0}$.

The solutions are then given by:

\begin{equation}
    \begin{pmatrix}
        y_{1} \\
        y_{2}
    \end{pmatrix} =
    A \cos{(\Omega_{1}t + \phi_{1})}
    \begin{pmatrix}
        \frac{-1-\sqrt{5}}{2} \\
        1
    \end{pmatrix}
    +
    B \cos{(\Omega_{2}t + \phi_{2})}
    \begin{pmatrix}
        \frac{-1+\sqrt{5}}{2} \\
        1
    \end{pmatrix}
\end{equation}

If $y_{2} = 2a$ at rest, by symmetry, for the upper mass to be at rest, $y_{1} = a$. This constitutes the initial conditions $Y(0) = (a, 2a)$ and $\dot{Y}(0) = (0, 0)$. Substituting the initial conditions yields $A = \frac{5 - 3\sqrt{5}}{10} a$, $B = \frac{5 + 3\sqrt{5}}{10} a$, $\phi_{1} = \phi_{2} = 0$. Therefore:

\begin{equation}
    \begin{split}
        y_{1}(t) &= \frac{5 - \sqrt{5}}{10} a \cos{\Omega_{1}t} + \frac{5 + \sqrt{5}}{10} a \cos{\Omega_{2}t} \\
        y_{2}(t) &= a \cos{\Omega_{1}t} + a \cos{\Omega_{2}t}
    \end{split}
\end{equation}

By conservation of energy, the total energy is given by:

\begin{equation}
    E = \frac{1}{2} k y_{1}(0)^{2} + \frac{1}{2} k y_{2}(0)^{2} = ka^{2}
\end{equation}

The ratio is given by:

\begin{equation}
    E_{q_{1}}/E_{q_{2}} = \frac{\left( \frac{-1-\sqrt{5}}{2} \right)^{2} + 1}{\left( \frac{-1+\sqrt{5}}{2} \right)^{2} + 1}
\end{equation}
\qed


\problem{6.3}{Sliding Masses}

\subproblem{a}{}
Let $y_{1}$ be the displacement of the left mass, with rightwards taken as positive, and let $y_{2}$ be the displacement of the right mass, with leftwards taken as positive. Consider the vector $Y(t) = (y_{1}, y_{2})^{\intercal}$, with the following system of equations:

\begin{equation}
    \ddot{Y} =
    -\frac{1}{m}
    \begin{pmatrix}
        k_{0} + k_{1} & k_{1}         \\
        k_{1}         & k_{0} + k_{1}
    \end{pmatrix}
    Y
\end{equation}

Define $\omega_{0} = \sqrt{k_{0}/m}$ and $\omega_{1} = \sqrt{k_{1}/m}$. Solving for the eigenvalues yields $\lambda_{1} = \omega_{0}^{2}$, $\Omega_{1} = \omega_{0}$ and $\mathbf{e}_{1} = (1, -1)^{\intercal}$; $\lambda_{2} = \omega_{0}^{2} + 2\omega_{1}^{2}$, $\Omega_{2} = \sqrt{\omega_{0}^{2} + 2\omega_{1}^{2}}$ and $\mathbf{e}_{2} = (1, 1)^{\intercal}$. We also know that $\Omega_{0} = \sqrt{(k_{0} + k_{1})/m} = \sqrt{\omega_{0}^{2} + \omega_{1}^{2}}$. Therefore:

\begin{equation}
    \Omega_{2} = \sqrt{2\Omega_{0}^{2} - \Omega_{1}^{2}}
\end{equation}

and:

\begin{equation}
    \frac{k_{1}}{k_{0}} = \frac{\sqrt{\Omega_{0}^{2} - \Omega_{1}^{2}}}{\Omega_{0}}
\end{equation}

\subproblem{b}{}
The normal modes are $p_{1} = y_{1} - y_{2}$ and $p_{2} = y_{1} + y_{2}$. $p_{1}$ corresponds to the masses moving in same directions with the same displacement (lower effective spring constant), and $p_{2}$ corresponds to the masses moving in the opposite direction with the same displacement (higher spring constant).

\subproblem{c}{}
With friction taken into account, the system of equations becomes:

\begin{equation}
    \ddot{Y} =
    -\frac{1}{m}
    \begin{pmatrix}
        k_{0} + k_{1} & k_{1}         \\
        k_{1}         & k_{0} + k_{1}
    \end{pmatrix}
    Y
    -2\gamma
    \dot{Y}
\end{equation}

The general solution has the form

\begin{equation}
    \begin{pmatrix}
        y_{1} \\
        y_{2}
    \end{pmatrix} =
    (Ae^{r_{1+}t} + Be^{r_{1-}t})
    \begin{pmatrix}
        1 \\
        -1
    \end{pmatrix}
    +
    (Ce^{r_{2+}t} + De^{r_{2-}t})
    \begin{pmatrix}
        1 \\
        1
    \end{pmatrix}
\end{equation}

where $r_{1,2}$ satisfy the characteristic equation:

\begin{equation}
    r_{1,2}^{2} + 2\gamma r_{1,2} + \Omega_{1,2}^{2} = 0
\end{equation}

Depending on the sign of $\gamma^{2} - \Omega_{1,2}^{2}$, the solution may be overdamped, critically damped, or underdamped.

\subproblem{d}{}
In the limit $\omega_{0} \ll \gamma \ll \omega_{1}$, we have $r_{1\pm} = -\gamma \pm w_{1}$, where $w_{1}^{2} = \gamma^{2} - \Omega_{1}^{2}$, and $r_{2\pm} = -\gamma \pm iw_{2}$, where $w_{2}^{2} = \Omega_{2}^{2} - \gamma^{2}$. The solution associated with $p_{1}$ is overdamped, and the solution associated with $p_{2}$ is underdamped. Given the initial conditions $Y(0) = (a, 0)^{\intercal}$ and $\dot{Y}(0) = (0, 0)^{\intercal}$, the solutions are:

\begin{equation}
    \begin{split}
        y_{1}(t) = \frac{a}{4} e^{-\gamma t} \left( \frac{w_{1} + \gamma}{w_{1}} e^{w_{1}t} + \frac{w_{1} - \gamma}{w_{1}} e^{-w_{1}t} \right) + \frac{a}{2} e^{-\gamma t} \cos{w_{2}t} \\
        y_{2}(t) = -\frac{a}{4} e^{-\gamma t} \left( \frac{w_{1} + \gamma}{w_{1}} e^{w_{1}t} + \frac{w_{1} - \gamma}{w_{1}} e^{-w_{1}t} \right) + \frac{a}{2} e^{-\gamma t} \cos{w_{2}t}
    \end{split}
\end{equation}

Essentially, this tells us that $p_{1}$ component (moving in same directions) decays much faster than the $p_{2}$ component (moving in opposite direction). This is because in the $p_{1}$ mode, the central spring is not stretched, and the damping effect is much stronger than the spring effect.

\subproblem{e}{}
In the limit $\gamma \ll \omega_{0}$, we have $r_{1\pm} = -\gamma \pm iw_{1}$, where $w_{1}^{2} = \Omega_{1}^{2} - \gamma^{2}$, and $r_{2\pm} = -\gamma \pm iw_{2}$, where $w_{2}^{2} = \Omega_{2}^{2} - \gamma^{2}$. Both modes are underdamped. If the right wall oscillates with $y_{0} = a \sin{\Omega_{2}t}$, the equation of motion becomes:

\begin{equation}
    \ddot{Y} =
    -\frac{1}{m}
    \begin{pmatrix}
        k_{0} + k_{1} & k_{1}         \\
        k_{1}         & k_{0} + k_{1}
    \end{pmatrix}
    Y
    -2\gamma
    \dot{Y}
    -
    \frac{k_{0}}{m}y_{0}
    \begin{pmatrix}
        0 \\
        1
    \end{pmatrix}
\end{equation}

If we transform to the normal coordinates, the following equation results:

\begin{equation}
    \ddot{Y'} =
    -\begin{pmatrix}
        \lambda_{1} & 0           \\
        0           & \lambda_{2}
    \end{pmatrix}
    Y'
    -2\gamma
    \dot{Y'}
    -
    \frac{k_{0}}{m}y_{0}
    \begin{pmatrix}
        -1 \\
        1
    \end{pmatrix}
\end{equation}

The long term behaviour would be the particular solution to the set of equations:

\begin{equation}
    \begin{split}
        \ddot{p}_{1} + 2\gamma \dot{p}_{1} + \Omega_{1}^{2} p_{1} &= \omega_{0}^{2}a \sin{\Omega_{2}t} \\
        \ddot{p}_{2} + 2\gamma \dot{p}_{2} + \Omega_{2}^{2} p_{2} &= -\omega_{0}^{2}a \sin{\Omega_{2}t}
    \end{split}
\end{equation}

The particular solutions are of the form:

\begin{equation}
    \begin{split}
        p_{1p}(t) &= C_{1} \sin{(\Omega_{2}t + \phi_{1})} \\
        p_{2p}(t) &= C_{2} \sin{(\Omega_{2}t + \phi_{2})}
    \end{split}
\end{equation}

The constants can be solved by substitution, although the algebra is too nasty to carry out. $y_{1}(t)$ and $y_{2}(t)$ can be obtained by noting $y_{1} = (p_{1} + p_{2})/2$ and $y_{2} = (p_{1} - p_{2})/2$. It is worth noting that $C_{2}$ would likely be much higher as the driving frequency is close to the eigenfrequency of $q_{2}$.
\qed


\problem{6.4}{Sliding Masses Unbound}

Let the the coordinate of the masses be $y_{1}$ and $y_{2}$ respectively (measured in the same direction) and let the original length of the spring be $l$. The system of equations is:

\begin{equation}
    \ddot{Y} =
    -\begin{pmatrix}
        k/m_{1}  & -k/m_{1} \\
        -k/m_{2} & k/m_{2}
    \end{pmatrix}
    Y
    +
    \begin{pmatrix}
        -lk/m_{1} \\
        lk/m_{2}
    \end{pmatrix}
\end{equation}

Solving for the eigenvalues and eigenvectors, we have $\lambda_{1} = 0$ with eigenvector $\mathbf{e}_{1} = (1, 1)^{\intercal}$, and $\lambda_{2} = k(1/m_{1} + 1/m_{2})$ with eigenvector $\mathbf{e}_{2} = (-m_{2}/m_{1}, 1)^{\intercal}$.

The normal modes are $p_{1} = y_{1} + y_{2}$ and $p_{2} = -(m_{2}/m_{1})y_{1} + y_{2}$. $p_{1}$ is the normal mode where the masses have the same displacement so that the spring is not stretched; $p_{2}$ is the normal mode where the masses have opposite displacements such that the centre of mass is stationary.

Disregarding the inhomogeneous term, the general solution is:

\begin{equation}
    \begin{pmatrix}
        y_{1} \\
        y_{2}
    \end{pmatrix} =
    (A\cos{\omega t} + B\sin{\omega t})
    \begin{pmatrix}
        -m_{2}/m_{1} \\
        1
    \end{pmatrix}
    +
    (Ct + D)
    \begin{pmatrix}
        1 \\
        1
    \end{pmatrix}
\end{equation}

where $\omega = \sqrt{k(1/m_{1} + 1/m_{2})}$.

Consider the initial conditions $Y(0) = (0, l)$ and $\dot{Y}(0) = (v, 0)$. The full solutions are:

\begin{equation}
    \begin{split}
        y_{1}(t) &= \frac{m_{2}}{m_{1} + m_{2}} \left( \frac{v}{\omega} \sin{\omega t} - l\cos{\omega t} \right) + \frac{m_{1}}{m_{1} + m_{2}} vt + \frac{m_{2}}{m_{1} + m_{2}} l \\
        y_{2}(t) &= \frac{m_{1}}{m_{1} + m_{2}} \left( l\cos{\omega t} - \frac{v}{\omega} \sin{\omega t} \right) + \frac{m_{1}}{m_{1} + m_{2}} vt + \frac{m_{2}}{m_{1} + m_{2}} l
    \end{split}
\end{equation}

The total energy is $mv^{2}/2$.
\qed


\end{document}
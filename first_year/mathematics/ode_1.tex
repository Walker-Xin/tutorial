\documentclass[12pt]{article}
\usepackage{homework}
\pagestyle{fancy}

% assignment information
\def\course{Ordinary Differential Equations}
\def\assignmentno{Problem Set 1}
\def\assignmentname{First-Order ODEs}
\def\name{Xin, Wenkang}
\def\time{\today}

\begin{document}

\begin{titlepage}
    \begin{center}
        \large
        \textbf{\course}

        \vfill

        \Huge
        \textbf{\assignmentno}

        \vspace{1.5cm}

        \large{\assignmentname}

        \vfill

        \large
        \name

        \time
    \end{center}
\end{titlepage}


%==========
\pagebreak
\section*{Minimal Set}
%==========


\problem{1.1}{}

\subproblem{i}
Second order linear DE.

\subproblem{ii}
Third order non-linear DE.

\subproblem{iii}
First order non-linear DE.
\qed


\problem{1.2}{}
All $C$ appearing in the following solutions are arbitrary constants unless otherwise stated.

\subproblem{a}
Let $\partial \Phi/\partial y = Q(x, y)$, such that:

\begin{equation}
    \Phi(x, y) = y^{2} \cos{x} + 2xy \cos{x} + \phi(x)
\end{equation}

for some function $\phi(x)$.

Differentiating with respect to $x$ and comparing with $P(x, y)$:

\begin{equation}
    \begin{split}
        -y^{2} \sin{x} + 2y (\cos{x} - x \sin{x}) + \phi'(x) &= P(x, y) \\
        \phi'(x) = 6x \cos{x} - 3x^{2} \sin{x} \\
        \phi(x) = 3x^{2} \cos{x} + C
    \end{split}
\end{equation}

Therefore, the solution $\Phi(x, y)$ is given by:

\begin{equation}
    \Phi(x, y) = y^{2} \cos{x} + 2xy \cos{x} + 3x^{2} \cos{x} + C
\end{equation}

\subproblem{b}

\begin{correction}
    Consider the correction factor $\Lambda(x) = x^{-3}$.
\end{correction}

\subproblem{c}

\subsubproblem{i}

\begin{equation}
    \begin{split}
        \int_{0}^{y} \frac{1}{e^{y}} \, \mathrm{d}y &= \int_{0}^{x} \frac{x}{1 + x^{2}} \, \mathrm{d}x \\
        1 - e^{-y} &= \frac{\ln{\left\lvert 1 + x^{2} \right\rvert}}{2} \\
        y &= \ln{\left( \frac{2}{2 - \ln{\left\lvert 1 + x^{2} \right\rvert}} \right)}
    \end{split}
\end{equation}

\subsubproblem{ii}

\begin{equation}
    \begin{split}
        y' &= \frac{x(2y^{2} + 1)}{y(x^{2} - 1)} \\
        \int \frac{y}{2y^{2} + 1} \, \mathrm{d}y &= \int \frac{x}{x^{2} - 1} \, \mathrm{d}x \\
        \frac{1}{4} \ln{\left\lvert 2y^{2} + 1 \right\rvert} &= \frac{1}{2} \ln{\left\lvert x^{2} - 1 \right\rvert} + C \\
        \ln{\left\lvert 2y^{2} + 1 \right\rvert} &= \ln{\left[ C \left\lvert x^{2} - 1 \right\rvert^{2} \right]}
    \end{split}
\end{equation}

\begin{equation}
    y = \sqrt{\frac{C(x^{2} - 1)^{2} - 1}{2}}
\end{equation}

\subproblem{d}
Let $z = 2x + y$, such that $\mathrm{d}z/\mathrm{d}x = 2 + \mathrm{d}y/\mathrm{d}x$ and thus:

\begin{equation}
    \begin{split}
        \frac{\mathrm{d}z}{\mathrm{d}x} - 2 &= 2z^{2} \\
        \int \frac{1}{z^{2} + 1} \, \mathrm{d}z &= \int 2 \, \mathrm{d}x \\
        \tan^{-1}{(z)} &= 2x + C \\
        y &= \tan{(2x + C)} - 2x
    \end{split}
\end{equation}

\subproblem{e}
As the RHS is homogeneous, dividing the numerator and denominator by $x^{2}$ yields:

\begin{equation}
    y' = \frac{y/x + (y/x)^{2}}{2}
\end{equation}

Let $z = y/x$, such that $\mathrm{d}z/\mathrm{d}x = y'/x - z/x$ and thus:

\begin{equation}
    \begin{split}
        x \frac{\mathrm{d}z}{\mathrm{d}x} + z &= \frac{z + z^{2}}{2} \\
        2\int \left( \frac{1}{z - 1} - \frac{1}{z} \right) \, \mathrm{d}z &= \int \frac{1}{x} \, \mathrm{d}x \\
        2\ln{\left\lvert z - 1 \right\rvert} - 2\ln{\left\lvert z \right\rvert} &= \ln{\left\lvert x \right\rvert} + C \\
    \end{split}
\end{equation}

\begin{equation}
    y =
    \begin{cases}
        \frac{x}{1 - C\sqrt{\left\lvert x \right\rvert}}, \, z > 1 \text{ or } z < 0 \\
        \frac{x}{1 + C\sqrt{\left\lvert x \right\rvert}}, \, 0 < z < 1
    \end{cases}
\end{equation}

where $x \ne 0$.

\subproblem{f}

Let $p = x - 3/2$ and $q = y + 1/2$, such that $y' = \mathrm{d}q/\mathrm{d}p$ and:

\begin{equation}
    \begin{split}
        \frac{\mathrm{d}q}{\mathrm{d}p} = \frac{p + q}{p - q} = \frac{1 + q/p}{1 - q/p}
    \end{split}
\end{equation}

Let $z = q/p$ so that following the standard procedure:

\begin{equation}
    \begin{split}
        \int \frac{1}{\frac{1 + z}{1 - z} - z} \, \mathrm{d}z &= \ln{\left\lvert p \right\rvert} + C \\
        \int \frac{1 - z}{z^{2} + 1} \, \mathrm{d}z &= \ln{\left\lvert p \right\rvert} + C \\
        \tan^{-1}{(z)} - \frac{1}{2} \ln{\left\lvert z^{2} + 1 \right\rvert} &= \ln{\left\lvert p \right\rvert} + C \\
        z &= \tan{\left[ \ln{(C \left\lvert p \right\rvert \sqrt{\left\lvert z^{2} + 1 \right\rvert})} \right]}
    \end{split}
\end{equation}

where $z \ne 1$.

This equation gives an implicit relationship between $x$ and $y$.

\subproblem{g}

\subsubproblem{i}
Consider the following integrating factor:

\begin{equation}
    \Lambda(x) = \exp(\int \frac{1}{x} \, \mathrm{d}x) = x
\end{equation}

Multiplying both sides by $\Lambda(x)$:

\begin{equation}
    \begin{split}
        xy' + y &= 3x \\
        \frac{\mathrm{d}}{\mathrm{d}x} (yx) &= 3x \\
        y &= \frac{3x}{2} + \frac{C}{x}
    \end{split}
\end{equation}

where $x \ne 0$.

\subsubproblem{ii}
Consider the following integrating factor:

\begin{equation}
    \Lambda(x) = \exp(\int \cos{x} \, \mathrm{d}x) = e^{\sin{x}}
\end{equation}

Multiplying both sides by $\Lambda(x)$:

\begin{equation}
    \begin{split}
        \frac{\mathrm{d}}{\mathrm{d}x} \left( e^{\sin{x}} y \right) &= 2e^{\sin{x}} \sin{x} \cos{x} \\
        e^{\sin{x}} y &= 2 \left( e^{\sin{x}} \sin{x} - \int e^{\sin{x}} \cos{x} \, \mathrm{d}x \right) \\
        e^{\sin{x}} y &= 2 \left( e^{\sin{x}} \sin{x} - e^{\sin{x}} + C \right) \\
        y &= 2\sin{x} - 2 + Ce^{-\sin{x}}
    \end{split}
\end{equation}

\subproblem{h}
For this Bernoulli's equation, divide the equation by $y^{2/3}$:

\begin{equation}
    \begin{split}
        y' y^{-2/3} + y^{1/3} &= x \\
        3 \left( \frac{\mathrm{d}y^{1/3}}{\mathrm{d}x} \right) + y^{1/3} &= x
    \end{split}
\end{equation}

Let $z = y^{1/3}$ so that:

\begin{equation}
    \begin{split}
        \frac{\mathrm{d}z}{\mathrm{d}x} + \frac{z}{3} &= \frac{x}{3} \\
        \frac{\mathrm{d}}{\mathrm{d}x} \left( e^{x/3} z \right) &= e^{x/3} \frac{x}{3} \\
        e^{x/3} z &= x e^{x/3} - 3e^{x/3} + C \\
        y &= \left( x - 3 + Ce^{-x/3} \right)^{3}
    \end{split}
\end{equation}

The trivial solution is $y = 0$.
\qed


\problem{1.3}{}
All $C$ appearing in the following solutions are arbitrary constants unless otherwise stated.

\subproblem{i}
Let $\partial \Phi/\partial y = Q(x, y)$, such that:

\begin{equation}
    \Phi(x, y) = y \sin{x} + \phi(x)
\end{equation}

for some function $\phi(x)$.

Differentiating with respect to $x$ and comparing with $P(x, y)$:

\begin{equation}
    \begin{split}
        y \cos{x} + \phi'(x) &= P(x, y) \\
        \phi'(x) = -x \\
        \phi(x) = -\frac{x^{2}}{2} + C
    \end{split}
\end{equation}

Therefore, the solution $\Phi(x, y)$ is given by:

\begin{equation}
    \Phi(x, y) = y \sin{x} - \frac{x^{2}}{2} + C
\end{equation}

where $x \ne 0$.

\subproblem{ii}

\begin{equation}
    \begin{split}
        \int \frac{1}{5y - 8} \, \mathrm{d}y &= \int \frac{1}{3x + x^{2}} \, \mathrm{d}x \\
        \frac{1}{5} \ln{\left\lvert 5y - 8 \right\rvert} &= \frac{1}{3} \ln{\left\lvert \frac{x}{x + 3} \right\rvert} + C \\
        y &= \frac{1}{5} \left( \pm C \left\lvert \frac{x}{x + 3} \right\rvert^{5/3} + 8 \right)
    \end{split}
\end{equation}

where $x \ne 0, -3$.

The trivial solution is $y = 8/5$.

\subproblem{iii}

\begin{equation}
    y' = 3 - \frac{2x}{y} = \frac{3y - 2x}{y} = \frac{3(y/x) - 2}{y/x}
\end{equation}

Let $z = y/x$ and following the standard procedure:

\begin{equation}
    \begin{split}
        \int \frac{z}{3z - 2 - z^{2}} \, \mathrm{d}z &= \ln{x} + C \\
        \frac{1}{2} \ln{(z^{2} - 3z + 2)} - \frac{3}{2} \left[ \ln{(z - 2)} - \ln{(z - 1)} \right] &= \ln{\frac{C}{\left\lvert x \right\rvert}} \\
        \frac{y/x - 1}{(y/x - 2)^{2}} &= \frac{C}{\left\lvert x \right\rvert}
    \end{split}
\end{equation}

where $x \ne 0$.

This equation gives an implicit relationship between $x$ and $y$.

The trivial solution is $y = x$.

\subproblem{iv}

\begin{equation}
    \begin{split}
        y' + \frac{y}{x} &= 2 x^{3/2} y^{1/2} \\
        y^{-1/2} y' + \frac{y^{1/2}}{x} &= 2 x^{3/2} \\
        2 \frac{\mathrm{d}}{\mathrm{d}x} \left( y^{1/2} \right) + \frac{y^{1/2}}{x} &= 2 x^{3/2}
    \end{split}
\end{equation}

Let $z = y^{1/2}$ and consider the integrating factor $\Lambda(x) = \sqrt{x}$:

\begin{equation}
    \begin{split}
        \frac{\mathrm{d}}{\mathrm{d}x} \left( \sqrt{x} z \right) &= 2x^{2} \\
        \sqrt{x} z &= \frac{2}{3} x^{3} + C \\
        y &= \left( \frac{2}{3} x^{5/2} + Cx^{-1/2} \right)^{2}
    \end{split}
\end{equation}

where $x \ne 0$.

\subproblem{v}

\begin{equation}
    \begin{split}
        2y' &= \frac{y}{x} + \frac{y^{3}}{x^{3}} \\
        y' &= \frac{(y/x)^{3} + (y/x)}{2}
    \end{split}
\end{equation}

Let $z = y/x$ and following the standard procedure:

\begin{equation}
    \begin{split}
        \int \frac{2}{z^{3} - z} \, \mathrm{d}z &= \ln{\left\lvert x \right\rvert} + C \\
        \frac{1}{2} \ln{\left\lvert z^{2} - 1 \right\rvert} - \ln{\left\lvert z \right\rvert} &= \ln{C \left\lvert x \right\rvert} \\
        \frac{\sqrt{y^{2} - x^{2}}}{y} &= \pm Cx
    \end{split}
\end{equation}

where $x \ne 0$.

This equation gives an implicit relationship between $x$ and $y$. The trivial solution is $y = x$.

\subproblem{vi}

\begin{equation}
    \begin{split}
        xy y' - y^{2} = (x + y)^{2} e^{-y/x} \\
        y' = \frac{y}{x} \left( \frac{x}{y} + 2 + \frac{y}{x} \right) e^{-y/x}
    \end{split}
\end{equation}

Let $z = y/x$ and following the standard procedure:

\begin{equation}
    \begin{split}
        \int \frac{e^{z}}{z + 2 + 1/z} \, \mathrm{d}z = \ln{x} + C \\
        \frac{e^{y/x}}{y/x + 1} = \ln{C\left\lvert x \right\rvert}
    \end{split}
\end{equation}

This equation gives an implicit relationship between $x$ and $y$.

\subproblem{vii}

\begin{equation}
    \begin{split}
        x(x - 1)y' + y &= x(x - 1)^{2} \\
        y' + \frac{y}{x(x - 1)} &= x - 1
    \end{split}
\end{equation}

Consider the integrating factor $\Lambda(x) = (x - 1)/x$:

\begin{equation}
    \begin{split}
        \frac{\mathrm{d}}{\mathrm{d}x} \left( \frac{x - 1}{x} y \right) &= \frac{(x - 1)^{2}}{x} \\
        \frac{x - 1}{x} y &= \frac{x^{2}}{2} - 2x + \ln{\left\lvert x \right\rvert} + C \\
        y &= \frac{x}{x - 1} \left( \frac{x^{2}}{2} - 2x + \ln{\left\lvert x \right\rvert} + C \right)
    \end{split}
\end{equation}

where $x \ne 0, 1$.

\subproblem{viii}

\begin{equation}
    \begin{split}
        2xy' - y = x^{2} \\
        y' - \frac{y}{2x} = \frac{x}{2}
    \end{split}
\end{equation}

Consider the integrating factor $\Lambda(x) = 1/\sqrt{x}$:

\begin{equation}
    \begin{split}
        \frac{\mathrm{d}}{\mathrm{d}x} \left( \frac{y}{\sqrt{x}} \right) &= \frac{\sqrt{x}}{2} \\
        \frac{y}{\sqrt{x}} &= \frac{x^{3/2}}{3} + C \\
        y &= \frac{x^{2}}{3} + C \sqrt{x}
    \end{split}
\end{equation}

where $x \ne 0$.

\subproblem{ix}
Let $z = y + x$ so that:

\begin{equation}
    \begin{split}
        z' &= \cos{z} + 1 \\
        \int_{\pi/2}^{z} \frac{1}{\cos{z} + 1} \, \mathrm{d}z &= \int_{0}^{x} 1 \, \mathrm{d}x \\
        \tan{z/2} - 1 &= x \\
        y &= 2\tan^{-1}{(x + 1)} - x
    \end{split}
\end{equation}

\subproblem{x}
Let $z = x - y$, such that $\mathrm{d}z/\mathrm{d}x = 1 - \mathrm{d}y/\mathrm{d}x$:

\begin{equation}
    \begin{split}
        1  - \frac{\mathrm{d}z}{\mathrm{d}x} &= \frac{z}{z + 1} \\
        \frac{\mathrm{d}z}{\mathrm{d}x} &= \frac{1}{z + 1} \\
        \frac{z^{2}}{2} + z &= x + C \\
        x^{2} - 2xy + y^{2} - 2y + C &= 0 \\
        y &= x + 1 \pm \sqrt{2x + 1 + C}
    \end{split}
\end{equation}

\subproblem{xi}

\begin{equation}
    y' + \frac{y}{\tan{x}} = \cos{2x}
\end{equation}

Consider the integrating factor $\Lambda(x) = \sin{x}$:

\begin{equation}
    \begin{split}
        \frac{\mathrm{d}}{\mathrm{d}x} (y \sin{x}) &= \sin{x} \left( 2\cos^{2}{x} - 1 \right) \\
        y \sin{x} &= -\frac{2}{3} \cos^{3}{x} + \cos{x} + C \\
        y &= \frac{1}{\sin{x}} \left( \cos{x} - \frac{2}{3} \cos^{3}{x} + C \right)
    \end{split}
\end{equation}

Given the boundary conditions $y = 1/2$ at $x = \pi/2$, have:

\begin{equation}
    y = \frac{1}{\sin{x}} \left( \cos{x} - \frac{2}{3} \cos^{3}{x} + \frac{1}{2} \right)
\end{equation}

where $x \ne 0$.

\subproblem{xii}

First consider the case where $n = 0$ and $y' + ky = \sin{x}$. Consider the integrating factor $\Lambda(x) = e^{kx}$:

\begin{equation}
    \begin{split}
        \frac{\mathrm{d}}{\mathrm{d}x} \left( y e^{kx} \right) = e^{kx} \sin{x} \\
        y e^{kx} = \int e^{kx} \sin{x} \, \mathrm{d}x
    \end{split}
\end{equation}

Let $I(x) = \int e^{kx} \sin{x} \, \mathrm{d}x$. Integrating by parts:

\begin{equation}
    \begin{split}
        I(x) &= \frac{1}{k} e^{kx} \sin{x} - \int \frac{1}{k} e^{kx} \cos{x} \, \mathrm{d}x \\
        &= \frac{1}{k} e^{kx} \sin{x} - \frac{1}{k^{2}} e^{kx} \cos{x} - \int \frac{1}{k^{2}} e^{kx} \sin{x} \, \mathrm{d}x \\
        &= \frac{1}{k} e^{kx} \sin{x} - \frac{1}{k^{2}} e^{kx} \cos{x} - I(x) \\
        I(x) &= \frac{1}{k^{2} + 1} e^{kx} \left( k \sin{x} - \cos{x} \right) + C
    \end{split}
\end{equation}

Therefore:

\begin{equation}
    y = \frac{1}{k^{2} + 1} \left( k \sin{x} - \cos{x} \right) + C e^{-kx}
\end{equation}

Next consider the case where $n > 1$ and the equation is now a Bernoulli's equation. Dividing by $y^{n}$ yields:

\begin{equation}
    \begin{split}
        y^{-n} y' + k y^{1-n} = \sin{x} \\
        \frac{1}{1 - n} \frac{\mathrm{d}}{\mathrm{d}x} \left( y^{1 - n} \right) + k y^{1-n} = \sin{x}
    \end{split}
\end{equation}

Let $z = y^{1 - n}$ and consider the integrating factor $\Lambda(x) = e^{(1 - n)kx}$:

\begin{equation}
    \begin{split}
        z' + (1 - n)kz &= (1 - n) \sin{x} \\
        z e^{(1 - n)kx} &= \int (1 - n) e^{(1 - n)kx} \sin{x} \, \mathrm{d}x \\
        &= \frac{1 - n}{(1 - n)^{2} k^{2} + 1} e^{(1 - n)kx} \left[ (1 - n)k \sin{x} - \cos{x} \right] + C\\
        y &= \left\{ \frac{1 - n}{(1 - n)^{2} k^{2} + 1} \left[ (1 - n)k \sin{x} - \cos{x} \right] + C e^{(n - 1)kx} \right\}^{1/(1 - n)}
    \end{split}
\end{equation}
\qed

%==========
\pagebreak
\section*{Supplementary Questions}
%==========


Left for revision.


%==========
\pagebreak
\section*{Extracurricular Questions}
%==========


\problem{1.8}{Integral curves and orthogonal curves}

\subproblem{a}
Taking the differential:

\begin{equation}
    \begin{split}
        \mathrm{d}y - \frac{1}{x} \sec^{2}{\left[ \ln{(Cx)} \right]} \mathrm{d}x - \frac{1}{C} \sec^{2}{\left[ \ln{(Cx)} \right]} \mathrm{d}C = 0 \\
        \mathrm{d}y - \sec^{2}{\left[ \ln{(Cx)} \right]} \left( \frac{1}{x} \mathrm{d}x  + \frac{1}{C} \frac{\partial C}{\partial x} \mathrm{d}x + \frac{1}{C} \frac{\partial C}{\partial y} \mathrm{d}y \right) \\
        \left( 1 - \frac{\sec^{2}{\left[ \ln{(Cx)} \right]}}{C} \frac{\partial C}{\partial y} \right) \mathrm{d}y - \sec^{2}{\left[ \ln{(Cx)} \right]} \left( \frac{1}{x} + \frac{1}{C} \frac{\partial C}{\partial x} \right) \mathrm{d}x = 0
    \end{split}
\end{equation}

This ODE, with an arbitrary function $C(x, y)$, has the integral curves of the specified form.

\subproblem{b}

Let $F(x, y, y') = 0$ have the integral curves:

\begin{equation}
    f(x, y, C) = 0
\end{equation}

where $C$ is an arbitrary constant.

Then it must be the case that the equation:

\begin{equation}
    \left( \frac{\partial f}{\partial y} + \frac{\partial f}{\partial C} \frac{\partial C}{\partial y} \right) \mathrm{d}y + \left( \frac{\partial f}{\partial x} + \frac{\partial f}{\partial C} \frac{\partial C}{\partial x} \right) \mathrm{d}x = 0
\end{equation}

is equivalent to $F(x, y, y') = 0$.

Making $y'$ the subject

\begin{equation}
    y' = -\left( \frac{\partial f}{\partial x} + \frac{\partial f}{\partial C} \frac{\partial C}{\partial x} \right) / \left( \frac{\partial f}{\partial y} + \frac{\partial f}{\partial C} \frac{\partial C}{\partial y} \right)
\end{equation}

Then the curves $g(x, y, C) = 0$ orthogonal to $f(x, y, C) = 0$ must satisfy the ODE:

\begin{equation}
    y' = \left( \frac{\partial f}{\partial y} + \frac{\partial f}{\partial C} \frac{\partial C}{\partial y} \right) / \left( \frac{\partial f}{\partial x} + \frac{\partial f}{\partial C} \frac{\partial C}{\partial x} \right)
\end{equation}

But this is equivalent to $F(x, y, -1/y') = 0$. Thus, the integral curves of $F(x, y, -1/y') = 0$ are orthogonal to those of $F(x, y, y') = 0$.

\subproblem{c}

\begin{equation}
    x + y'(y + 1) = 0
\end{equation}

Then the orthogonal curves satisfy the ODE:

\begin{equation}
    y' = \frac{y + 1}{x}
\end{equation}

The solution to the ODE is:

\begin{equation}
    y = Dx - 1
\end{equation}
\qed


\problem{1.9}{Riccati equations}

\subproblem{a}
Apparently, $y_{0}(x) = e^{x}$ is a particular solution to the ODE. Let $y(x) = z(x) + y_{0}(x)$:

\begin{equation}
    \begin{split}
        y' &= z' + y_{0}' = z^{2} + 2z y_{0} + y_{0}^{2} - 2e^{x}(z + y_{0}) + e^{2x} + e^{x} \\
        z' &= z^{2} \\
        z &= -\frac{1}{x} + C \\
        y &= e^{x} - \frac{1}{x} + C
    \end{split}
\end{equation}
\qed


\end{document}
\documentclass[12pt]{article}
\usepackage{homework}
\pagestyle{fancy}

% assignment information
\def\course{Multiple Integrals \& Vector Calculus}
\def\assignmentno{Problem Set 3}
\def\assignmentname{}
\def\name{Xin, Wenkang}
\def\time{\today}

\begin{document}

\begin{titlepage}
    \begin{center}
        \large
        \textbf{\course}

        \vfill

        \Huge
        \textbf{\assignmentno}

        \vspace{1.5cm}

        \large{\assignmentname}

        \vfill

        \large
        \name

        \time
    \end{center}
\end{titlepage}


\problem{1}{}

\begin{equation}
\begin{split}
    &\frac{\partial (x, y, z)}{\partial (r, \theta, \phi)} \\
    =& \sin{\theta} \cos{\phi} (r^{2} \sin^{2}\theta \cos{\phi}) - \sin{\theta} \sin{\phi} (-r^{2} \sin^{2}\theta \sin{\phi}) \\
    +& \cos{\theta} (r^{2} \cos^{2}{\phi} \cos{\theta} \sin{\theta} + r^{2} \sin^{2}{\phi} \cos{\theta} \sin{\theta}) \\
    =& r^{2} \sin{\theta}
\end{split}
\end{equation}
\qed


\problem{2}{}
The volume of a sphere is given by $\{(r, \theta, \phi) \mid r \in [0 ,a], \theta \in [0, \pi/2], \phi \in [0, 2\pi]\}$

\begin{equation}
    M = \int_{V} k r^{2} \sin{\theta} \, \mathrm{d}r \mathrm{d}\theta \mathrm{d}\phi = \frac{2}{3} k \pi a^{3}
\end{equation}

By symmetry, $\bar{x} = \bar{y} = 0$. For $\bar{z}$:

\begin{equation}
    \bar{z} = \frac{\int_{V} z \, \mathrm{d}r \mathrm{d}\theta \mathrm{d}\phi}{M} = \frac{\int_{0}^{\pi/2} \int_{0}^{a} r^{3} \cos{\theta} \sin{\theta} \, \mathrm{d}r \mathrm{d}\theta}{\int_{0}^{\pi/2} \int_{0}^{a} r^{2} \sin{\theta} \, \mathrm{d}r \mathrm{d}\theta} = \frac{3}{8} a
\end{equation}

so that $(\bar{x}, \bar{y}, \bar{z}) = (0, 0, \frac{3}{8} a)$.

For the principal moments of inertia:

\begin{equation}
\begin{split}
    I_{xx} &= \int_{V} k r^{4} (\sin^{2}{\theta} \cos^{2}{\phi} + \cos^{2}{\theta}) \sin{\theta} \, \mathrm{d}r \mathrm{d}\theta \mathrm{d}\phi = \frac{4}{15} \pi k a^{5} \\
    I_{yy} &= I_{xx} = \frac{4}{15} \pi k a^{5} \\
    I_{zz} &= \int_{V} k r^{4} \sin^{3}{\theta} \, \mathrm{d}r \mathrm{d}\theta \mathrm{d}\phi = \frac{4}{15} \pi k a^{5}
\end{split}
\end{equation}

For the product of inertia:

\begin{equation}
\begin{split}
    I_{xy} &= -\int_{V} k r^{4} \sin^{3}{\theta} \cos{\phi} \sin{\phi} \, \mathrm{d}r \mathrm{d}\theta \mathrm{d}\phi = 0 \\
    I_{yz} &= -\int_{V} k r^{4} \cos{\theta} \sin^{2}{\theta} \sin{\phi} \, \mathrm{d}r \mathrm{d}\theta \mathrm{d}\phi = 0 \\
    I_{xz} &= -\int_{V} k r^{4} \cos{\theta} \sin^{2}{\theta} \cos{\phi} \, \mathrm{d}r \mathrm{d}\theta \mathrm{d}\phi = 0
\end{split}
\end{equation}
\qed


\problem{3}{}

\begin{equation}
    \frac{\mathrm{d}V}{\mathrm{d}t} = \int_{S} \mathbf{F} \cdot \, \mathrm{d}\mathbf{S}
\end{equation}

where $\mathbf{F} = (0.4/\sqrt{3}) (-1, -1, 1)$.

We can divide the surface into three parts: a triangle bound by $y = x$, $y = 0$ and $x = 1$; a triangle formed by $(0, 0,0)$, $(0, 1, 1)$ and $(1, 1, 0)$; a triangle formed by bound by $z = 1 - x$, $z = 1$ and $x = 1$. On the first triangle, the surface integral evaluates to $0.2/\sqrt{3}$. On the second triangle, which is formed by the surface $z = -x + y$

\begin{equation}
    \int_{A} \frac{0.4}{\sqrt{3}} \, \mathrm{d}y \mathrm{d}y = \frac{0.2}{\sqrt{3}}
\end{equation}

On the third triangle, the surface integral evaluates to $0.2/\sqrt{3}$. Therefore, the total volume of air flow per unit time is $0.2 \sqrt{3} \unit{m^{3}s^{-1}}$.
\qed


\problem{4}{}

\subproblem{a}

\begin{equation}
    \int_{S} \mathbf{r} \cdot \mathbf{n} \, \mathrm{d}S = \int_{x=1} x \, \mathrm{d}S + \int_{y=1} y \, \mathrm{d}S + \int_{z=1} z \, \mathrm{d}S = 3
\end{equation}

\subproblem{b}

\begin{equation}
    \int_{S} \mathbf{r} \cdot \mathbf{n} \, \mathrm{d}S = \int_{S} a \, \mathrm{d}S = 4\pi a^{3}
\end{equation}
\qed


\problem{5}{}

\subproblem{a}

\begin{equation}
    \int_{S} \mathbf{A} \cdot \mathbf{n} \, \mathrm{d}S = \mistake{\int_{V} -1 \, \mathrm{d}V = -36}
\end{equation}

\begin{correction}
    Integrating on the surface:

    \begin{equation}
        \int_{S} \mathbf{A} \cdot \mathbf{n} \, \mathrm{d}S = \int_{2x+y=6} (2x + 2y) \, \mathrm{d}x \mathrm{d}z = 108
    \end{equation}
\end{correction}

\subproblem{b}

\begin{equation}
    \int_{S} \mathbf{A} \cdot \mathbf{n} \, \mathrm{d}S = -\int_{A_{1}} y^{2} \, \mathrm{d}A - \int_{A_{2}} -2x \, \mathrm{d}A + \int_{A_{3}} \frac{2x^{2} - 4xy - 2x + 12y}{2} \, \mathrm{d}A = 18
\end{equation}

where the regions are:

\begin{equation}
\begin{split}
    A_{1} &= \{(x, y, z) \in \mathbb{R}^{3} \mid x = 0, 0 \leq y \leq 6, 0 \leq z \leq 3 - y/2 \} \\
    A_{2} &= \{(x, y, z) \in \mathbb{R}^{3} \mid y = 0, 0 \leq x \leq 3, 0 \leq z \leq 1 - x \} \\
    A_{3} &= \{(x, y, z) \in \mathbb{R}^{3} \mid z = 0, 0 \leq x \leq 3, 0 \leq y \leq 6 - 2x \}
\end{split}
\end{equation}

\subproblem{c}

\begin{equation}
    \int_{S} \mathbf{A} \cdot \mathbf{n} \, \mathrm{d}S = \int_{V} 1 \, \mathrm{d}V = 18 \pi
\end{equation}
\qed


\problem{6}{}

\begin{equation}
    \int_{S} \mathbf{A} \cdot \mathrm{d}\mathbf{S} = -\int_{y=0} x^{2}  \, \mathrm{d}S + \int_{z=1} yz \, \mathrm{d}S + \int_{x+y=1} (xy^{2} + x^{2}) \, \mathrm{d}S = \frac{1}{4} \ne 0
\end{equation}

\begin{equation}
    \int_{V} \nabla \cdot \mathbf{A} \, \mathrm{d}V = \int_{V} (y^{2} + y) \, \mathrm{d}x \mathrm{d}y \mathrm{d}z = \frac{1}{4}
\end{equation}

This is a consequence of the divergence theorem.
\qed


\problem{7}{}

\begin{equation}
\begin{split}
    \nabla \cdot \mathbf{A} &= 3yz^{2} + 6xy^{2} - x^{2}y \\
    \mathbf{A} \cdot \nabla \phi &= 3xyz^{2} \times 6x + 2xy^{3} \times (-z) - x^{2}yz \times (-y) = 18x^{2}yz^{2} - 2xy^{3}z + 3x^{2}y^{2}z \\
    \nabla \cdot (\phi \mathbf{A}) &= \phi \nabla \cdot \mathbf{A} + \mathbf{A} \cdot \nabla \phi = (3x^{2} - yz)(3yz^{2} + 6xy^{2} - x^{2}y) + 18x^{2}yz^{2} - 2xy^{3}z + 3x^{2}y^{2}z \\
    \nabla \cdot (\nabla \phi) &= 6
\end{split}
\end{equation}
\qed


\problem{8}{}
In spherical coordinates, the field is given by:

\begin{equation}
    \mathbf{B} = \frac{\mu_{0}I}{2\pi} \frac{1}{r} \hat{\phi}
\end{equation}

so that the divergence is:

\begin{equation}
    \nabla \cdot \mathbf{B} = \frac{\mu_{0}I}{2\pi} \nabla \cdot (r^{-1} \hat{\phi}) = 0
\end{equation}
\qed


\end{document}
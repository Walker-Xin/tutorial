\documentclass[12pt]{article}
\usepackage{homework}
\pagestyle{fancy}

% assignment information
\def\course{Electromagnetism}
\def\assignmentno{Problem Set 2}
\def\assignmentname{Method of Image Charges \& Electric Fields derived from Gauss' Law}
\def\name{Xin, Wenkang}
\def\time{\today}

\begin{document}

\begin{titlepage}
    \begin{center}
        \large
        \textbf{\course}

        \vfill

        \Huge
        \textbf{\assignmentno}

        \vspace{1.5cm}

        \large{\assignmentname}

        \vfill

        \large
        \name

        \time
    \end{center}
\end{titlepage}


%==========
\pagebreak
\section*{Method of Image Charges}
%==========


\problem{0}{Background}
Near a conductor, the parallel component of the electric field is continuous and influenced by external field. The perpendicular component of the electric field is discontinuous and influenced by the surface charge deposited on the conductor.
\qed


\problem{1}{Charge monopole near a flat metal surface}

\subproblem{a}{}
Using the method of image charges, the radial component of the field cancels out so that the electric field is only in the polar direction:

\begin{equation}
    \mathbf{E} = -2 \frac{1}{4\pi \epsilon_{0}} \frac{q}{d^{2} + r^{2}} \frac{d}{\sqrt{d^{2} + r^{2}}} \hat{z} = - \frac{qd}{2\pi \epsilon_{0} (d^{2} + r^{2})^{3/2}} \hat{z}
\end{equation}

\subproblem{c}{}
We have the surface charge density $\sigma = E_{+} - E_{-}$, so that the total charge is:

\begin{equation}
    Q = \int_{0}^{\infty} \sigma 2\pi r \, \mathrm{d}r = -\int_{0}^{\infty} \frac{qd}{2\pi \epsilon_{0} (d^{2} + r^{2})^{3/2}} 2\pi r \, \mathrm{d}r = -q
\end{equation}

\subproblem{d}{}
The force exerted on the charge is:

\begin{equation}
    \mathbf{F} = -\frac{1}{4\pi \epsilon_{0}} \frac{q^{2}}{4d^{2}} \hat{z}
\end{equation}

so that the work down is given by:

\begin{equation}
    W = \int_{d}^{\infty} -\frac{1}{4\pi \epsilon_{0}} \frac{q^{2}}{4s^{2}} \, \mathrm{d}s = \frac{1}{4\pi \epsilon_{0}} \frac{q^{2}}{4d}
\end{equation}

and the minimum escape kinetic energy is:

\begin{equation}
    T = W = \frac{1}{4\pi \epsilon_{0}} \frac{e^{2}}{4d} = \qty{3.6}{eV}
\end{equation}
\qed


\problem{2}{Two charges near a flat metal surface}
Using two image charges, the force on the positive charge is:

\begin{equation}
\begin{split}
    F_{+} &= \frac{1}{4\pi \epsilon_{0}} \left( \frac{Q^{2}}{a^{2}} - \frac{Q^{2}}{a^{2} + 4b^{2}} \frac{a}{\sqrt{a^{2} + 4b^{2}}} \right) \hat{x} - \frac{1}{4\pi \epsilon_{0}} \left( \frac{Q^{2}}{4b^{2}} - \frac{Q^{2}}{a^{2} + 4b^{2}} \frac{2b}{\sqrt{a^{2} + 4b^{2}}} \right) \hat{z} \\
    &= \frac{Q^{2}}{4\pi \epsilon_{0}} \left[ \frac{1}{a^{2}} - \frac{a}{(a^{2} + 4b^{2})^{3/2}} \right] \hat{x} - \frac{Q^{2}}{4\pi \epsilon_{0}} \left[ \frac{1}{4b^{2}} - \frac{2b}{(a^{2} + 4b^{2})^{3/2}} \right] \hat{z}
\end{split}
\end{equation}

By symmetry, the force on the negative charge is:

\begin{equation}
    F_{-} = -\frac{Q^{2}}{4\pi \epsilon_{0}} \left[ \frac{1}{a^{2}} - \frac{a}{(a^{2} + 4b^{2})^{3/2}} \right] \hat{x} + \frac{Q^{2}}{4\pi \epsilon_{0}} \left[ \frac{1}{4b^{2}} - \frac{2b}{(a^{2} + 4b^{2})^{3/2}} \right] \hat{z}
\end{equation}
\qed


\problem{3}{Charge monopole near two orthogonal metal surfaces}
Let us orient the coordinate axis such that $Q$ is situated at $(a, 0, a)$ Consider two negative charges at $(a, 0, -a)$ and $(-a, 0, a)$, and a positive charge at $(-a, 0, -a)$.

\subproblem{a}{}
Along the join, which is the y-axis, the electric field produced by the charges are diagonally pair-wise equal and opposite, so that the net electric field is zero.

\subproblem{b}{}
Just above the point $(a, 0, 0)$, we have the approximation $z \ll a$. The electric potential is given by:

\begin{equation}
    \phi(z) = \frac{Q}{4\pi \epsilon_{0}} \left[ \frac{1}{\sqrt{4a^{2} + (a+z)^{2}}} - \frac{1}{\sqrt{4a^{2} + (a-z)^{2}}} + \frac{1}{a-z} - \frac{1}{a+z} \right] \approx \frac{Q}{4\pi \epsilon_{0}} \frac{z}{a^{2}} \left( 2 - \frac{4}{5\sqrt{5}} \right)
\end{equation}

so that the field just above the point $(a, 0, 0)$ is:

\begin{equation}
    \mathbf{E} = -\frac{\partial \phi}{\partial z} = -\frac{Q}{2\pi \epsilon_{0}} \frac{1}{a^{2}} \left( 1 - \frac{2}{5\sqrt{5}} \right) \hat{z}
\end{equation}

For the other plate, by symmetry, we have:

\begin{equation}
    \mathbf{E} = -\frac{Q}{2\pi \epsilon_{0}} \left( 1 - \frac{2}{5\sqrt{5}} \right) \hat{x}
\end{equation}

\subproblem{d}{}
The surface charge density is given by:

\begin{equation}
    \sigma = -\frac{\partial \phi}{\partial z}\vert_{z=0} \epsilon_{0} = -\frac{Q}{2\pi a^{2}} \left( 1 - \frac{2}{5\sqrt{5}} \right)
\end{equation}
\qed


\problem{4}{Uniformly charged rod near a metal surface}
Consider a charged rod with linear charge density $\lambda$ and an image charged rod with linear charge density $-\lambda$. By symmetry, the electric field produced is given by:

\begin{equation}
    \mathbf{E}(x) = -2 \frac{\lambda}{2\pi \epsilon_{0}} \frac{1}{\sqrt{x^{2} + a^{2}}} \frac{d}{\sqrt{x^{2} + a^{2}}} \hat{z} = -\frac{\lambda}{\pi \epsilon_{0}} \frac{d}{\sqrt{x^{2} + a^{2}}} \hat{z}
\end{equation}
\qed


%==========
\pagebreak
\section*{Electric Fields derived from Gauss' Law}
%==========


\problem{0}{Background}
Gauss' law states that the electric flux through a closed surface is proportional to the net charge enclosed by the surface. This helps us to derive the electric field from a charge distribution with a high degree of symmetry, as the surface integral can be simplified.
\qed


\problem{1}{Uniformly charged sphere}

\subproblem{a}{}
For $r \le a$, the field and potential are trivial. For $0 < r < a$, since the charge is uniformly distributed, by Gauss' law, the electric field satisfies:

\begin{equation}
    E 4\pi r^{2} = \frac{q}{\epsilon_{0}} \frac{r^{3}}{a^{3}}
\end{equation}

so that $\mathbf{E} = qr/(4\pi \epsilon_{0} a^{3}) \hat{r}$ for $0 < r < a$. The potential is given by the integral:

\begin{equation}
    V(r) = -\int_{\infty}^{r} E(r) \, \mathrm{d}r = -\int_{\infty}^{a} \frac{q}{4\pi \epsilon_{0} r^{2}} \, \mathrm{d}r - \int_{a}^{r} \frac{qr}{4\pi \epsilon_{0} a^{3}} \, \mathrm{d}r = \frac{q}{4\pi \epsilon_{0} a^{3}} \left( \frac{3a^{2}}{2} - \frac{r^{2}}{2} \right)
\end{equation}

\subproblem{b}{}
For a surface charge distribution over a sphere, the electric field is given by Gauss' law

\begin{equation}
    \mathbf{E}(r) =
    \begin{cases}
        \mathbf{0} & r \le a \\
        \frac{q}{4\pi \epsilon_{0}} \frac{1}{r^{2}} \hat{r} & r > a
    \end{cases}
\end{equation}

The potential is given by:

\begin{equation}
    V(r) =
    \begin{cases}
        \frac{q}{4\pi \epsilon_{0}} \frac{1}{a} & r \le a \\
        \frac{q}{4\pi \epsilon_{0}} \frac{1}{r} & r > a
    \end{cases}
\end{equation}


\problem{2}{Coulomb energy of the nucleus}
The charge density can be written as $\rho = (3Ze)/(4\pi a^{3})$ which is a constant. The assembly energy is given by the integral:

\begin{equation}
    W = \frac{1}{2} \int_{V} \phi \, \mathrm{d}q
\end{equation}

where $V$ is the sphere occupied by the nucleus and $\phi$ is the electric potential satisfying the expression given in the previous problem.

Carrying out the integration:

\begin{equation}
\begin{split}
    W = \frac{1}{2} \int_{0}^{2\pi} \int_{0}^{\pi} \int_{0}^{a} \frac{Ze}{4\pi a^{3}} \left( \frac{3a^{2}}{2} - \frac{r^{2}}{2} \right) \rho r^{2} \sin{\theta} \, \mathrm{d}r \mathrm{d}\theta \mathrm{d}\phi = \frac{3 (Ze)^{2}}{20\pi \epsilon a}
\end{split}
\end{equation}

If the charge is distributed over the surface instead, then $\phi$ is a constant so that the energy is given by:

\begin{equation}
    W = \frac{1}{2} \phi Q = \frac{(Ze)^{2}}{8\pi \epsilon a}
\end{equation}
\qed


\problem{3}{Electron in a hydrogen atom}

\subproblem{a}{}
$V(r)$ has a horizontal asymptote as $r \to \infty$ and has a minimum at $r = 0$.

\begin{center}
    % This file was created with tikzplotlib v0.10.1.
\begin{tikzpicture}

\definecolor{darkgray176}{RGB}{176,176,176}
\definecolor{steelblue31119180}{RGB}{31,119,180}

\begin{axis}[
tick align=outside,
tick pos=left,
unbounded coords=jump,
x grid style={darkgray176},
xlabel={r},
xmin=-0.48948948948949, xmax=10.4994994994995,
xtick style={color=black},
y grid style={darkgray176},
ylabel={V},
ymin=-1.2, ymax=0.5,
ytick style={color=black}
]
\addplot [semithick, steelblue31119180]
table {%
0 nan
0.01001001001001 -0.999933864472175
0.02002002002002 -0.999738084875358
0.03003003003003 -0.999416531302046
0.04004004004004 -0.998972981649783
0.05005005005005 -0.998411123653367
0.0600600600600601 -0.997734556873914
0.0700700700700701 -0.996946794645589
0.0800800800800801 -0.996051265980973
0.0900900900900901 -0.995051317435858
0.1001001001001 -0.99395021493437
0.11011011011011 -0.992751145555217
0.12012012012012 -0.991457219279869
0.13013013013013 -0.990071470703488
0.14014014014014 -0.988596860709364
0.15015015015015 -0.987036278107626
0.16016016016016 -0.985392541238973
0.17017017017017 -0.983668399544158
0.18018018018018 -0.981866535099932
0.19019019019019 -0.979989564122164
0.2002002002002 -0.978040038436802
0.21021021021021 -0.976020446919363
0.22022022022022 -0.973933216903613
0.23023023023023 -0.971780715560056
0.24024024024024 -0.969565251244899
0.25025025025025 -0.967289074820084
0.26026026026026 -0.964954380945002
0.27027027027027 -0.962563309340485
0.28028028028028 -0.960117946025652
0.29029029029029 -0.957620324528176
0.3003003003003 -0.95507242706854
0.31031031031031 -0.952476185718808
0.32032032032032 -0.94983348353647
0.33033033033033 -0.947146155673862
0.34034034034034 -0.944415990463685
0.35035035035035 -0.941644730481124
0.36036036036036 -0.938834073583051
0.37037037037037 -0.935985673924812
0.38038038038038 -0.933101142955036
0.39039039039039 -0.930182050388971
0.4004004004004 -0.927229925160755
0.41041041041041 -0.9242462563551
0.42042042042042 -0.921232494118803
0.43043043043043 -0.918190050552509
0.44044044044044 -0.915120300583158
0.45045045045045 -0.912024582817495
0.46046046046046 -0.908904200377076
0.47047047047047 -0.90576042171513
0.48048048048048 -0.902594481415675
0.49049049049049 -0.89940758097526
0.500500500500501 -0.896200889567701
0.510510510510511 -0.892975544792167
0.520520520520521 -0.889732653404964
0.530530530530531 -0.88647329203537
0.540540540540541 -0.883198507885858
0.550550550550551 -0.879909319417021
0.560560560560561 -0.876606717017554
0.570570570570571 -0.873291663659564
0.580580580580581 -0.869965095539571
0.590590590590591 -0.866627922705459
0.600600600600601 -0.863281029669695
0.610610610610611 -0.859925276009111
0.620620620620621 -0.85656149695152
0.630630630630631 -0.853190503949455
0.640640640640641 -0.8498130852413
0.650650650650651 -0.846430006400086
0.660660660660661 -0.843042010870203
0.670670670670671 -0.8396498204923
0.680680680680681 -0.836254136016607
0.690690690690691 -0.832855637604947
0.700700700700701 -0.829454985321646
0.710710710710711 -0.826052819613619
0.720720720720721 -0.822649761779825
0.730730730730731 -0.819246414430337
0.740740740740741 -0.815843361935246
0.750750750750751 -0.812441170863615
0.760760760760761 -0.809040390412686
0.770770770770771 -0.805641552827567
0.780780780780781 -0.802245173811588
0.790790790790791 -0.798851752927519
0.800800800800801 -0.795461773989873
0.810810810810811 -0.79207570544845
0.820820820820821 -0.788694000763335
0.830830830830831 -0.785317098771523
0.840840840840841 -0.781945424045349
0.850850850850851 -0.7785793872429
0.860860860860861 -0.775219385450588
0.870870870870871 -0.771865802518034
0.880880880880881 -0.768519009385454
0.890890890890891 -0.765179364403681
0.900900900900901 -0.761847213647003
0.910910910910911 -0.758522891218957
0.920920920920921 -0.755206719551234
0.930930930930931 -0.751899009695855
0.940940940940941 -0.748600061610738
0.950950950950951 -0.745310164438826
0.960960960960961 -0.742029596780895
0.970970970970971 -0.738758626962184
0.980980980980981 -0.735497513292989
0.990990990990991 -0.732246504323331
1.001001001001 -0.729005839091854
1.01101101101101 -0.725775747369047
1.02102102102102 -0.722556449894941
1.03103103103103 -0.719348158611375
1.04104104104104 -0.716151076888976
1.05105105105105 -0.712965399748935
1.06106106106106 -0.709791314079727
1.07107107107107 -0.706628998848853
1.08108108108108 -0.703478625309732
1.09109109109109 -0.700340357203843
1.1011011011011 -0.697214350958221
1.11111111111111 -0.694100755878398
1.12112112112112 -0.690999714336907
1.13113113113113 -0.687911361957429
1.14114114114114 -0.684835827794686
1.15115115115115 -0.68177323451017
1.16116116116116 -0.678723698543804
1.17117117117117 -0.675687330281609
1.18118118118118 -0.672664234219488
1.19119119119119 -0.669654509123187
1.2012012012012 -0.666658248184538
1.21121121121121 -0.663675539174048
1.22122122122122 -0.660706464589922
1.23123123123123 -0.657751101803605
1.24124124124124 -0.654809523201906
1.25125125125125 -0.651881796325784
1.26126126126126 -0.648967984005872
1.27127127127127 -0.646068144494817
1.28128128128128 -0.643182331596482
1.29129129129129 -0.64031059479211
1.3013013013013 -0.637452979363496
1.31131131131131 -0.634609526513238
1.32132132132132 -0.631780273482137
1.33133133133133 -0.628965253663801
1.34134134134134 -0.626164496716517
1.35135135135135 -0.623378028672461
1.36136136136136 -0.620605872044287
1.37137137137137 -0.617848045929166
1.38138138138138 -0.615104566110335
1.39139139139139 -0.612375445156195
1.4014014014014 -0.609660692517024
1.41141141141141 -0.606960314619367
1.42142142142142 -0.604274314958124
1.43143143143143 -0.601602694186428
1.44144144144144 -0.598945450203325
1.45145145145145 -0.596302578239334
1.46146146146146 -0.59367407093991
1.47147147147147 -0.591059918446878
1.48148148148148 -0.588460108477868
1.49149149149149 -0.585874626403809
1.5015015015015 -0.583303455324509
1.51151151151151 -0.580746576142376
1.52152152152152 -0.578203967634327
1.53153153153153 -0.575675606521908
1.54154154154154 -0.573161467539677
1.55155155155155 -0.57066152350189
1.56156156156156 -0.568175745367525
1.57157157157157 -0.565704102303679
1.58158158158158 -0.563246561747378
1.59159159159159 -0.560803089465842
1.6016016016016 -0.558373649615221
1.61161161161161 -0.555958204797864
1.62162162162162 -0.553556716118127
1.63163163163163 -0.551169143236771
1.64164164164164 -0.54879544442398
1.65165165165165 -0.546435576611019
1.66166166166166 -0.544089495440578
1.67167167167167 -0.541757155315824
1.68168168168168 -0.539438509448185
1.69169169169169 -0.537133509903915
1.7017017017017 -0.534842107649439
1.71171171171171 -0.532564252595535
1.72172172172172 -0.530299893640355
1.73173173173173 -0.528048978711334
1.74174174174174 -0.525811454805991
1.75175175175175 -0.523587268031663
1.76176176176176 -0.521376363644197
1.77177177177177 -0.519178686085605
1.78178178178178 -0.516994179020734
1.79179179179179 -0.514822785372952
1.8018018018018 -0.512664447358881
1.81181181181181 -0.510519106522197
1.82182182182182 -0.508386703766529
1.83183183183183 -0.506267179387456
1.84184184184184 -0.504160473103645
1.85185185185185 -0.502066524087143
1.86186186186186 -0.49998527099283
1.87187187187187 -0.497916651987078
1.88188188188188 -0.495860604775608
1.89189189189189 -0.493817066630585
1.9019019019019 -0.491785974416955
1.91191191191191 -0.48976726461805
1.92192192192192 -0.487760873360471
1.93193193193193 -0.485766736438278
1.94194194194194 -0.483784789336486
1.95195195195195 -0.481814967253902
1.96196196196196 -0.479857205125306
1.97197197197197 -0.477911437642992
1.98198198198198 -0.475977599277697
1.99199199199199 -0.474055624298913
2.002002002002 -0.472145446794613
2.01201201201201 -0.470247000690402
2.02202202202202 -0.468360219768096
2.03203203203203 -0.466485037683754
2.04204204204204 -0.464621387985177
2.05205205205205 -0.462769204128881
2.06206206206206 -0.460928419496551
2.07207207207207 -0.459098967411008
2.08208208208208 -0.457280781151679
2.09209209209209 -0.455473793969599
2.1021021021021 -0.453677939101946
2.11211211211211 -0.451893149786123
2.12212212212212 -0.450119359273403
2.13213213213213 -0.448356500842137
2.14214214214214 -0.44660450781055
2.15215215215215 -0.444863313549123
2.16216216216216 -0.443132851492576
2.17217217217217 -0.44141305515146
2.18218218218218 -0.439703858123377
2.19219219219219 -0.438005194103821
2.2022022022022 -0.43631699689666
2.21221221221221 -0.434639200424269
2.22222222222222 -0.432971738737319
2.23223223223223 -0.431314546024228
2.24224224224224 -0.429667556620288
2.25225225225225 -0.428030705016478
2.26226226226226 -0.426403925867957
2.27227227227227 -0.424787154002267
2.28228228228228 -0.423180324427233
2.29229229229229 -0.421583372338578
2.3023023023023 -0.419996233127263
2.31231231231231 -0.418418842386546
2.32232232232232 -0.416851135918785
2.33233233233233 -0.415293049741975
2.34234234234234 -0.413744520096038
2.35235235235235 -0.412205483448863
2.36236236236236 -0.410675876502116
2.37237237237237 -0.409155636196802
2.38238238238238 -0.407644699718618
2.39239239239239 -0.406143004503072
2.4024024024024 -0.404650488240394
2.41241241241241 -0.403167088880233
2.42242242242242 -0.401692744636156
2.43243243243243 -0.400227393989946
2.44244244244244 -0.398770975695704
2.45245245245245 -0.397323428783767
2.46246246246246 -0.395884692564446
2.47247247247247 -0.394454706631577
2.48248248248248 -0.393033410865908
2.49249249249249 -0.391620745438314
2.5025025025025 -0.390216650812846
2.51251251251251 -0.388821067749628
2.52252252252252 -0.38743393730759
2.53253253253253 -0.386055200847059
2.54254254254254 -0.384684800032194
2.55255255255255 -0.38332267683329
2.56256256256256 -0.381968773528936
2.57257257257257 -0.380623032708039
2.58258258258258 -0.379285397271719
2.59259259259259 -0.377955810435081
2.6026026026026 -0.376634215728851
2.61261261261261 -0.375320557000912
2.62262262262262 -0.374014778417701
2.63263263263263 -0.372716824465514
2.64264264264264 -0.371426639951684
2.65265265265265 -0.370144170005666
2.66266266266266 -0.368869360080008
2.67267267267267 -0.367602155951229
2.68268268268268 -0.366342503720598
2.69269269269269 -0.365090349814815
2.7027027027027 -0.363845640986603
2.71271271271271 -0.362608324315214
2.72272272272272 -0.361378347206842
2.73273273273273 -0.36015565739496
2.74274274274274 -0.358940202940568
2.75275275275275 -0.357731932232373
2.76276276276276 -0.356530793986881
2.77277277277277 -0.355336737248424
2.78278278278278 -0.354149711389113
2.79279279279279 -0.352969666108722
2.8028028028028 -0.351796551434503
2.81281281281281 -0.350630317720943
2.82282282282282 -0.349470915649449
2.83283283283283 -0.348318296227979
2.84284284284284 -0.347172410790614
2.85285285285285 -0.346033210997072
2.86286286286286 -0.344900648832164
2.87287287287287 -0.343774676605207
2.88288288288288 -0.342655246949375
2.89289289289289 -0.341542312821008
2.9029029029029 -0.340435827498867
2.91291291291291 -0.339335744583355
2.92292292292292 -0.338242017995675
2.93293293293293 -0.337154601976968
2.94294294294294 -0.336073451087389
2.95295295295295 -0.334998520205155
2.96296296296296 -0.333929764525556
2.97297297297297 -0.332867139559923
2.98298298298298 -0.331810601134561
2.99299299299299 -0.330760105389655
3.003003003003 -0.329715608778138
3.01301301301301 -0.328677068064525
3.02302302302302 -0.327644440323724
3.03303303303303 -0.326617682939816
3.04304304304304 -0.325596753604802
3.05305305305305 -0.324581610317329
3.06306306306306 -0.32357221138139
3.07307307307307 -0.322568515404999
3.08308308308308 -0.321570481298842
3.09309309309309 -0.320578068274906
3.1031031031031 -0.319591235845087
3.11311311311311 -0.318609943819782
3.12312312312312 -0.317634152306453
3.13313313313313 -0.316663821708178
3.14314314314314 -0.31569891272219
3.15315315315315 -0.314739386338386
3.16316316316316 -0.313785203837833
3.17317317317317 -0.312836326791255
3.18318318318318 -0.3118927170575
3.19319319319319 -0.310954336782006
3.2032032032032 -0.31002114839524
3.21321321321321 -0.309093114611142
3.22322322322322 -0.30817019842554
3.23323323323323 -0.30725236311457
3.24324324324324 -0.306339572233078
3.25325325325325 -0.305431789613012
3.26326326326326 -0.304528979361815
3.27327327327327 -0.303631105860799
3.28328328328328 -0.302738133763521
3.29329329329329 -0.301850027994145
3.3033033033033 -0.300966753745804
3.31331331331331 -0.300088276478954
3.32332332332332 -0.299214561919723
3.33333333333333 -0.298345576058258
3.34334334334334 -0.297481285147064
3.35335335335335 -0.296621655699344
3.36336336336336 -0.295766654487333
3.37337337337337 -0.294916248540635
3.38338338338338 -0.294070405144547
3.39339339339339 -0.293229091838397
3.4034034034034 -0.292392276413866
3.41341341341341 -0.29155992691332
3.42342342342342 -0.290732011628136
3.43343343343343 -0.289908499097031
3.44344344344344 -0.289089358104388
3.45345345345345 -0.288274557678589
3.46346346346346 -0.287464067090344
3.47347347347347 -0.286657855851022
3.48348348348348 -0.285855893710984
3.49349349349349 -0.285058150657924
3.5035035035035 -0.284264596915202
3.51351351351351 -0.283475202940189
3.52352352352352 -0.282689939422609
3.53353353353353 -0.281908777282889
3.54354354354354 -0.281131687670509
3.55355355355355 -0.280358641962358
3.56356356356356 -0.279589611761092
3.57357357357357 -0.278824568893499
3.58358358358358 -0.278063485408866
3.59359359359359 -0.277306333577354
3.6036036036036 -0.276553085888374
3.61361361361361 -0.275803715048969
3.62362362362362 -0.275058193982207
3.63363363363363 -0.274316495825571
3.64364364364364 -0.27357859392936
3.65365365365365 -0.272844461855095
3.66366366366366 -0.272114073373933
3.67367367367367 -0.271387402465079
3.68368368368368 -0.270664423314218
3.69369369369369 -0.269945110311941
3.7037037037037 -0.269229438052186
3.71371371371371 -0.268517381330679
3.72372372372372 -0.267808915143388
3.73373373373373 -0.267104014684983
3.74374374374374 -0.266402655347299
3.75375375375375 -0.26570481271781
3.76376376376376 -0.26501046257811
3.77377377377377 -0.264319580902401
3.78378378378378 -0.263632143855991
3.79379379379379 -0.262948127793791
3.8038038038038 -0.262267509258833
3.81381381381381 -0.261590264980787
3.82382382382382 -0.260916371874486
3.83383383383383 -0.260245807038462
3.84384384384384 -0.259578547753493
3.85385385385385 -0.258914571481148
3.86386386386386 -0.258253855862352
3.87387387387387 -0.257596378715948
3.88388388388388 -0.256942118037282
3.89389389389389 -0.256291051996778
3.9039039039039 -0.255643158938538
3.91391391391391 -0.254998417378938
3.92392392392392 -0.254356806005245
3.93393393393393 -0.253718303674227
3.94394394394394 -0.253082889410787
3.95395395395395 -0.252450542406596
3.96396396396396 -0.251821242018738
3.97397397397397 -0.251194967768362
3.98398398398398 -0.250571699339345
3.99399399399399 -0.249951416576961
4.004004004004 -0.249334099486563
4.01401401401401 -0.248719728232268
4.02402402402402 -0.248108283135654
4.03403403403403 -0.247499744674468
4.04404404404404 -0.246894093481337
4.05405405405405 -0.246291310342493
4.06406406406406 -0.245691376196505
4.07407407407407 -0.245094272133017
4.08408408408408 -0.2444999793915
4.09409409409409 -0.243908479360009
4.1041041041041 -0.24331975357395
4.11411411411411 -0.242733783714856
4.12412412412412 -0.242150551609168
4.13413413413413 -0.241570039227034
4.14414414414414 -0.240992228681107
4.15415415415415 -0.240417102225355
4.16416416416416 -0.239844642253882
4.17417417417417 -0.239274831299756
4.18418418418418 -0.238707652033842
4.19419419419419 -0.238143087263652
4.2042042042042 -0.237581119932196
4.21421421421421 -0.237021733116846
4.22422422422422 -0.236464910028204
4.23423423423423 -0.235910634008983
4.24424424424424 -0.235358888532898
4.25425425425425 -0.234809657203557
4.26426426426426 -0.234262923753369
4.27427427427427 -0.233718672042458
4.28428428428428 -0.233176886057583
4.29429429429429 -0.232637549911069
4.3043043043043 -0.232100647839745
4.31431431431431 -0.23156616420389
4.32432432432432 -0.231034083486192
4.33433433433433 -0.230504390290706
4.34434434434434 -0.229977069341827
4.35435435435435 -0.229452105483271
4.36436436436436 -0.228929483677064
4.37437437437437 -0.228409189002532
4.38438438438438 -0.227891206655313
4.39439439439439 -0.22737552194636
4.4044044044044 -0.226862120300969
4.41441441441441 -0.226350987257801
4.42442442442442 -0.225842108467921
4.43443443443443 -0.225335469693838
4.44444444444444 -0.224831056808563
4.45445445445445 -0.224328855794658
4.46446446446446 -0.223828852743312
4.47447447447447 -0.223331033853412
4.48448448448448 -0.222835385430621
4.49449449449449 -0.222341893886477
4.5045045045045 -0.22185054573748
4.51451451451451 -0.221361327604203
4.52452452452452 -0.220874226210405
4.53453453453453 -0.220389228382144
4.54454454454454 -0.219906321046911
4.55455455455455 -0.21942549123276
4.56456456456456 -0.218946726067452
4.57457457457457 -0.218470012777602
4.58458458458458 -0.217995338687836
4.59459459459459 -0.217522691219951
4.6046046046046 -0.217052057892093
4.61461461461461 -0.216583426317924
4.62462462462462 -0.216116784205814
4.63463463463463 -0.215652119358029
4.64464464464464 -0.215189419669929
4.65465465465465 -0.214728673129172
4.66466466466466 -0.214269867814931
4.67467467467467 -0.213812991897104
4.68468468468468 -0.213358033635546
4.69469469469469 -0.2129049813793
4.7047047047047 -0.21245382356583
4.71471471471471 -0.212004548720275
4.72472472472472 -0.211557145454691
4.73473473473473 -0.211111602467317
4.74474474474474 -0.210667908541835
4.75475475475475 -0.210226052546643
4.76476476476476 -0.20978602343413
4.77477477477477 -0.209347810239963
4.78478478478478 -0.208911402082371
4.79479479479479 -0.208476788161449
4.8048048048048 -0.208043957758452
4.81481481481481 -0.207612900235109
4.82482482482482 -0.207183605032934
4.83483483483483 -0.206756061672549
4.84484484484484 -0.206330259753006
4.85485485485485 -0.205906188951126
4.86486486486486 -0.205483839020828
4.87487487487487 -0.205063199792482
4.88488488488488 -0.204644261172251
4.89489489489489 -0.204227013141452
4.9049049049049 -0.203811445755915
4.91491491491491 -0.203397549145348
4.92492492492492 -0.202985313512713
4.93493493493493 -0.2025747291336
4.94494494494494 -0.202165786355613
4.95495495495495 -0.201758475597758
4.96496496496496 -0.201352787349838
4.97497497497497 -0.200948712171852
4.98498498498498 -0.200546240693399
4.99499499499499 -0.200145363613092
5.00500500500501 -0.199746071697968
5.01501501501502 -0.199348355782913
5.02502502502503 -0.198952206770088
5.03503503503504 -0.198557615628356
5.04504504504505 -0.198164573392724
5.05505505505506 -0.197773071163779
5.06506506506507 -0.197383100107136
5.07507507507508 -0.196994651452891
5.08508508508509 -0.196607716495073
5.0950950950951 -0.19622228659111
5.10510510510511 -0.195838353161289
5.11511511511512 -0.195455907688232
5.12512512512513 -0.195074941716367
5.13513513513514 -0.194695446851411
5.14514514514515 -0.19431741475985
5.15515515515516 -0.193940837168436
5.16516516516517 -0.193565705863673
5.17517517517518 -0.193192012691321
5.18518518518519 -0.192819749555893
5.1951951951952 -0.19244890842017
5.20520520520521 -0.192079481304707
5.21521521521522 -0.191711460287349
5.22522522522523 -0.191344837502755
5.23523523523524 -0.190979605141922
5.24524524524525 -0.190615755451712
5.25525525525526 -0.190253280734387
5.26526526526527 -0.189892173347145
5.27527527527528 -0.189532425701665
5.28528528528529 -0.189174030263649
5.2952952952953 -0.188816979552373
5.30530530530531 -0.188461266140244
5.31531531531532 -0.188106882652354
5.32532532532533 -0.187753821766043
5.33533533533534 -0.187402076210466
5.34534534534535 -0.187051638766162
5.35535535535536 -0.18670250226463
5.36536536536537 -0.186354659587902
5.37537537537538 -0.186008103668128
5.38538538538539 -0.185662827487159
5.3953953953954 -0.18531882407614
5.40540540540541 -0.184976086515096
5.41541541541542 -0.184634607932534
5.42542542542543 -0.184294381505039
5.43543543543544 -0.183955400456881
5.44544544544545 -0.18361765805962
5.45545545545546 -0.183281147631715
5.46546546546547 -0.182945862538142
5.47547547547548 -0.182611796190009
5.48548548548549 -0.182278942044178
5.4954954954955 -0.181947293602889
5.50550550550551 -0.181616844413388
5.51551551551552 -0.181287588067558
5.52552552552553 -0.180959518201553
5.53553553553554 -0.18063262849544
5.54554554554555 -0.180306912672832
5.55555555555556 -0.179982364500541
5.56556556556557 -0.179658977788218
5.57557557557558 -0.17933674638801
5.58558558558559 -0.17901566419421
5.5955955955956 -0.178695725142915
5.60560560560561 -0.178376923211685
5.61561561561562 -0.178059252419208
5.62562562562563 -0.177742706824967
5.63563563563564 -0.177427280528904
5.64564564564565 -0.177112967671098
5.65565565565566 -0.176799762431436
5.66566566566567 -0.176487659029292
5.67567567567568 -0.176176651723211
5.68568568568569 -0.175866734810587
5.6956956956957 -0.175557902627354
5.70570570570571 -0.175250149547673
5.71571571571572 -0.174943469983627
5.72572572572573 -0.174637858384912
5.73573573573574 -0.174333309238538
5.74574574574575 -0.174029817068526
5.75575575575576 -0.173727376435615
5.76576576576577 -0.173425981936965
5.77577577577578 -0.173125628205866
5.78578578578579 -0.172826309911449
5.7957957957958 -0.172528021758399
5.80580580580581 -0.172230758486673
5.81581581581582 -0.171934514871216
5.82582582582583 -0.171639285721684
5.83583583583584 -0.171345065882166
5.84584584584585 -0.171051850230911
5.85585585585586 -0.170759633680059
5.86586586586587 -0.170468411175367
5.87587587587588 -0.170178177695945
5.88588588588589 -0.169888928253993
5.8958958958959 -0.169600657894536
5.90590590590591 -0.169313361695169
5.91591591591592 -0.169027034765793
5.92592592592593 -0.168741672248367
5.93593593593594 -0.16845726931665
5.94594594594595 -0.168173821175954
5.95595595595596 -0.167891323062894
5.96596596596597 -0.167609770245141
5.97597597597598 -0.167329158021182
5.98598598598599 -0.167049481720072
5.995995995996 -0.166770736701203
6.00600600600601 -0.166492918354058
6.01601601601602 -0.166216022097979
6.02602602602603 -0.165940043381938
6.03603603603604 -0.165664977684297
6.04604604604605 -0.165390820512587
6.05605605605606 -0.165117567403276
6.06606606606607 -0.164845213921547
6.07607607607608 -0.164573755661071
6.08608608608609 -0.16430318824379
6.0960960960961 -0.164033507319694
6.10610610610611 -0.163764708566608
6.11611611611612 -0.163496787689971
6.12612612612613 -0.163229740422629
6.13613613613614 -0.162963562524617
6.14614614614615 -0.162698249782953
6.15615615615616 -0.162433798011431
6.16616616616617 -0.162170203050412
6.17617617617618 -0.161907460766619
6.18618618618619 -0.16164556705294
6.1961961961962 -0.161384517828221
6.20620620620621 -0.16112430903707
6.21621621621622 -0.16086493664966
6.22622622622623 -0.160606396661532
6.23623623623624 -0.160348685093404
6.24624624624625 -0.160091797990974
6.25625625625626 -0.159835731424735
6.26626626626627 -0.159580481489782
6.27627627627628 -0.159326044305627
6.28628628628629 -0.159072416016015
6.2962962962963 -0.158819592788734
6.30630630630631 -0.15856757081544
6.31631631631632 -0.158316346311471
6.32632632632633 -0.158065915515673
6.33633633633634 -0.157816274690216
6.34634634634635 -0.157567420120421
6.35635635635636 -0.15731934811459
6.36636636636637 -0.157072055003823
6.37637637637638 -0.156825537141856
6.38638638638639 -0.156579790904885
6.3963963963964 -0.1563348126914
6.40640640640641 -0.156090598922018
6.41641641641642 -0.155847146039315
6.42642642642643 -0.155604450507665
6.43643643643644 -0.155362508813074
6.44644644644645 -0.155121317463019
6.45645645645646 -0.154880872986292
6.46646646646647 -0.154641171932836
6.47647647647648 -0.15440221087359
6.48648648648649 -0.154163986400333
6.4964964964965 -0.153926495125529
6.50650650650651 -0.153689733682173
6.51651651651652 -0.153453698723642
6.52652652652653 -0.153218386923541
6.53653653653654 -0.152983794975552
6.54654654654655 -0.152749919593293
6.55655655655656 -0.152516757510163
6.56656656656657 -0.152284305479201
6.57657657657658 -0.15205256027294
6.58658658658659 -0.151821518683266
6.5965965965966 -0.151591177521271
6.60660660660661 -0.151361533617119
6.61661661661662 -0.151132583819901
6.62662662662663 -0.150904324997498
6.63663663663664 -0.150676754036446
6.64664664664665 -0.150449867841795
6.65665665665666 -0.150223663336982
6.66666666666667 -0.149998137463689
6.67667667667668 -0.149773287181714
6.68668668668669 -0.149549109468841
6.6966966966967 -0.149325601320706
6.70670670670671 -0.149102759750672
6.71671671671672 -0.148880581789695
6.72672672672673 -0.148659064486203
6.73673673673674 -0.148438204905966
6.74674674674675 -0.148218000131971
6.75675675675676 -0.147998447264301
6.76676676676677 -0.147779543420008
6.77677677677678 -0.147561285732994
6.78678678678679 -0.147343671353889
6.7967967967968 -0.147126697449932
6.80680680680681 -0.14691036120485
6.81681681681682 -0.146694659818742
6.82682682682683 -0.146479590507963
6.83683683683684 -0.146265150505004
6.84684684684685 -0.146051337058381
6.85685685685686 -0.145838147432518
6.86686686686687 -0.145625578907639
6.87687687687688 -0.145413628779647
6.88688688688689 -0.145202294360022
6.8968968968969 -0.144991572975705
6.90690690690691 -0.144781461968991
6.91691691691692 -0.144571958697419
6.92692692692693 -0.144363060533665
6.93693693693694 -0.144154764865437
6.94694694694695 -0.143947069095364
6.95695695695696 -0.143739970640899
6.96696696696697 -0.143533466934207
6.97697697697698 -0.143327555422066
6.98698698698699 -0.143122233565766
6.996996996997 -0.142917498841003
7.00700700700701 -0.14271334873778
7.01701701701702 -0.142509780760309
7.02702702702703 -0.14230679242691
7.03703703703704 -0.142104381269912
7.04704704704705 -0.141902544835558
7.05705705705706 -0.141701280683907
7.06706706706707 -0.141500586388737
7.07707707707708 -0.141300459537451
7.08708708708709 -0.141100897730983
7.0970970970971 -0.140901898583704
7.10710710710711 -0.140703459723329
7.11711711711712 -0.140505578790823
7.12712712712713 -0.140308253440315
7.13713713713714 -0.140111481339001
7.14714714714715 -0.13991526016706
7.15715715715716 -0.139719587617558
7.16716716716717 -0.139524461396368
7.17717717717718 -0.139329879222074
7.18718718718719 -0.13913583882589
7.1971971971972 -0.138942337951572
7.20720720720721 -0.138749374355329
7.21721721721722 -0.138556945805743
7.22722722722723 -0.138365050083683
7.23723723723724 -0.138173684982219
7.24724724724725 -0.137982848306544
7.25725725725726 -0.137792537873886
7.26726726726727 -0.13760275151343
7.27727727727728 -0.137413487066238
7.28728728728729 -0.137224742385165
7.2972972972973 -0.137036515334783
7.30730730730731 -0.136848803791298
7.31731731731732 -0.136661605642476
7.32732732732733 -0.136474918787561
7.33733733733734 -0.136288741137203
7.34734734734735 -0.136103070613376
7.35735735735736 -0.135917905149304
7.36736736736737 -0.135733242689388
7.37737737737738 -0.135549081189127
7.38738738738739 -0.135365418615048
7.3973973973974 -0.13518225294463
7.40740740740741 -0.13499958216623
7.41741741741742 -0.134817404279015
7.42742742742743 -0.134635717292883
7.43743743743744 -0.1344545192284
7.44744744744745 -0.134273808116723
7.45745745745746 -0.134093581999532
7.46746746746747 -0.13391383892896
7.47747747747748 -0.133734576967525
7.48748748748749 -0.133555794188059
7.4974974974975 -0.133377488673645
7.50750750750751 -0.133199658517542
7.51751751751752 -0.133022301823124
7.52752752752753 -0.132845416703811
7.53753753753754 -0.132669001283004
7.54754754754755 -0.13249305369402
7.55755755755756 -0.132317572080025
7.56756756756757 -0.132142554593971
7.57757757757758 -0.131967999398532
7.58758758758759 -0.131793904666041
7.5975975975976 -0.131620268578427
7.60760760760761 -0.131447089327151
7.61761761761762 -0.131274365113144
7.62762762762763 -0.13110209414675
7.63763763763764 -0.130930274647658
7.64764764764765 -0.130758904844848
7.65765765765766 -0.130587982976525
7.66766766766767 -0.130417507290065
7.67767767767768 -0.130247476041952
7.68768768768769 -0.130077887497719
7.6976976976977 -0.129908739931892
7.70770770770771 -0.129740031627933
7.71771771771772 -0.129571760878178
7.72772772772773 -0.129403925983784
7.73773773773774 -0.129236525254672
7.74774774774775 -0.129069557009469
7.75775775775776 -0.128903019575454
7.76776776776777 -0.128736911288503
7.77777777777778 -0.128571230493033
7.78778778778779 -0.128405975541946
7.7977977977978 -0.12824114479658
7.80780780780781 -0.128076736626651
7.81781781781782 -0.127912749410201
7.82782782782783 -0.127749181533546
7.83783783783784 -0.127586031391222
7.84784784784785 -0.127423297385936
7.85785785785786 -0.127260977928509
7.86786786786787 -0.127099071437831
7.87787787787788 -0.126937576340805
7.88788788788789 -0.126776491072302
7.8978978978979 -0.126615814075104
7.90790790790791 -0.12645554379986
7.91791791791792 -0.126295678705034
7.92792792792793 -0.126136217256857
7.93793793793794 -0.125977157929278
7.94794794794795 -0.125818499203914
7.95795795795796 -0.125660239570006
7.96796796796797 -0.125502377524369
7.97797797797798 -0.125344911571344
7.98798798798799 -0.125187840222753
7.997997997998 -0.125031161997851
8.00800800800801 -0.124874875423281
8.01801801801802 -0.124718979033029
8.02802802802803 -0.124563471368375
8.03803803803804 -0.124408350977851
8.04804804804805 -0.124253616417196
8.05805805805806 -0.12409926624931
8.06806806806807 -0.123945299044211
8.07807807807808 -0.123791713378991
8.08808808808809 -0.123638507837772
8.0980980980981 -0.123485681011663
8.10810810810811 -0.123333231498718
8.11811811811812 -0.123181157903892
8.12812812812813 -0.123029458838999
8.13813813813814 -0.122878132922671
8.14814814814815 -0.122727178780316
8.15815815815816 -0.122576595044077
8.16816816816817 -0.122426380352788
8.17817817817818 -0.122276533351939
8.18818818818819 -0.122127052693629
8.1981981981982 -0.121977937036533
8.20820820820821 -0.121829185045853
8.21821821821822 -0.121680795393288
8.22822822822823 -0.121532766756989
8.23823823823824 -0.12138509782152
8.24824824824825 -0.121237787277823
8.25825825825826 -0.121090833823174
8.26826826826827 -0.120944236161151
8.27827827827828 -0.120797993001591
8.28828828828829 -0.120652103060556
8.2982982982983 -0.120506565060292
8.30830830830831 -0.120361377729196
8.31831831831832 -0.120216539801776
8.32832832832833 -0.120072050018615
8.33833833833834 -0.119927907126337
8.34834834834835 -0.119784109877567
8.35835835835836 -0.119640657030899
8.36836836836837 -0.119497547350858
8.37837837837838 -0.119354779607867
8.38838838838839 -0.119212352578208
8.3983983983984 -0.119070265043991
8.40840840840841 -0.11892851579312
8.41841841841842 -0.118787103619253
8.42842842842843 -0.118646027321776
8.43843843843844 -0.118505285705761
8.44844844844845 -0.118364877581939
8.45845845845846 -0.118224801766664
8.46846846846847 -0.118085057081879
8.47847847847848 -0.117945642355084
8.48848848848849 -0.117806556419304
8.4984984984985 -0.117667798113055
8.50850850850851 -0.117529366280316
8.51851851851852 -0.11739125977049
8.52852852852853 -0.117253477438377
8.53853853853854 -0.117116018144144
8.54854854854855 -0.116978880753287
8.55855855855856 -0.116842064136609
8.56856856856857 -0.116705567170181
8.57857857857858 -0.116569388735315
8.58858858858859 -0.116433527718534
8.5985985985986 -0.116297983011542
8.60860860860861 -0.116162753511191
8.61861861861862 -0.116027838119454
8.62862862862863 -0.115893235743394
8.63863863863864 -0.115758945295136
8.64864864864865 -0.115624965691837
8.65865865865866 -0.115491295855655
8.66866866866867 -0.115357934713725
8.67867867867868 -0.115224881198125
8.68868868868869 -0.115092134245852
8.6986986986987 -0.114959692798791
8.70870870870871 -0.114827555803687
8.71871871871872 -0.11469572221212
8.72872872872873 -0.114564190980474
8.73873873873874 -0.114432961069911
8.74874874874875 -0.114302031446344
8.75875875875876 -0.114171401080411
8.76876876876877 -0.114041068947443
8.77877877877878 -0.113911034027445
8.78878878878879 -0.113781295305064
8.7987987987988 -0.113651851769563
8.80880880880881 -0.113522702414797
8.81881881881882 -0.113393846239186
8.82882882882883 -0.113265282245689
8.83883883883884 -0.113137009441779
8.84884884884885 -0.113009026839416
8.85885885885886 -0.112881333455023
8.86886886886887 -0.112753928309461
8.87887887887888 -0.112626810428004
8.88888888888889 -0.112499978840313
8.8988988988989 -0.112373432580413
8.90890890890891 -0.112247170686666
8.91891891891892 -0.112121192201751
8.92892892892893 -0.111995496172636
8.93893893893894 -0.111870081650554
8.94894894894895 -0.111744947690984
8.95895895895896 -0.111620093353621
8.96896896896897 -0.111495517702355
8.97897897897898 -0.111371219805252
8.98898898898899 -0.111247198734521
8.998998998999 -0.111123453566503
9.00900900900901 -0.110999983381636
9.01901901901902 -0.110876787264444
9.02902902902903 -0.110753864303503
9.03903903903904 -0.110631213591429
9.04904904904905 -0.110508834224849
9.05905905905906 -0.110386725304379
9.06906906906907 -0.110264885934607
9.07907907907908 -0.110143315224066
9.08908908908909 -0.110022012285214
9.0990990990991 -0.109900976234415
9.10910910910911 -0.109780206191913
9.11911911911912 -0.109659701281814
9.12912912912913 -0.109539460632064
9.13913913913914 -0.109419483374427
9.14914914914915 -0.109299768644466
9.15915915915916 -0.10918031558152
9.16916916916917 -0.109061123328688
9.17917917917918 -0.108942191032801
9.18918918918919 -0.108823517844408
9.1991991991992 -0.108705102917755
9.20920920920921 -0.108586945410761
9.21921921921922 -0.108469044485003
9.22922922922923 -0.108351399305693
9.23923923923924 -0.10823400904166
9.24924924924925 -0.108116872865328
9.25925925925926 -0.1079999899527
9.26926926926927 -0.107883359483338
9.27927927927928 -0.107766980640339
9.28928928928929 -0.107650852610325
9.2992992992993 -0.107534974583414
9.30930930930931 -0.107419345753211
9.31931931931932 -0.10730396531678
9.32932932932933 -0.107188832474632
9.33933933933934 -0.107073946430705
9.34934934934935 -0.106959306392343
9.35935935935936 -0.106844911570284
9.36936936936937 -0.106730761178632
9.37937937937938 -0.106616854434851
9.38938938938939 -0.106503190559735
9.3993993993994 -0.106389768777402
9.40940940940941 -0.106276588315267
9.41941941941942 -0.106163648404029
9.42942942942943 -0.106050948277652
9.43943943943944 -0.10593848717335
9.44944944944945 -0.105826264331568
9.45945945945946 -0.105714278995963
9.46946946946947 -0.105602530413393
9.47947947947948 -0.105491017833892
9.48948948948949 -0.10537974051066
9.4994994994995 -0.105268697700043
9.50950950950951 -0.10515788866152
9.51951951951952 -0.10504731265768
9.52952952952953 -0.104936968954211
9.53953953953954 -0.104826856819884
9.54954954954955 -0.104716975526533
9.55955955955956 -0.104607324349042
9.56956956956957 -0.104497902565328
9.57957957957958 -0.104388709456326
9.58958958958959 -0.104279744305973
9.5995995995996 -0.104171006401191
9.60960960960961 -0.104062495031872
9.61961961961962 -0.103954209490864
9.62962962962963 -0.103846149073955
9.63963963963964 -0.103738313079857
9.64964964964965 -0.103630700810191
9.65965965965966 -0.103523311569473
9.66966966966967 -0.103416144665097
9.67967967967968 -0.103309199407323
9.68968968968969 -0.103202475109259
9.6996996996997 -0.103095971086848
9.70970970970971 -0.102989686658856
9.71971971971972 -0.102883621146851
9.72972972972973 -0.102777773875195
9.73973973973974 -0.102672144171025
9.74974974974975 -0.102566731364244
9.75975975975976 -0.102461534787499
9.76976976976977 -0.102356553776175
9.77977977977978 -0.102251787668378
9.78978978978979 -0.102147235804918
9.7997997997998 -0.102042897529299
9.80980980980981 -0.101938772187705
9.81981981981982 -0.101834859128984
9.82982982982983 -0.101731157704637
9.83983983983984 -0.101627667268804
9.84984984984985 -0.101524387178248
9.85985985985986 -0.101421316792345
9.86986986986987 -0.10131845547307
9.87987987987988 -0.101215802584982
9.88988988988989 -0.101113357495213
9.8998998998999 -0.101011119573455
9.90990990990991 -0.100909088191944
9.91991991991992 -0.100807262725453
9.92992992992993 -0.100705642551271
9.93993993993994 -0.100604227049201
9.94994994994995 -0.100503015601535
9.95995995995996 -0.100402007593054
9.96996996996997 -0.100301202411004
9.97997997997998 -0.100200599445093
9.98998998998999 -0.100100198087474
10 -0.099999997732731
};
\addplot [semithick, black]
table {%
-0.489489489489489 0
10.4994994994995 0
};
\addplot [semithick, black]
table {%
0 -1.2
0 0.5
};
\end{axis}

\end{tikzpicture}

\end{center}

\subproblem{b}{}
In the limit $r \ll a$, we have the approximation:

\begin{equation}
    e^{-2r/a} \approx 1 - \frac{2r}{a} + \frac{2r^{2}}{a^{2}} - \frac{4r^{3}}{3a^{3}}
\end{equation}

so that up to second order in $r/a$, the potential is given by:

\begin{equation}
    V(r) \approx \frac{q}{4\pi \epsilon_{0}} \left( -\frac{2}{a} + \frac{2r}{a^{2}} - \frac{4r^{2}}{3a^{3}} + \frac{1}{a} - \frac{2r}{a^{2}} + \frac{2r^{2}}{a^{3}} \right) = \frac{q}{4\pi \epsilon_{0}} \left( -\frac{1}{a} + \frac{2r^{2}}{3a^{3}} \right)
\end{equation}

Therefore, the electric field is given by:

\begin{equation}
    E(r) = -\frac{\partial V}{\partial r} \approx -\frac{qr}{3\pi \epsilon_{0} a^{3}}
\end{equation}

\subproblem{c}{}
We need $E(d) + E_{ext} = 0$, leading to:

\begin{equation}
    \frac{qd}{3\pi \epsilon_{0} a^{3}} = E_{ext}
\end{equation}

Taking $p = qd$, we have $p/E_{ext} = 3\pi \epsilon_{0} a^{3}$

\subproblem{d}{}
We have:

\begin{equation}
    d = \frac{3\pi \epsilon_{0} a^{3} E}{q} \approx \qty{5e-17}{m}
\end{equation}

which is much less than $a$.

\subproblem{e}{}
We know that $\rho = -\epsilon_{0} \nabla^{2} V$, so the charge density is given by:

\begin{equation}
    \rho(r) = -\frac{q}{4\pi} \nabla \cdot \left( \frac{1 - e^{-2r/a}}{r^{2}} - \frac{2e^{-2r/a}}{ar} - \frac{2e^{-2r/a}}{a^{2}} \right) = -\frac{q}{\pi a^{3}} e^{-2r/a}
\end{equation}

The total charge is hence given by the integral:

\begin{equation}
    Q = \int_{0}^{2\pi} \int_{0}^{\pi} \int_{0}^{\infty} \rho(r) r^{2} \sin{\theta} \, \mathrm{d}r \mathrm{d}\theta \mathrm{d}\phi = -\frac{4q}{a^{3}} \int_{0}^{\infty} r^{2} e^{-2r/a} \, \mathrm{d}r = -q
\end{equation}
\qed

\end{document}
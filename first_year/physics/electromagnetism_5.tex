\documentclass[12pt]{article}
\usepackage{homework}
\pagestyle{fancy}

% assignment information
\def\course{Electromagnetism}
\def\assignmentno{Problem Set 5}
\def\assignmentname{Motion of Charged Particles \& Electro-Magnetic Fields and Maxwell's Equations}
\def\name{Xin, Wenkang}
\def\time{\today}

\begin{document}

\begin{titlepage}
    \begin{center}
        \large
        \textbf{\course}

        \vfill

        \Huge
        \textbf{\assignmentno}

        \vspace{1.5cm}

        \large{\assignmentname}

        \vfill

        \large
        \name

        \time
    \end{center}
\end{titlepage}


%==========
\pagebreak
\section*{Motion of Charged Particles}
%==========


\problem{0}{Background}
The Lorentz force is given by:

\begin{equation}
    \mathbf{F} = q \left( \mathbf{E} + \mathbf{v} \times \mathbf{B} \right)
\end{equation}
\qed


\problem{1}{Bainbridge mass spectrometer}

\subproblem{a}{}

\subproblem{b}{}
We need $Eq = Bvq$, or:

\begin{equation}
    v = \frac{E}{B} = \qty{500}{ms^{-1}}
\end{equation}

\subproblem{c}{}
\qed


\problem{2}{Charged particles moving in a constant magnetic field}

\subproblem{a}{}
The equation of motion is:

\begin{equation}
    m \ddot{\mathbf{r}} = q \dot{\mathbf{r}} \times \mathbf{B}
\end{equation}

where we take $\mathbf{B} = (0, 0, B)^{\intercal}$ without loss of generality as we can always align the z-axis with the magnetic field.

Writing out the components:

\begin{equation}
\begin{split}
    m \ddot{x} &= q \dot{y} B \\
    m \ddot{y} &= -q \dot{x} B \\
    m \ddot{z} &= 0
\end{split}
\end{equation}

Thus the speed in the z-direction is constant and in the x-y plane, $\ddot{x}^{2} + \ddot{y}^{2} = \frac{q^{2}B^{2}}{m^{2}} (\dot{x}^{2} + \dot{y}^{2})$, which is uniform circular motion. This demonstrates that the particle moves in a helical path.

\subproblem{b}{}
Let the angle between the magnetic field and the velocity be $\delta \theta$. The velocity component parallel to the magnetic field is $v_{\parallel} = v \cos{\delta \theta}$, while the perpendicular component determines the radius of the helix:

\begin{equation}
    R = \frac{mv_{\perp}}{qB} = \frac{mv \sin{\delta \theta}}{qB}
\end{equation}

For every revolution, the time taken is $T = 2\pi R/v_{\perp} = 2\pi m/qB$. In this time, the particle has travelled a distance $d$ in the perpendicular direction given by:

\begin{equation}
    D = v_{\parallel} T = \frac{2\pi m}{qB} v \cos{\delta \theta} \approx 2\pi \frac{mv}{Be}
\end{equation}

where $q$ is taken as $e$.

\subproblem{c}{}
The total number of revolutions is:

\begin{equation}
    N = \frac{T}{2\pi m/qB} = \frac{\qty{100}{y}}{\qty{0.0357}{s}} = \qty{8.8e10}{}
\end{equation}
\qed


\problem{3}{Magnetic quadrupole lens}

\subproblem{a}{}
The equations of motion are:

\begin{equation}
\begin{split}
    m \ddot{x} &= -qAx \dot{z} \\
    m \ddot{y} &= qAy \dot{z} \\
    m \ddot{z} &= qA(x\dot{x} - y\dot{y})
\end{split}
\end{equation}

Now let $\dot{x}$ and $\dot{y}$ be very small so that we ignore the z-direction acceleration and $\dot{z} = v$ is constant. We have $\ddot{x} = -\kappa^{2} x$ and $\ddot{y} = \kappa^{2} y$, where $\kappa^{2} = qA/mv$ is a positive constant.

\subproblem{b}{}
Obviously, the equation of motion in x-direction gives a harmonic oscillator (focusing), while y-direction is unstable as any small perturbation will grow exponentially (defocusing). 

\subproblem{c}{}
The period of the x-direction is $2\pi/\kappa$, so that after one quarter period, the distance travelled is:

\begin{equation}
    D = \frac{\pi}{2\kappa} v = \frac{\pi}{2} \sqrt{\frac{mv}{\left\lvert qA \right\rvert}}
\end{equation}
\qed


%==========
\pagebreak
\section*{Electro-Magnetic Fields and Maxwell's Equations}
%==========


\problem{0}{Background}
The Maxwell equations are:

\begin{equation}
\begin{split}
    \nabla \cdot \mathbf{E} &= \frac{\rho}{\epsilon_{0}} \\
    \nabla \cdot \mathbf{B} &= 0 \\
    \nabla \times \mathbf{E} &= -\frac{\partial \mathbf{B}}{\partial t} \\
    \nabla \times \mathbf{B} &= \mu_{0} \left( \mathbf{J} + \epsilon_{0} \frac{\partial \mathbf{E}}{\partial t} \right)
\end{split}
\end{equation}

The first one is Gauss's law for electric field while the second one is Gauss's law for magnetic field, which implies the non-existence of magnetic monopoles. The third one is Faraday's law of induction, which implies that a changing electric field induces a magnetic field. The fourth one is Ampere's law, which implies that a current induces a magnetic field.
\qed


\problem{1}{Displacement current}

\subproblem{a}{}
For a long wire, Ampere's law gives:

\begin{equation}
    B = \mu_{0} \frac{I}{2\pi r}
\end{equation}

This cannot be used for a short wire because a current flowing in a short wire causes a build-up of charges at the ends, which causes a time-varying electric field that induces a magnetic field.

\subproblem{b}{}
Given a steady current $I$ flowing in a wire of length $2b$, charges accumulate at the ends according to $q = It$ (the end where the current is flowing into). Let the current be flowing in the z-direction. By symmetry, the electric field created by the charges at the centre is:

\begin{equation}
    \mathbf{E}(r) = -\frac{q}{2\pi \epsilon_{0}} \frac{b}{(r^{2} + b^{2})^{3/2}} \hat{z}
\end{equation}

The modified Ampere's law reads:

\begin{equation}
    \oint \mathbf{B} \cdot d\mathbf{l} = \mu_{0} I_{\text{encl}} + \mu_{0} \int_{S} \epsilon_{0} \frac{\partial \mathbf{E}}{\partial t} \cdot \mathrm{d}\mathbf{A}
\end{equation}

The surface integral evaluates to:

\begin{equation}
    \int_{S} \epsilon_{0} \frac{\partial \mathbf{E}}{\partial t} \cdot \mathrm{d}\mathbf{A} = - \int_{0}^{r} \frac{I}{2\pi} \frac{b}{(r^{2} + b^{2})^{3/2}} 2\pi r \mathrm{d}r = -I \left( 1 - \frac{b}{\sqrt{r^{2} + b^{2}}} \right)
\end{equation}

Combining this with the first term, we have:

\begin{equation}
    \mathbf{B}(r)= \frac{\mu_{0} I}{2\pi r} \frac{b}{\sqrt{r^{2} + b^{2}}} \hat{\phi}
\end{equation}

which agrees with the result from Biot-Savart law.
\qed


\problem{2}{Electro-magnetic waves in vacuo}

\subproblem{a}{}
In vacuum, we have the Maxwell equations:

\begin{equation}
\begin{split}
    \nabla \cdot \mathbf{E} &= 0 \\
    \nabla \cdot \mathbf{B} &= 0 \\
    \nabla \times \mathbf{E} &= -\frac{\partial \mathbf{B}}{\partial t} \\
    \nabla \times \mathbf{B} &= \mu_{0} \epsilon_{0} \frac{\partial \mathbf{E}}{\partial t}
\end{split}
\end{equation}

Taking the curl of the curl terms, we can decouple the equations into two sets:

\begin{equation}
\begin{split}
    \nabla \times ( \nabla \times \mathbf{E} ) = \nabla (\nabla \cdot \mathbf{E}) - \nabla^{2} \mathbf{E} = -\mu_{0} \epsilon_{0} \frac{\partial^{2} \mathbf{E}}{\partial t^{2}} \\
    \nabla \times ( \nabla \times \mathbf{B} ) = \nabla (\nabla \cdot \mathbf{B}) - \nabla^{2} \mathbf{B} = -\mu_{0} \epsilon_{0} \frac{\partial^{2} \mathbf{B}}{\partial t^{2}}
\end{split}
\end{equation}

Since the divergence terms are zero:

\begin{equation}
\begin{split}
    \nabla^{2} \mathbf{E} &= \mu_{0} \epsilon_{0} \frac{\partial^{2} \mathbf{E}}{\partial t^{2}} \\
    \nabla^{2} \mathbf{B} &= \mu_{0} \epsilon_{0} \frac{\partial^{2} \mathbf{B}}{\partial t^{2}}
\end{split}
\end{equation}

Written in components form in Cartesian coordinates, we have:

\begin{equation}
\begin{split}
    \nabla^{2} E_{i} &= \mu_{0} \epsilon_{0} \frac{\partial^{2} E_{i}}{\partial t^{2}} \\
    \nabla^{2} B_{i} &= \mu_{0} \epsilon_{0} \frac{\partial^{2} B_{i}}{\partial t^{2}}
\end{split}
\end{equation}

which are the wave equations.

\subproblem{b}{}
We can check that the following solutions satisfy the wave equations:

\begin{equation}
\begin{split}
    \tilde{\mathbf{E}}(z, t) = \tilde{\mathbf{E}}_{0} e^{i(kz - \omega t)} \\
    \tilde{\mathbf{B}}(z, t) = \tilde{\mathbf{B}}_{0} e^{i(kz - \omega t)}
\end{split}
\end{equation}

where $\tilde{\mathbf{E}}_{0}$ and $\tilde{\mathbf{B}}_{0}$ are some amplitudes, $\omega = ck$ and $c = 1/\sqrt{\mu_{0} \epsilon_{0}}$ is the speed of the wave (speed of light in vacuum).

They are called plane waves for they do not have x- or y-dependence. Since the divergence of $\tilde{\mathbf{E}}$ and $\tilde{\mathbf{B}}$ are zero, we need $\tilde{E}_{0z} = \tilde{B}_{0z} = 0$. By Faraday's law, we need $\nabla \times \tilde{\mathbf{E}} = -\partial \tilde{\mathbf{B}}/\partial t$, which gives $-k \tilde{E}_{0y} = \omega \tilde{B}_{0x}$ and $k \tilde{E}_{0x} = \omega \tilde{B}_{0y}$. This implies:

\begin{equation}
    B_{0} = \frac{k}{\omega} E_{0} = \frac{1}{c} E_{0}
\end{equation}

Therefore, given an electric field of the form $\mathbf{E} = E_{0}(\sin{(kz - \omega t)}, 0, 0)^{\intercal}$, the magnetic field is given by:

\begin{equation}
    \mathbf{B} = \frac{1}{c} E_{0}(0, \sin{(kz - \omega t)}, 0)^{\intercal}
\end{equation}

\subproblem{c}{}
The characteristic impedance of free space is just $E_{0}/B_{0} = c$.
\qed


\problem{3}{Poynting vector of an electro-magnetic wave}

\subproblem{a}{}
The Poynting vector in a plane electromagnetic wave is given by:

\begin{equation}
    \mathbf{S} = \frac{1}{\mu_{0}} \mathbf{E} \times \mathbf{B} = \frac{1}{\mu_{0}c} E_{0}^{2} \sin^{2}{(kz - \omega t + \phi)} \hat{z} = c \epsilon_{0} E_{0}^{2} \sin^{2}{(kz - \omega t + \phi)} \hat{z}
\end{equation}

the average magnitude of which is $c \epsilon_{0} E_{0}^{2}/2$.

\subproblem{b}{}
The distance from the sun to the earth is $cT$. The total radiative power is distributed over the sphere of radius $cT$
\qed


\problem{4}{Poynting vector for a long resistive rod}
The electric field in the rod is given by:

\begin{equation}
    \mathbf{E} = \frac{V}{l} \hat{z} = \frac{IR}{l} \hat{z}
\end{equation}

where the z-direction is defined as the direction of the current.

The magnetic field is given by:

\begin{equation}
    \mathbf{B}(r) = \frac{\mu_{0}I}{2\pi r} \hat{\phi}
\end{equation}

The Poynting vector (outside or on the surface of the rod) is thus given by:

\begin{equation}
    \mathbf{S} = \frac{1}{\mu_{0}} \mathbf{E} \times \mathbf{B} = -\frac{I^{2}R}{2\pi rl} \hat{r}
\end{equation}

which points radially inwards.

Integrating the Poynting vector over the surface of the rod, we get:

\begin{equation}
    -\int_{S} \mathbf{S} \cdot \mathrm{d}\mathbf{A} = I^{2}R
\end{equation}

which is the power radiated by the rod as expected.
\qed

\end{document}
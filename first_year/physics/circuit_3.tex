\documentclass[12pt]{article}
\usepackage{homework}
\pagestyle{fancy}

% assignment information
\def\course{Circuit Theory}
\def\assignmentno{Problem Set 3}
\def\assignmentname{Complex Impedances and Response of Linear Circuits to AC}
\def\name{Xin, Wenkang}
\def\time{\today}

\begin{document}

\begin{titlepage}
    \begin{center}
        \large
        \textbf{\course}

        \vfill

        \Huge
        \textbf{\assignmentno}

        \vspace{1.5cm}

        \large{\assignmentname}

        \vfill

        \large
        \name

        \time
    \end{center}
\end{titlepage}


%==========
\pagebreak
\section*{Complex Impedances and Response of Linear Circuits to AC}
%==========


\problem{14}{}

\subproblem{a}{}

\begin{equation}
    V_{\text{rms}} = V_{C}
\end{equation}

\subproblem{b}{}

\begin{equation}
    V_{\text{rms}} = \frac{V_{C}}{\sqrt{2}}
\end{equation}

\subproblem{c}{}

\begin{equation}
    V_{\text{rms}} = \frac{V_{C}}{2}
\end{equation}

(b) and (c) are different because (b) has a higher extreme absolute value.

\subproblem{d}{}

\begin{equation}
    V_{\text{rms}} = \sqrt{\frac{1}{2T} \int_{-T}^{T} \left( \frac{V_{0}}{T} \right)^{2} t^{2} \, \mathrm{d}t} = \frac{V_{0}}{\sqrt{3}}
\end{equation}
\qed


\problem{15}{}
For all the networks we have $\tilde{V} = V_{0} e^{i(\omega t - \pi/2)}$

\subproblem{a}{}

\begin{equation}
    Z = R - \frac{j}{\omega C}
\end{equation}

\begin{equation}
    \tilde{I} = \frac{\tilde{V}}{Z} = \frac{V_{0}}{\sqrt{R^{2} + (1/\omega C)^{2}}} e^{i(\omega t - \pi/2 + \phi)}
\end{equation}

where $\phi = \tan^{-1}{\left[ 1/(\omega C R) \right]} = \qty{0.016}{rad}$.

The current leads the voltage by $\phi$.

\begin{equation}
    \tilde{V_{R}} = R \tilde{I} = V_{0} \frac{R}{\sqrt{R^{2} + (1/\omega C)^{2}}} e^{i(\omega t - \pi/2 + \phi)}
\end{equation}

Thus, $V_{R, \text{max}} = V_{0} R/(\sqrt{R^{2} + (1/\omega C)^{2}}) = \qty{10}{V}$.

\subproblem{b}{}

\begin{equation}
    Z = j \omega L - \frac{j}{\omega C}
\end{equation}

\begin{equation}
    \tilde{I} = \frac{\tilde{V}}{Z} = \frac{V_{0}}{\sqrt{(\omega L)^{2} + (1/\omega C)^{2}}} e^{i(\omega t - \pi)}
\end{equation}

The current lags the voltage by $\pi/2$.

\begin{equation}
    V_{C, \text{max}} = V_{0} \frac{(1/\omega C)}{\sqrt{(\omega L)^{2} + (1/\omega C)^{2}}} = \qty{10}{V}
\end{equation}

\subproblem{c}{}

\begin{equation}
    Z = R + j \omega L - \frac{j}{\omega C}
\end{equation}

\begin{equation}
    \tilde{I} = \frac{\tilde{V}}{Z} = \frac{V_{0}}{\sqrt{R^{2} + (\omega L)^{2} + (1/\omega C)^{2}}} e^{i(\omega t - \pi/2 + \phi)}
\end{equation}

where:

\begin{equation}
    \phi = \tan^{-1}{\left( \frac{1/(\omega C) - \omega L}{R} \right)} = \qty{5.5e-3}{rad}
\end{equation}

The current leads the voltage by $\phi$.

\begin{equation}
    V_{L, \text{max}} = V_{0} \frac{\omega L}{\sqrt{R^{2} + (\omega L - 1/\omega C)^{2}}} = \qty{1.3e-3}{V}
\end{equation}

\subproblem{d}{}

\begin{equation}
    Z = \frac{1}{1/R + 1/(j \omega L)} = X e^{i\pi/2}
\end{equation}

where $X = \qty{3.95}{\Omega}$.

\begin{equation}
    \tilde{I} = \frac{\tilde{V}}{Z} = \frac{V_{0}}{X} e^{i(\omega t - \pi)}
\end{equation}

The current lags the voltage by $\pi/2$.

\begin{equation}
    V_{R, \text{max}} = V_{0} = \qty{10}{V}
\end{equation}

\subproblem{e}{}

\begin{equation}
    Z = \frac{1}{1/(j \omega L) + j \omega C} = X e^{i\pi/2}
\end{equation}

where $X = \qty{39.5}{\Omega}$.

\begin{equation}
    \tilde{I} = \frac{\tilde{V}}{Z} = \frac{V_{0}}{X} e^{i(\omega t - \pi)}
\end{equation}

The current lags the voltage by $\pi/2$.

\begin{equation}
    V_{C, \text{max}} = V_{0} = \qty{10}{V}
\end{equation}

\subproblem{f}{}

\begin{equation}
    Z = \frac{1}{1/R + \frac{1}{j \omega L - j/(\omega C)}} = \frac{R(\omega L - 1/(\omega C))}{\sqrt{R^{2} + (\omega L - 1/(\omega C))^{2}}} e^{i\phi} = X e^{i\phi}
\end{equation}

where:

\begin{equation}
    \phi = \tan^{-1}{\left( \frac{R}{1/(\omega C) - \omega L} \right)} = \qty{-1.54}{rad}
\end{equation}

\begin{equation}
    \tilde{I} = \frac{\tilde{V}}{Z} = \frac{V_{0}}{X} e^{i(\omega t - \pi/2 - \phi)}
\end{equation}

The current lags the voltage by $\phi$.

\begin{equation}
    V_{L, \text{max}} = V_{0} \frac{\omega L}{\omega L - 1/(\omega C)} = \qty{10.3}{V}
\end{equation}
\qed


\problem{16}{}
For both circuits, $\tilde{V_{1}} = V_{0} e^{i(\omega t - \pi/2)}$.

\subproblem{I}{}

\begin{equation}
    \tilde{V_{2}} = \tilde{V_{1}} \frac{\frac{1}{j \omega C}}{R + \frac{1}{j \omega C}} = \frac{V_{0}}{\sqrt{1 + (\omega C R)^{2}}} e^{i(\omega t - \pi/2 - \phi)}
\end{equation}

where:

\begin{equation}
    \phi = \tan^{-1}{\left( \omega C R \right)}
\end{equation}

\begin{equation}
    \frac{V_{2, \text{max}}}{V_{1, \text{max}}} = \frac{1}{\sqrt{1 + (\omega C R)^{2}}}
\end{equation}

\subproblem{II}{}

\begin{equation}
    \tilde{V_{2}} = \tilde{V_{1}} \frac{j \omega L}{R + j \omega L} = \frac{V_{0}}{\sqrt{1 + (R/\omega L)^{2}}} e^{i(\omega t - \pi/2 + \phi)}
\end{equation}

where:

\begin{equation}
    \phi = \tan^{-1}{\left( \frac{R}{\omega L} \right)}
\end{equation}

\begin{equation}
    \frac{V_{2, \text{max}}}{V_{1, \text{max}}} = \frac{1}{\sqrt{1 + (R/\omega L)^{2}}}
\end{equation}

For both circuits, the ratio is given by:

\begin{equation}
    \frac{1}{\sqrt{1 + (\omega/\omega_{0})^{2}}}
\end{equation}

and the phase is given by:

\begin{equation}
    \tan^{-1}{\left( \frac{\omega}{\omega_{0}} \right)}
\end{equation}
\qed


\problem{17}{}
We have $\tilde{I} = I_{0} e^{i(\omega t - \pi/2)}$.

\subproblem{a}{}

\begin{equation}
    \tilde{V_{C}} = \frac{\tilde{I}}{j \omega C} = \frac{I_{0}}{\omega C} e^{i(\omega t - \pi)}
\end{equation}

where $\frac{I_{0}}{\omega C} = \qty{10}{V}$.

\begin{equation}
    \tilde{V_{L}} = \tilde{I} j \omega L = I_{0} \omega L e^{i\omega t}
\end{equation}

where $I_{0} \omega L = \qty{20}{V}$.

\begin{equation}
    \tilde{V_{R}} = \tilde{I} R = I_{0} R e^{i(\omega t - \pi/2)}
\end{equation}

where $I_{0} R = \qty{10}{V}$.

\begin{equation}
    \tilde{V} = I_{0} X e^{i(\omega t - \pi/2 + \phi)}
\end{equation}

where:

\begin{equation}
    X = \sqrt{R^{2} + (\omega L - 1/(\omega C))^{2}} = \qty{141}{\Omega}
\end{equation}

and:

\begin{equation}
    \phi = \tan^{-1}{\left( \frac{\omega L - 1/(\omega C)}{R} \right)} = \pi/2
\end{equation}

and $I_{0} X = \qty{14.1}{V}$

\subproblem{b}{}

\begin{equation}
    P = \Re(\tilde{V}) \Re(\tilde{I}) = I_{0}^{2} X \sin{(\omega t + \phi)} \sin{\omega t}
\end{equation}

\subproblem{c}{}

\begin{equation}
    W_{L} = \frac{1}{2} L \Re(\tilde{I})^{2} = \frac{1}{2} L I_{0}^{2} \sin^{2}{\omega t}
\end{equation}

\begin{equation}
    W_{C} = \frac{1}{2} C \Im(\tilde{V_{C}})^{2} = \frac{1}{2} \frac{1}{\omega^{2}C} I_{0}^{2} \cos^{2}{\omega t}
\end{equation}

\subproblem{d}

Their maximum values are $\frac{1}{2} L I_{0}^{2}$ and $\frac{1}{2} \frac{1}{\omega^{2}C} I_{0}^{2}$ respectively.

\subproblem{f}

For the sum to be constant, $L = 1/(\omega^{2} C) = \qty{0.1}{mH}$.
\qed


\problem{18}{}
For this circuit:

\begin{equation}
    Z = R + j \omega L - \frac{j}{\omega C} = \sqrt{R^{2} + (\omega L - \frac{1}{\omega C})^{2}} e^{i\phi}
\end{equation}

For the amplitude to be minimum:

\begin{equation}
    \frac{\mathrm{d}}{\mathrm{d}\omega} \left[ \sqrt{R^{2} + (\omega L - \frac{1}{\omega C})^{2}} \right] = \frac{2(\omega L - \frac{1}{\omega C})(L + \frac{1}{\omega^{2} C})}{2\sqrt{R^{2} + (\omega L - \frac{1}{\omega C})^{2}}} = 0
\end{equation}

This happens when $\omega L - \frac{1}{\omega C} = 0$ or $\omega = \frac{1}{\sqrt{LC}}$. Then $Z = R e^{i\phi}$, and $V_{C, \text{max}} = V_{0}/(\omega C R) = \qty{3.16}{V}$.
\qed


\problem{19}{}

\begin{equation}
    Z = j \omega L - \frac{j}{2\omega C} + \frac{1}{j \omega C + \frac{1}{\frac{1}{2j \omega C} + j \omega L}} = j \left( \omega L - \frac{1}{2\omega C} - \frac{1}{\omega C + \frac{2\omega C}{1 - 2\omega^{2} C L}} \right) = 0
\end{equation}

Solving this equation (avoiding the asymptote) gives $\omega = \sqrt{5/(2LC)}$. The other root $\omega = \sqrt{1/(2LC)}$ is obtained by substituting the asymptote value directly into the original expression.
\qed


\problem{20}{}

\subproblem{a}{}
For the bridge to be balanced:

\begin{equation}
    Z_{1} \frac{1}{\frac{1}{R_{4}} + j \omega C_{4}} = \frac{1}{j \omega C_{3}} R_{2}
\end{equation}

Solving for $Z_{1}$ yields:

\begin{equation}
    Z_{1} = R_{2} \frac{C_{4}}{C_{3}} - j \frac{R_{2}}{R_{4}} \frac{1}{\omega C_{3}}
\end{equation}

\subproblem{b}{}
For a series combination:

\begin{equation}
    R_{1} - j \frac{1}{\omega C_{1}} = R_{2} \frac{C_{4}}{C_{3}} - j \frac{R_{2}}{R_{4}} \frac{1}{\omega C_{3}}
\end{equation}

so that $R_{1} = R_{2} C_{4}/C_{3}$ and $C_{1} = C_{3} R_{4}/R_{2}$.

\subproblem{c}{}
For a parallel combination:

\begin{equation}
    \frac{1}{\frac{1}{R_{1}} + j \frac{1}{\omega C_{1}}} = R_{2} \frac{C_{4}}{C_{3}} - j \frac{R_{2}}{R_{4}} \frac{1}{\omega C_{3}}
\end{equation}

Solving this equation yields:

\begin{equation}
\begin{split}
    R_{1} &= \frac{R_{2}\left[ 1 + (\omega C_{4} R_{4})^{2} \right]}{\omega^{2} C_{3} C_{4} R_{4}^{2}} \\
    C_{1} &= \frac{C_{3} R_{4}}{R_{2}\left[ 1 + (\omega C_{4} R_{4})^{2} \right]}
\end{split}
\end{equation}
\qed


\problem{21}{}

\subproblem{a}{}
Note that $\omega_{0} L = 1/(\omega_{0} C) = \sqrt{3}R$

\begin{equation}
    Z = R + \frac{1}{\frac{1}{R + j \omega_{0} L} + \frac{1}{R - j/(\omega_{0} C)}} = R + \frac{(R + j\sqrt{3}R)(R - j\sqrt{3}R)}{2R} = 3R
\end{equation}

\subproblem{b}{}

\begin{equation}
    \tilde{V_{AX}} = \frac{R}{3R} \tilde{V_{AB}} = \frac{\tilde{V_{AB}}}{3}
\end{equation}

\begin{equation}
    \tilde{V_{XY}} = \frac{2}{3} \tilde{V_{AB}} \frac{R}{R + j\sqrt{3}R} = \frac{\tilde{V_{AB}}}{3} e^{i\pi/3}
\end{equation}

\begin{equation}
    \tilde{V_{XZ}} = \frac{1}{3} \tilde{V_{AB}} \frac{R}{R - j/(\sqrt{3}R)} = \frac{\tilde{V_{AB}}}{3} e^{-i\pi/3}
\end{equation}
\qed


\problem{22}{}
We have:

\begin{equation}
    V_{xy} = \left( \frac{1}{2} - \frac{1}{1 + j\omega R C} \right) V = \frac{j\omega R C - 1}{j\omega R C + 1} \frac{V}{2}
\end{equation}

so the amplitude is $\left\lvert V_{xy} = V/2 \right\rvert$ which is independent of $R$ and the phase is $\pi - 2\phi$ where $\phi = \tan^{-1}{(\omega R C)}$.

Therefore, when $R = 1/(\omega C)$, $\pi - 2\phi = \pi/2$.


\end{document}
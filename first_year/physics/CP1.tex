\documentclass[12pt]{article}
\usepackage{homework}
\pagestyle{fancy}

% assignment information
\def\course{Preliminary Examination 2023}
\def\assignmentno{CP 1}
\def\assignmentname{Classical Mechanics \& Special Relativity}
\def\name{}
\def\time{\today}

\begin{document}

\begin{titlepage}
    \begin{center}
        \large
        \textbf{\course}

        \vfill

        \Huge
        \textbf{\assignmentno}

        \vspace{1.5cm}

        \large{\assignmentname}

        \vfill

        \large
        \name

        \time
    \end{center}
\end{titlepage}


%==========
\pagebreak
\section*{Section A}
%==========


\problem{7}{}

\subproblem{a}{}
Let us set the origin of the coordinate system at the geometric centre of the rod, with the positive pointing at the $2m$ end. The new centre of mass is given by:

\begin{equation}
    x_{\text{cm}} = \frac{2ml/2 - ml/2}{2m + m} = \frac{l}{6}
\end{equation}

The momentum of inertia about the geometric centre is:

\begin{equation}
    I = 3m \left( \frac{l}{2} \right)^{2} = \frac{3}{4} ml^{2}
\end{equation}

so that by the parallel axis theorem, the momentum of inertia about the new centre of mass is:

\begin{equation}
    I_{\text{cm}} = I - 3m \left( \frac{l}{6} \right)^{2} = \frac{2}{3} ml^{2}
\end{equation}

By conservation of angular momentum, we have:

\begin{equation}
    I_{\text{cm}} \omega = mv \left( \frac{l}{2} - \frac{l}{6} \right)
\end{equation}

so that $\omega = v/2l$.

\subproblem{b}{}
The linear speed of the centre of mass is given by conservation of momentum:

\begin{equation}
    mv = (2m + m) v_{\text{cm}}
\end{equation}

so that $v_{\text{cm}} = v/3$.

After one half rotation, the linear speed due to $\omega$ is against $v_{\text{cm}}$ for the $2m$ end so that the total speed is:

\begin{equation}
    v_{\text{total}} = v_{\text{cm}} - \omega \left( \frac{l}{2} - \frac{l}{6} \right) = v/6
\end{equation}


%==========
\pagebreak
\section*{Section B}
%==========


\problem{8}{}

\subproblem{a}{}
Given the central potential of the form $V(r) = \beta/r^{2}$, the effective potential is:

\begin{equation}
    V_{\text{eff}}(r) = \frac{L^{2}}{2mr^{2}} + \frac{\beta}{r^{2}}
\end{equation}

where $L \equiv mr^{2}\dot{\theta}$ is the angular momentum of the particle.

\subproblem{b}{}
With $\beta > -L^{2}/2m$, the effective potential is always positive. We have the energy conservation equation:

\begin{equation}
    E = \frac{1}{2} m \dot{r}^{2} + \frac{L^{2}}{2mr^{2}} + \frac{\beta}{r^{2}}
\end{equation}

where $E$ is the conserved energy of the particle.

Differentiating with respect to time, we have:

\begin{equation}
    0 = m \dot{r} \ddot{r} - \frac{L^{2}}{mr^{3}} \dot{r} - \frac{2\beta}{r^{3}} \dot{r}
\end{equation}

or, assuming non-zero $\dot{r}$:

\begin{equation}
    \ddot{r} = \frac{L^{2}}{mr^{3}} + \frac{2\beta}{r^{3}}
\end{equation}

Returning to Equation (2), with the substitution $\dot{r} = \dot{\theta} \mathrm{d}r/\mathrm{d}\theta$, we have:

\begin{equation}
    \dot{r} = \dot{\theta} \frac{\mathrm{d}r}{\mathrm{d}\theta} = \pm\sqrt{\frac{2}{m} \left( E - \frac{L^{2}}{2mr^{2}} - \frac{\beta}{r^{2}} \right)}
\end{equation}

But $\dot{\theta} = L/mr^{2}$, so that:

\begin{equation}
    \frac{1}{r^{2}} \frac{\mathrm{d}r}{\mathrm{d}\theta} = \pm\frac{1}{L} \sqrt{2m \left( E - \frac{L^{2}}{2mr^{2}} - \frac{\beta}{r^{2}} \right)}
\end{equation}

Now use the substitution $u = 1/r$, so that:

\begin{equation}
    -\frac{\mathrm{d}u}{\mathrm{d}\theta} = \pm\frac{1}{L} \sqrt{2mE - (L^{2} + 2m\beta)u^{2}}
\end{equation}

This is a separable differential equation with the solution:

\begin{equation}
    \frac{L}{\sqrt{L^{2} + 2m\beta}} \sin^{-1}{\left( r_{0}u \right)} = \pm\theta + \theta_{0}
\end{equation}

where $r_{0} = \sqrt{L^{2}/2mE + \beta/E}$.

The plus-minus sign corresponds to clock- and counter-clockwise orbits so let us choose the positive case for simplicity. We may set $\theta_{0} = 0$ without loss of generality as this is just a rotation of the coordinate system. Further simplification gives:

\begin{equation}
    \frac{1}{r} = \frac{1}{a} \sqrt{\frac{2mE}{L^{2}}} \sin{(a\theta)}
\end{equation}

where $a^{2} = 1 + 2m\beta/L^{2}$ as expected.

The minimum of $r$ is apparently $r_{\text{min}} = \sqrt{2mE/L^{2}}/a$.

If $\beta = 0$, $a = 1$ and the equation becomes:

\begin{equation}
    \frac{1}{r} = \sqrt{\frac{2mE}{L^{2}}} \sin{\theta}
\end{equation}

which is a straight line as expected for a free particle.

\subproblem{c}{}
With $\beta = -L^{2}/2m$, the effective potential is zero and Equation (7) becomes:

\begin{equation}
    \frac{\mathrm{d}u}{\mathrm{d}\theta} = \pm\frac{1}{L} \sqrt{2mE}
\end{equation}

Taking the positive case, we have the solution:

\begin{equation}
    r = \frac{1}{\theta} \sqrt{\frac{L^{2}}{2mE}}
\end{equation}

Although for $r$ to reach zero, $\theta$ must approach infinity, implying an infinite number of revolutions, this is still possible in finite time. To see this, consider Equation (4):

\begin{equation}
    \ddot{r} = 0
\end{equation}

which means $\dot{r}$ is constant and if $\dot{r} <0$ initially, $r$ always reaches zero in finite time.
\qed


\problem{9}{}

\subproblem{a}{}
The Lagrangian of the system can be written as:

\begin{equation}
    \mathcal{L} = \frac{1}{2} m \left( r^{2} \dot{\theta}^{2} + \dot{r}^{2} \right) + \frac{1}{2} M \dot{r}^{2} + Mg(l-r)
\end{equation}

where the constant term $Mgl$ can be ignored.

The Euler-Lagrange equation gives the equations of motion:

\begin{equation}
\begin{split}
    (m + M) \ddot{r} &= mr\dot{\theta}^{2} - Mg \\
    L \equiv mr^{2}\dot{\theta} &= \text{constant}
\end{split}
\end{equation}

For circular motion, we impose the conditions $\dot{r} = 0$ and $\ddot{r} = 0$ for some $r = r_{0}$ and $\dot{\theta} = \omega$. The equation for $r$ gives us:

\begin{equation}
    mr_{0}\omega^{2} - Mg = 0
\end{equation}

This means that given some initial radius $r_{0}$, $\dot{\theta}$ must satisfy the above equation for circular motion to occur. Under this circular motion, the angular momentum is:

\begin{equation}
    L = mr_{0}^{2}\omega = \sqrt{mMg r_{0}^{3}}
\end{equation}

which is a constant.

Returning to the equation for $r$, we use the substitution $\dot{\theta} = L/mr_{0}^{2}$ to obtain:

\begin{equation}
    (m + M) \ddot{r} = \frac{L^{2}}{mr^{3}} - Mg
\end{equation}

We can expand the right-hand side as a Taylor series about $r = r_{0}$:

\begin{equation}
    \frac{L^{2}}{mr^{3}} - Mg = \frac{L^{2}}{m} \left[ \frac{1}{r_{0}^{3}} - \frac{3(r-r_{0})}{r_{0}^{4}} + \cdots \right] - Mg
\end{equation}

We may set the origin at $r_{0}$ so that $r' \equiv r - r_{0}$ and $\ddot{r}' = \ddot{r}$. Collecting the coefficients of $r'$ and ignoring any constant and higher-order terms, we have:

\begin{equation}
\begin{split}
    (m + M) \ddot{r}' &= -\frac{3L^{2}}{mr_{0}^{4}} r' \\
    \ddot{r}' &= -\frac{3M}{m} \frac{g}{r_{0}} r'
\end{split}
\end{equation}

which is simple harmonic motion with angular frequency:

\begin{equation}
    \Omega = \sqrt{\frac{3M}{m} \frac{g}{r_{0}}}
\end{equation}

For $m \gg M$, this tends to zero as the effect of $M$ can be ignored; for $m \ll M$, small oscillation approximation is no longer true; and for $M = 2m$, this becomes $\sqrt{6g/r_{0}}$.

\subproblem{b}{}
The coordinates of the mass are $X(t) = l \sin{\theta} + A \cos{\omega t}$ and $Y(t) = -l \cos{\theta}$ so that the Lagrangian is:

\begin{equation}
    \frac{\mathcal{L}}{ml^{2}} = \frac{1}{2} \dot{\theta}^{2} + \frac{1}{2} \kappa^{2} \sin^{2}{\omega t} - \kappa \sin{\omega t} \cos{\theta} \dot{\theta} + \frac{g}{l} \cos{\theta}
\end{equation}

where $\kappa \equiv \omega A/l$.

The equation of motion is:

\begin{equation}
    \ddot{\theta} = \kappa \omega \cos{\omega t} \cos{\theta} - \frac{g}{l} \sin{\theta}
\end{equation}

For small oscillations, we may approximate $\cos{\theta} \approx 1$ and $\sin{\theta} \approx \theta$ so that:

\begin{equation}
    \ddot{\theta} = \kappa \omega \cos{\omega t} - \frac{g}{l} \theta
\end{equation}

This is a forced harmonic oscillator with the complementary solution:

\begin{equation}
    \theta_{c} = C \cos{\left( \sqrt{\frac{g}{l}} t + \phi \right)}
\end{equation}

and the particular solution:

\begin{equation}
    \theta_{p} = \frac{\kappa \omega}{g/l - \omega^{2}} \cos{\omega t}
\end{equation}

Assuming that $\omega \neq \sqrt{g/l}$, the general solution is then:

\begin{equation}
    \theta(t) = C \cos{\left( \omega_{0} t + \phi \right)} + \frac{A}{l} \frac{\omega^{2}}{\omega_{0}^{2} - \omega^{2}} \cos{\omega t}
\end{equation}

where $\omega_{0} \equiv \sqrt{g/l}$ is the natural frequency and $C$ and $\phi$ are constants determined by the initial conditions.


\problem{10}{}

\subproblem{a}{}
Consider the moment when the balloon has a mass $m(t)$ and velocity $v(t)$. In a short time $\delta t$, we can write the conservation of momentum as:

\begin{equation}
    mv + (F - mg) \delta t = (m + \delta m)(v + \delta v) + (-\delta m)v \\
\end{equation}

where $F = (M + m_{0})g$ is the constant buoyancy force.

Suppose $\delta m = -\alpha \delta t$ for some positive constant $\alpha$. Then, ignoring second order terms:

\begin{equation}
    \frac{F - mg}{\alpha} \delta m = m \delta v 
\end{equation}

This is a separable equation with the solution:

\begin{equation}
    v = \frac{1}{\alpha} \left[ F \ln{\left( \frac{m}{M} \right)} - g (m - M)\right]
\end{equation}

But $m(t) = M - \alpha t$, so that:

\begin{equation}
    v(t) = \frac{F}{\alpha} \ln{\left( 1 - \frac{\alpha}{M}t \right)} - gt
\end{equation}

The height as a function of time is given by:

\begin{equation}
    h(t) = \int_{0}^{t} v \, \mathrm{d}t = \frac{mF}{\alpha^{2}} \left( 1 - \frac{\alpha}{M}t \right) \left[ 1 - \ln{\left( 1 - \frac{\alpha}{M}t \right)} \right] - \frac{1}{2}gt^{2}
\end{equation}

\subproblem{b}{}
Consider the moment when the object has a mass $m$ and velocity $v$. In a short time $\delta t$, we can write the conservation of momentum as:

\begin{equation}
    mv + mg \sin{\theta} \delta t = (m + \delta m)(v + \delta v)
\end{equation}

Ignoring any second order terms gives:

\begin{equation}
    mg \sin{\theta} \delta t = m \delta v + \delta m v
\end{equation}

By definition $\delta x = v \delta t$. We also have $m = \sigma x$ so that $\delta m = \sigma \delta x$. Substituting:

\begin{equation}
    \sigma g \sin{\theta} \frac{x}{v} \delta x = \sigma x \delta v + \sigma v \delta x
\end{equation}

Or:

\begin{equation}
\frac{\mathrm{d}v}{\mathrm{d}x} = \left( \frac{g \sin{\theta}}{v} - \frac{v}{x} \right)
\end{equation}

Consider the trial solution $x = kt^{2}/2$. Substitution leads to:

\begin{equation}
    \frac{1}{t} = \frac{1}{t} \left( \frac{g \sin{\theta}}{k} - 2 \right)
\end{equation}

which means $k = g\sin{\theta}/3$ as required.

\end{document}
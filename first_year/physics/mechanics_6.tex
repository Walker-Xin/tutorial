\documentclass[12pt]{article}
\usepackage{homework}
\pagestyle{fancy}

% assignment information
\def\course{Classical Mechanics}
\def\assignmentno{Problem Set 6}
\def\assignmentname{Lagrangian Dynamics}
\def\name{Xin, Wenkang}
\def\time{\today}

\begin{document}

\begin{titlepage}
    \begin{center}
        \large
        \textbf{\course}

        \vfill

        \Huge
        \textbf{\assignmentno}

        \vspace{1.5cm}

        \large{\assignmentname}

        \vfill

        \large
        \name

        \time
    \end{center}
\end{titlepage}


%==========
\pagebreak
\section*{Lagrangian Dynamics}
%==========


\problem{1}{Fermat's principle}

\subproblem{a}
Suppose that the trajectory of the beam has the form $y(x)$ from $x_{0}$ to $x_{1}$. Fermat's principle means that $y(x)$ will have such a form that the time taken for the beam to travel from $x_{0}$ to $x_{1}$ is a minimum. That is, the functional of the form

\begin{equation}
    S = \int_{x_{0}}^{x_{1}} \frac{\sqrt{1 + \left( \frac{\mathrm{d}y}{\mathrm{d}x} \right)^{2}}}{c/n(y)} dx = c^{-1} \int_{x_{0}}^{x_{1}} n(y) \sqrt{1 + y'^{2}} dx
\end{equation}

is minimised.

\subproblem{b}
Given that $n$ is a constant, we may split the integral into two parts:

\begin{equation}
    S = S_{1} + S_{2} = \frac{n}{c} \int_{x_{0}}^{x} \sqrt{1 + y_{1}'^{2}} dx + \frac{n}{c} \int_{x}^{x_{1}} \sqrt{1 + y_{2}'^{2}} dx
\end{equation}

Now, we can assert that minimising the two separate functional integrals will give us a minimised $S$. Imposing Euler-Lagrange equation on $S_{1}$ and $S_{2}$, we get:

\begin{equation}
    \frac{\mathrm{d}}{\mathrm{d}x} \left( \frac{y_{1, 2}'}{\sqrt{1 + y_{1, 2}'^{2}}} \right) = 0
\end{equation}

or simply, $y'_{1, 2}$ are constants.

This means that both $y_{1}$ and $y_{2}$ are linear functions of $x$. We have thus proven that light moves in a straight line in a homogeneous medium.

Now we seek a $x \in (x_{0}, x_{1})$ that minimises the following function:

\begin{equation}
    L = \sqrt{(x - x_{0})^{2} + y_{0}^{2}} + \sqrt{(x - x_{1})^{2} + y_{1}^{2}}
\end{equation}

Differentiating, we have the following equation:

\begin{equation}
    \frac{x - x_{0}}{\sqrt{(x - x_{0})^{2} + y_{0}^{2}}} + \frac{x - x_{1}}{\sqrt{(x - x_{1})^{2} + y_{1}^{2}}} = 0
\end{equation}

We can verify that $x = (x_{1}y_{0} + x_{0}y_{1})/(y_{0} + y_{1})$ satisfies this equation, and this is exactly the point where the incident and reflected angles are equal. Thus, we have proven the law of reflection.
\qed


\problem{2}{Motion in two dimensions}
We have the Lagrangian:

\begin{equation}
    \mathcal{L} = \frac{1}{2} m \left\lvert \dot{\mathbf{r}} \right\rvert^{2} - V(r) = \frac{1}{2} m (\dot{r}^{2} + r^{2} \dot{\theta}^{2}) - V(r)
\end{equation}

The generalised momenta are $p_{r} = \partial \mathcal{L}/\partial \dot{r} = m \dot{r}$ and $p_{\theta} = \partial \mathcal{L}/\partial \dot{\theta} = m r^{2} \dot{\theta}$. The Hamiltonian is thus:

\begin{equation}
    \mathcal{H} = p_{r} \dot{r} + p_{\theta} \dot{\theta} - \mathcal{L} = \frac{1}{2} m (\dot{r}^{2} + r^{2} \dot{\theta}^{2}) + V(r)
\end{equation}

which in this case is the usual energy.

Euler-Lagrange equation gives us:

\begin{equation}
\begin{split}
    \frac{\mathrm{d}}{\mathrm{d}t} \left( m \dot{r} \right) &= -\frac{\partial V}{\partial r} \\
    \frac{\mathrm{d}}{\mathrm{d}t} \left( m r^{2} \dot{\theta} \right) &= 0
\end{split}
\end{equation}

$\theta$ is a cyclic coordinate, so we have $p_{\theta}$ is conserved, corresponding to the conservation of angular momentum. The total time derivative of the Hamiltonian is:

\begin{equation}
    \frac{\mathrm{d}\mathcal{H}}{\mathrm{d}t} = \frac{\partial \mathcal{H}}{\partial r} \dot{r} + \frac{\partial \mathcal{H}}{\partial \dot{r}} \ddot{r} + \frac{\partial \mathcal{H}}{\partial \dot{\theta}} \ddot{\theta} = \left( \frac{\partial V}{\partial r} + mr \dot{\theta}^{2} \right) \dot{r} + m \dot{r} \ddot{r} + m r^{2} \dot{\theta} \ddot{\theta} = 0
\end{equation}

This means that the Hamiltonian or the energy is conserved.
\qed


\problem{3}{The simple pendulum}
We have the Lagrangian:

\begin{equation}
    \mathcal{L} = \frac{1}{2} m l^{2} \dot{\theta}^{2} - mgl \cos{\theta}
\end{equation}

where the zero potential is taken as the plane containing the pivot.

Euler-Lagrange equation gives us:

\begin{equation}
    ml^{2} \ddot{\theta} + mgl \sin{\theta} = 0
\end{equation}

or in the small angle approximation:

\begin{equation}
    \ddot{\theta} = -\frac{g}{l} \sin{\theta} \approx -\frac{g}{l} \theta
\end{equation}

The period is thus $T = 2\pi \sqrt{l/g}$.

If the system is accelerated downwards in the vertical direction at a rate $a$, the effective potential becomes $m(g - a)l \cos{\theta}$. The oscillatory behaviour vanishes when $a = g$, i.e., free fall. This is hinting at the equivalence principle.
\qed


\problem{4}{A sliding block}
Following the described coordinate system, we have the Lagrangian:

\begin{equation}
    \mathcal{L} = \frac{1}{2} m \dot{s}^{2} + \frac{1}{2} M \dot{x}^{2} + mgs \sin{\alpha}
\end{equation}

where $\alpha$ is the angle between the incline and the horizontal.

However, we note that there is a constraint on the system such that the block cannot leave the incline. This can be expressed as $f(x, s) = x - s \cos{\alpha} = 0$. We include this constraint in the Lagrangian with a Lagrange multiplier $\lambda$:

\begin{equation}
    \mathcal{L}' = \frac{1}{2} m \dot{s}^{2} + \frac{1}{2} M \dot{x}^{2} + mgs \sin{\alpha} - \lambda \left( x - s\cos{\alpha} \right)
\end{equation}

We have the following equations of motion, along with the constraint:

\begin{equation}
\begin{split}
    m \ddot{s} &= mg \sin{\alpha} + \lambda \cos{\alpha} \\
    M \ddot{x} &= -\lambda \\
    \ddot{x} &= \ddot{s} \cos{\alpha}
\end{split}
\end{equation}

Solving them together, we have:

\begin{equation}
\begin{split}
    \ddot{s} &= \frac{m \sin{\alpha}}{M \cos^{2}{\alpha}} g \\
    \ddot{x} &= \frac{m \sin{\alpha}}{M \cos{\alpha}} g \\
\end{split}
\end{equation}
\qed


\problem{5}{Atwood's machine}
Let us take downwards as positive and maintain this direction throughout. Imposing conservation of string on the second pulley, we have the constraint:

\begin{equation}
    f(z_{i}) = 2z_{1} + z_{2} + z_{3} - 3L = 0
\end{equation}

This reduces the number of degrees of freedom to two, and we can write the Lagrangian as:

\begin{equation}
\begin{split}
    \mathcal{L} &= \frac{1}{2} m (5 \dot{z}_{1}^{2} + 2 \dot{z}_{2}^{2} + 3 \dot{z}_{3}^{2}) + mg(5z_{1} + 2z_{2} + 3z_{3}) \\
                &= \frac{1}{2} m \left[ 5 \dot{z}_{1}^{2} + 2 \dot{z}_{2}^{2} + 3 (2\dot{z}_{1} + \dot{z}_{2})^{2} \right] + mg \left[ 5z_{1} + 2z_{2} + 3(3L - 2z_{1} - z_{2}) \right] \\
                &= \frac{1}{2} m \left( 17 \dot{z}_{1}^{2} + 5 \dot{z}_{2}^{2} + 12 \dot{z}_{1} \dot{z}_{2} \right) - mg \left( z_{1} + z_{2} \right)
\end{split}
\end{equation}

where the constant term proportional to $mgL$ is dropped at the last step.

We have the following equations of motion, along with the constraint:

\begin{equation}
\begin{split}
    17 \ddot{z}_{1} + 6 \ddot{z}_{2} &= -g \\
    6 \ddot{z}_{1} + 5 \ddot{z}_{2} &= -g \\
    2\ddot{z}_{1} + \ddot{z}_{2} + \ddot{z}_{3} &= 0
\end{split}
\end{equation}

Solving them together, we have:

\begin{equation}
\begin{split}
    \ddot{z}_{1} &= \frac{1}{49} g \\
    \ddot{z}_{2} &= -\frac{11}{49} g \\
    \ddot{z}_{3} &= \frac{9}{49} g
\end{split}
\end{equation}

The tension in the upper string is given by:

\begin{equation}
    T_{\text{upper}} = 5mg - 5m \ddot{z}_{1} = \frac{240}{49} g
\end{equation}

The tension in the lower string is given by:

\begin{equation}
    T_{\text{lower}} = 2mg - 2m \ddot{z}_{2} = \frac{120}{49} g
\end{equation}

Verify that:

\begin{equation}
    3mg - 3m \ddot{z}_{3} = T_{\text{lower}} = T_{\text{upper}}/2
\end{equation}

The system is not in equilibrium because the second pulley is not in equilibrium. Once $2m$ and $3m$ start to accelerate, the lower tension is no longer $5mg/2$ and the upper tension is no longer $5mg$. Then the system is no longer in equilibrium.
\qed


\problem{6}{Particle sliding on a sphere}

\subproblem{a}
First let $r = R$ and we have the Lagrangian:

\begin{equation}
    \mathcal{L} = \frac{1}{2} m R^{2} \dot{\theta}^{2} - mRg \cos{\theta}
\end{equation}

which leads to the equation of motion:

\begin{equation}
    \ddot{\theta} = \frac{g}{R} \sin{\theta}
\end{equation}

\subproblem{b}
If we relax the constraint $r = R$ exactly, but instead model it as an infinite potential barrier $V(r)$ satisfying:

\begin{equation}
    V(r) = \infty \text{ for } r < R, \quad 0 \text{ for } r \ge R
\end{equation}

then we have the Lagrangian:

\begin{equation}
    \mathcal{L}' = \frac{1}{2} m (\dot{r}^{2} + r^{2} \dot{\theta}^{2}) - mrg \cos{\theta} - V(r)
\end{equation}

The following modified equations of motion result:

\begin{equation}
\begin{split}
    m\ddot{r} &= mr \dot{\theta}^{2} - mg \cos{\theta} - \frac{\mathrm{d}V}{\mathrm{d}r} \\
    m (r^{2} \ddot{\theta} + 2r \dot{r} \dot{\theta}) &= mrg \sin{\theta}
\end{split}
\end{equation}

We set $r = R$, and the reaction force $F_{N}$ is given by:

\begin{equation}
    F_{N} = \frac{\mathrm{d}V}{\mathrm{d}r} = mr \dot{\theta}^{2} - mg \cos{\theta}
\end{equation}

\subproblem{c}
Before the particle leaves the sphere, the system can be described the original Lagrangian $\mathcal{L}$. We can use the identity $\ddot{\theta} = \dot{\theta} \mathrm{d}\dot{\theta}/\mathrm{d}\theta$ to integrate the equation:

\begin{equation}
\begin{split}
    \dot{\theta} \frac{\mathrm{d}\dot{\theta}}{\mathrm{d}\theta} &= \frac{g}{R} \sin{\theta} \\
    \dot{\theta}^{2} &= 2\frac{g}{R} (1 - \cos{\theta})
\end{split}
\end{equation}

At the moment of leaving the sphere, we return to the modified equations of motion and set $r = R$, $\dot{r} = 0$ and $\mathrm{d}V/\mathrm{d}r = 0$, so that:

\begin{equation}
    m\ddot{r} = mR \dot{\theta}^{2} - mg \cos{\theta} = mg(2 - 3\cos{\theta})
\end{equation}

Now we demand that $\ddot{r} > 0$ for the particle to start leaving the sphere, this means that $\cos{\theta} < 2/3$ or $\theta > \cos^{-1}{(2/3)} \approx \ang{48.2}$.

\qed


\problem{7}{A bead on a rotating hoop}

\subproblem{a}
There is only one degree of motion. Accounting for both motion due to $\dot{\theta}$ and due to $\omega$, the Lagrangian is given by:

\begin{equation}
    \mathcal{L} = \frac{1}{2} m (R^{2} \dot{\theta}^{2} + R^{2} \sin^{2}{\theta} \omega^{2}) + mRg \cos{\theta}
\end{equation}

and the equation of motion is:

\begin{equation}
    \ddot{\theta} = \omega^{2} \sin{\theta} \cos{\theta} - \frac{g}{R} \sin{\theta}
\end{equation}

\subproblem{b}
The equilibrium positions are given by $\ddot{\theta} = 0$. They are $\theta_{0} = 0$ and $\theta_{\pm} = \pm \cos^{-1}{(g/R \omega^{2})}$. The first one is the trivial point at the bottom, and the remaining ones only exist if $\omega^{2} > g/R$.

Let us expand $\ddot{\theta}$ around $\theta = 0$:

\begin{equation}
    \ddot{\theta} = \frac{\omega^{2}}{2} \left( 2\theta - \frac{8\theta^{3}}{6} + \dots \right) - \frac{g}{R} \left( \theta - \frac{\theta^{3}}{6} + \dots \right)
\end{equation}

so that $\mathrm{d}\ddot{\theta}/\mathrm{d}\theta \approx \omega^{2} - g/R$ in the vicinity of $\theta = 0$.

This means that if the non-trivial equilibrium points exist, i.e., $\omega^{2} > g/R$, any small perturbation $\delta \theta$ will lead to an acceleration that is in the same direction as $\delta \theta$, this means that the trivial equilibrium point is unstable. The trivial equilibrium is apparently stable if $\omega^{2} < g/R$, as the particle has nowhere to go anyway.

We can also expand $\mathrm{d}\ddot{\theta}/\mathrm{d}\theta$ around $\theta_{\pm}$:

\begin{equation}
    \frac{\mathrm{d}\ddot{\theta}}{\mathrm{d}\theta} \approx \omega^{2} \cos{2\theta_{\pm}} - \frac{g}{R} \cos{\theta_{\pm}} = 2 \frac{g^{2}}{R^{2}\omega^{2}} - \omega^{2} - \frac{g^{2}}{R^{2} \omega^{2}} = \frac{g^{2}}{R^{2}\omega^{2}} \left( 1 - \frac{\omega^{4}}{g^{2}/R^{2}} \right) < 0
\end{equation}

which means that any small perturbation $\delta \theta$ will lead to an acceleration that is in the opposite direction, implying that the non-trivial equilibrium points are stable.

\subproblem{c}
The total energy is given by:

\begin{equation}
    E = \frac{1}{2} m (R^{2} \dot{\theta}^{2} + R^{2} \sin^{2}{\theta} \omega^{2}) - mRg \cos{\theta}
\end{equation}

whereas the Hamiltonian is given by:

\begin{equation}
    \mathcal{H} = \frac{\partial \mathcal{L}}{\partial \dot{\theta}} \dot{\theta} - \mathcal{L} = \frac{1}{2} m (R^{2} \dot{\theta}^{2} - R^{2} \sin^{2}{\theta} \omega^{2}) - mRg \cos{\theta}
\end{equation}

Taking the time derivative of the energy:

\begin{equation}
    \dot{E} = mR^{2} \dot{\theta} \ddot{\theta} + mR^{2} \omega^{2} \sin{\theta} \cos{\theta} \dot{\theta} + mRg \sin{\theta} \dot{\theta} = 2mR^{2} \omega^{2} \sin{\theta} \cos{\theta} \dot{\theta}
\end{equation}

which is not always zero.

On the other hand, the Hamiltonian is constant:

\begin{equation}
    \dot{\mathcal{H}} = \frac{\partial \mathcal{L}}{\partial \dot{\theta}} \ddot{\theta} + \frac{\mathrm{d}}{\mathrm{d}t} \left( \frac{\partial \mathcal{L}}{\partial \dot{\theta}} \right) \dot{\theta} - \frac{\partial \mathcal{L}}{\partial \theta} \dot{\theta} - \frac{\partial \mathcal{L}}{\partial \dot{\theta}} \ddot{\theta} = 0
\end{equation}

where the middle two terms cancel out by the Euler-Lagrange equation.

We can view the Hamiltonian as the conserved generalised energy of the system, while the usual energy is not conserved as the hoop is constantly doing work on the bead.
\qed


\problem{8}{A pendulum with accelerated support}

\subproblem{a}
The total motion of the particle is a superposition of the pendulum swinging and the box moving. The Lagrangian of the whole system is given by:

\begin{equation}
\begin{split}
    \mathcal{L} &= \frac{1}{2} M \dot{x}^{2} + \frac{1}{2} m \left[ (l \dot{\theta} \cos{\theta} + \dot{x})^{2} + (l \dot{\theta} \sin{\theta})^{2} \right] + mgl \cos{\theta} \\
    &= \frac{1}{2} M \dot{x}^{2} + \frac{1}{2} m \left( l^{2} \dot{\theta}^{2} + 2l \dot{x} \dot{\theta} \cos{\theta} + \dot{x}^{2} \right) + mgl \cos{\theta}
\end{split}
\end{equation}

and the equations of motion are:

\begin{equation}
\begin{split}
    \frac{\mathrm{d}}{\mathrm{d}t} \left[ (M + m) \dot{x} + ml \dot{\theta} \cos{\theta} \right] &= 0 \\
    l \ddot{\theta} + \ddot{x} \cos{\theta} &= -g \sin{\theta}
\end{split}
\end{equation}

The first equation implies the conservation of momentum. Let us denote the quantity by $P$, such that:

\begin{equation}
    P =  (M + m) \dot{x} + ml \dot{\theta} \cos{\theta} = 0
\end{equation}

since $P$ is zero at the initial moment.

\subproblem{b}
In the small angle approximation, we have the equations:

\begin{equation}
\begin{split}
    (M + m) \dot{x} + ml \dot{\theta} &= 0 \\
    l \ddot{\theta} + \ddot{x} &= -g \theta
\end{split}
\end{equation}

We could differentiate the first equation to yield:

\begin{equation}
    (M + m) \ddot{x} + ml \ddot{\theta} = 0
\end{equation}

so that we have a second order differential equation for $\ddot{\theta}$:

\begin{equation}
    \frac{M}{M + m} l \ddot{\theta} = -g \theta
\end{equation}

This is simple harmonic motion with the angular frequency:

\begin{equation}
    \omega = \sqrt{\frac{M + m}{M} \frac{g}{l}}
\end{equation}

On the other hand, we could integrate the first equation to yield:

\begin{equation}
    (M + m) x + ml \theta = C
\end{equation}

where $C$ is an integration constant that can be set as zero as it is equivalent to a translation of the origin.

This means that (the numerical value of) $x$ is directly proportional to $\theta$, and it can be thus argued that $x$ must also undergo simple harmonic motion with the same angular frequency as $\theta$.
\qed


\problem{9}{Normal modes}

\subproblem{a}
The Lagrangian is given by:

\begin{equation}
    \mathcal{L} = \frac{1}{2} m (\dot{x}_{1}^{2} + \dot{x}_{2}^{2}) - \frac{1}{2} k_{1} x_{1}^{2} - \frac{1}{2} k_{2} (x_{2} - x_{1})^{2}
\end{equation}

and the equations of motion are:

\begin{equation}
\begin{split}
    m \ddot{x}_{1} &= -k_{1} x_{1} + k_{2}(x_{2} - x_{1}) \\
    m \ddot{x}_{2} &= -k_{2}(x_{2} - x_{1})
\end{split}
\end{equation}

\subproblem{b}
Define the displacement vector $X = (x_{1}, x_{2})^{\intercal}$, we have the equations in matrix form:

\begin{equation}
    \ddot{X}
    =
    -\frac{1}{m}
    \begin{pmatrix}
        k_{1} + k_{2} & -k_{2} \\
        -k_{2} & k_{2}
    \end{pmatrix}
    X
\end{equation}

so that the eigenfrequencies are:

\begin{equation}
\begin{split}
    \Omega_{+}^{2} = \frac{1}{2m} \left( k_{1} + 2 k_{2} + \sqrt{k_{1}^2+4 k_{2}^2} \right) \\
    \Omega_{-}^{2} = \frac{1}{2m} \left( k_{1} + 2 k_{2} - \sqrt{k_{1}^2+4 k_{2}^2} \right)
\end{split}
\end{equation}

corresponding to the eigenvectors:

\begin{equation}
\begin{split}
    \mathbf{e}_{+} = \left( -\frac{k_{1} + \sqrt{k_{1}^2 + 4k_{2}^2}}{2k_{2}}, 1 \right)^{\intercal} \\
    \mathbf{e}_{-} = \left( -\frac{k_{1} - \sqrt{k_{1}^2 + 4k_{2}^2}}{2k_{2}}, 1 \right)^{\intercal}
\end{split}
\end{equation}

The solution is given by:

\begin{equation}
    X(t) = A \mathbf{e}_{+} \cos{(\Omega_{+} t + \phi_{1})} + B \mathbf{e}_{-} \cos{(\Omega_{-} t + \phi_{2})} 
\end{equation}

where $A$, $B$, $\phi_{1}$ and $\phi_{2}$ are constants to be determined by the initial conditions.

In the limit $k_{1} \ll k_{2}$, $\Omega_{+}^{2} \to 2k_{2}/m$ and $\Omega_{-}^{2} \to 0$. In the limit $k_{1} \gg k_{2}$, $\Omega_{+}^{2} \to k_{1}/m$ and $\Omega_{-}^{2} \to 0$.
\qed


%==========
\pagebreak
\section*{Additional questions}
%==========


\problem{}{}


\end{document}
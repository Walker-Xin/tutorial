\documentclass[12pt]{article}
\usepackage{homework}
\pagestyle{fancy}

% assignment information
\def\course{Circuit Theory}
\def\assignmentno{Problem Set 1}
\def\assignmentname{Introduction to Simple Circuits of Resistors}
\def\name{Xin, Wenkang}
\def\time{\today}

\begin{document}

\begin{titlepage}
    \begin{center}
        \large
        \textbf{\course}

        \vfill

        \Huge
        \textbf{\assignmentno}

        \vspace{1.5cm}

        \large{\assignmentname}

        \vfill

        \large
        \name

        \time
    \end{center}
\end{titlepage}


%==========
\pagebreak
\section*{Introduction to Simple Circuits of Resistors}
%==========


\problem{1}{}

\begin{equation}
    R = R_{1} + R_{2} + \frac{1}{1/R_{3} + 1/R_{4}} = \qty{6}{k\Omega}
\end{equation}

\begin{equation}
    I_{1} = I_{2} = \frac{V_{0}}{R} = \qty{2}{mA}
\end{equation}

\begin{equation}
\begin{split}
    V_{1} = I_{1} R_{1} = \qty{2}{V} \\
    V_{2} = I_{2} R_{2} = \qty{6}{V}
\end{split}
\end{equation}

\begin{equation}
\begin{split}
    V_{3} = V_{4} = V_{0} - V_{1} - V_{2} = \qty{4}{V} \\
    I_{3} = \frac{V_{3}}{R_{3}} = \qty{1.3}{mA} \\
    I_{4} = \frac{V_{4}}{R_{4}} = \qty{0.67}{mA}
\end{split}
\end{equation}
\qed


\problem{2}{}
By symmetry, the equivalent circuit consists of three parallel parts (3-6-3) connected in series. The net resistance is:

\begin{equation}
    R_{\text{net}} = \frac{R}{3} + \frac{R}{4} + \frac{R}{3} = \frac{5}{6} R
\end{equation}
\qed


\problem{3}{}
For nodes on the same edge, note that the symmetry of the system demands that the two adjacent nodes to the connection nodes (excluding the other connection node) must have the same potential. Thus any resistors in-between can be treated as parallel. This simplifies the circuit significantly and the eventual result is:

\begin{equation}
    R_{\text{net}} = \mistake{\frac{2}{5}} R
\end{equation}

\begin{correction}
    \begin{equation}
        R_{\text{net}} = \frac{7}{12} R
    \end{equation}
\end{correction}

For nodes on opposite nodes on the same face, note on this face, the other pair of adjacent nodes must have the same potential. Similarly, the same pair of nodes on the opposite face share the same potential. The result is:

\begin{equation}
    R_{\text{net}} = \frac{3}{4} R
\end{equation}


\problem{4}{}

\begin{equation}
    V_{2} = \frac{R_{2}}{R_{1} + R_{2}} V_{1} = \qty{4}{V}
\end{equation}

After the load has been applied:

\begin{equation}
    V_{2}' = \frac{R'}{R_{1} + R'} V_{1} = 0.95 V_{2}
\end{equation}

where $V_{2}'$ is new the voltage drop across $R_{2}$ and $R'$ is the net resistance of $R_{2}$ and $R_{L}$.

Apparently $1/R' = 1/R_{2} + 1/R_{L}$ and:

\begin{equation}
    \frac{R'}{R_{1} + R'} = 0.95 \frac{R_{2}}{R_{1} + R_{2}}
\end{equation}

Solving this equation for $R_{L}$ yields:

\begin{equation}
    R_{L} = 19 \frac{R_{1} R_{2}}{R_{1} + R_{2}} = \qty{1.52}{k\Omega}
\end{equation}
\qed


\problem{5}{}
A mesh analysis with passive sign convention yields a system of linear equations:

\begin{equation}
\begin{split}
    5 - 10I_{1} - 20(I_{1} + I_{2}) = 0 \\
    10 - 15I_{1} - 20(I_{1} + I_{2}) = 0
\end{split}
\end{equation}

Solving the equations yields:

\begin{equation}
\begin{split}
    I_{1} &= -\frac{1}{26} \unit{A} \\
    I_{2} &= \frac{4}{13} \unit{A} \\
    I_{3} &= I_{1} + I_{2} = \frac{7}{26} \unit{A}
\end{split}
\end{equation}
\qed


\problem{6}{}
Choose the bottom nodes to have zero potential and let the top nodes have potential $\phi$. KCL applied at the node above $R_{2}$ yields the equation:

\begin{equation}
    \frac{\varepsilon - \phi}{R_{1}} + I_{0} = \frac{\phi}{R_{2}}
\end{equation}

Solving this gives \mistake{$\phi = \qty{24}{V}$}, which is the potential difference between A and B.

\begin{equation}
\begin{split}
    I_{1} = \frac{\phi - \varepsilon}{R_{1}} = \mistake{\qty{14}{mA}} \\
    I_{2} = \frac{\phi}{R_{2}} = \mistake{\qty{24}{mA}}
\end{split}
\end{equation}

\begin{correction}
    Solving this gives $\phi = \qty{6}{V}$, which is the potential difference between A and B.
    
    \begin{equation}
    \begin{split}
        I_{1} = \frac{\epsilon - \phi}{R_{1}} = \qty{4}{mA} \\
        I_{2} = \frac{\phi}{R_{2}} = \qty{6}{mA}
    \end{split}
    \end{equation}

    Applying Thevenin's theorem, $R_{\text{eq}} = 1/(1/R_{1} + 1/R_{2}) = \qty{0.5}{k\Omega}$ and $V_{\text{eq}} = \phi = \qty{6}{V}$.

    Applying Norton's theorem, $R_{\text{eq}} = \qty{0.5}{k\Omega}$ and $I_{eq} = V_{\text{eq}}/R_{\text{eq}} = \qty{12}{mA}$.
\end{correction}
\qed

\problem{7}{}
A mesh analysis applied at the central loop yields the equation:

\begin{equation}
    V_{0} - (I_{3} + I_{1}) R_{4} - I_{3} R_{3} - (I_{3} + I_{2}) R_{2} - I_{3} R_{1} = 0
\end{equation}

Solving this gives $I_{3} = \qty{-0.8}{mA}$.

Thus:

\begin{equation}
\begin{split}
    I_{R1} &= \left\lvert I_{3} \right\rvert = \qty{0.8}{mA}, \text{ pointing right} \\
    I_{R2} &= \left\lvert I_{3} + I_{2} \right\rvert = \qty{3.2}{mA}, \text{ pointing up} \\
    I_{R3} &= \left\lvert I_{3} \right\rvert = \qty{0.8}{mA}, \text{ pointing left} \\
    I_{R4} &= \left\lvert I_{3} \right\rvert = \qty{0.2}{mA}, \text{ pointing down}
\end{split}
\end{equation}


\end{document}
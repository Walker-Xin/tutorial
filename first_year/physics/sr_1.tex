\documentclass[12pt]{article}
\usepackage{homework}
\pagestyle{fancy}

% assignment information
\def\course{Special Relativity}
\def\assignmentno{Problems 1}
\def\assignmentname{Lorentz Transformation, Velocity Addition and Energy \& Mass}
\def\name{Xin, Wenkang}
\def\time{\today}

\begin{document}

\begin{titlepage}
    \begin{center}
        \large
        \textbf{\course}

        \vfill

        \Huge
        \textbf{\assignmentno}

        \vspace{1.5cm}

        \large{\assignmentname}

        \vfill

        \large
        \name

        \time
    \end{center}
\end{titlepage}


%==========
\pagebreak
\section*{Lorentz Transformation and Elementary Consequences}
%==========


\problem{1}{}

In frame $S$, the two events have coordinates $(x_{0}/c, x_{0}, 0, 0)$ and $(x_{0}/2c, 2x_{0}, 0, 0)$ respectively. In frame $S'$, we demand that they have coordinates $(t', x_{1}', 0, 0)$ and $(t', x_{2}', 0, 0)$. Thus, by Lorentz transformation:

\begin{equation}
    t' = \gamma (\frac{x_{0}}{c} - \frac{v x_{0}}{c}) = \gamma (\frac{x_{0}}{2c} - \frac{2v x_{0}}{c})
\end{equation}

Hence, solving the equation yield $v = -c/2$ and $t' = \sqrt{3}x_{0}/c$.
\qed


\problem{2}{}

\begin{equation}
    v = \omega R \sin{\ang{30}} = \qty{3.3e8}{m s^{-1}}
\end{equation}

This does not violate the limitation of speed of light, as no new information is transmitted at a superluminal speed.
\qed


\problem{3}{}

In frame $S_{A}$, the first signal is produced at coordinates $(t, 0, 0, 0)$, which corresponds to $(t', x', 0, 0)$ in $S_{B}$.

After a time $\Delta t = \left\lvert x' \right\rvert/c$ in $S_{B}$, $B$ receives the signal and after another $2t$, the return signal is produced. This event is marked as $(t' + \Delta t + 2t, 0, 0, 0)$ in $S_{B}$, which corresponds to $(t_{A}, x_{A}, 0, 0)$ in $S_{A}$.

Then the in $S_{A}$, the time $T$ when $A$ receives the return signal is given by:

\begin{equation}
    \begin{split}
        T &= t_{A} + \frac{1}{c} \left\lvert x_{A} \right\rvert \\
        &= \gamma (t' + \Delta t + 2t) + \frac{1}{c} \gamma v (t' + \Delta t + 2t) \\
        &= \gamma (1 + \frac{v}{c}) (t' + \Delta t + 2t)
    \end{split}
\end{equation}

But by Lorentz transformation, $t' = \gamma t$ and $\Delta t = \left\lvert x' \right\rvert/c = \gamma v t/c$. Hence, substituting the values, we have:

\begin{equation}
    T = (3 + 2\sqrt{3}) t
\end{equation}
\qed


\problem{4}{}
In the pilot's frame, the total distance to be travelled is $D' = D/\gamma = \qty{2.52}{ly}$ due to length contraction. Thus:

\begin{equation}
    T = \frac{D'}{v} = \frac{D}{\gamma v} = \qty{3.15}{years}
\end{equation}
\qed


\problem{5}{}
In the Galaxy frame, the energy of the proton is given by $E = \gamma m c^{2}$. We have the equation:

\begin{equation}
    \beta = \sqrt{1 - \frac{1}{\gamma^{2}}} = \sqrt{1 - \left( \frac{m_{0} c^{2}}{E} \right)^{2}} \approx 1
\end{equation}

as the rest energy of the proton is too small compared to $E$.

Hence, the time to travel across the Galaxy measured in the Galaxy frame is $T_{G} = \qty{10e5}{years}$. Then by time dilation, the time measured in the proton's frame is:

\begin{equation}
    T_{P} = \frac{1}{\gamma} T = \sqrt{1 - \beta^{2}} T_{G} = \qty{4.93}{min}
\end{equation}
\qed


\problem{6}{}

\subproblem{a}
We have the relationship between proper time and time measured in a frame:

\begin{equation}
    \frac{\mathrm{d}\tau}{\mathrm{d}t} = \frac{1}{\gamma}
\end{equation}

Then, as measured in the rest frame:

\begin{equation}
    t = \int_{0}^{\tau} \gamma \, \mathrm{d}\tau = \gamma \tau = \qty{38}{ns}
\end{equation}

\subproblem{b}

\begin{equation}
    D = \beta c t = \qty{8.32}{m}
\end{equation}

\subproblem{c}

\begin{equation}
    D' = \beta c \tau = \qty{5.69}{m}
\end{equation}
\qed


\problem{7}{}
Assume that the 'dying' of muons follows a Poisson process, such that the time for one muon to die follows an exponential distribution with mean $\tau_{0}$. We have:

\begin{equation}
    T \sim \exp(1/\tau_{0})
\end{equation}

Since $1\%$ of the muons survive, we have:

\begin{equation}
    P(T > \tau) = 0.01
\end{equation}

where $\tau$ is the proper time for the muon to travel down to the ground.

Then, from the cumulative distribution function:

\begin{equation}
    P(T \le \tau) = 1 - e^{-\tau/\tau_{0}} = 1 - 0.01
\end{equation}

Solving this equation yields $\tau = \tau_{0} \ln{100}$. The distance travelled measured in muons' frame is thus $D_{0} = v \tau = \qty{3.01e3}{m}$ and the distance travelled measured in the ground frame is $D = \gamma v \tau = \qty{2.13e4}{m}$.
\qed


%==========
\pagebreak
\section*{Addition of velocities; Energy \& Mass}
%==========


\problem{9}{}
Suppose an initial velocity 4-vector $U = \gamma_{u} (c, u, 0, 0)^{\intercal}$. Two successive Lorentz transformations yield:

\begin{equation}
    \begin{split}
        U' &= \Lambda_{v_{2}} \Lambda_{v_{1}} U \\
        &= \Lambda_{v_{2}} \gamma_{u} \begin{pmatrix} \gamma_{v_{1}}c - \gamma_{v_{1}} \beta_{v_{1}} u \\ -\gamma_{v_{1}} \beta_{v_{1}} c + \gamma_{v_{1}} u \\ 0 \\ 0 \end{pmatrix} \\
        &= \gamma_{u} \gamma_{v_{1}} \gamma_{v_{2}} \begin{pmatrix} (1 + \beta_{v_{1}} \beta_{v_{2}})c - (\beta_{v_{1}} + \beta_{v_{2}}) u \\ (1 - \beta_{v_{1}} \beta_{v_{2}})c - (\beta_{v_{1}} + \beta_{v_{2}}) u \\ 0 \\ 0 \end{pmatrix}
    \end{split}
\end{equation}

On the other hand, with $v = (v_{1} + v_{2})/(1 + v_{1}v_{2}/c^{2})$:

\begin{equation}
    \begin{split}
        \Lambda_{v} U = \gamma_{u} \gamma_{v} \begin{pmatrix} c - \beta_{v} u \\ -\beta_{v}c + u \\ 0 \\ 0 \end{pmatrix}
    \end{split}
\end{equation}

To verify $\Lambda_{v_{2}} \Lambda_{v_{1}} = \Lambda_{v}$, we only need to show the following relationship:

\begin{equation}
    \frac{\gamma_{v_{1}} \gamma_{v_{2}}}{1 + \beta_{v_{1}} \beta_{v_{2}}} \stackrel{?}{=} \gamma_{v}
\end{equation}

For $\gamma_{v}$:

\begin{equation}
    \gamma_{v} = \frac{1}{\sqrt{1 - \frac{(v_{1} + v_{2})^{2}}{c^{2} (1 + v_{1}v_{2}/c^{2})^{2}}}} = \frac{1}{1 + \beta_{v_{1}} \beta_{v_{2}}} \frac{1 + \beta_{v_{1}} \beta_{v_{2}}}{\sqrt{\left( 1 + \frac{\beta_{v_{1}} + \beta_{v_{2}}}{1 + \beta_{v_{1}} \beta_{v_{2}}} \right) \left( 1 - \frac{\beta_{v_{1}} + \beta_{v_{2}}}{1 + \beta_{v_{1}} \beta_{v_{2}}} \right)}} = \frac{\gamma_{v_{1}} \gamma_{v_{2}}}{1 + \beta_{v_{1}} \beta_{v_{2}}}
\end{equation}

This verifies the proposed relationship.
\qed


\problem{10}{}
It is known that the molar mass of trinitrotoluene (TNT) is $M_{\text{TNT}} = \qty{227}{g mol^{-1}}$. Thus:

\begin{equation}
    \Delta m = N \frac{\Delta E}{c^{2}} = N_{A} \frac{m_{\text{tot}}}{M_{\text{TNT}}} \frac{\Delta E}{c^{2}} = \qty{2.6e34}{eV c^{-2}} = \qty{47}{g}
\end{equation}

where $N_{A}$ is the Avogadro constant.
\qed


\problem{11}{}
We have:

\begin{equation}
    P = 4\pi R^{2} I = \frac{\mathrm{d}E}{\mathrm{d}t}
\end{equation}

Therefore:

\begin{equation}
    \frac{\mathrm{d}m}{\mathrm{d}t} = \frac{\mathrm{d}}{\mathrm{d}t} \left( \frac{E}{c^{2}} \right) = \frac{P}{c^{2}} = \frac{4\pi R^{2} I}{c^{2}} = \qty{4.4e9}{kg s^{-1}}
\end{equation}
\qed


\end{document}
\documentclass[12pt]{article}
\usepackage{homework}
\pagestyle{fancy}

% assignment information
\def\course{Classical Mechanics}
\def\assignmentno{Problem Set 3}
\def\assignmentname{Projectiles, Rocket Motion and Motion in E \& B Fields}
\def\name{Xin, Wenkang}
\def\time{\today}

\begin{document}

\begin{titlepage}
    \begin{center}
        \large
        \textbf{\course}

        \vfill

        \Huge
        \textbf{\assignmentno}

        \vspace{1.5cm}

        \large{\assignmentname}

        \vfill

        \large
        \name

        \time
    \end{center}
\end{titlepage}


%==========
\pagebreak
\section*{More on Projectiles}
%==========


\problem{1}{Projectile with no air resistance}
Let axes be parallel and perpendicular to the plane. By conservation of energy, the impact speed is $v = \sqrt{2gh}$. Resolving the velocity along the axes:

\begin{equation}
\begin{split}
    v_{x} = v \cos{\theta} \\
    v_{y} = -v \sin{\theta} \\
\end{split}
\end{equation}

Given that no energy is lost, it is implied that $v_{y}$ switches direction and $v_{x}$ remains unchanged. Therefore, the ball moves off the plane at an angle $\theta$ with respect to the normal.

The time taken for the next impact to occur is:

\begin{equation}
    t = 2 \frac{v \sin{\theta}}{g \sin{\theta}} = 2 \frac{v}{g}
\end{equation}

In this time, the distance travelled down the plane is:

\begin{equation}
    D = v_{x} t + \frac{1}{2} g \sin{\theta} t^{2} = 8h\sin{\theta}
\end{equation}
\qed


\problem{2}{Projectiles from a rotating wheel}

The vertical component of the velocity of the water drop is $v_{y} = a\omega \cos{\theta}$ and the horizontal component is $v_{x} = a\omega \sin{\theta}$. By conservation of energy:

\begin{equation}
\begin{split}
    mga \sin{\theta} + \frac{1}{2} m (a\omega)^{2} &= \frac{1}{2} m (a\omega \sin{\theta})^{2} + mg y_{\text{max}} \\
    y_{\text{max}}(\theta) &= a \sin{\theta} + \frac{a^{2}\omega^{2}}{2g} \cos^{2}{\theta}
\end{split}
\end{equation}

Differentiating $y_{\text{max}}(\theta)$ yields $\sin{\theta} = g/(a\omega)$ for maximum height. The maximum height is:

\begin{equation}
    y_{\text{max}} = \frac{g}{2\omega^{2}} + \frac{a^2 \omega^{2}}{2g} < h
\end{equation}

This leads to a quadratic inequality for $\omega^{2}$ that has the solution:

\begin{equation}
    (a\omega)^{2} < gh + g \sqrt{h^{2} - a^{2}}
\end{equation}
\qed


%==========
\pagebreak
\section*{Motion of Rockets}
%==========


\problem{3}{}
Consider conservation of momentum before and after the detachment:

\begin{equation}
\begin{split}
    (m + \delta m)v = m(v + \delta v) + (v - w) \delta m \\
    m \delta v = w \delta m \\
\end{split}
\end{equation}

For the case where the small mass is projected forwards, the sign of $w$ needs to be reversed. Since $\delta m$, being a positive quantity, change treated as the magnitude of mass loss for the rocket, we have for the rocket $-w\mathrm{d}m = m \mathrm{d}v$ or:

\begin{equation}
    m \frac{\mathrm{d}v}{\mathrm{d}t} + w \frac{\mathrm{d}m}{\mathrm{d}t} = 0
\end{equation}

This is essentially the expansion of $F = \mathrm{d}(mv)/\mathrm{d}t$ in the absence of any force. Given an external force of $F = -mg$, we have the expression:

\begin{equation}
    m \frac{\mathrm{d}v}{\mathrm{d}t} + w \frac{\mathrm{d}m}{\mathrm{d}t} + mg = 0
\end{equation}
\qed


\problem{4}{}
Consider conservation of momentum in conjunction with the impulse due to gravity:

\begin{equation}
    mv - g \delta t = (m - \alpha \delta t)(v + \delta v) + \alpha \delta t (v - V)
\end{equation}

Ignoring any second order terms and taking the limit $\delta t, \delta v \to 0$, we have:

\begin{equation}
    \frac{\mathrm{d}v}{\mathrm{d}t} = -g + \frac{\alpha V}{m} = -g + \frac{\alpha V}{M - \alpha t}
\end{equation}

Consider the acceleration of the rocket at $t = 0$, which must be positive for successful launch:

\begin{equation}
    \frac{\mathrm{d}v}{\mathrm{d}t}(0) = -g + \frac{\alpha V}{M} > 0
\end{equation}

This implies $\alpha V > Mg$. Since half of $M$ is fuel, we have $t = M/2\alpha$. The expression for speed at a time $t$ is:

\begin{equation}
    v(t) = \int_{0}^{t} \left( -g + \frac{\alpha V}{M - \alpha t} \right) \, \mathrm{d}t = V \ln{\frac{M}{M - \alpha t}} - gt
\end{equation}

Substituting $t = M/2\alpha$ yields:

\begin{equation}
    v(M/2\alpha) = V \ln{2} - \frac{gM}{2\alpha}
\end{equation}

The height achieved at burn-out is:

\begin{equation}
    h = \int_{0}^{M/2\alpha} \left( V \ln{\frac{M}{M - \alpha t}} - gt \right) \, \mathrm{d}t = -\frac{gM^{2}}{8\alpha^{2}} + \frac{MV}{2\alpha} (1 - \ln{2})
\end{equation}

and the maximum height is:

\begin{equation}
    h + \frac{v(M/2\alpha)^{2}}{2g} = (1 - 2\ln{2}) \frac{MV}{2\alpha} + \frac{(V\ln{2})^{2}}{2g}
\end{equation}
\qed


\problem{5}{Two stage rocket}

\subproblem{i}{}
For a rocket, we have the mass-speed relation:

\begin{equation}
    v = u \ln{\frac{m_{i}}{m_{f}}}
\end{equation}

For the one stage rocket, mass of casing is $m_{c} = (N-1)rm$ and mass of fuel is $(N-1)(1-r)m$. Thus:

\begin{equation}
    v_{1} = u \ln{\frac{N}{N - (N-1)(1-r)}} = u \ln{\frac{N}{rN + (1-r)}}
\end{equation}

\subproblem{ii}
For the first stage, the mass of fuel is $(N-1)(1-r)(N-n)m$ and the final speed after stage one is:

\begin{equation}
    v_{1} = u \ln{\frac{N}{N - (N-1)(1-r)(1-n)}} = u \ln{\frac{N}{rN + n(1-r)}}
\end{equation}

For the second stage, we apply the same logic as in part (i) with the substitution $N \to n$. Therefore:

\begin{equation}
    v_{2} = u \ln{\frac{N}{rN + n(1-r)}} + u \ln{\frac{n}{rn + (1-r)}}
\end{equation}

\subproblem{iii}

\begin{equation}
    \frac{\mathrm{d}v_{2}}{\mathrm{d}n} = u \left( \frac{1}{n} - \frac{r}{rn + (1-r)} - \frac{1-r}{rN + (1-r)n} \right) = 0
\end{equation}

which has a solution of $n^{2} = N$. Thus the maximum of $v_{2}$ is:

\begin{equation}
    u \ln \left[ \frac{n^{3}}{n[rn + (1-r)]} \right] = 2u \ln{\frac{n}{rN + n(1-r)}}
\end{equation}

\subproblem{iv}
The escape velocity from Earth is approximately $\qty{11.2}{kms^{-1}}$. Given $r = 0.1$ and a large $N$ or $n$, the logarithm term approaches $\ln{(1/r)} = \ln{10}$. With $u = \qty{2.9}{kms^{-1}}$, the upper limit of $v_{1}$ is $\qty{6.7}{kms^{-1}}$ and that of $v_{2}$ is $\qty{13}{kms^{-1}}$. Thus, only the two stage rocket can reach escape velocity.
\qed


%==========
\pagebreak
\section*{Work and Energy}
%==========


\problem{6}{Calculation of work along a path}
The work done are:

\begin{equation}
\begin{split}
    W_{a} &= \int_{0}^{1} 2xy \, \mathrm{d}x + \int_{0}^{1} x^{2} \, \mathrm{d}y = \alpha \\
    W_{b} &= \int_{0}^{1} x^{2} \, \mathrm{d}y + \int_{0}^{1} 2xy \, \mathrm{d}x = \alpha \\
    W_{c} &= \int_{0}^{1} 2xy \, \mathrm{d}x + x^{2} \, \mathrm{d}y = \alpha \\
\end{split}
\end{equation}

This implies $\mathbf{F}$ is conservative. For $\mathbf{G}$:

\begin{equation}
\begin{split}
    W_{a} &= \frac{\alpha^{3}}{3} + \alpha^{2} \\
    W_{b} &= \frac{\alpha^{3}}{3} \\
    W_{c} &= \alpha^{3} \\
\end{split}
\end{equation}

This implies $\mathbf{G}$ is not conservative.
\qed


\problem{7}{Pendulum oscillations with large amplitude}

\subproblem{a}
Consider the simple mathematical pendulum and the torque exerted on the mass. We have the equation of motion:

\begin{equation}
    ml^{2} \alpha = -mgl \sin{\theta}
\end{equation}

or

\begin{equation}
    \ddot{\theta} = -\frac{g}{l} \sin{\theta}
\end{equation}

For small $\theta$, we have $\sin{\theta} \approx \theta$ and the motion is simple harmonic with the period $T = 2\pi/\omega = 2\pi \sqrt{l/g}$. Without this approximation, consider the substitution $\ddot{\theta} = \dot{\theta} \mathrm{d}\dot{\theta}/\mathrm{d}\theta$:

\begin{equation}
    \dot{\theta} \frac{\mathrm{d}\dot{\theta}}{\mathrm{d}\theta} = -\frac{g}{l} \sin{\theta}
\end{equation}

Integrating the equation from $\theta_{0}$ to $\theta$ leads to $\dot{\theta}^{2} = 2g(\cos{\theta} - \cos{\theta_{0}})/l$, or:

\begin{equation}
    \frac{\mathrm{d}\theta}{\mathrm{d}t} = \sqrt{\frac{2g}{l}} \sqrt{\cos{\theta} - \cos{\theta_{0}}}
\end{equation}

Integrating again leads to:

\begin{equation}
    T = 4 \sqrt{\frac{l}{2g}} \int_{0}^{\theta_{0}} \frac{1}{\sqrt{\cos{\theta} - \cos{\theta_{0}}}} \, \mathrm{d}\theta
\end{equation}

Comparing with $T = kf(\theta_{0}) \sqrt{l/g}$ gives $k = 2\sqrt{2}$ and:

\begin{equation}
    f(\theta_{0}) = \int_{0}^{\theta_{0}} \frac{1}{\sqrt{\cos{\theta} - \cos{\theta_{0}}}} \, \mathrm{d}\theta
\end{equation}

\subproblem{b}
With the substitution $\sin{(\theta/2)} = \sin{(\theta_{0}/2)} \sin{\phi}$ and $\alpha = \sin{(\theta_{0}/2)}$, we have:

\begin{equation}
    \mathrm{d}\theta = \frac{2\alpha \cos{\phi}}{\sqrt{1 - \alpha^{2} \sin^{2}{\phi}}} \mathrm{d}\phi
\end{equation}

Making the substitution and simplifying yields a complete elliptic integral:

\begin{equation}
    f(\theta_{0}) = \int_{0}^{\sin{\phi} = 0} \sqrt{2} \frac{1}{\sqrt{1 - \alpha^{2} \sin^{2}{\phi}}} \, \mathrm{d}\phi
\end{equation}

or:

\begin{equation}
    T = 4 \sqrt{\frac{l}{g}} \int_{0}^{\pi/2} \frac{1}{\sqrt{1 - \alpha^{2} \sin^{2}{\phi}}} \, \mathrm{d}\phi
\end{equation}

\subproblem{c}
Writing out the integrand with binomial expansion:

\begin{equation}
\begin{split}
    \int_{0}^{\pi/2} \frac{1}{\sqrt{1 - \alpha^{2} \sin^{2}{\phi}}} \, \mathrm{d}\phi &= \int_{0}^{\pi/2} \left( 1 + \frac{1}{2} \alpha^{2} \sin^{2}{\phi} - \frac{1}{8} \alpha^{4} \sin^{4}{\phi} + \dots \right) \, \mathrm{d}\phi \\
    &\approx \int_{0}^{\pi/2} \left( 1 + \frac{1}{2} \alpha^{2} \sin^{2}{\phi} \right) \, \mathrm{d}\phi \\
    &= \frac{\pi}{2} + \frac{\pi}{4} \alpha^{2}
\end{split}
\end{equation}

Thus the percentage error incurred is at least $\frac{\pi}{4} \alpha^{2}/(\frac{\pi}{2} + \frac{\pi}{4} \alpha^{2}) = 3/11 = 27\%$.
\qed


%==========
\pagebreak
\section*{Motion of Particles in E \& B Fields}
%==========


\problem{8}{A charged particle in magnetic field}
The three components of $m \ddot{\mathbf{r}} = q \dot{\mathbf{r}} \cross \mathbf{B}$ are:

\begin{equation}
\begin{split}
    m \ddot{r}_{x} &= q \left( \dot{r}_{y} B_{z} - \dot{r}_{z} B_{y} \right) \\
    m \ddot{r}_{y} &= q \left( \dot{r}_{z} B_{x} - \dot{r}_{x} B_{z} \right) \\
    m \ddot{r}_{z} &= q \left( \dot{r}_{x} B_{y} - \dot{r}_{y} B_{x} \right)
\end{split}
\end{equation}

With $\mathbf{B} = (0, 0, B)^{\intercal}$, we have:

\begin{equation}
    \ddot{\mathbf{r}}
    =
    \frac{q}{m}
    \begin{pmatrix}
        0 & B & 0 \\
        -B & 0 & 0 \\
        0 & 0 & 0
    \end{pmatrix}
    \dot{\mathbf{r}}
\end{equation}

Diagonalising the coefficient matrix, we have $\lambda_{1} = 0$ with $\mathbf{e}_{1} = (0, 0, 1)^{\intercal}$, $\lambda_{2} = i(qB/m)$ with $\mathbf{e}_{2} = (-i, 1, 0)^{\intercal}$ and $\lambda_{3} = -i(qB/m)$ with $\mathbf{e}_{3} = (i, 1, 0)^{\intercal}$. Set $\omega \equiv (qB/m)$, we have the solution for $\dot{\mathbf{r}}$:

\begin{equation}
    \dot{\mathbf{r}}(t) = 
    A
    \begin{pmatrix}
        0 \\
        0 \\
        1
    \end{pmatrix}
    +
    B e^{i\omega t}
    \begin{pmatrix}
        -i \\
        1 \\
        0
    \end{pmatrix}
    +
    C e^{-i\omega t}
    \begin{pmatrix}
        i \\
        1 \\
        0
    \end{pmatrix}
\end{equation}

With the initial condition $\dot{\mathbf{r}}(0) = (0, v_{0}, w_{0})^{\intercal}$, we have $A = w_{0}$ and $B = C = v_{0}/2$. Integrating again:

\begin{equation}
    \mathbf{r}(t) = 
    At
    \begin{pmatrix}
        0 \\
        0 \\
        1
    \end{pmatrix}
    +
    \frac{B}{i\omega} e^{i\omega t}
    \begin{pmatrix}
        -i \\
        1 \\
        0
    \end{pmatrix}
    -
    \frac{C}{i\omega} e^{-i\omega t}
    \begin{pmatrix}
        i \\
        1 \\
        0
    \end{pmatrix}
    +
    \mathbf{K}
\end{equation}

With the initial condition $\mathbf{r}(0) = \mathbf{0}$, we have $\mathbf{K} = (v_{0}/\omega, 0, 0)^{\intercal}$. Thus the full solutions are:

\begin{equation}
\begin{split}
    x(t) &= \frac{v_{0}}{\omega} (1 - \cos{\omega t}) \\
    y(t) &= \frac{v_{0}}{\omega} \sin{\omega t} \\
    z(t) &= w_{0} t
\end{split}
\end{equation}

The path is an upwards helix.
\qed


\problem{9}{Motion in both E \& B fields}
With the addition of $\mathbf{E} = (0, E, 0)^{\intercal}$, the equation of motion becomes:

\begin{equation}
    \ddot{\mathbf{r}}
    =
    \frac{q}{m}
    \begin{pmatrix}
        0 & B & 0 \\
        -B & 0 & 0 \\
        0 & 0 & 0
    \end{pmatrix}
    \dot{\mathbf{r}}
    +
    \frac{q}{m}
    \begin{pmatrix}
        0 \\
        E \\
        0
    \end{pmatrix}
\end{equation}

This is just an inhomogeneous version of the previous problem. Since the inhomogeneous term is constant, the solution only differs by a constant term. Thus, we still consider the general solution:

\begin{equation}
    \dot{\mathbf{r}}(t) = 
    A
    \begin{pmatrix}
        0 \\
        0 \\
        1
    \end{pmatrix}
    +
    B \cos{\omega t}
    \begin{pmatrix}
        -i \\
        1 \\
        0
    \end{pmatrix}
    +
    C \sin{\omega t}
    \begin{pmatrix}
        i \\
        1 \\
        0
    \end{pmatrix}
    +
    \mathbf{K}_{1}
\end{equation}

Integrating again:

\begin{equation}
    \mathbf{r}(t) = 
    At
    \begin{pmatrix}
        0 \\
        0 \\
        1
    \end{pmatrix}
    +
    \frac{B}{\omega} \sin{\omega t}
    \begin{pmatrix}
        -i \\
        1 \\
        0
    \end{pmatrix}
    -
    \frac{C}{\omega} \cos{\omega t}
    \begin{pmatrix}
        i \\
        1 \\
        0
    \end{pmatrix}
    +
    \mathbf{K}_{1}t
    +
    \mathbf{K}_{2}
\end{equation}

With the initial condition $\mathbf{r}(0) = \mathbf{0}$ and $\dot{\mathbf{r}}(0) = (v_0, 0, 0)^{\intercal}$, we have $\mathbf{K}_{1} = (E/B, 0, -A)^{\intercal}$ and $\mathbf{K}_{2} = (v_{0}/\omega, v_{0}/\omega, 0)^{\intercal}$. Thus the full solutions are:

\begin{equation}
\begin{split}
    x(t) &= \frac{v_{0}}{\omega} \sin{\omega t} + \frac{E}{B} t\\
    y(t) &= \frac{v_{0}}{\omega} (1 - \cos{\omega t}) \\
    z(t) &= 0
\end{split}
\end{equation}
\qed


%==========
\pagebreak
\section*{Non-Inertial Reference Frames}
%==========


\problem{10}{}

\subproblem{a}{}

\begin{equation}
    a = \frac{F}{M}
\end{equation}

\subproblem{b}{}

\begin{equation}
    F_{\text{apparent}} = m(g + a) = m(g + \frac{F}{M})
\end{equation}

\subproblem{c}{}
In the non-inertial lift frame:

\begin{equation}
    t = \sqrt{\frac{2h}{g + a}} = \sqrt{\frac{2h}{g + F/M}}
\end{equation}

In the inertial frame that has the same upward speed as the lift at the moment of dropping:

\begin{equation}
\begin{split}
    \frac{1}{2} gt^{2} + \frac{1}{2} at^{2} = h \\
    t = \sqrt{\frac{2h}{g + a}} = \sqrt{\frac{2h}{g + F/M}}
\end{split}
\end{equation}

which agrees with the result from the non-inertial frame.

\subproblem{d}{}
For no acceleration, $F = M/g$.
\qed


%==========
\pagebreak
\section*{Additional Questions}
%==========


\problem{11}{Curvilinear motion}

\subproblem{a}{}
For a general three-dimensional motion described by the position vector $\mathbf{r}(t)$, spherical coordinates can be used to describe the motion so that $\mathbf{r} = r \hat{r}$ and $\mathbf{v} \equiv \dot{\mathbf{r}} = \dot{r} \hat{r} + r \dot{\theta} \hat{\theta} + r \sin{\theta} \dot{\phi} \hat{\phi}$. For a uniform circular motion, we might choose the coordinate system such that the motion happens in a plane that is perpendicular to the z-axis. Then $v_{r} = v_{\theta} = 0$ for circular motion, implying that $\dot{r} = \dot{\theta} = 0$ and $r, \theta$ are constants. For uniform motion, we also require $\dot{\phi}$ to be a constant. This implies:

\begin{equation}
    \dot{\mathbf{r}} = r \sin{\theta} \dot{\phi} \hat{\phi} = (\sin{\theta} \dot{\phi} \hat{\theta}) \times \mathbf{r}
\end{equation}

\subproblem{b}{}
Define the normal unit vector as $\hat{n} = \dot{\hat{\theta}}/ \left\lvert \dot{\hat{\theta}} \right\rvert$. We have:

\begin{equation}
    \frac{\mathrm{d}\hat{\theta}}{\mathrm{d}s} = \frac{\mathrm{d}\hat{\theta}}{\mathrm{d}t} / \frac{\mathrm{d}s}{\mathrm{d}t} = \frac{\lvert \dot{\hat{\theta}} \rvert}{v} \hat{n} = \frac{1}{\rho} \hat{n}
\end{equation}

where $\rho \equiv v/\left\lvert \dot{\hat{\theta}} \right\rvert$ is the radius of curvature.

\subproblem{c}{}
Given $\mathbf{v} = v \hat{\theta}$, differentiating with respect to $t$ gives:

\begin{equation}
    \mathbf{a} = \frac{\mathrm{d}\mathbf{v}}{\mathrm{d}t} = \dot{v} \hat{\theta} + v \dot{\hat{\theta}} = \dot{v} \hat{\theta} + \frac{v^{2}}{\rho} \hat{n}
\end{equation}
\qed


\problem{12}{Projectile with resistance}
We have two equations of motion on the x- and y-axes:

\begin{equation}
\begin{split}
    \ddot{x} &= -k \dot{x} \\
    \ddot{y} &= -g \mp k \dot{y}
\end{split}
\end{equation}

where minus sign is for the upward motion and plus sign is for the downward motion.

For the upward motion, we have the time to reach the maximum height given by the integral:

\begin{equation}
    T_{1} = \int_{0}^{T_{1}} \, \mathrm{d}t = -\int_{v_{0}}^{0} \frac{1}{g + k\dot{y}} \, \mathrm{d}\dot{y} = \frac{1}{k} \ln{\left( 1 + \frac{k v_{0}}{g} \right)}
\end{equation}

Using the identity $\ddot{y} = \dot{y} \mathrm{d}\dot{y}/\mathrm{d}y$, the maximum height is given by:

\begin{equation}
    H = \int_{0}^{H} \, \mathrm{d}y = -\int_{v_{0}}^{0} \frac{\dot{y}}{g + k\dot{y}} \, \mathrm{d}\dot{y} = \frac{v_{0}}{k} - \frac{g}{k^{2}} \ln{\left( 1 + \frac{k v_{0}}{g} \right)}
\end{equation}

For the downward motion:

\begin{equation}
\begin{split}
    t &= \int_{0}^{t} \, \mathrm{d}t = -\int_{0}^{v} \frac{1}{g - k\dot{y}} \, \mathrm{d}\dot{y} = \frac{1}{k} \ln{\left( 1 - \frac{kv}{g} \right)} \\
    v(t) &= \frac{g}{k} \left( 1 - e^{-kt} \right)
\end{split}
\end{equation}

so that the downward flight time is given by:

\begin{equation}
    \int_{H}^{0} \, \mathrm{d}y = \int_{0}^{T_{2}} \frac{g}{k} \left( 1 - e^{-kt} \right) \, \mathrm{d}t
\end{equation}

    





\end{document}
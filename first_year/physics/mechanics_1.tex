\documentclass[12pt]{article}
\usepackage{homework}
\pagestyle{fancy}

% assignment information
\def\course{Classical Mechanics}
\def\assignmentno{Problem Set 1}
\def\assignmentname{Introductory Problems \& Collisions in One Dimension}
\def\name{Xin, Wenkang}
\def\time{\today}

\begin{document}

\begin{titlepage}
    \begin{center}
        \large
        \textbf{\course}

        \vfill

        \Huge
        \textbf{\assignmentno}

        \vspace{1.5cm}

        \large{\assignmentname}

        \vfill

        \large
        \name

        \time
    \end{center}
\end{titlepage}


%==========
\pagebreak
\section*{Introductory Problems}
%==========


\problem{1}{Vectors in two dimensions}

\subproblem{a}
The vector $\mathbf{r}$ makes an angle $\tan^{-1}{(3/4)}$ with the x-axis. Therefore the unit vector $\hat{\mathbf{u}}$ can make an angle $\theta = \tan^{-1}{(3/4)} + \ang{20}$ with the x-axis so that:

\begin{equation}
    \hat{\mathbf{u}} = \cos{\theta} \mathbf{i} + \sin{\theta} \mathbf{j} = 0.547 \mathbf{i} + 0.837 \mathbf{j}
\end{equation}

\subproblem{b}
Since $\hat{\mathbf{u}}$ and $\mathbf{r}$ are linearly independent, they span the whole $\mathbb{R}^{2}$.

\subproblem{c}
The direction of the particle's velocity changes while its magnitude stays constant, so the motion is accelerated.

\subproblem{d}

\begin{equation}
    \theta(t) = \frac{d(t)}{r} + \theta_{0} = \frac{vt}{r} + \tan^{-1}{(3/4)} = 3t + \qty{0.64}{rad}
\end{equation}
\qed


\problem{2}{Dimensional analysis}

\subproblem{a}
Suppose a power law relationship between $T$, $l$, $m$, $g$ and $\theta_{0}$:

\begin{equation}
\begin{split}
    T = k f(\theta_{0}) l^{a} m^{b} g^{c} \\
    T = L^{a} M^{b} (L T^{-2})^{c}
\end{split}
\end{equation}

where $k$ is a dimensionless constant and $f(\theta_{0})$ is an arbitrary function of $\theta_{0}$, which is also dimensionless.

Solving the resulting system of linear equations yields $a = 1/2$, $b = 0$ and $c = -1/2$. Thus:

\begin{equation}
    T = k f(\theta_{0}) \sqrt{\frac{l}{g}}
\end{equation}

\subproblem{b}
Suppose a power law relationship between $T$, $M$, $G$ and $R$. $m$ is ignored because the system is inherently kinematic so only the acceleration of the satellite is of interest. We have:

\begin{equation}
\begin{split}
    T = k M^{a} G^{b} R^{c} \\
    T = M^{a} (L^{3} M^{-1} T^{-2})^{b} L^{c}
\end{split}
\end{equation}

Solving the resulting system of linear equations yields $a = b = -1/2$ and $c = 3/2$. Thus:

\begin{equation}
    T^{2} = k \frac{R^{3}}{GM}
\end{equation}

\subproblem{c}
For vacuum permittivity:

\begin{equation}
\begin{split}
    F_{E} &= \frac{1}{4\pi \epsilon_{0}} \frac{q_{1} q_{2}}{r^{2}} \\
    [\epsilon_{0}] &= \frac{C^{2}}{L^{2}} \frac{1}{M L T^{-2}} = C^{2} T^{2} L^{-1} M^{-1} = I^{2} T^{4} L^{-3} M^{-1}
\end{split}
\end{equation}

For vacuum permeability:

\begin{equation}
\begin{split}
    F_{M} &= \frac{\mu_{0}}{4\pi} \frac{I_{1} I_{2}}{d} l \\
    [\mu_{0}] &= M L T^{-2} \frac{L}{I^{2} L} = A^{-2} T^{-2} L M
\end{split}
\end{equation}

Thus:

\begin{equation}
\begin{split}
    \left[ \sqrt{\frac{1}{\mu_{0} \epsilon_{0}}} \right] = \left( T^{2} L^{-2}\right)^{-1/2} = L T^{-1}
\end{split}
\end{equation}

$(1/\mu_{0} \epsilon_{0})^{1/2}$ is by definition the speed of light in vacuum based on wave equations for electromagnetic waves derived from \text{Max}well's equations.
\qed


\problem{3}{Energy conservation}

\subproblem{a}
By conservation of energy:

\begin{equation}
\begin{split}
    \frac{1}{2} m v^{2} &= mgl \\
    v &= \sqrt{2gl}
\end{split}
\end{equation}

\subproblem{b}

\begin{equation}
    T = mg + m \frac{v^{2}}{l} = 3mg
\end{equation}

\subproblem{c}
As the collision is inelastic:

\begin{equation}
    v = v_{2m} - v_{m}
\end{equation}

By conservation of momentum:

\begin{equation}
    mv = 2m v_{2m} + m v_{m}
\end{equation}

Solving the two linear equations yields $v_{2m} = 2v/3 = 2\sqrt{2gl}/3$ and $v_{m} = -v/3 = -\sqrt{2gl}/3$.

\subproblem{d}
By conservation of energy:

\begin{equation}
\begin{split}
    \frac{1}{2} m v_{m}^{2} &= mgh = mgl \cos{\theta} \\
    \theta &= \cos^{-1}{\left( \frac{v_{m}^{2}}{2gl} \right)} = \ang{27.3} 
\end{split}
\end{equation}
\qed


\problem{4}{The simple harmonic oscillator}

\subproblem{b}

\begin{equation}
    U(x) - U(0) = \frac{1}{2} k \left[ x_{\text{max}} \cos{(\omega_{0}t + \phi)} \right]^{2} = \frac{1}{2} \omega_{0}^{2} m \left[ x_{\text{max}} \cos{(\omega_{0}t + \phi)} \right]^{2}
\end{equation}

The change is always positive, meaning that as long as the particle deviates from the origin, it possesses more potential energy than at the origin.

\subproblem{c}
First note that $v = \dot{x} = -\omega_{0} x_{\text{max}} \sin{(\omega_{0}t + \phi)} $

\begin{equation}
\begin{split}
    E &= \frac{1}{2} m v^{2} + \frac{1}{2} k x^{2} \\
      &= \frac{1}{2} m \omega_{0}^{2} x_{\text{max}}^{2} \sin^{2}{(\omega_{0}t + \phi)} + \frac{1}{2} \omega_{0}^{2} m x_{\text{max}}^{2} \cos^{2}{(\omega_{0}t + \phi)} \\
      &= \frac{1}{2} m \omega_{0}^{2} x_{\text{max}}^{2} = \frac{1}{2} m v_{\text{max}}^{2}
\end{split}
\end{equation}

which is a constant.
\qed


\problem{5}{The potential energy function}

\subproblem{a}
By definition, $F = -\mathrm{d}U/\mathrm{d}x$:

\begin{equation}
    F = -\frac{\mathrm{d}}{\mathrm{d}x} \left( \frac{U_{0} a^{2}}{x^{2} + a^{2}} \right) = U_{0} a^{2} \frac{2x}{(x^{2} + a^{2})^{2}}
\end{equation}

\subproblem{c}
$F$ is always repulsive, as $F$ is positive in the +x-axis and negative in the -x-axis.

\subproblem{d}
By conservation of energy:

\begin{equation}
\begin{split}
    U(0) &= U(\infty) + \frac{1}{2} m v^{2} = \frac{1}{2} m v^{2} \\
    v &= \sqrt{\frac{2U(0)}{m}} = \sqrt{\frac{2U_{0}}{m}}
\end{split}
\end{equation}

\subproblem{e}
For the particle to reach $+\infty$, it just needs to overcome the potential barrier at $x = 0$, i.e., have a speed slightly larger than zero at $x = 0$. By conservation of energy:

\begin{equation}
\begin{split}
    \frac{1}{2} m v_{0}^{2} &= U(0) \\
    v_{0} &= \sqrt{\frac{2U_{0}}{m}}
\end{split}
\end{equation}
\qed


\problem{6}{A two-particle problem in 1-D - the centre of mass system}

\subproblem{a}
By Newton's second and third law:

\begin{equation}
\begin{split}
    m_{1} \ddot{x}_{1} = F_{1} + F_{\text{int}} \\
    m_{2} \ddot{x}_{2} = F_{2} - F_{\text{int}}
\end{split}
\end{equation}

Adding the equations:

\begin{equation}
    m_{1} \ddot{x}_{1} + m_{2} \ddot{x}_{2} = F_{1} + F_{2}
\end{equation}

\subproblem{b}
Given that $F_{1} = F_{2} = 0$, have:

\begin{equation}
    m_{1} \ddot{x}_{1} + m_{2} \ddot{x}_{2} = 0
\end{equation}

Integrating once with respect to time:

\begin{equation}
\begin{split}
    m_{1} \dot{x}_{1} + m_{2} \dot{x}_{2} = P
\end{split}
\end{equation}

where $P$ is an arbitrary constant.

This implies that the total momentum of the system is a constant, i.e., the momentum of the system is conserved.

\subproblem{c}

\begin{equation}
    X_{\text{CM}} = \frac{m_{1} x_{1} + m_{2} x_{2}}{m_{1} + m_{2}}
\end{equation}

Differentiating with respect to time:

\begin{equation}
\begin{split}
    \dot{X}_{\text{CM}} &= \frac{m_{1} \dot{x_{1}} + m_{2} \dot{x_{2}}}{m_{1} + m_{2}} = \frac{P}{m_{1} + m_{2}} \\
    (m_{1} + m_{2}) \dot{X}_{\text{CM}} &= P
\end{split}
\end{equation}

\subproblem{d}

Differentiating again with respect to time:

\begin{equation}
    (m_{1} + m_{2}) \ddot{X}_{\text{CM}} = m_{1} \ddot{x}_{1} + m_{2} \ddot{x}_{2} = F_{1} + F_{2}
\end{equation}

If there is no external force such that $F_{1} = F_{2} = 0$, then $\ddot{X}_{\text{CM}} = 0$ and $X_{\text{CM}}$ is either stationary or moving in a straight line at constant speed.

\subproblem{e}

Given:

\begin{equation}
\begin{split}
    m_{1} \ddot{x}_{1} &= F_{\text{int}} \\
    m_{2} \ddot{x}_{2} &= -F_{\text{int}}
\end{split}
\end{equation}

Make the substitution $x_{i} = x_{i}' + X_{\text{CM}}$:

\begin{equation}
\begin{split}
    \ddot{x}'_{1} + \ddot{X}_{\text{CM}} &= \frac{F_{\text{int}}}{m_{1}} \\
    \ddot{x}'_{2} + \ddot{X}_{\text{CM}} &= -\frac{F_{\text{int}}}{m_{2}}
\end{split}
\end{equation}

Subtracting:

\begin{equation}
\begin{split}
    \ddot{x}'_{1} - \ddot{x}'_{2} &= F_{\text{int}} \left( \frac{1}{m_{1}} + \frac{1}{m_{2}} \right) \\
    \mu \ddot{x}' &= F_{\text{int}}
\end{split}
\end{equation}
\qed


%==========
\pagebreak
\section*{Additional Problems}
%==========


\problem{10}{Energy loss to rest}
By conservation of energy, we have a relation between $h_{n}$ and $h_{n-1}$:

\begin{equation}
    mg h_{n} = (1 - f) mg h_{n-1}
\end{equation}

This recursive relation can be solved with the initial height $h_{0} = h$, so that:

\begin{equation}
    h_{n} = (1 - f)^{n} h
\end{equation}

The time taken from $h_{n}$ to $h_{n+1}$ is:

\begin{equation}
    t_{n} = \sqrt{\frac{2h_{n}}{g}} + \sqrt{\frac{2h_{n+1}}{g}} = \left[ (1 - f)^{n/2} + (1 - f)^{(n + 1)/2} \right] \sqrt{\frac{2h}{g}} = (1 + \sqrt{1 - f}) (1 - f)^{n/2} \sqrt{\frac{2h}{g}}
\end{equation}

This is a geometric series, so the total time is:

\begin{equation}
    T = \sum_{n = 0}^{\infty} t_{n} = \frac{1 + \sqrt{1 - f}}{1 - \sqrt{1 - f}} \sqrt{\frac{2h}{g}}
\end{equation}

Given $h = \qty{5}{m}$ and $f = 0.1$, we have $T = \qty{38.3}{s}$.
\qed


\problem{11}{Force and momentum - falling chain}
 


\end{document}
\documentclass[12pt]{article}
\usepackage{homework}
\pagestyle{fancy}

% assignment information
\def\course{Statistical Physics}
\def\assignmentno{Problem Set 5}
\def\assignmentname{Maximum Entropy Inference \break Canonical Ensemble \break Classical Monatomic Ideal Gas}
\def\name{Xin, Wenkang}
\def\time{\today}

\begin{document}

% title page
\begin{titlepage}
    \begin{center}
        \large
        \textbf{\course}

        \vfill

        \Huge
        \textbf{\assignmentno}

        \vspace{1.5cm}

        \large{\assignmentname}

        \vfill

        \large
        \name

        \time
    \end{center}
\end{titlepage}


%==========
\pagebreak
\section*{Maximum Entropy Inference}
%==========


\problem{5.2}{Tossing coins and assigning probabilities.}

\subproblem{a}
Out of the total $\mathcal{N}$ tosses, choose $\mathcal{N}_1$ to be heads and the rest to be tails. Thus the total number of ways of assigning is:

\begin{equation}
W(\mathcal{N}_{1}) = C_{\mathcal{N}_{1}}^{\mathcal{N}} = \frac{\mathcal{N}!}{\mathcal{N}_{1}!(\mathcal{N} - \mathcal{N}_{1})!}
\end{equation}

\subproblem{b}
Extremising $W(\mathcal{N}_{1})$ is equivalent to extremising the numerator $f(\mathcal{N}_{1}) \equiv \mathcal{N}_{1}!(\mathcal{N} - \mathcal{N}_{1})!$, as the denominator is a constant. Taking the log of $f$:

\begin{equation}
\begin{split}
\ln{f} &= \ln{\mathcal{N}_{1}!} + \ln{(\mathcal{N} - \mathcal{N}_{1})!} \\
       &\approx \mathcal{N}_{1} \ln{\mathcal{N}_{1}} - \mathcal{N}_{1} + (\mathcal{N} - \mathcal{N}_{1}) \ln{(\mathcal{N} - \mathcal{N}_{1})} - (\mathcal{N} - \mathcal{N}_{1}) \\
       &= \mathcal{N}_{1} \ln{\mathcal{N}_{1}} - \mathcal{N}_{1} \ln{(\mathcal{N} - \mathcal{N}_{1})} + \mathcal{N} \ln{(\mathcal{N} - \mathcal{N}_{1})} - \mathcal{N}
\end{split}
\end{equation}

Differentiating with respect to $\mathcal{N}_{1}$:

\begin{equation}
\begin{split}
\deri{f}{\mathcal{N}_{1}} &= 1 + \ln{\mathcal{N}_{1}} + \frac{\mathcal{N}_{1}}{\mathcal{N} - \mathcal{N}_{1}} - \ln{(\mathcal{N} - \mathcal{N}_{1})} - \frac{\mathcal{N}}{\mathcal{N} - \mathcal{N}_{1}} \\
                          &= \ln{\frac{\mathcal{N}_{1}}{\mathcal{N} - \mathcal{N}_{1}}} = 0
\end{split}
\end{equation}

Obviously, the equality is achieved when $\mathcal{N}_{1} = \mathcal{N}/2$. This is a minimum for $f$ because the second derivative equals $1/\mathcal{N}_{1} + 1/(\mathcal{N}-\mathcal{N}_{1})$, which is positive at $\mathcal{N}_{1} = \mathcal{N}/2$. This implies that $W$ will achieve a maximum, with the probabilities $p_{1} = p_{2} = 1/2$. The Gibbs entropy is given by

\begin{equation}
S_{G} = -p_{1} \ln{p_{1}} - p_{2} \ln{p_{2}} = \ln{2}
\end{equation}

\subproblem{c}
The expressions for $W(0)$ and $W(m)$ are:

\begin{equation}
W(0)  = \frac{\mathcal{N}!}{(\mathcal{N}/2)!^{2}} \quad W(m) = \frac{\mathcal{N}!}{(\mathcal{N}/2 - m)!(\mathcal{N}/2 + m)!}
\end{equation}

Taking the log of their ratio:

\begin{equation}
\begin{split}
\ln{\frac{W(m)}{W(0)}} &= (\frac{\mathcal{N}}{2} \ln{\frac{\mathcal{N}}{2}} - \frac{\mathcal{N}}{2}) \times 2 - \left[ (\frac{\mathcal{N}}{2} - m) \ln{(\frac{\mathcal{N}}{2} - m)} - (\frac{\mathcal{N}}{2} - m) \right] \\
                       &- \left[ (\frac{\mathcal{N}}{2} + m) \ln{(\frac{\mathcal{N}}{2} + m)} - (\frac{\mathcal{N}}{2} + m) \right] \\
                       &= \mathcal{N} \ln{\frac{\mathcal{N}}{2}} - \frac{\mathcal{N}}{2} \ln{\left[ (\frac{\mathcal{N}}{2})^{2} - m^{2} \right]} + m \ln{\frac{\mathcal{N}/2 - m}{\mathcal{N}/2 + m}} \\
                       &= \mathcal{N} \ln{\frac{\mathcal{N}}{2}} - \frac{\mathcal{N}}{2} \ln{\mathcal{N}^{2} \left[ \frac{1}{4} - (\frac{m}{\mathcal{N}})^{2} \right]} + m \ln{\frac{(1/2 - m/\mathcal{N})^{2}}{(1/2)^{2} - (m/\mathcal{N})^{2}}}
\end{split}
\end{equation}

At the limit $m \ll \mathcal{N}$, the following approximation can be made:

\begin{equation}
\begin{split}
\ln{\frac{W(m)}{W(0)}} &\approx \mathcal{N} \ln{\frac{\mathcal{N}}{2}} - \frac{\mathcal{N}}{2} \ln{\frac{\mathcal{N}^{2}}{4}} + m \ln{\left( 1 - \frac{4m}{\mathcal{N}} \right)} \\
                       &= m \ln{\left( 1 - \frac{4m}{\mathcal{N}} \right)} \\
                       &\approx -m \left( \frac{4m}{\mathcal{N}} \right)
\end{split}
\end{equation}

Thus:

\begin{equation}
\frac{W(m)}{W(0)} \approx \exp{-\frac{4m^{2}}{\mathcal{N}}}
\end{equation}
\qed

\problem{5.3}{Loaded die.}

\subproblem{a}
We expect $p_{\alpha} = 1/6$ for all cases, and the expectation $\ave{\alpha}$ is obviously $3.5$.

\subproblem{b}
Given $\ave{\alpha} \equiv \sum_{\alpha} p_{\alpha} \alpha = 3.667$, this is a maximisation problem similar to the derivation of canonical distribution. We wish to find a set $\{p_{\alpha}\}$ of probabilities that extremises the Gibbs entropy $S_{G}$ given by:

\begin{equation}
S_{G} = - \sum p_{\alpha} \ln{p_{\alpha}}
\end{equation}

under the constraints:

\begin{equation}
\sum p_{\alpha} - 1 = 0
\end{equation}

and:

\begin{equation}
\sum p_{\alpha} \alpha = \ave{\alpha}
\end{equation}

Following the usual procedure of Lagrangian multipliers, we find $p_{\alpha}$ to follow the distribution:

\begin{equation}
p_{\alpha} = \frac{e^{-\beta \alpha}}{Z(\beta)}
\end{equation}

where $Z(\beta)$ is the normalisation function defined as:

\begin{equation}
Z(\beta) = \sum_{\alpha} e^{-\beta \alpha}
\end{equation}

Then $\beta$ is determined by the equation (derived from the second constant):

\begin{equation}
- \pderi{\ln{Z}}{\beta} = \ave{\alpha} = 3.667
\end{equation}

which can be solved numerically to find $\beta$ and thus all $p_{\alpha}$.
\qed


%==========
\pagebreak
\section*{Canonical Ensemble}
%==========


\problem{5.4}{Heat Capacity, Thermal Stability and Fluctuations.}

\subproblem{a}
By definition, the constant volume heat capacity is given by:

\begin{equation}
\begin{split}
C_{V} &\equiv \left( \pderi{U}{T} \right)_{V} \\
      &= \left( \pderi{U}{\beta} \pderi{\beta}{T} \right)_{V} \\
      &= \left[ (-\pderi[2]{\ln{Z}}{\beta}) (-\frac{1}{k_{B} T^2}) \right]_{V} \\
      &= \left( k_{B} \beta^{2} \pderi[2]{\ln{Z}}{\beta} \right)_{V}
\end{split}
\end{equation}

\subproblem{b}
The partition function of a classical monatomic ideal gas is given by:

\begin{equation}
Z = \frac{1}{N!} \left( \frac{V}{\lambda_{\text{th}}^{3}} \right)^{N}
\end{equation}

where $\lambda_{\text{th}} = \hbar \sqrt{\frac{2\pi}{m}} \sqrt{\beta}$.

Hence the log of the partition function is:

\begin{equation}
\begin{split}
\ln{Z} &= N (\ln{V} - 3\ln{\lambda_{\text{th}}}) - \ln{N!} \\
       &\approx N (\ln{N} - \ln{n} - 3\ln{\lambda_{\text{th}}}) - N \ln{N!} + N \\
       &= N \left( -\frac{3}{2} \ln{\beta} + \text{constant} \right)
\end{split}
\end{equation}

Substituting into the expression in (a), we have:

\begin{equation}
C_{V} = \frac{3}{2} k_{B} N
\end{equation}

\subproblem{c}
The expression for $C_{V}$ can be rewritten as:

\begin{equation}
\begin{split}
C_{V} &= \pderi{U}{\beta} \pderi{\beta}{T} \\
      &= \left[ -\frac{1}{Z} \sum E_{\alpha}^{2} e^{-\beta E_{\alpha}} - \frac{1}{Z} \pderi{Z}{\beta} \sum E_{\alpha} e^{-\beta E_{\alpha}} \right] (-k_{B} \beta^{2}) \\
      &= k_{B} \beta^{2} \left[ \frac{1}{Z} \sum E_{\alpha}^{2} e^{-\beta E_{\alpha}} - \left( \frac{1}{Z} \sum E_{\alpha} e^{-\beta E_{\alpha}} \right)^{2} \right] \\
      &= k_{B} \beta^{2} (\ave{E_{\alpha}^{2}} - \ave{E_{\alpha}}^{2})
\end{split}
\end{equation}

which is greater than or equal to zero by the definition of variance.

\subproblem{d}
null.
\qed

\problem{5.5}{Elastic Chain.}

\subproblem{a}
Denote the vertical single-segment state by $+$ and the horizontal single-segment state by $-$. The microstate of the whole chain is determined by the number $m$ of segments in $+$ state, where $m$ ranges from zero to $N$.

Using the canonical ensemble, the single-segment partition function $Z_{1}$ is obviously:

\begin{equation}
Z_{1} = e^{-\beta E_{-}} + e^{-\beta E_{+}} = 1 + e^{-\beta E_{0}}
\end{equation}

where $E_{0} \equiv \gamma a$ is the energy of the $+$ state and the $-$ state has zero energy.

The energies $E_{m}$ of the microstates $m$ of the whole chain are given by:

\begin{equation}
E_{m} = m \gamma a
\end{equation}

Thus the partition function $Z$ of the whole chain is:

\begin{equation}
Z = \sum_{m=0}^{N} e^{-\beta E_{m}} = \frac{1 - e^{-(N+1)\beta E_{0}}}{1 - e^{-\beta E_{0}}}
\end{equation}

\subproblem{b}
The lengths $L_{m}$ of the microstates are:

\begin{equation}
L_{m} = (N - m)a = Na - \frac{E_{m}}{\gamma}
\end{equation}

Thus the mean length $L$ is:

\begin{equation}
L = \sum p_{m} L_{m} = \sum p_{m} (Na - \frac{E_{m}}{\gamma}) = Na - \frac{U}{\gamma}
\end{equation}

where $U$ is the mean energy by definition.

Thus we have the relation $\gamma = -U/(L - L_{0})$ where $L_{0} \equiv Na$ is the natural length of the chain if there is no tension and all segments are in $-$ state. On the other hand, $U$ is given by the partition function via:

\begin{equation}
\begin{split}
-U &= \pderi{\ln{Z}}{\beta} \\
   &= \pderi{}{\beta} \left[ \ln{(1 - e^{-(N+1)\beta E_{0}})} - \ln{(1 - e^{-\beta E_{0}})} \right] \\
   &= - \frac{E_{0} e^{-\beta E_{0}}}{1 - e^{-\beta E_{0}}} + \frac{(N+1)E_{0} e^{-(N+1)\beta E_{0}}}{1 - e^{-(N+1)\beta E_{0}}} \\
   &\approx - \frac{E_{0} e^{-\beta E_{0}}}{1 - e^{-\beta E_{0}}} \\
   &= E_{0} \frac{1}{1 - e^{\beta E_{0}}}
\end{split}
\end{equation}

where the approximation is made as $N$ is very large.

Now we assume that the temperature is high so that $\beta E_{0} = \beta \ll 1$, and the following approximation can be made:

\begin{equation}
\begin{split}
\gamma = \frac{1}{L - L_{0}} \gamma a \frac{1}{e^{\beta E_{0}} - 1} \\
\frac{a}{e^{\beta E_{0}} - 1} = L - L_{0}
\end{split}
\end{equation}

\subproblem{d}
Restricting $U$ as a given effectively turns the system into an isolated system with energy $U$, i.e., a microcanonical ensemble. Out of all the $2^{N}$ possible states, only a part of them satisfy the condition $U = E_{m} = m \gamma a$. Thus, the number $\Omega$ of possible states is given by:

\begin{equation}
\Omega(U) = C_{m}^{N} = \frac{N!}{m!(N-m)!}
\end{equation}

where $m(U) = U/(\gamma a)$ is a function of the mean energy. Now, the mean energy is no longer a constraint that the probability distribution must satisfy, and thus maximisation of the thermodynamic entropy means that all states are equiprobable:

\begin{equation}
p(U) = \frac{1}{\Omega}
\end{equation}

with the thermodynamic entropy given by:

\begin{equation}
S(U) = k_{B} \ln{\Omega}
\end{equation}

Then, by definition and using the Stirling's formula, the temperature is given by:

\begin{equation}
\begin{split}
\frac{1}{T} &= \pderi{S}{U} \\
            &= \pderi{S}{m} \pderi{m}{U} \\
            &= \frac{k_{B}}{\gamma a} \pderi{}{m} [\ln{N!} - \ln{m!} - ln{(N-m)!}] \\
            &\approx \frac{k_{B}}{\gamma a} \ln{\frac{N-m}{m}} \\
            &= \frac{k_{B}}{\gamma a} \ln{\left( \frac{N \gamma a}{U} - 1 \right)}
\end{split}
\end{equation}

Thus the temperature is given by:

\begin{equation}
T = \frac{\gamma a}{k_{B}} \frac{1}{\ln{\left( \frac{N \gamma a}{U} - 1 \right)}}
\end{equation}

Note that the temperature is positive for $U < N \gamma a/2$ and negative for $U > N \gamma a/2$. A vertical asymptote exists at $U = N \gamma a/2$. 

The dynamical stability arguments do not apply as the model does not take the motion of the chain segments into consideration, which is also the reason why the model differs from the ideal gas model.
\qed


%==========
\pagebreak
\section*{Classical Monatomic Ideal Gas}
%==========


\problem{5.6}{}

\subproblem{a}
Particles in the ideal gas model behave like free particles of mass $m$ in a box of volume $V = L_{x}L_{y}L_{z}$. The single-particle wave function (in position representation)is given by:

\begin{equation}
\psi(x, y, z) = A \sin{(k_{x}x)} \sin{(k_{y}y)} \sin{(k_{z}z)}
\end{equation}

where $A$ is the normalisation constant.

The components of the wave vector $\mathbf{k} = (k_{x}, k_{y}, k_{z})^{T}$ satisfy the condition $k_{i} = 2\pi n_{i}/L_{i}$. On the other hand, the momentum of the particle is given by $\mathbf{P} = \hbar \mathbf{k}$, so the energy levels of a single particle are:

\begin{equation}
E_{k} = \frac{\hbar^{2} k^{2}}{2m}
\end{equation}

Then the single-particle partition function $Z_{1}$ is:

\begin{equation}
\begin{split}
Z_{1} &= \sum_{k} e^{-\beta E_{k}} \\
      &= \sum_{k} e^{-\beta \hbar^{2} k^{2}/2m} \\
      &\approx \frac{V}{(2\pi)^{3}} \int_{0}^{\infty} e^{-\beta \hbar^{2} k^{2}/2m} 4\pi k^{2} \diff{k} \\
      &= \frac{V}{\lambda_{\text{th}}^3}
\end{split}
\end{equation}

where $\lambda_{\text{th}} \equiv \hbar \sqrt{(2\pi)/(m k_{B} T)}$ is the thermal wavelength.

If we assume the particles to be distinguishable, the total partition function $Z$ is simply:

\begin{equation}
Z = Z_{1}^{N} = \left( \frac{V}{\lambda_{\text{th}}^3} \right)^{N}
\end{equation}

so that its log is given by:

\begin{equation}
\ln{Z} = N \left(\ln{V} + \frac{1}{2} \ln{\frac{m}{2\pi \hbar^{2}}} - \frac{1}{2} \ln{\beta}\right)
\end{equation}

This leads to the entropy:

\begin{equation}
\begin{split}
S &= - \left( \pderi{F}{T} \right)_{V} \\
  &= k_{B} \left( \pderi{(T \ln{Z})}{T} \right)_{V} \\
  &= k_{B} \left( \ln{Z} + T \pderi{\ln{Z}}{\beta} \pderi{\beta}{T} \right)_{V} \\
  &= k_{B} N \left( \ln{V} - 3\ln{\lambda_{\text{th}}} + \frac{1}{2} \right) \\
\end{split}
\end{equation}

This expression of entropy is not additive, as if we consider two systems with identical temperature (that sets $\ln{\lambda_{\text{th}}}$ zero), number of particle and volume, then they have the same individual entropy:

\begin{equation}
S_{0} = k_{B} N \left( \ln{V} + \frac{1}{2} \right)
\end{equation}

If we combine the two system, the number of particles and the volume doubles, but the temperature remains the same, and the new entropy is:

\begin{equation}
S_{1} = 2 k_{B} N \left( \ln{2V} + \frac{1}{2} \right) = k_{B} N \left( \ln{4V^2} + 1 \right)
\end{equation}

But this new entropy does not equal the sum of the individual entropies, which is given by:

\begin{equation}
2S_{0} = k_{B} N \left( \ln{V^2} + 1 \right)
\end{equation}

Thus this expression of entropy is non-additive and thus invalid. To account for the indistinguishability of the gas particles, we recognise that the state of the system is given by the set of occupation numbers $\{n_{i}\}$ for all the single-particle states enumerated by $i$. If the numbers of available single-particle states are much greater than the number of particles, then the probability of a particle occupying any single-particle state is very small and the probability of more than one particle occupying the same single-particle state is thus negligible. Therefore, we assume that for any single-particle state, its occupation number is either zero or unity. Thus the new partition function is:

\begin{equation}
Z = \frac{1}{N!} Z_{1}^{N} = \frac{1}{N!} \left( \frac{V}{\lambda_{\text{th}}^3} \right)^{N} 
\end{equation}

This expression of the partition function leads to the entropy of the form:

\begin{equation}
S = k_{B} N \left[ \frac{5}{2} - \ln{(n \lambda_{\text{th}}^3)} \right]
\end{equation}

where $n = N/V$ is the overall particle density.

\subproblem{b}

Given:

\begin{equation}
S = k_{B} N \left[ \frac{5}{2} - \ln{(n \lambda_{\text{th}}^3)} \right]
\end{equation}
\qed


\problem{5.7}{Relativistic Ideal Gas.}

\subproblem{a}
At relativistic speeds, the energy levels of a single particle become:

\begin{equation}
E_{k} = \sqrt{m^{2} c^{4} + \hbar^{2} k^{2} c^{2}} = m c^{2} \sqrt{1 + \frac{\hbar^{2}}{m^{2} c^{2}} k^{2}}
\end{equation}

so that the single-particle partition function $Z_{1}$ now becomes:

\begin{equation}
Z_{1} = \frac{V}{2\pi^{2}} \int_{0}^{\infty} k^{2} e^{-\beta m c^{2} \sqrt{1 + \hbar^{2}k^{2}/m^{2}c^{2}}} \diff{k} = \frac{V}{2\pi^{2}} I(k, T)
\end{equation}

where $I$ represents the integral as a function of $k$ and $T$ only.

The total partition function $Z$ thus is:

\begin{equation}
Z = \frac{1}{N!} Z_{1}^{N} = \frac{1}{N!} \left( \frac{V}{2\pi^{2}} \right)^{N} I^{N}
\end{equation}

Since the equation of state is given by:

\begin{equation}
P = -\left( \pderi{F}{V} \right)_{T} = k_{B} T \left( \pderi{\ln{Z}}{V} \right)_{T}
\end{equation}

The change in the form of the integral does not affect the equation of state, and we still obtain $PV = Nk_{B}T$.

\subproblem{b}
Returning to the integral $I$, at ultra-relativistic limits where $pc = \hbar k c \gg mc^{2}$:

\begin{equation}
\begin{split}
I &= \int_{0}^{\infty} k^{2} e^{-\beta m c^{2} \sqrt{1 + \hbar^{2}k^{2}/m^{2}c^{2}}} \diff{k} \\
  &\approx \int_{0}^{\infty} k^{2} e^{-\beta m c^{2}  (\hbar k/mc)} \diff{k} \\
  &=\int_{0}^{\infty} k^{2} e^{-\beta \hbar k/c} \diff{k} \\
  &= 2 \left( \frac{c}{\beta \hbar} \right)^3
\end{split}
\end{equation}

where the approximation is made using the condition $\hbar k/mc \gg 1$.

Then the log of the total partition function is:

\begin{equation}
\begin{split}
\ln{Z} &= \ln{\left[ \frac{1}{N!} \left( \frac{c^3}{\pi \hbar^{3}} \frac{V}{\beta^3} \right)^{N} \right]} \\
       &\approx N (\ln{V} - \ln{N} + 3\ln{T} + \kappa)
\end{split}
\end{equation}

where Stirling formula has been used and $\kappa$ is a constant.

Then from Question 5.6 Part (a), the entropy is:

\begin{equation}
\begin{split}
S &= k_{B} \left( \ln{Z} + T \pderi{\ln{Z}}{T} \right)_{V} \\
  &= k_{B} N (\ln{V} - \ln{N} + 3\ln{T} + \kappa + 3)
\end{split}
\end{equation}

and thus in an adiabatic process:

\begin{equation}
\diff{S} = \pderi{S}{V} \diff{V} + \pderi{S}{T} \diff{T} = \frac{1}{V} \diff{V} + \frac{3}{T} \diff{T} = 0
\end{equation}

Integrating and using the equation of state leads to the desired relation $PV^{4/3} = \text{constant}$.

\subproblem{c}
The internal energy density is given by:

\begin{equation}
\varepsilon = \frac{U}{V} = - \frac{1}{V} \pderi{\ln{Z}}{\beta} = \frac{1}{V} \frac{3N}{\beta}
\end{equation}

Coupled with the equation of state, we obtain $P = \varepsilon/3$. The ratio depends on the coefficient of $N\ln{\beta}$ in the expression of $\ln{Z}$, and it is the same for both ultra-relativistic and non-relativistic cases. Thus there is no difference.
\qed

\end{document}
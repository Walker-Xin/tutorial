\documentclass[12pt]{article}
\usepackage{homework}
\pagestyle{fancy}

% assignment information
\def\course{Electromagnetism}
\def\assignmentno{Problem Set 1}
\def\assignmentname{Electric Fields, Potentials and the Principle of Superposition}
\def\name{Xin, Wenkang}
\def\time{\today}

\begin{document}

\begin{titlepage}
    \begin{center}
        \large
        \textbf{\course}

        \vfill

        \Huge
        \textbf{\assignmentno}

        \vspace{1.5cm}

        \large{\assignmentname}

        \vfill

        \large
        \name

        \time
    \end{center}
\end{titlepage}


%==========
\pagebreak
\section*{Electric Fields, Potentials and the Principle of Superposition}
%==========


\problem{0}{Background}
An electric field at a point in space generated by a distribution of charges is the force per unit charge that a test charge would experience if placed on that point.

The electric potential at a point in space is the work done per unit charge to move a test charge from a reference point to that point.

We have the relationship between an electric field $\mathbf{E}(\mathbf{r})$ and its electric potential $\phi(\mathbf{r})$:

\begin{equation}
    \mathbf{E}(\mathbf{r}) = -\nabla \phi(\mathbf{r})
\end{equation}

For both the electric field and the electric potential, the net effect of a distribution of charges is the sum of the effects of each individual charge, i.e.:

\begin{equation}
\begin{split}
    \mathbf{E}(\mathbf{r}) &= \sum_{i} \mathbf{E}_{i}(\mathbf{r}) \\
    \phi(\mathbf{r}) &= \sum_{i} \phi_{i}(\mathbf{r})
\end{split}
\end{equation}
\qed


\problem{1}{Assembly of point charges in the corners of a square}

\subproblem{a}
By symmetry, we only consider the electric field along the line AC:

\begin{equation}
    E = \frac{1}{4\pi\epsilon_{0}} \left( \frac{5q}{a^{2}/2} + \frac{q}{a^{2}/2} \right) = \frac{1}{4\pi\epsilon_{0}} \frac{12q}{a^{2}}
\end{equation}

where the direction of the field is from A to C.

\subproblem{b}
The assembly energy is given by:

\begin{equation}
    W = \frac{1}{2} \sum_{i} q_{i} \phi_{i} = -\frac{1}{4\pi\epsilon_{0}} (16 + \sqrt{2}/2)q^{2}
\end{equation}
\qed


\problem{2}{Electric dipole}

\subproblem{a}
Under spherical coordinates, the potential $V$ produced by the dipole is given by:

\begin{equation}
\begin{split}
    V(\mathbf{r}) &= \frac{q}{4\pi\epsilon_{0}} \left( \frac{1}{\left\lvert \mathbf{r} - \mathbf{d}/2 \right\rvert} - \frac{1}{\left\lvert \mathbf{r} + \mathbf{d}/2 \right\rvert}\right) \\
    &= \frac{q}{4\pi\epsilon_{0}} \left[ \left( r^{2} - \mathbf{r} \cdot \mathbf{d} + \frac{d^{2}}{4} \right)^{-1/2} - \left( r^{2} + \mathbf{r} \cdot \mathbf{d} + \frac{d^{2}}{4} \right)^{-1/2} \right] \\
    &\approx \frac{q}{4\pi\epsilon_{0}} \frac{1}{r} \left[ \left( 1 + \frac{\mathbf{r} \cdot \mathbf{d}}{r^{2}} \right) - \left( 1 - \frac{\mathbf{r} \cdot \mathbf{d}}{r^{2}} \right) \right] \\
    &= \frac{\mathbf{p} \cdot \mathbf{r}}{2\pi\epsilon_{0}r^{3}} \\
    &= \frac{p \cos{\theta}}{2\pi\epsilon_{0}r^{2}}
\end{split}
\end{equation}

where $\mathbf{p} = q\mathbf{d}$ is the dipole moment and $\theta$ is the polar angle.

\subproblem{b}
The electric field is given by:

\begin{equation}
\begin{split}
    E &= -\nabla V \\
    &= - \frac{\partial V}{\partial r} \hat{r} - \frac{1}{r} \frac{\partial V}{\partial \theta} \hat{\theta} \\
    &= \frac{p \cos{\theta}}{2\pi\epsilon_{0}r^{3}} \hat{r} + \frac{p \sin{\theta}}{4\pi\epsilon_{0}r^{3}} \hat{\theta}
\end{split}
\end{equation}

\subproblem{c}
The torque experienced by the dipole is $\mathbf{T} = \mathbf{p} \times \mathbf{E}_{ext}$ so that the energy can be defined as the work done against the torque in rotating the dipole from $\pi/2$ to $\alpha$:

\begin{equation}
    W(\alpha) \equiv \int_{\pi/2}^{\alpha} p E_{ext} \sin{\theta} \, \mathrm{d}\theta = -p E_{ext} \cos{\alpha} = - \mathbf{p} \cdot \mathbf{E}_{ext}
\end{equation}
\qed


\problem{3}{Assembly of point charges on a line; multipoles}
Under spherical coordinates, the potential $V$ produced by the dipole is given by:

\begin{equation}
\begin{split}
    V(\mathbf{r}) &= \frac{1}{4\pi\epsilon_{0}} \left( \frac{q_{2}}{r} - \frac{q_{1}}{\left\lvert \mathbf{r} - \mathbf{a} \right\rvert} - \frac{q_{1}}{\left\lvert \mathbf{r} + \mathbf{a} \right\rvert} \right) \\
    &= \frac{q_{2}}{4\pi\epsilon_{0}} \frac{1}{r} - \frac{q_{1}}{4\pi\epsilon_{0}r} \left[ \left( 1 - 2\frac{a}{r} \cos{\theta} + \frac{a^{2}}{r^{2}} \right)^{-1/2} + \left( 1 + 2\frac{a}{r} \cos{\theta} + \frac{a^{2}}{r^{2}} \right)^{-1/2} \right]
\end{split}
\end{equation}

The term $[1 \mp 2 (a/r) \cos{\theta} + (a/r)^{2}]^{-1/2}$ can be treated as a binomial expansion and further expansion of the resulting items can yield the desired result. Alternatively, we note that the term is the generating function of the Legendre polynomials:

\begin{equation}
    \left( 1 \mp 2\frac{a}{r} \cos{\theta} + \frac{a^{2}}{r^{2}} \right)^{-1/2} = \sum_{i=0}^{\infty} P_{i}(\cos{\theta}) \left( \mp \frac{a}{r} \right)^{i}
\end{equation}

where $P_{i}(\cos{\theta})$ is the $i$th Legendre polynomial.

The first three Legendre polynomials are $P_{0}(x) = 1$, $P_{1}(x) = x$ and $P_{2}(x) = 3x^{2}/2 - 1/2$. Expanding the terms up to $P_{2}(\cos{\theta})$ and simplifying:

\begin{equation}
    V(\mathbf{r}) = \frac{q_{2}}{4\pi\epsilon_{0}r} - \frac{q_{1}}{2\pi\epsilon_{0}r} - \frac{q_{1}}{4\pi\epsilon_{0}r} (3\cos^{2}{\theta} - 1) \frac{a^{2}}{r^{2}}
\end{equation}

where the terms are monopole, dipole and quadrupole contributions, respectively.

With $q_{2} = 2q_{1}$, we can write the potential as:

\begin{equation}
    V(\mathbf{r}) = \frac{q_{1}}{4\pi\epsilon_{0}r} (1 - 3\cos^{2}{\theta}) \frac{a^{2}}{r^{2}}
\end{equation}

and the associated electric field is:

\begin{equation}
    \mathbf{E}(\mathbf{r}) = - \nabla V = \frac{3q_{1}a^{2}}{4\pi\epsilon_{0}r^{4}} \left[ (3\cos^{2}{\theta} - 1) \hat{r} + (6\cos{\theta} \sin{\theta}) \hat{\theta} \right]
\end{equation}
\qed


\problem{4}{Uniformly charged rod}

\subproblem{a}
By symmetry, the electric field only has a z-component given by:

\begin{equation}
    \mathbf{E}(z) = \hat{z} \int_{-l}^{l} \frac{1}{4\pi\epsilon_{0}} \frac{\lambda}{(z - h)^{2}} \, \mathrm{d}h = \frac{\lambda}{4\pi\epsilon_{0}} \frac{2l}{z^{2} - l^{2}} \hat{z}
\end{equation}

\subproblem{b}
By symmetry, the electric field only has an x-component given by:

\begin{equation}
    \mathbf{E}(x) = \hat{x} \int_{-l}^{l} \frac{1}{4\pi\epsilon_{0}} \frac{\lambda}{x^{2} + h^{2}} \, \mathrm{d}h = \frac{\lambda}{2\pi\epsilon_{0}x} \tan^{-1}{(h/x)} \hat{x}
\end{equation}
\qed


\problem{5}{Uniformly charged disk}

\subproblem{a}
Considering the contributions of infinitesimal ring elements:

\begin{equation}
\begin{split}
    \mathbf{E}(z) &= \hat{z} \int \mathrm{d}E_{z} = \hat{z} \int_{0}^{b} \frac{1}{4\pi\epsilon_{0}} \frac{\sigma 2\pi r \mathrm{d}r}{r^{2} + z^{2}} \frac{z}{\sqrt{r^{2} + z^{2}}} \\
    &= \hat{z} \frac{\sigma z}{4\epsilon_{0}} \int_{0}^{b} \frac{2r}{(r^{2} + z^{2})^{3/2}} \, \mathrm{d}r \\
    &= \frac{\sigma}{2\epsilon_{0}} \left[ 1 - \left( 1 + \frac{b}{z} \right)^{-1/2} \right] \hat{z}
\end{split}
\end{equation}

\subproblem{b}
In the limit $z \ll b$:

\begin{equation}
    E(z) = \frac{\sigma}{2\epsilon_{0}} \left[ 1 - \sqrt{\frac{z}{b}} \left( 1 + \frac{z}{b} \right)^{-1/2} \right] \approx \frac{\sigma}{2\epsilon_{0}} \left( 1 - \sqrt{\frac{z}{b}} \right)
\end{equation}

The first term is expected as it is the perpendicular component of the boundary electric field close to a thin charge distribution.

In the limit $z \gg b$:

\begin{equation}
    E(z) \approx \frac{\sigma b}{2\epsilon_{0}z}
\end{equation}
\qed


\problem{6}{Uniformly charged sphere}

\subproblem{a}
The electric field along the axis of the thin ring is given by:

\begin{equation}
    E(z) = \frac{q}{4\pi\epsilon_{0}} \frac{z}{(a^{2} + z^{2})^{3/2}}
\end{equation}

For a maximum, we compute the first derivative:

\begin{equation}
    \frac{\mathrm{d}E}{\mathrm{d}z} = \frac{q}{4\pi\epsilon_{0}} \frac{(a^{2} + z^{2})^{3/2} - (a^{2} + z^{2})^{1/2} z^{2}}{(a^{2} + z^{2})^{3/2}} = 0
\end{equation}

The solutions are $z = \pm a/\sqrt{2}$.

\subproblem{b}
The force exerted on an electron is given by:

\begin{equation}
    F(z) = -e E(z) = -\frac{eqz}{4\pi\epsilon_{0} a^{3}} \left[ 1 + \left( \frac{z}{a} \right)^{2} \right]^{-3/2} \approx -\frac{eqz}{4\pi\epsilon_{0} a^{3}}
\end{equation}

if $z \ll a$ and terms beyond the first order are neglected.

Thus the motion is approximately simple harmonic with frequency:

\begin{equation}
    f = \frac{1}{2\pi} \sqrt{\frac{eq}{4\pi\epsilon_{0} a^{3} m}} = \sqrt{\frac{eq}{16\pi^{3} \epsilon_{0} a^{3} m}}
\end{equation}

The oscillation is possible because for th electron, the point $z = 0$ is a stable equilibrium. In fact, any small oscillation around a stable equilibrium is simple harmonic.
\qed


\problem{7}{Uniformly charged hollow sphere}
In spherical coordinates, the infinitesimal area element can be expressed as $\mathrm{d}A = r^{2} \sin{\theta} \mathrm{d}\theta \mathrm{d}\phi$, where $r$ can be treated as a constant $a$. Orienting the axis so that $\mathbf{P}$ lies on the $z$-axis, the potential is given by:

\begin{equation}
\begin{split}
    \phi(\mathbf{P}) &= \iint_{S} \, \mathrm{d}\phi = \int_{0}^{2\pi} \int_{0}^{\pi} \frac{1}{4\pi\epsilon_{0}} \frac{\sigma a^{2} \sin{\theta}}{\sqrt{a^{2} + p^{2} - 2ap\cos{\theta}}} \, \mathrm{d}\theta \mathrm{d}\phi \\
    &= \frac{\sigma a^{2}}{2\epsilon_{0}} \int_{0}^{\pi} \frac{\sin{\theta}}{\sqrt{a^{2} + p^{2} - 2ap\cos{\theta}}} \, \mathrm{d}\theta \\
    &= \frac{\sigma a}{\epsilon_{0}}
\end{split}
\end{equation}

This demonstrates that the potential inside a conductor is constant so that the field is zero.
\qed


\end{document}
\documentclass[12pt]{article}
\usepackage{homework}
\pagestyle{fancy}

% assignment information
\def\course{Mathematical Methods}
\def\assignmentno{Problem Sheet 2}
\def\assignmentname{Fourier Series and Fourier Integrals}
\def\name{Xin, Wenkang}
\def\time{\today}

\begin{document}

\begin{titlepage}
    \begin{center}
        \large
        \textbf{\course}

        \vfill

        \Huge
        \textbf{\assignmentno}

        \vspace{1.5cm}

        \large{\assignmentname}

        \vfill

        \large
        \name

        \time
    \end{center}
\end{titlepage}


%==========
\pagebreak
\section*{Fourier Series and Fourier Integrals}
%==========


\problem{1}{Fourier Series}

\subproblem{a}
We have the coefficients for the cosine series as:

\begin{equation}
    a_{r} = \frac{2}{2\pi} \int_{0}^{\pi} \sin{x} \cos{rx} \, \mathrm{d}x
\end{equation}

where apparently $a_{0} = 2/\pi$.

For $r \ge 1$, the integral denoted as $I_{r}$ can be evaluated by parts:

\begin{equation}
\begin{split}
    I_{r} &= \left[ -\cos{x} \cos{rx} \right]_{0}^{\pi} - r \int_{0}^{\pi} \cos{x} \sin{rx} \, \mathrm{d}x \\
    &= \left[ \cos{x} \cos{rx} \right]_{\pi}^{0} - r \left\{ \left[ \sin{x} \sin{rx} \right]_{0}^{\pi} - r \int_{0}^{\pi} \sin{x} \cos{rx} \, \mathrm{d}x \right\} \\
    &= 1 + \cos{\pi r} + r^{2} I_{r}
\end{split}
\end{equation}

Therefore, the coefficients are:

\begin{equation}
    a_{r} = \frac{1}{\pi} \frac{1 + \cos{\pi r}}{1 - r^{2}} =
    \begin{cases}
        0 & \text{if } r \text{ is odd} \\
        2/\pi(1 - r^{2}) & \text{if } r \text{ is even}
    \end{cases}
\end{equation}

On the other hand, the coefficients for the sine series are:

\begin{equation}
    b_{r} = \frac{2}{2\pi} \int_{0}^{\pi} \sin{x} \sin{rx} \, \mathrm{d}x
\end{equation}

which are zero except for $r = 1$:

\begin{equation}
    b_{1} = \frac{1}{\pi} \int_{0}^{\pi} \sin^{2}{x} \, \mathrm{d}x = \frac{1}{2}
\end{equation}

Hence, the Fourier series is:

\begin{equation}
    f(x) = \frac{1}{2} \sin{x} + \frac{2}{\pi} \sum_{\text{even }r \ge 0}^{\infty} \frac{1}{1 - r^{2}} \cos{rx}
\end{equation}

\subproblem{b}
Since the function is even, we only need the coefficients for the cosine series:

\begin{equation}
\begin{split}
    a_{r} &= \frac{1}{\pi} \int_{-\pi}^{\pi} x^{2} \cos{rx} \, \mathrm{d}x \\
    &= \frac{2}{\pi} \int_{0}^{\pi} x^{2} \cos{rx} \, \mathrm{d}x \\
    &= \frac{2}{\pi} \left\{ \left[ x^{2} \frac{1}{r} \sin{rx} \right]_{0}^{\pi} -  \frac{2}{r} \int_{0}^{\pi} x \sin{rx} \, \mathrm{d}x \right\} \\
    &= -\frac{4}{r\pi} \left\{ \left[ -x \frac{1}{r} \cos{rx} \right]_{0}^{\pi} + \frac{1}{r} \int_{0}^{\pi} \cos{rx} \, \mathrm{d}x \right\} \\
    &= \frac{4}{r^{2}} \cos{r\pi} = \frac{4}{r^{2}} (-1)^{r}
\end{split}
\end{equation}

for $r \ge 1$.

Apparently, $a_{0} = 2\pi^{2}/3$ and the Fourier series is:

\begin{equation}
    f(x) = \frac{\pi^{2}}{3} + \sum_{r = 1}^{\infty} (-1)^{r} \frac{4}{r^{2}} \cos{rx}
\end{equation}

\subproblem{c}
Consider the norm of the function $f(x) = x^{2}$ on the interval $[-\pi, \pi]$:

\begin{equation}
    \norm{f}^{2} = \int_{-\pi}^{\pi} x^{4} \, \mathrm{d}x = \frac{2\pi^{5}}{5}
\end{equation}

By Parseval's equation, we have:

\begin{equation}
    \frac{\norm{f}^{2}}{\pi} = \frac{1}{2} \left( \frac{\pi^{2}}{3} \right)^{2} + \sum_{r = 1}^{\infty} \frac{16}{r^{4}}
\end{equation}

so that:

\begin{equation}
    \sum_{r = 1}^{\infty} \frac{1}{r^{4}} = \frac{1}{16} \left( \frac{2\pi^{4}}{5} - \frac{\pi^{4}}{18} \right) = 
\end{equation}

\problem{2}{Sine and cosine Fourier series}

\subproblem{a}
The coefficients for the cosine series are:

\begin{equation}
    a_{r} = \frac{2}{\pi} \int_{0}^{\pi} x \sin{x} \cos{rx} \, \mathrm{d}x
\end{equation}

For the case $r = 0$ and $r = 1$, we have:

\begin{equation}
    a_{0} = \frac{2}{\pi} \int_{0}^{\pi} x \sin{x} \, \mathrm{d}x = \frac{2}{\pi} \left\{ \left[ -x \cos{x} \right]_{0}^{\pi} + \int_{0}^{\pi} \cos{x} \, \mathrm{d}x \right\} = 2
\end{equation}

and:

\begin{equation}
    a_{1} = \frac{2}{\pi} \int_{0}^{\pi} x \sin{x} \cos{x} \, \mathrm{d}x = \frac{1}{\pi} \int_{0}^{\pi} x \sin{2x} \, \mathrm{d}x = \frac{1}{2\pi} \left\{ \left[ -x \cos{2x} \right]_{0}^{\pi} + \int_{0}^{\pi} \cos{2x} \, \mathrm{d}x \right\} = -\frac{1}{2}
\end{equation}

For $r \ge 2$, the integral, which is denoted as $I_{r}$, can be evaluated by parts:

\begin{equation}
    I_{r} = \left[ -x \cos{x} \cos{rx} \right]_{0}^{\pi} + \int_{0}^{\pi} \cos{x} \cos{rx} \, \mathrm{d}x - r \int_{0}^{\pi} x \cos{x} \sin{rx} \, \mathrm{d}x
\end{equation}

For the middle term, the only non-zero contribution is when $r = 1$ as $\cos{rx}$ are orthogonal. Therefore we may neglect the middle term for $r \ge 2$:

\begin{equation}
\begin{split}
    I_{r} &= \pi \cos{r\pi} - r \left\{ \left[ x \sin{x} \sin{rx} \right]_{0}^{\pi} - \int_{0}^{\pi} \sin{x} \sin{rx} \, \mathrm{d}x - r \int_{0}^{\pi} x \sin{x} \cos{rx} \, \mathrm{d}x \right\} \\
    &= \pi \cos{r\pi} + r^{2} I_{r}
\end{split}
\end{equation}

where in the last step we have again used the orthogonality of $\sin{rx}$.

This means that for $r \ge 2$, the coefficients are:

\begin{equation}
    a_{r} = (-1)^{r} \frac{2}{1 - r^{2}}
\end{equation}

Hence the cosine Fourier series is:

\begin{equation}
    f(x) = 2 - \frac{1}{2} \cos{x} + \sum_{r = 2}^{\infty} (-1)^{r} \frac{2}{1 - r^{2}} \cos{rx}
\end{equation}

\subproblem{b}
The coefficients for the sine series are:

\begin{equation}
    b_{r} = \frac{2}{\pi} \int_{0}^{\pi} x \sin{x} \sin{rx} \, \mathrm{d}x
\end{equation}

where $b_{1} = \pi/2$.

For $r \ge 2$, the integral, which is denoted as $I_{r}$, can be evaluated as:

\begin{equation}
\begin{split}
    I_{r} &= \frac{1}{2} \int_{0}^{\pi} x \cos{(1 + r)x} \, \mathrm{d}x - \frac{1}{2} \int_{0}^{\pi} x \cos{(1 - r)x} \, \mathrm{d}x \\
    &= \frac{1}{2} \left[ \frac{\cos{(1 + r)\pi} - 1}{(1 + r)^{2}} - \frac{\cos{(1 - r)\pi} - 1}{(1 - r)^{2}} \right] \\
\end{split}
\end{equation}

where we used the integral result:

\begin{equation}
    \int_{0}^{\pi} x \cos{kx} \, \mathrm{d}x = \frac{\cos{k\pi} - 1}{k^{2}}
\end{equation}

This means that the coefficients are:

\begin{equation}
    b_{r} =
    \begin{cases}
        0 & \text{if } r \text{ is odd} \\
        2 \left[ (1 + r)^{-2} - (1 - r)^{-2} \right]/\pi & \text{if } r \text{ is even}
    \end{cases}
\end{equation}

Hence the sine Fourier series is:

\begin{equation}
    f(x) = \frac{\pi}{2} \sin{x} + \sum_{\text{even }r \ge 2}^{\infty} \frac{2}{\pi} \left[ \frac{1}{(1 + r)^{2}} - \frac{1}{(1 - r)^{2}} \right] \sin{rx}
\end{equation}

\subproblem{c}
The coefficients for the cosine series are:

\begin{equation}
    a_{r} = \frac{2}{\pi} \int_{0}^{\pi} x \cos{rx} \, \mathrm{d}x = \frac{2}{\pi} \frac{\cos{k\pi} - 1}{k^{2}} =
    \begin{cases}
        -4/\pi r^{2} & \text{if } r \text{ is odd} \\
        0 & \text{if } r \text{ is even}
    \end{cases}
\end{equation}

except for $r = 0$ where $a_{0} = \pi$.

Hence the cosine Fourier series is:

\begin{equation}
    f(x) = \pi - \frac{4}{\pi} \sum_{\text{odd }r \ge 1}^{\infty} \frac{1}{r^{2}} \cos{rx}
\end{equation}

\subproblem{d}
The coefficients for the sine series are:

\begin{equation}
    b_{r} = \frac{2}{\pi} \int_{0}^{\pi} x \sin{rx} \, \mathrm{d}x = -\frac{2}{\pi} \frac{\pi \cos{r\pi}}{r} =
    \begin{cases}
        2/r & \text{if } r \text{ is odd} \\
        -2/r & \text{if } r \text{ is even}
    \end{cases}
\end{equation}

Hence the sine Fourier series is:

\begin{equation}
    f(x) = \sum_{r = 1}^{\infty} (-1)^{r + 1} \frac{2}{r} \sin{rx}
\end{equation}

The cosine series does not converge to $f(x)$ near zero because $f(x)$ is not even, whereas the sine series converges to zero.

Consider the norm of the function $f(x) = x$ on the interval $[0, \pi]$:

\begin{equation}
    \norm{f}^{2} = \int_{0}^{\pi} x^{2} \, \mathrm{d}x = \frac{\pi^{3}}{3}
\end{equation}

The Parseval's equation gives:

\begin{equation}
    \frac{2\norm{f}^{2}}{\pi} = \sum_{r = 1}^{\infty} \frac{4}{r^{2}}
\end{equation}

so that:

\begin{equation}
    \sum_{r = 1}^{\infty} \frac{1}{r^{2}} = \frac{\pi^{2}}{6}
\end{equation}
\qed


\problem{3}{Legendre polynomials as an orthogonal basis}

\subproblem{a}
We know that the Legendre polynomials can be defined as:

\begin{equation}
    P_{l}(x) = \frac{1}{2^{l} l!} \frac{\mathrm{d}^{l}}{\mathrm{d}x^{l}} (x^{2} - 1)^{l}
\end{equation}

Consider the inner product $\left\langle P_{n}, x^{k} \right\rangle$ for $k < n$:

\begin{equation}
\begin{split}
    \left\langle P_{n}, x^{k} \right\rangle &\propto \int_{-1}^{1} x^{k} \frac{\mathrm{d}^{l}}{\mathrm{d}x^{l}} (x^{2} - 1)^{l} \, \mathrm{d}x \\
    &= \left[ x^{k} \frac{\mathrm{d}^{l - 1}}{\mathrm{d}x^{l - 1}} (x^{2} - 1)^{l} \right]_{-1}^{1} - \int_{-1}^{1} k x^{k - 1} \frac{\mathrm{d}^{l - 1}}{\mathrm{d}x^{l - 1}} (x^{2} - 1)^{l} \, \mathrm{d}x \\
    &= \dots \\
    &\propto (-1)^{k} \int_{-1}^{1} \frac{\mathrm{d}^{l - k}}{\mathrm{d}x^{l - k}} (x^{2} - 1)^{l} \, \mathrm{d}x \\
    &= 0
\end{split}
\end{equation}

where we have used the fact that $k < n$ and $l - k \ge 0$.

Any polynomial $p$ of degree $k < n$ can be written as $p(x) = \sum_{r = 0}^{k} a_{r} x^{r}$, so that:

\begin{equation}
    \left\langle P_{n}, p \right\rangle = \sum_{r = 0}^{k} a_{r} \left\langle P_{n}, x^{r} \right\rangle = 0
\end{equation}

\subproblem{b}
Using the above definition, the first five Legendre polynomials are:

\begin{equation}
\begin{split}
    P_{0} &= 1 \\
    P_{1} &= x \\
    P_{2} &= \frac{1}{2} (3x^{2} - 1) \\
    P_{3} &= \frac{1}{2} (5x^{3} - 3x) \\
    P_{4} &= \frac{1}{8} (35x^{4} - 30x^{2} + 3)
\end{split}
\end{equation}

\subproblem{c}
Decomposing the function $f(x) = x^{4}$ in terms of the Legendre polynomials:

\begin{equation}
\begin{split}
    a_{0} &= \left\langle P_{0}, f \right\rangle = \int_{-1}^{1} x^{4} \, \mathrm{d}x = \frac{2}{5} \\
    a_{1} &= \left\langle P_{1}, f \right\rangle = \int_{-1}^{1} x^{5} \, \mathrm{d}x = 0 \\
    a_{2} &= \left\langle P_{2}, f \right\rangle = \int_{-1}^{1} \frac{3x^{6} - x^{4}}{2} \, \mathrm{d}x = \frac{1}{35} \\
    a_{3} &= \left\langle P_{3}, f \right\rangle = \int_{-1}^{1} \frac{5x^{7} - 3x^{5}}{2} \, \mathrm{d}x = 0 \\
    a_{4} &= \left\langle P_{4}, f \right\rangle = \int_{-1}^{1} \frac{35x^{8} - 30x^{6} + 3x^{4}}{8} \, \mathrm{d}x = {}
\end{split}
\end{equation}

\subproblem{d}
The given function can be written as a generating function of the Legendre polynomials:

\begin{equation}
    f(x) = \frac{1}{\sqrt{(1/2)^2 - x + 1}} = \sum_{r = 0}^{\infty} P_{r}(x) \left( \frac{1}{2} \right)^{r}
\end{equation}
\qed


\problem{4}{Examples of Fourier transforms}

\subproblem{a}
For the given square pulse, its Fourier transform is:

\begin{equation}
    \hat{\chi}(k) = \frac{1}{\sqrt{2\pi}} \int_{-1}^{1} e^{-ikx} \, \mathrm{d}x = \frac{1}{\sqrt{2\pi}} \frac{e^{-ik} - e^{ik}}{-ik} = \sqrt{\frac{2}{\pi}} \frac{\sin{k}}{k}
\end{equation}

\subproblem{c}
The convolution is:

\begin{equation}
    f(\mu) = \chi \ast \chi = \int_{\mathbb{R}} \chi(x) \chi(\mu - x) \, \mathrm{d}x
\end{equation}

But $\chi$ is symmetric about the origin and $\chi(\mu - x) = \chi(x - \mu)$, so that:

\begin{equation}
    f(\mu) = \int_{-1}^{1} \chi(x) \chi(x - \mu) \, \mathrm{d}x
\end{equation}

where the range of integration is restricted to $[-1, 1]$ because $\chi(x) = 0$ otherwise.

Consider the case $0 < \mu < 2$, the integrand is non-zero only when $x \in [\mu - 1, 1]$. Therefore:

\begin{equation}
    f(\mu) = \int_{\mu - 1}^{1} \, \mathrm{d}x = 2 - \mu
\end{equation}

Similarly, for $-2 < \mu \le 0$:

\begin{equation}
    f(\mu) = \int_{-1}^{\mu + 1} \, \mathrm{d}x = \mu + 2
\end{equation}

Thus, the convolution is:

\begin{equation}
    f(\mu) =
    \begin{cases}
        2 - \mu & \text{if } 0 < \mu < 2 \\
        \mu + 2 & \text{if } -2 < \mu \le 0 \\
        0 & \text{otherwise}
    \end{cases}
\end{equation}

which is a triangular pulse.

\subproblem{d}
The Fourier transform of the convolution is:

\begin{equation}
\begin{split}
    \hat{f}(k) &= \frac{1}{\sqrt{2\pi}} \int_{-2}^{2} (\mu + 2) e^{-ik\mu} \, \mathrm{d}\mu + \frac{1}{\sqrt{2\pi}} \int_{0}^{2} (2 - \mu) e^{-ik\mu} \, \mathrm{d}\mu \\
    &= \frac{2}{\sqrt{2\pi}} \int_{-2}^{2} (\mu + 2) \cos{(k\mu)} \, \mathrm{d}\mu \\
    &= \frac{4}{\sqrt{2\pi}} \frac{\sin^{2}{k}}{k^{2}} \\
    &= \sqrt{2\pi} \left[ \hat{\chi}(k) \right]^{2}
\end{split}
\end{equation}

as expected.

\subproblem{e}
Consider the inner product $\left\langle \mathcal{F}(f), \mathcal{F}(f) \right\rangle$:

\begin{equation}
\begin{split}
    \left\langle \mathcal{F}(f), \mathcal{F}(f) \right\rangle &= \left\langle \mathcal{F}^{\dagger}[\mathcal{F}(f)], f \right\rangle \\
    &= \left\langle f, f \right\rangle \\
    &= \int_{\mathbb{R}} f(x) f(x) \, \mathrm{d}x \\
    &= \int_{-2}^{0} (2 + \mu)^{2} \, \mathrm{d}\mu + \int_{0}^{2} (2 - \mu)^{2} \, \mathrm{d}\mu \\
    &= \frac{16}{3}
\end{split}
\end{equation}

where the second step is due to the fact that $\mathcal{F}$ is unitary.

On the other hand, the norm of $\mathcal{F}(f)$ is just:

\begin{equation}
    \norm{\mathcal{F}(f)}^{2} = \frac{8}{\pi} \int_{\mathbb{R}} \frac{\sin^{4}{k}}{k^{4}} \, \mathrm{d}k
\end{equation}

so that we have the result:

\begin{equation}
    \int_{\mathbb{R}} \frac{\sin^{4}{k}}{k^{4}} \, \mathrm{d}k = \frac{2\pi}{3}
\end{equation}
\qed


\problem{5}{Some properties of Fourier transforms}

\subproblem{a}
Consider $\mathcal{F} \circ T_{a}$ acting on $f$:

\begin{equation}
\begin{split}
    \mathcal{F} \circ T_{a}(f) &= \mathcal{F} [f(x - a)] \\
    &= \frac{1}{\sqrt{2\pi}} \int_{\mathbb{R}} f(x - a) e^{-ikx} \, \mathrm{d}x \\
    &= \frac{1}{\sqrt{2\pi}} \int_{\mathbb{R}} f(y) e^{-ik(y + a)} \, \mathrm{d}y \\
    &= e^{-ika} \mathcal{F}(f) \\
    &= E_{-a} \mathcal{F}(f)
\end{split}
\end{equation}

where we have used the substitution $y = x - a$ in the third step.

Further consider $\mathcal{F} \circ E_{a}$:

\begin{equation}
\begin{split}
    \mathcal{F} \circ E_{a}(f) &= \frac{1}{\sqrt{2\pi}} \int_{\mathbb{R}} e^{iax} f(x) e^{-ikx} \, \mathrm{d}x \\
    &= \frac{1}{\sqrt{2\pi}} \int_{\mathbb{R}} f(x) e^{-i(k - a)x} \, \mathrm{d}x \\
    &= T_{a}(\hat{f})(k) \\
    &= T_{a} \circ \mathcal{F}(f)
\end{split}
\end{equation}

\subproblem{b}
Consider $\mathcal{F} \circ \mathcal{D}_{a}$ acting on $f$:

\begin{equation}
\begin{split}
    \mathcal{F} \circ \mathcal{D}_{a}(f) &= \frac{1}{\sqrt{2\pi}} \int_{\mathbb{R}} f(ax) e^{-ikx} \, \mathrm{d}x \\
    &= \frac{1}{\sqrt{2\pi}} \frac{1}{a} \int_{\mathbb{R}} f(y) e^{-iky/a} \, \mathrm{d}y \\
    &= \frac{1}{a} \hat{f}(k/a) \\
    &= \frac{1}{a} \mathcal{D}_{1/a} \circ \mathcal{F}(f)
\end{split}
\end{equation}

\subproblem{c}
Consider $\mathcal{F} \circ D_{x}$ acting on $f$:

\begin{equation}
\begin{split}
    \mathcal{F} \circ D_{x}(f) &= \frac{1}{\sqrt{2\pi}} \int_{\mathbb{R}} \frac{\mathrm{d}f}{\mathrm{d}x} e^{-ikx} \, \mathrm{d}x \\
    &= \frac{1}{\sqrt{2\pi}} \left[ f(x) e^{-ikx} \right]_{-\infty}^{\infty} + \frac{ik}{\sqrt{2\pi}} \int_{\mathbb{R}} f(x) e^{-ikx} \, \mathrm{d}x \\
    &= M_{ik} \circ \mathcal{F}(f)
\end{split}
\end{equation}

where the boundary term vanishes because $f(x) \to 0$ as $x \to \pm \infty$ for $f$ to be bounded.

Further consider $iD_{k} \circ \mathcal{F}(f)$:

\begin{equation}
\begin{split}
    iD_{k} \circ \mathcal{F}(f) &= \frac{i}{\sqrt{2\pi}} \frac{\mathrm{d}}{\mathrm{d}k} \left[ \int_{\mathbb{R}} f(x) e^{-ikx} \, \mathrm{d}x \right] \\
    &= \frac{i}{\sqrt{2\pi}} \int_{\mathbb{R}} f(x) \frac{\mathrm{d}}{\mathrm{d}k} e^{-ikx} \, \mathrm{d}x \\
    &= \frac{i}{\sqrt{2\pi}} \int_{\mathbb{R}} f(x) (-ix) e^{-ikx} \, \mathrm{d}x \\
    &= M_{x} \circ \mathcal{F}(f)
\end{split}
\end{equation}
\qed


\problem{6}{More Fourier transforms}

\subproblem{a}
Consider the Fourier transform of the standard Gaussian:

\begin{equation}
\begin{split}
    \hat{f}(k) &= \frac{1}{\sqrt{2\pi}} \int_{\mathbb{R}} e^{-(x^{2}/2 + ikx)} \, \mathrm{d}x \\
    &= \frac{1}{\sqrt{2\pi}} \int_{\mathbb{R}} e^{-((x + ik)^{2} - k^{2})/2} \, \mathrm{d}x \\
    &= \frac{1}{\sqrt{2\pi}} e^{-k^{2}/2} \int_{\mathbb{R}} e^{-(x + ik)^{2}/2} \, \mathrm{d}x \\
    &= e^{-k^{2}/2}
\end{split}
\end{equation}

so that the standard Gaussian is invariant under Fourier transform.

\subproblem{b}
We have $\mathcal{D}_{a}(f_{a})(x) = \exp(-x^{2}/2)$ and:

\begin{equation}
    \mathcal{F} \circ \mathcal{D}_{a}(f_{a}) = \frac{1}{a} \mathcal{D}_{1/a} \circ \mathcal{F}(f_{a})
\end{equation}

But we already established that $\mathcal{F}(f_{a}(ax)) = e^{-k^{2}/2}$, so that:

\begin{equation}
    \mathcal{F}(f_{a})(k) = \mathcal{D}_{a} \left( a e^{-k^{2}/2} \right) = a e^{-a^{2}k^{2}/2}
\end{equation}

\subproblem{c}
We have $\mathcal{D}_{a} \circ T_{-c}(f_{a})(x) = \exp(-x^{2}/2)$ and:

\begin{equation}
    \mathcal{F} \circ \mathcal{D}_{a} \circ T_{-c}(f_{a}) = \frac{1}{a} \mathcal{D}_{1/a} \circ \mathcal{F} \circ T_{-c}(f_{a}) = \frac{1}{a} \mathcal{D}_{1/a} \circ E_{c} \circ \mathcal{F}(f_{a})
\end{equation}

Therefore:

\begin{equation}
    \mathcal{F}(f_{a})(k) = a e^{-a^{2}k^{2}/2 + ick}
\end{equation}
\qed


\problem{7}{Hermite polynomials and Fourier transform}

\subproblem{a}
We can rewrite the given Gaussian:

\begin{equation}
    g(x) = \exp\left[ -\left( \frac{x}{\sqrt{2}} - \sqrt{2}z \right)^{2} + z^{2} \right] = \exp\left( z^{2} \right) \exp\left[ -\frac{(x - \sqrt{2}z)^{2}}{2} \right]
\end{equation}

so that its Fourier transform is:

\begin{equation}
\begin{split}
    \hat{g}(k) &= \exp\left( z^{2} \right) \mathcal{F}(f_{2, \sqrt{2}z}) \\
    &= \exp\left( -\frac{k^{2}}{2} - 2ikz + z^{2} \right)
\end{split}
\end{equation}

where we have used the result from the previous problem.

\subproblem{b}
We know the generating function of the Hermite polynomials:

\begin{equation}
    \exp(2yz - z^{2}) = \sum_{n = 0}^{\infty} H_{n}(y) \frac{z^{n}}{n!}
\end{equation}

Consider the Fourier transform of $h_{n}(x) = \exp(-x^{2}/2) H_{n}(x)$:

\begin{equation}
\begin{split}
    \mathcal{F}(h_{n})(k) &= \frac{1}{\sqrt{2\pi}} \int_{\mathbb{R}} \exp\left( -\frac{x^{2}}{2} -ikx \right) H_{n}(x) \, \mathrm{d}x \\
    &= \frac{1}{\sqrt{2\pi}} \int_{\mathbb{R}} \exp\left( \frac{x^{2}}{2} -ikx \right) \exp(-x^{2}) H_{n}(x) \, \mathrm{d}x \\
    &= \frac{1}{\sqrt{2\pi}} \int_{\mathbb{R}} \left[ \sum_{j}^{\infty} H_{j}(-ik/\sqrt{2}) \frac{(x/\sqrt{2})^{j}}{j!} \right] \exp(-x^{2}) H_{n}(x) \, \mathrm{d}x \\
    &= \sum_{j}^{\infty} H_{j}(-ik/\sqrt{2}) \frac{1}{j! 2^{j/2}} \frac{1}{\sqrt{2\pi}} \int_{\mathbb{R}} \exp(-x^{2}) H_{n}(x) x^{j} \, \mathrm{d}x
\end{split}
\end{equation}

Consider the integral term. We can always write $x^{j}$ as a linear combination of $H_{n}(x)$:

\begin{equation}
    x^{j} = \sum_{r = 0}^{j} a_{r} H_{r}(x)
\end{equation}

so that:

\begin{equation}
    \int_{\mathbb{R}} \exp(-x^{2}) H_{n}(x) x^{j} \, \mathrm{d}x = \sum_{r = 0}^{j} a_{r} \int_{\mathbb{R}} \exp(-x^{2}) H_{n}(x) H_{r}(x) \, \mathrm{d}x = a_{n} \sqrt{\pi} 2^{n} n!
\end{equation}

where $a_{n} = \left\langle H_{n}, x^{j} \right\rangle$.


\end{document}
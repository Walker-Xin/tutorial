\documentclass[12pt]{article}
\usepackage{homework}
\pagestyle{fancy}

% assignment information
\def\course{Quantum Mechanics}
\def\assignmentno{Problem Sheet 3}
\def\assignmentname{The Simple Harmonic Oscillator \& Problems on Basic Quantum Mechanics}
\def\name{Xin, Wenkang}
\def\time{\today}

\begin{document}

\begin{titlepage}
    \begin{center}
        \large
        \textbf{\course}

        \vfill

        \Huge
        \textbf{\assignmentno}

        \vspace{1.5cm}

        \large{\assignmentname}

        \vfill

        \large
        \name

        \time
    \end{center}
\end{titlepage}


%==========
\pagebreak
\section*{The simple harmonic oscillator}
%==========


\problem{3.1}{}
Given that $\hat{H} = (\hat{p}^2 + \hat{x}^2)/2$ and $[\hat{x}, \hat{p}] = i$ and an energy eigenstate $\ket{\psi}$ with energy $E$, we have:

\begin{equation}
\begin{split}
    \hat{H} (\hat{x} \mp i \hat{p}) \ket{\psi} &= \frac{1}{2} (\hat{p}^2 + \hat{x}^2) (\hat{x} \mp i \hat{p}) \ket{\psi} \\
    &= \frac{1}{2} (\hat{p}^2 \hat{x} \mp i \hat{p}^3 + \hat{x}^3 \mp i \hat{x} \hat{p}^2) \ket{\psi} \\
    &= \frac{1}{2} [\hat{p} (\hat{x} \hat{p} - [\hat{x}, \hat{p}]) \mp i \hat{p}^3 + \hat{x}^3 \mp i (\hat{p} \hat{x} + [\hat{x}, \hat{p}]) \hat{p}] \ket{\psi} \\
    &= \frac{1}{2} [\hat{p} \hat{x} \hat{p} - i\hat{p} \mp i \hat{p}^3 + \hat{x}^3 \mp i \hat{p} \hat{x} \hat{p} \pm  \hat{p}^2] \ket{\psi} \\
\end{split}
\end{equation}


\problem{3.2}{}
Consider the Hermitian conjugate of the annihilation operator $\hat{a}$:

\begin{equation}
\hat{a}^{\dagger} = \frac{m\omega \hat{x} - i \hat{p}}{\sqrt{2m\omega \hbar}}
\end{equation}

We have:

\begin{equation}
    \hat{a}^{\dagger} \hat{a} = \frac{m\omega \hat{x} - i \hat{p}}{\sqrt{2m\omega \hbar}} \frac{m\omega \hat{x} + i \hat{p}}{\sqrt{2m\omega \hbar}} = \frac{m^2 \omega^2 \hat{x}^2 + \hat{p}^2}{2m\omega \hbar} + \frac{i}{2\hbar} [\hat{x}, \hat{p}] = \hat{H}/\hbar \omega - 1/2
\end{equation}

This allows us to calculate:

\begin{equation}
    \left\lvert \hat{a}\ket{n} \right\rvert = \mel{n}{\hat{a}^{\dagger} \hat{a}}{n} = \frac{E_{n}}{\hbar \omega} - \frac{1}{2} = n
\end{equation}

On the other hand, this is just $\left\lvert \alpha \ket{n - 1} \right\rvert^{2} = \alpha^{2}$, which gives us $\alpha = \sqrt{n}$.
\qed


\problem{3.3}{}
The pendulum follows an approximately harmonic potential of the form:

\begin{equation}
    V(x) = \frac{1}{2} m \omega^{2} x^{2}
\end{equation}

Given that $A = \qty{3}{cm}$, we require:

\begin{equation}
    \left( n + \frac{1}{2} \right) \hbar \omega = \frac{1}{2} m \omega^{2} A^{2}
\end{equation}

solving which gives the enormous energy level $n = \qty{5.5e30}{}$.
\qed


\problem{3.4}{}
We minimise the function $E(p, x) = p^{2}/2m + m\omega^{2} x^{2}/2$ under the constraint $xp = \hbar/2$. Consider the function $f(p, x, \lambda) = E(p, x) + \lambda (xp - \hbar/2)$, we have need:

\begin{equation}
\begin{split}
    \frac{\partial f}{\partial p} = \frac{p}{m} + \lambda x &= 0 \\
    \frac{\partial f}{\partial x} = m\omega^{2} x + \lambda p &= 0 \\
    \frac{\partial f}{\partial \lambda} = xp - \frac{\hbar}{2} &= 0
\end{split}
\end{equation}

Solving which gives us $E_{\text{min}} = \hbar \omega / 2$, which is indeed the ground state energy of the harmonic oscillator.
\qed


\problem{3.5}{}
The position representation of the $n$-th energy eigenstate of the harmonic oscillator is given by:

\begin{equation}
    \psi_{n}(x) = A_{n} H_{n}(\xi) e^{-\xi^{2}/2}
\end{equation}

where $\xi = \sqrt{m\omega/\hbar} x$ and $A_{n}$ is a normalisation constant.

The nodes of the function are due to the Hermite polynomial $H_{n}(\xi)$, which is of degree $n$. By the fundamental theorem of algebra, it has $n$ roots, which are the nodes of the wave function.
\qed


\problem{3.6}{}
The ground state wave function of the harmonic oscillator is given by:

\begin{equation}
    \psi_{0}(x) = A_{0} e^{-\xi^{2}/2}
\end{equation}

To obtain wave functions of higher energy states, we can apply the raising operator $\hat{a}^{\dagger}$ to the ground state wave function. Consider:

\begin{equation}
\begin{split}
    \mel{x}{\hat{a}^{\dagger}}{0} &= \frac{1}{\sqrt{2m\omega \hbar}} \mel{x}{m\omega \hat{x} - i \hat{p}}{0} \\
    &= \left( \frac{x}{2l} - l \frac{\mathrm{d}}{\mathrm{d}x} \right) \psi_{0}(x) 
\end{split}
\end{equation}

or, raising the state again:

\begin{equation}
    \psi_{2} = \left( \frac{x}{2l} - l \frac{\mathrm{d}}{\mathrm{d}x} \right)^{2} \psi_{0}(x)
\end{equation}
\qed


\problem{3.7}{}
Consider the matrix element $\hat{x}_{jk} \equiv \mel{j}{\hat{x}}{k}$:

\begin{equation}
\begin{split}
    \hat{x}_{jk} &= \mel{j}{\hat{x}}{k} \\
    &= l \mel{j}{\hat{a}^{\dagger} + \hat{a}}{k} \\
    &= l (\sqrt{j}\braket{j}{k - 1} + \sqrt{j + 1} \braket{j}{k + 1}) \\
    &= l (\delta_{j, k - 1} \sqrt{j} + \delta_{j, k + 1} \sqrt{j + 1})
\end{split}
\end{equation}

Thus, $\hat{x}_{jk}$ is non-zero only when $k = j \pm 1$, i,e., $\hat{x}$ is a tridiagonal matrix with the diagonal elements being zero.

For $\hat{p}$, we have the identity:

\begin{equation}
    \hat{p} = i \sqrt{\frac{m\omega \hbar}{2}} (\hat{a}^{\dagger} - \hat{a})
\end{equation}

which gives us:

\begin{equation}
\begin{split}
    \hat{p}_{jk} &= \mel{j}{\hat{p}}{k} \\
    &= i \sqrt{\frac{m\omega \hbar}{2}} (\sqrt{j + 1} \braket{j}{k + 1} - \sqrt{j} \braket{j}{k - 1}) \\
    &= i \sqrt{\frac{m\omega \hbar}{2}} (\delta_{j, k + 1} \sqrt{j + 1} - \delta_{j, k - 1} \sqrt{j})
\end{split}
\end{equation}

which is also a tridiagonal matrix with the upper 'diagonal' elements switching their signs.
\qed


\problem{3.8}{}
Since $\hat{x}$ and $\hat{H}$ commute and the Hamiltonian of a harmonic oscillator is time independent, we have by Ehrenfest's theorem that the expectation value of $\hat{x}$ is time independent. We can evaluate the ket $\hat{x} \ket{\psi}$:

\begin{equation}
\begin{split}
    \hat{x} \ket{\psi} &= l(\hat{a} + \hat{a}^{\dagger}) \left( \frac{1}{2} \ket{N - 1} + \frac{1}{\sqrt{2}} \ket{N} + \frac{1}{2} \ket{N + 1} \right) \\
    &= l \left( \frac{1}{2} \sqrt{N} \ket{N} + \frac{1}{\sqrt{2}} \sqrt{N} \ket{N - 1} + \frac{1}{\sqrt{2}} \sqrt{N + 1} \ket{N + 1} + \frac{1}{2} \sqrt{N + 1} \ket{N} \right) \\
\end{split}
\end{equation}

where we have ignored $\ket{N - 2}$ and $\ket{N + 2}$ since they are orthogonal to $\ket{\psi}$.

We thus have the expectation value of $\hat{x}$:

\begin{equation}
    \mel{\psi}{\hat{x}}{\psi} = \frac{l}{\sqrt{2}} (\sqrt{N} + \sqrt{N + 1})
\end{equation}

where $l = \sqrt{\hbar/2m\omega}$.

This shows that while the position expectation of a single `pure' state $\ket{N}$ is zero, that of a mixed state is not.
\qed


%==========
\pagebreak
\section*{Problems on basic quantum mechanics}
%==========


\problem{3.9}{}
$\hat{H}$ is obviously Hermitian since its complex conjugate is itself. $\hat{B}$ is not for the same reason. 

Apparently the eigenvalues of $\hat{H}$ are $\hbar \omega$ and $-\hbar \omega$, with the former having the eigenstate $\ket{1}$ and the latter (degenerate) corresponding to $\ket{2}$ and $\ket{3}$. It is trivial to show that the eigenvalues of $\hat{B}$ are $1$ and $-1$. The former has the eigenstate $\ket{1}$ and $\ket{2} + \ket{3}$, while the latter has the eigenstate $\ket{2} - \ket{3}$.

Both $\hat{H}$ and $\hat{B}$ have degenerate eigenvalues so they cannot uniquely specify the eigenstates. Consider the commutator $[\hat{H}, \hat{B}]$:

\begin{equation}
\begin{split}
    [\hat{H}, \hat{B}] &= \hat{H} \hat{B} - \hat{B} \hat{H} \\
    &= 
    \hbar \omega b
    \begin{pmatrix}
        1 & 0 & 0 \\
        0 & -1 & 0 \\
        0 & 0 & -1
    \end{pmatrix}
    \begin{pmatrix}
        1 & 0 & 0 \\
        0 & 0 & 1 \\
        0 & 1 & 0
    \end{pmatrix}
    -
    \hbar \omega b
    \begin{pmatrix}
        1 & 0 & 0 \\
        0 & 0 & 1 \\
        0 & 1 & 0
    \end{pmatrix}
    \begin{pmatrix}
        1 & 0 & 0 \\
        0 & -1 & 0 \\
        0 & 0 & -1
    \end{pmatrix} \\
    &= \hbar \omega b
    \begin{pmatrix}
        1 & 0 & 0 \\
        0 & 0 & -1 \\
        0 & -1 & 0
    \end{pmatrix}
    -
    \hbar \omega b
    \begin{pmatrix}
        1 & 0 & 0 \\
        0 & 0 & -1 \\
        0 & -1 & 0
    \end{pmatrix} \\
    &= 0    
\end{split}
\end{equation}

Since $[\hat{H}, \hat{B}] = 0$, the two operators share a common set of eigenstates. It is easy to see that the eigenstates of $\hat{B}$ are just linear combinations of those of $\hat{H}$, so we choose $\ket{1}$, $\ket{2} + \ket{3}$ and $\ket{2} - \ket{3}$ as the shared eigenstates.
\qed


\problem{3.10}{}
By Erfhenfest's theorem, we have the time derivative of the expectation value of an operator $\hat{A}$:

\begin{equation}
    \frac{\mathrm{d}}{\mathrm{d}t} \mel{\psi}{\hat{A}}{\psi} = \frac{i}{\hbar} \mel{\psi}{[\hat{H}, \hat{A}]}{\psi} + \mel{\psi}{\frac{\partial \hat{A}}{\partial t}}{\psi}
\end{equation}

The second term is zero for a time-independent operator. The probability of measuring energy $E_{k}$ is given by:

\begin{equation}
    P_{k} = \left\lvert \braket{k}{\psi} \right\rvert^{2}
\end{equation}

Consider the projection operator $A_{k}$ onto the $k$-th energy eigenstate acting on the state $\ket{\psi}$:

\begin{equation}
    A_{k} \ket{\psi} = \ket{k} \braket{k}{\psi}
\end{equation}

The expectation value of $A_{k}$ is thus:

\begin{equation}
    \mel{\psi}{A_{k}}{\psi} = \braket{\psi}{k} \braket{k}{\psi} = \left\lvert \braket{k}{\psi} \right\rvert^{2}
\end{equation}

Since the projection operator commutes with the Hamiltonian, we have:

\begin{equation}
    \frac{\mathrm{d}P_{k}}{\mathrm{d}t} = \frac{\mathrm{d}}{\mathrm{d}t} \mel{\psi}{A_{k}}{\psi} = \frac{i}{\hbar} \mel{\psi}{[\hat{H}, A_{k}]}{\psi} = 0
\end{equation}
\qed


\problem{3.11}{}
The probability of measuring $q_{r}$ is given by:

\begin{equation}
    P(q_{r} | \psi) = \left\lvert \braket{q_{r}}{\psi} \right\rvert^{2}
\end{equation}

The summation of all probabilities is:

\begin{equation}
\begin{split}
    \sum_{r} P(q_{r} | \psi) &= \sum_{r} \left\lvert \braket{q_{r}}{\psi} \right\rvert^{2} \\
    &= \sum_{r} \braket{\psi}{q_{r}} \braket{q_{r}}{\psi} \\
    &= \braket{\psi}{\psi} \\
    &= 1
\end{split}
\end{equation}

where we have used the completeness relation $\sum_{r} \ket{q_{r}} \bra{q_{r}} = \mathbb{I}$.

Note that we can express a state in its position representation:

\begin{equation}
    \ket{\psi} = \int \braket{x}{\psi} \ket{x} \, \mathrm{d}x
\end{equation}

where $\psi(x) \equiv \braket{x}{\psi}$ is the wave function.

The expectation value of $\hat{Q}$ is given by:

\begin{equation}
\begin{split}
    \mel{\psi}{\hat{Q}}{\psi} &= \int \int \braket{\psi}{x'} \bra{x'} \hat{Q} \ket{x} \braket{x}{\psi} \, \mathrm{d}x \, \mathrm{d}x' \\
    &= \int \psi^{*}(x) \hat{Q} \psi(x) \, \mathrm{d}x
\end{split}
\end{equation}
\qed


\problem{3.12}{}

\subproblem{a}
Given the TISE in the position representation:

\begin{equation}
    -\frac{\hbar^{2}}{2m} \frac{\mathrm{d}^{2} \psi(x)}{\mathrm{d}x^{2}} = E \psi(x)
\end{equation}

we have the general solution:

\begin{equation}
    \psi(x) = A \sin{kx} + B \cos{kx}
\end{equation}

where $k \equiv \sqrt{2mE}/\hbar$.

For $\psi(0) = 0$, we have $B = 0$. For $\psi(a) = 0$, we have $k = n\pi/a$ where $n$ is an integer. Thus, the energy levels are:

\begin{equation}
    E_{n} = \frac{\hbar^{2} n^{2} \pi^{2}}{2ma^{2}}
\end{equation}

To fix the normalisation constant $A$, we have:

\begin{equation}
    \left\lvert A \right\rvert^{2} \int_{0}^{a} \sin^{2} \frac{n\pi x}{a} \, \mathrm{d}x = 1
\end{equation}

which gives us $A = \sqrt{2/a}$.


\subproblem{b}
The expectation value of the position is given by:

\begin{equation}
\begin{split}
    \mel{\psi}{\hat{x}}{\psi} &= \int_{0}^{a} \psi^{*}(x) x \psi(x) \, \mathrm{d}x \\
    &= \frac{2}{a} \int_{0}^{a} x \sin^{2} \frac{n\pi x}{a} \, \mathrm{d}x \\
    &= \frac{a}{2}
\end{split}
\end{equation}


\subproblem{c}
The variance of the position is given by:

\begin{equation}
    \left\langle (x - \left\langle x \right\rangle)^{2} \right\rangle = \left\langle x^{2} - 2x \left\langle x \right\rangle + \left\langle x \right\rangle^{2} \right\rangle = \left\langle x^{2} \right\rangle - \left\langle x \right\rangle^{2} = \left\langle x^{2} \right\rangle - \frac{a^{2}}{4}
\end{equation}

where we treat $\left\langle x \right\rangle$ as a constant.

We evaluate $\left\langle x^{2} \right\rangle$:

\begin{equation}
\begin{split}
    \left\langle x^{2} \right\rangle &= \int_{0}^{a} \psi^{*}(x) x^{2} \psi(x) \, \mathrm{d}x \\
    &= \frac{2}{a} \int_{0}^{a} x^{2} \sin^{2} \frac{n\pi x}{a} \, \mathrm{d}x \\
    &= a^{2} \left( \frac{1}{3} - \frac{1}{2n^{2}\pi^{2}} \right)
\end{split}
\end{equation}

which gives us the variance:

\begin{equation}
    \left\langle (x - \left\langle x \right\rangle)^{2} \right\rangle = \frac{a^{2}}{12} \left( 1 - \frac{6}{n^{2}\pi^{2}} \right)
\end{equation}

\subproblem{d}
Consider a particle undergoing elastic collisions with the walls of the box. Suppose that the particle starts from $x = 0$ with a velocity $v$ at $t = 0$. The position of the particle at time $t$ is given by:

\begin{equation}
    x(t) = 
    \begin{cases}
        vt & 2na/v \leq t \leq (2n + 1)a/v \\
        a - vt & (2n + 1)a/v \leq t \leq (2n + 2)a/v
    \end{cases}
\end{equation}

The average position of the particle is given by the integral:

\begin{equation}
\begin{split}
    \left\langle x \right\rangle &= \sum_{n = 0}^{\infty} \frac{1}{a/v} \int_{2na/v}^{(2n + 1)a/v} vt \, \mathrm{d}t + \sum_{n = 0}^{\infty} \frac{1}{a/v} \int_{(2n + 1)a/v}^{(2n + 2)a/v} (a - vt) \, \mathrm{d}t \\
    &= \frac{a}{2} \sum_{n = 0}^{\infty} \left[ (2n + 1)^{2} - (2n)^{2} \right] + \frac{a}{2} \sum_{n = 0}^{\infty} \left[ 2 - (2n + 2)^{2} + (2n + 1)^{2} \right] \\
    &= 0
\end{split}
\end{equation}

where as the variance is given by:

\begin{equation}
\begin{split}
    \left\langle (x - \left\langle x \right\rangle)^{2} \right\rangle &= \sum_{n = 0}^{\infty} \frac{1}{a/v} \int_{2na/v}^{(2n + 1)a/v} (vt)^{2} \, \mathrm{d}t + \sum_{n = 0}^{\infty} \frac{1}{a/v} \int_{(2n + 1)a/v}^{(2n + 2)a/v} (a - vt)^{2} \, \mathrm{d}t \\
    &= \frac{a^{2}}{3} \sum_{n = 0}^{\infty} \left[ (2n + 1)^{3} - (2n)^{3} \right] + \frac{a^{2}}{3} \sum_{n = 0}^{\infty} \left[ (2n + 1)^{3} - (2n)^{3} \right] \\
    &= \frac{2a^{2}}{3} \sum_{n = 0}^{\infty} \left[ (2n + 1)^{3} - (2n)^{3} \right]
\end{split}
\end{equation}

For the moment let $n$ tend to a finite value $N$. We have:

\begin{equation}
\begin{split}
    \sum_{n = 0}^{N} \left[ (2n + 1)^{3} - (2n)^{3} \right] &= \sum_{n = 0}^{N} \left( 12n^{2} + 6n + 1 \right) \\
    &= 12 \frac{N(N + 1)(2N + 1)}{6} + 6 \frac{N(N + 1)}{2} + (N + 1) \\
\end{split}
\end{equation}







\end{document}
\documentclass[12pt]{article}
\usepackage{homework}
\pagestyle{fancy}

% assignment information
\def\course{Quantum Mechanics}
\def\assignmentno{Problem Sheet 4}
\def\assignmentname{Transformations \& Orbital Angular Momentum}
\def\name{Xin, Wenkang}
\def\time{\today}

\begin{document}

\begin{titlepage}
    \begin{center}
        \large
        \textbf{\course}

        \vfill

        \Huge
        \textbf{\assignmentno}

        \vspace{1.5cm}

        \large{\assignmentname}

        \vfill

        \large
        \name

        \time
    \end{center}
\end{titlepage}


%==========
\pagebreak
\section*{Transformations}
%==========


\problem{4.1}{Reflection symmetry around a point $\mathbf{x}_{0}$}
Let $\ket{\mathbf{x}_{0} + \mathbf{x}}$ be a position eigenstate that yields $\mathbf{x}_{0} + \mathbf{x}$ upon measurement of position. On physical grounds, reflecting the eigenstate about the point $\mathbf{x}_{0}$ should yield the eigenstate $\ket{\mathbf{x}_{0} - \mathbf{x}}$:

\begin{equation}
    \hat{P}_{\mathbf{x}_{0}} \ket{\mathbf{x}_{0} + \mathbf{x}} = \ket{\mathbf{x}_{0} - \mathbf{x}}
\end{equation}

With this, consider the effect of $\hat{P}_{\mathbf{x}_{0}} \hat{x} \hat{P}_{\mathbf{x}_{0}}$ on a position eigenstate:

\begin{equation}
\begin{split}
    \hat{P}_{\mathbf{x}_{0}} \hat{x} \hat{P}_{\mathbf{x}_{0}}\ket{\mathbf{x}_{0} + \mathbf{x}} &= \hat{P}_{\mathbf{x}_{0}} \hat{x} \ket{\mathbf{x}_{0} - \mathbf{x}} \\
    &= (\mathbf{x}_{0} - \mathbf{x}) \hat{P}_{\mathbf{x}_{0}} \ket{\mathbf{x}_{0} - \mathbf{x}} \\
    &= (\mathbf{x}_{0} - \mathbf{x}) \ket{\mathbf{x}_{0} + \mathbf{x}} \\
    &= (2\mathbf{x}_{0} - \mathbf{x}_{0} - \mathbf{x}) \ket{\mathbf{x}_{0} + \mathbf{x}} \\
    &= (2\mathbf{x}_{0} \mathbb{I} - \hat{x}) \ket{\mathbf{x}_{0} + \mathbf{x}}
\end{split}
\end{equation}

Further consider $\hat{P}_{\mathbf{x}_{0}} \hat{p} \hat{P}_{\mathbf{x}_{0}}$. Apparently $\hat{p}$ anticommutes with $\hat{P}_{\mathbf{x}_{0}}$ so that $\hat{p} \hat{P}_{\mathbf{x}_{0}} = -\hat{P}_{\mathbf{x}_{0}} \hat{p}$. Thus:

\begin{equation}
\begin{split}
    \hat{P}_{\mathbf{x}_{0}} \hat{p} \hat{P}_{\mathbf{x}_{0}} &= -\hat{P}_{\mathbf{x}_{0}} \hat{P}_{\mathbf{x}_{0}} \hat{p} \\
    &= -\hat{p}
\end{split}
\end{equation}

since two successive reflections about the same point is equivalent to no reflection at all.

Consider the position wave function after the reflection:

\begin{equation}
\begin{split}
    \psi'(\mathbf{x}) &\equiv \mel{\hat{x}}{\hat{P}_{\mathbf{x}_{0}}}{\psi} \\
    &= \int \mel{\hat{x}}{\hat{P}_{\mathbf{x}_{0}}}{\mathbf{x}_{0} + \mathbf{x}'} \braket{\mathbf{x}_{0} + \mathbf{x}'}{\psi} \, \mathrm{d}^{3}x' \\
    &= \int \braket{\hat{x}}{\mathbf{x}_{0} - \mathbf{x}'} \psi(\mathbf{x}_{0} + \mathbf{x}') \, \mathrm{d}^{3}x' \\
    &= \int (\mathbf{x}_{0} - \mathbf{x}') \psi(\mathbf{x}_{0} + \mathbf{x}') \, \mathrm{d}^{3}x' \\
\end{split}
\end{equation}

Consider the change of variable $\mathbf{x}' \to \mathbf{x}_{0} - \mathbf{x}'$:

\begin{equation}
    \psi'(\mathbf{x}) = \int \mathbf{x}' \psi(2\mathbf{x}_{0} - \mathbf{x}') \, \mathrm{d}^{3}x' = \psi(2\mathbf{x}_{0} - \mathbf{x})
\end{equation}
\qed


\problem{4.2}{}
For translation invariance, $\hat{H}$ must commute with $\hat{p}$. Since $\hat{x}$ and $\hat{p}$ generally do not commute, the only form $V(\hat{x})$ can take is a constant.
\qed


\problem{4.3}{}
We define the orbital angular momentum operator $\hat{L}_{i}$ as:

\begin{equation}
    \hat{L}_{i} \equiv \epsilon_{ijk} \hat{x}_{j} \hat{p}_{k}
\end{equation}

Its Hermitian conjugate is:

\begin{equation}
    \hat{L}_{i}^{\dagger} = \epsilon_{ijk} \hat{p}_{k}^{\dagger} \hat{x}_{j}^{\dagger} = \epsilon_{ijk} \hat{p}_{k} \hat{x}_{j}
\end{equation}

On the other hand, from the canonical commutation relation:

\begin{equation}
    [\hat{x}_{j}, \hat{p}_{k}] = i\hbar \delta_{jk} \mathbb{I}
\end{equation}

which implies that $\hat{x}_{j}$ and $\hat{p}_{k}$ commute if $j \neq k$. 

Therefore:

\begin{equation}
    \hat{L}_{i}^{\dagger} = \epsilon_{ijk} \hat{p}_{k} \hat{x}_{j} = \epsilon_{ijk} \hat{x}_{j} \hat{p}_{k} = \hat{L}_{i}
\end{equation}
\qed


\problem{4.4}{}
For a central potential, we write the Hamiltonian as:

\begin{equation}
    \hat{H} = \frac{\hat{p}^{2}}{2m} + V(\hat{r}^{2})
\end{equation}

where we define the radial position operator $\hat{r}^{2}$ as:

\begin{equation}
    \hat{r}^{2} \equiv \hat{x}_{1}^{2} + \hat{x}_{2}^{2} + \hat{x}_{3}^{2}
\end{equation}

Let us write the potential as an expansion in terms of $\hat{r}^{2}$:

\begin{equation}
    V(\hat{r}^{2}) = \sum_{n=0}^{\infty} a_{n} \hat{r}^{2n}
\end{equation}

Consider the commutator $[\hat{H}, \hat{L}_{i}]$:

\begin{equation}
\begin{split}
    [\hat{H}, \hat{L}_{i}] &= \frac{1}{2m} [\hat{p}^{2}, \hat{L}_{i}] + \sum_{n = 0}^{\infty} a_{n} [\hat{r}^{2n}, \hat{L}_{i}] \\
    &= \frac{1}{2m} \sum_{j = 1, 2, 3} [\hat{p}_{j}^{2}, \hat{L}_{i}] + \sum_{n = 0}^{\infty} a_{n} \sum_{j = 1, 2, 3} [\hat{x}_{j}^{2n}, \hat{L}_{i}] \\
    &= \frac{1}{2m} \epsilon_{ikl} \sum_{j = 1, 2, 3} [\hat{p}_{j}^{2}, \hat{x}_{k} \hat{p}_{l}] + \sum_{n = 0}^{\infty} a_{n} \epsilon_{ikl} \sum_{j = 1, 2, 3} [\hat{x}_{j}^{2n}, \hat{x}_{k} \hat{p}_{l}] \\
\end{split}
\end{equation}

Let us consider the commutators separately. Note the following commutation relations:

\begin{equation}
\begin{split}
    [AB, C] &= A[B, C] + [A, C]B \\
    [A, BC] &= [A, B]C + B[A, C]
\end{split}
\end{equation}

For $[\hat{p}_{j}^{2}, \hat{x}_{k} \hat{p}_{l}]$:

\begin{equation}
\begin{split}
    [\hat{p}_{j}^{2}, \hat{x}_{k} \hat{p}_{l}] &= \hat{p}_{j} [\hat{p}_{j}, \hat{x}_{k} \hat{p}_{l}] + [\hat{p}_{j}, \hat{x}_{k} \hat{p}_{l}] \hat{p}_{j} \\
    &= \hat{p}_{j} [\hat{p}_{j}, \hat{x}_{k}] \hat{p}_{l} + \hat{p}_{j} \hat{x}_{k} [\hat{p}_{j}, \hat{p}_{l}] + [\hat{p}_{j}, \hat{x}_{k}] \hat{p}_{l} \hat{p}_{j} + \hat{x}_{k} [\hat{p}_{j}, \hat{p}_{l}] \hat{p}_{j} \\
    &= \hat{p}_{j} [\hat{p}_{j}, \hat{x}_{k}] \hat{p}_{l} + [\hat{p}_{j}, \hat{x}_{k}] \hat{p}_{l} \hat{p}_{j}
\end{split}
\end{equation}

where we have used the fact that $\hat{p}_{j}$ and $\hat{p}_{l}$ commute.

This commutator is not zero only when $k = j$, in which case:

\begin{equation}
    [\hat{p}_{j}^{2}, \hat{x}_{k} \hat{p}_{l}] = -2i\hbar \hat{p}_{j} \hat{p}_{l}
\end{equation}

Then the first term in the commutator $[\hat{H}, \hat{L}_{i}]$ becomes:

\begin{equation}
    \frac{1}{2m} \epsilon_{ijl} \sum_{j = 1, 2, 3} (-2i\hbar \hat{p}_{j} \hat{p}_{l}) = \frac{1}{im} \epsilon_{ijl} \hat{p}_{j} \hat{p}_{l}
\end{equation}

This is zero since $\hat{p}_{j}$ and $\hat{p}_{l}$ commute. We then consider the second commutator $[\hat{x}_{j}^{2n}, \hat{x}_{k} \hat{p}_{l}]$:

\begin{equation}
    [\hat{x}_{j}^{2n}, \hat{x}_{k} \hat{p}_{l}] = [\hat{x}_{j}^{2n}, \hat{x}_{k}] \hat{p}_{l} + \hat{x}_{k} [\hat{x}_{j}^{2n}, \hat{p}_{l}]
\end{equation}

where the first term is always zero since $\hat{x}_{j}^{2n}$ and $\hat{x}_{k}$ commute and the second term is not zero only when $l = j$, in which case:

\begin{equation}
\begin{split}
    [\hat{x}_{j}^{2n}, \hat{p}_{l}] &= \hat{x}_{k} [\hat{x}_{j}^{2n}, \hat{p}_{j}] \\
    &= \hat{x}_{k} \left\{ \hat{x}_{j}^{2n-1} [\hat{x}_{j}, \hat{p}_{j}] + [\hat{x}_{j}^{2n-1}, \hat{p}_{j}] \hat{x}_{j} \right\} \\
    &= \hat{x}_{k} \left\{ \hat{x}_{j}^{2n-1} [\hat{x}_{j}, \hat{p}_{j}] + x_{j}^{2n-2} [\hat{x}_{j}, \hat{p}_{j}] \hat{x}_{j} + \cdots + [\hat{x}_{j}, \hat{p}_{j}] \hat{x}_{j}^{2n-1} \right\} \\
    &= i \hbar (2n) \hat{x}_{k} \hat{x}_{j}^{2n-1}
\end{split}
\end{equation}

Therefore the second term in the commutator $[\hat{H}, \hat{L}_{i}]$ becomes:

\begin{equation}
    \sum_{n = 0}^{\infty} a_{n} \epsilon_{ikj} \sum_{j = 1, 2, 3} i \hbar (2n) \hat{x}_{k} \hat{x}_{j}^{2n-1}
\end{equation}

which is always zero since $\hat{x}_{k}$ and $\hat{x}_{j}^{2n-1}$ commute.

Therefore $[\hat{H}, \hat{L}_{i}] = 0$ for a central potential and the angular momentum is conserved.

Furthermore, consider a potential that has azimuthal symmetry, i.e. $V(\mathbf{x}) = V(\hat{x}_{1}^{2} + \hat{x}_{2}^{2})$. In this case, we can write the potential as:

\begin{equation}
    V = \sum_{n=0}^{\infty} a_{n} (\hat{x}_{1}^{2} + \hat{x}_{2}^{2})^{n}
\end{equation}

The change from previous results occurs on the second term, where we only let $j$ run over $1$ and $2$. Due to the presence of the $\epsilon_{ikj}$ term, the sum is zero only for $i = 3$, since for the other two cases we will miss one term in the sum due to $j = 3$ missing. Therefore, the Hamiltonian only commutes with $\hat{L}_{3}$, which is the $z$-component of the angular momentum. The $x$- and $y$-components of the angular momentum are not conserved.
\qed


\problem{4.5}{}
Let us expand the expression using binomial theorem:

\begin{equation}
\begin{split}
    \lim_{N \to \infty} \left( 1 + \frac{x}{N} \right)^{N} &= \lim_{N \to \infty} \sum_{n = 0}^{N} \binom{N}{n} \left( \frac{x}{N} \right)^{n} \\
    &= \lim_{N \to \infty} \sum_{n = 0}^{N} \frac{N!}{n!(N-n)!} \frac{1}{N^{n}} x^{n} \\
    &= \lim_{N \to \infty} \sum_{n = 0}^{N} \frac{N(N-1)(N-2) \cdots (N-n+1)}{N^{n}} \frac{1}{n!} x^{n} \\
    &= \lim_{N \to \infty} \sum_{n = 0}^{N} \frac{1}{n!} x^{n} \\
    &= e^{x}
\end{split}
\end{equation}

where at the last step we have used the fact that the sum is the Taylor series of $e^{x}$.

This can in some way be viewed as a definition of $e^{x}$. Indeed, the definition of exponential for an operator is just this limit:

\begin{equation}
    \exp(\hat{A}) \equiv \lim_{N \to \infty} \left( 1 + \frac{\hat{A}}{N} \right)^{N} = \left( 1 + \frac{\hat{A}}{N} \right) \left( 1 + \frac{\hat{A}}{N} \right) \cdots \left( 1 + \frac{\hat{A}}{N} \right)
\end{equation}

which can be viewed as applying the operator $(1 + \hat{A}/N)$ to the state $N$ times.
\qed


\problem{4.6}{Heisenberg equations of motion for the SHO}
In Heisenberg picture, we replace an operator $\hat{A}_{S}$ in Schr\"{o}dinger picture with:

\begin{equation}
    \hat{A}_{H}(t) = \hat{U}^{\dagger}(t) \hat{A}_{S} \hat{U}(t)
\end{equation}

where $\hat{U} = e^{-i\hat{H}t/\hbar}$ is the time evolution operator.

For the creation and annihilation operators, we have:

\begin{equation}
\begin{split}
\hat{a}(t) &= \sqrt{\frac{2\hbar}{m\omega}} \hat{U}^{\dagger}(t) \left( \hat{x}_{S} + \frac{i}{m\omega} \hat{p}_{S} \right) \hat{U}(t) \\
\hat{a}^{\dagger}(t) &= \sqrt{\frac{2\hbar}{m\omega}} \hat{U}^{\dagger}(t) \left( \hat{x}_{S} - \frac{i}{m\omega} \hat{p}_{S} \right) \hat{U}(t)
\end{split}
\end{equation}

The Heisenberg equations of motion for the annihilation operator is:

\begin{equation}
\begin{split}
\frac{\mathrm{d}}{\mathrm{d}t} \hat{a}(t) &= \frac{i}{\hbar} [\hat{H}, U^{\dagger}(t) \hat{a}_{S} U(t)] \\
&= \frac{i}{\hbar} \hat{U}^{\dagger}(t) [\hat{H}, \hat{a}_{S}] \hat{U}(t) \\
&= i \omega \hat{U}^{\dagger}(t) [\hat{a}_{S}^{\dagger} \hat{a}_{S}, \hat{a}_{S}] \hat{U}(t) \\
&= i \omega \hat{U}^{\dagger}(t) \left( \hat{a}_{S}^{\dagger} [\hat{a}_{S}, \hat{a}_{S}] + [\hat{a}_{S}^{\dagger}, \hat{a}_{S}] \hat{a}_{S} \right) \hat{U}(t) \\
&= -i \omega \hat{U}^{\dagger}(t) \hat{a}_{S} \hat{U}(t) \\
&= -i \omega \hat{a}(t)
\end{split}
\end{equation}

This is a differential equation with the solution:

\begin{equation}
    \hat{a}(t) = \hat{a}(0) e^{-i\omega t}
\end{equation}

Similarly for the creation operator, we have an equation $\mathrm{d}\hat{a}^{\dagger}/\mathrm{d}t = i\omega \hat{a}^{\dagger}(t)$ with the solution:

\begin{equation}
    \hat{a}^{\dagger}(t) = \hat{a}^{\dagger}(0) e^{i\omega t}
\end{equation}

Consider the following:

\begin{equation}
\begin{split}
    \hat{a}(t) &= \sqrt{\frac{2\hbar}{m\omega}} \left[ \hat{x}(t) + \frac{i}{m\omega} \hat{p}(t) \right] \\
    \hat{a}^{\dagger}(t) &= \sqrt{\frac{2\hbar}{m\omega}} \left[ \hat{x}(t) - \frac{i}{m\omega} \hat{p}(t) \right]
\end{split}
\end{equation}

which allow us to solve for $\hat{x}(t)$ and $\hat{p}(t)$:

\begin{equation}
\begin{split}
    \hat{x}(t) &= \frac{1}{2} \sqrt{\frac{m\omega}{2\hbar}} \left[ \hat{a}(t) + \hat{a}^{\dagger}(t) \right] \\
    \hat{p}(t) &= \frac{m\omega}{2i} \sqrt{\frac{m\omega}{2\hbar}} \left[ \hat{a}(t) - \hat{a}^{\dagger}(t) \right]
\end{split}
\end{equation}

Focus on the equation for $\hat{x}(t)$. We can take the time derivative of both sides:

\begin{equation}
\begin{split}
    \frac{\mathrm{d}}{\mathrm{d}t} \hat{x}(t) &= \frac{1}{2} \sqrt{\frac{m\omega}{2\hbar}} \left[ \frac{\mathrm{d}}{\mathrm{d}t} \hat{a}(t) + \frac{\mathrm{d}}{\mathrm{d}t} \hat{a}^{\dagger}(t) \right] \\
    &= -\frac{i\omega}{2} \sqrt{\frac{m\omega}{2\hbar}} \left[ \hat{a}(t) - \hat{a}^{\dagger}(t) \right] \\
    &= \frac{\hat{p}(t)}{m}
\end{split}
\end{equation}

which is the Heisenberg equation of motion for $\hat{x}(t)$.

Similarly, we can take the time derivative of the equation for $\hat{p}(t)$ and obtain:

\begin{equation}
    \frac{\mathrm{d}}{\mathrm{d}t} \hat{p}(t) = -m\omega^{2} \hat{x}(t)
\end{equation}

These are exactly the classical equations of motion for a harmonic oscillator, which demonstrates the correspondence between classical and quantum mechanics, i.e., Ehrenfest's theorem.
\qed


%==========
\pagebreak
\section*{Orbital Angular Momentum}
%==========


\problem{4.7}{}

\subproblem{a}

\begin{equation}
\begin{split}
    [\hat{L}_{i}, \hat{x}_{j}] &= \epsilon_{ikl} [\hat{x}_{k} \hat{p}_{l}, \hat{x}_{j}] \\
    &= \epsilon_{ikl} \hat{x}_{k} [\hat{p}_{l}, \hat{x}_{j}] + \epsilon_{ikl} [\hat{x}_{k}, \hat{x}_{j}] \hat{p}_{l} \\
    &= -i\hbar \epsilon_{ikl} \hat{x}_{k} \\
    &= i\hbar \epsilon_{ijk} \hat{x}_{k} \\
\end{split}
\end{equation}

\begin{equation}
\begin{split}
    [\hat{L}_{i}, \hat{p}_{j}] &= \epsilon_{ikl} [\hat{x}_{k} \hat{p}_{l}, \hat{p}_{j}] \\
    &= \epsilon_{ikl} \hat{x}_{k} [\hat{p}_{l}, \hat{p}_{j}] + \epsilon_{ikl} [\hat{x}_{k}, \hat{p}_{j}] \hat{p}_{l} \\
    &= i\hbar \epsilon_{ijl} \hat{p}_{l} \\
    &= i\hbar \epsilon_{ijk} \hat{p}_{k} \\
\end{split}
\end{equation}

\subproblem{b}

\begin{equation}
\begin{split}
    [\hat{L}_{x}, \hat{L}_{y}] &= [\hat{L}_{x}, \hat{z} \hat{p}_{x} - \hat{x} \hat{p}_{z}] \\
    &= [\hat{L}_{x}, \hat{z} \hat{p}_{x}] - [\hat{L}_{x}, \hat{x} \hat{p}_{z}] \\
    &= [\hat{L}_{x}, \hat{z}] \hat{p}_{x} + \hat{z} [\hat{L}_{x}, \hat{p}_{x}] - [\hat{L}_{x}, \hat{x}] \hat{p}_{z} - \hat{x} [\hat{L}_{x}, \hat{p}_{z}] \\
    &= i\hbar (\hat{x} \hat{p}_{y} - \hat{y} \hat{p}_{x}) \\
    &= i\hbar \hat{L}_{z}
\end{split}
\end{equation}

which can be generalised to:

\begin{equation}
    [\hat{L}_{i}, \hat{L}_{j}] = i\hbar \epsilon_{ijk} \hat{L}_{k}
\end{equation}

\subproblem{c}
In position representation, the angular momentum operator have the form:

\begin{equation}
\begin{split}
    \mel{\mathbf{x}}{\hat{L}_{i}}{\psi} &= \epsilon_{ijk} \mel{\mathbf{x}}{\hat{x}_{j} \hat{p}_{k}}{\psi} \\
    &= \epsilon_{ijk} x_{j} \mel{\mathbf{x}}{\hat{p}_{k}}{\psi} \\
    &= -i\hbar \epsilon_{ijk} x_{j} \frac{\partial}{\partial x_{k}} \psi(\mathbf{x})
\end{split}
\end{equation}

\subproblem{d}
Consider the commutator $[\hat{L}_{i}, \hat{L}^{2}]$:

\begin{equation}
\begin{split}
    [\hat{L}_{i}, \hat{L}^{2}] &= \sum_{r = 1, 2, 3} [\hat{L}_{i}, \hat{L}_{r}^{2}] \\
    &= \sum_{r = 1, 2, 3} \left( [\hat{L}_{i}, \hat{L}_{r}] \hat{L}_{r} + \hat{L}_{r} [\hat{L}_{i}, \hat{L}_{r}] \right) \\
    &= i\hbar \epsilon_{jir} (\hat{L}_{j} \hat{L}_{r} + \hat{L}_{r} \hat{L}_{j}) \\
    &= 0
\end{split}
\end{equation}
\qed


\problem{4.8}{}
We have the expression of $\hat{L}^{2}$ in spherical coordinates:

\begin{equation}
    \hat{L}^{2} = -\hbar^{2} \left[ \frac{1}{\sin \theta} \frac{\partial}{\partial \theta} \left( \sin \theta \frac{\partial}{\partial \theta} \right) + \frac{1}{\sin^{2} \theta} \frac{\partial^{2}}{\partial \phi^{2}} \right]
\end{equation}

\subproblem{a}

The following calculations follow:

\begin{equation}
\begin{split}
    \hat{L}^{2} (\cos{\theta}) &= -\hbar^{2} \frac{1}{\sin{\theta}} \frac{\partial}{\partial \theta} \left( -\sin^{2}{\theta} \right) \\
    &= 2\hbar^{2} \cos{\theta} \\
    \hat{L}^{2} \left( \sin{\theta} e^{\pm i\phi} \right) &= -\hbar^{2} \left[ \frac{e^{\pm i\phi}}{\sin{\theta}} \frac{\partial}{\partial \theta} \left( \sin{\theta} \frac{\partial}{\partial \theta} \sin{\theta} \right) + \frac{1}{\sin{\theta}} \frac{\partial}{\partial \phi} \left( \frac{\partial}{\partial \phi} e^{\pm i\phi} \right) \right] \\
    &= -\hbar^{2} \left[ \frac{e^{\pm i\phi}}{\sin{\theta}} \frac{\partial}{\partial \theta} \left( \sin{\theta} \cos{\theta} \right) + \frac{1}{\sin{\theta}} \frac{\partial}{\partial \phi} \left( \pm i e^{\pm i\phi} \right) \right] \\
    &= -\hbar^{2} \left[ \frac{e^{\pm i\phi}}{\sin{\theta}} \left( \cos^{2}{\theta} - \sin^{2}{\theta} \right) - \frac{1}{\sin{\theta}} e^{\pm i\phi} \right] \\
    &= 2\hbar^{2} e^{\pm i\phi}
\end{split}
\end{equation}

\begin{equation}
\begin{split}
    \hat{L}_{z} (\cos{\theta}) &= -i \cos{\theta} \\
    \hat{L}_{z} \left( \sin{\theta} e^{\pm i\phi} \right) &= \mp i \sin{\theta} e^{\pm i\phi}
\end{split}
\end{equation}

\subproblem{b}
For $Y_{1}^{0}$, the normalisation condition is:

\begin{equation}
    \left\lvert N \right\rvert^{2} \int_{0}^{2\pi} \int_{0}^{\pi} \cos^{2}{\theta} \sin{\theta} \, \mathrm{d}\theta \, \mathrm{d}\phi = 1
\end{equation}

which gives $N = \pm \sqrt{3/4\pi}$. 

For $Y_{1}^{\pm 1}$, the normalisation condition is:

\begin{equation}
    \left\lvert N \right\rvert^{2} \int_{0}^{2\pi} \int_{0}^{\pi} \sin^{2}{\theta} \sin{\theta} \, \mathrm{d}\theta \, \mathrm{d}\phi = 1
\end{equation}

which gives $N = \pm \sqrt{3/8\pi}$.

\subproblem{c}
In Cartesian coordinates, the above spherical harmonics are:

\begin{equation}
\begin{split}
    Y_{1}^{0} &= \sqrt{\frac{3}{4\pi}} \frac{z}{\sqrt{x^{2} + y^{2} + z^{2}}} \\
    Y_{1}^{\pm 1} &= \mp \sqrt{\frac{3}{8\pi}} \frac{\sqrt{x^{2} + y^{2}}}{\sqrt{x^{2} + y^{2} + z^{2}}} \exp \left[ \pm i \tan^{-1}{\left( \frac{y}{x} \right)} \right]
\end{split}
\end{equation}
\qed


\problem{4.9}{}
The wave function can be identified as:

\begin{equation}
    \braket{\theta, \phi}{\psi} \propto \sqrt{2} \sqrt{\frac{4\pi}{3}} Y_{1}^{0} + \sqrt{\frac{8\pi}{3}} Y_{1}^{1} + \sqrt{\frac{8\pi}{3}} Y_{1}^{-1}
\end{equation}

so that $\hat{L}^{2}$ always yields $2\hbar^{2}$ and $\hat{L}_{z}$ yields zero with probability $1/3$ and $\pm \hbar$ with probability $1/3$.

The expectation of $\hat{L}_{z}$ is zero.
\qed


\problem{4.10}{}
We recognise $\sin^{2}{\theta} e^{2i\phi}$ as $Y_{2}^{2}$ up to a constant factor. Therefore, $\hat{L}^{2}$ yields $6\hbar^{2}$ and $\hat{L}_{z}$ yields $2\hbar$ each with probability $1$.
\qed


\problem{4.11}{}
Consider the the wave function $\braket{\theta, \phi}{\psi} = A \sin^{2}{\theta}$:

\begin{equation}
\begin{split}
    \braket{\theta, \phi}{\psi} &= A \sin^{2}{\theta} \\
    &= A (1 - \cos^{2}{\theta}) \\
    &= A \left( 1 - \frac{\sqrt{16\pi/5}Y_{2}^{0} + 1}{3} \right) \\
    &= A \left( - \sqrt{\frac{16\pi}{45}} Y_{2}^{0} + \frac{2}{3} \right) \\
    &= A \left( - \sqrt{\frac{16\pi}{45}} Y_{2}^{0} + \frac{2\sqrt{4\pi}}{3} Y_{0}^{0} \right) \\
\end{split}
\end{equation}

where we have made use of the following spherical harmonics:

\begin{equation}
\begin{split}
    Y_{2}^{0} &= \sqrt{\frac{5}{16\pi}} (3\cos^{2}{\theta} - 1) \\
    Y_{0}^{0} &= \frac{1}{\sqrt{4\pi}}
\end{split}
\end{equation}

Therefore, measurement with $\hat{L}_{z}$ always yields $0$ and measurement with $\hat{L}^{2}$ yields $6\hbar^{2}$ with probability $1$.
\qed


\problem{4.12}{}

\subproblem{a}
The term $-e\varepsilon \hat{x}$ in the Hamiltonian suggests some kind of position dependent (linear) potential. In light of the charge factor, this can be interpreted as the potential due to a uniform electric field of strength $\varepsilon$ in the x-direction.

\subproblem{b}
The Hamiltonian of the form:

\begin{equation}
    \hat{H} = -\frac{\hbar^{2}}{2m} \nabla^{2} - \frac{e^{2}}{4\pi \epsilon_{0}r} - e\varepsilon \hat{x}
\end{equation}

is spherically symmetrical if $\varepsilon = 0$ and symmetrical only about the x-axis if $\varepsilon \neq 0$.

Therefore, in the case of $\varepsilon = 0$, $\hat{L}^{2}$ and $\hat{L}_{i}$ are conserved. In the case of $\varepsilon \neq 0$, only $\hat{L}_{x}$ is conserved.
\qed


\problem{4.13}{}

\begin{equation}
\begin{split}
    [\hat{L}_{i}, \hat{x} \cdot \hat{p}] &= \sum_{r = 1, 2, 3} [\hat{L}_{i}, \hat{x}_{r} \hat{p}_{r}] \\
    &= \sum_{r = 1, 2, 3} \left( [\hat{L}_{i}, \hat{x}_{r}] \hat{p}_{r} + \hat{x}_{r} [\hat{L}_{i}, \hat{p}_{r}] \right) \\
    &= i\hbar \sum_{r = 1, 2, 3} \left( \epsilon_{irl} \hat{x}_{l} \hat{p}_{r} + \epsilon_{irl} \hat{x}_{r} \hat{p}_{l} \right) \\
    &= 0
\end{split}
\end{equation}
\qed


\end{document}
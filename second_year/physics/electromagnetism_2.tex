\documentclass[12pt]{article}
\usepackage{homework}
\pagestyle{fancy}

% assignment information
\def\course{Electromagnetism 2}
\def\assignmentno{Problem Sheet 2}
\def\assignmentname{Electric and Magnetic Fields in Matter}
\def\name{Xin, Wenkang}
\def\time{\today}

\begin{document}

\begin{titlepage}
    \begin{center}
        \large
        \textbf{\course}

        \vfill

        \Huge
        \textbf{\assignmentno}

        \vspace{1.5cm}

        \large{\assignmentname}

        \vfill

        \large
        \name

        \time
    \end{center}
\end{titlepage}


%==========
\pagebreak
\section*{Electric Field in Matter}
%==========


\problem{2.1}{Capacitance and dielectrics}

\subproblem{a}
Using a small box-shape Gaussian surface, the field produced by a large plate carrying a uniform surface charge density $\sigma$ is $\sigma/2\epsilon_{0}$ as computed via Gauss' law. The parallel plate capacitor has two such plates, so the field between the plates is a superposition of the fields produced by each plate:

\begin{equation}
    E_{0} = \frac{Q_{0}/A}{2\epsilon_{0}} - \left( \frac{-Q_{0}/A}{2\epsilon_{0}} \right) = \frac{Q_{0}}{A\epsilon_{0}}
\end{equation}

As the field is uniform, the potential difference between the plates is simply $V_{0} = E_{0}d = Q_{0}d/A\epsilon_{0}$. The capacitance is then $C = Q_{0}/V_{0} = A\epsilon_{0}/d$.

The potential energy stored in the capacitor can be computed by integrating the energy density $\epsilon_{0}E^{2}/2$ over the volume between the plates:

\begin{equation}
    U_{0} = \frac{\epsilon_{0}}{2} \int_{V} E_{0}^{2} \, \mathrm{d}V = \frac{\epsilon_{0}}{2} \frac{Q_{0}^{2}}{A^{2}\epsilon_{0}^{2}} \int_{V} \, \mathrm{d}V = \frac{Q_{0}^{2}d}{2A\epsilon_{0}}
\end{equation}

\subproblem{b}
Consider the relationship between the potential difference and the charge on the capacitor:

\begin{equation}
    V = \frac{Qd}{\epsilon_{0}A}
\end{equation}

The dielectric material inserted causes a change in the permittivity $\epsilon_{0} \to \epsilon_{r} \epsilon_{0} > \epsilon_{0}$. Given constant $V$, the charge on the capacitor is then $Q \to \epsilon_{r}Q > Q$.

Apply Gauss' law for electric displacement $\mathbf{D}$ with a small box-shape Gaussian surface:

\begin{equation}
\begin{split}
    \oint_{S} \mathbf{D} \cdot \mathrm{d}\mathbf{S} &= Q_{f} \\
    D \alpha &= \frac{Q\alpha}{A} \\
    \epsilon_{0} E + P &= \frac{Q}{A}
\end{split}
\end{equation}

which gives $E = (Q/A - P)/\epsilon_{0}$.

Assuming a linear dielectric with the polarization $P = \epsilon_{0} \chi_{e} E$, the electric field is then:

\begin{equation}
    E = \frac{Q}{A\epsilon_{0}(1 + \chi_{e})}
\end{equation}

Since the potential is kept constant, the field is also constant so that $E = E_{0}$, which gives the relation:

\begin{equation}
    \frac{Q}{A\epsilon_{0}(1 + \chi_{e})} = \frac{Q_{0}}{A\epsilon_{0}}
\end{equation}

or $Q = (1 + \chi_{e}) Q_{0}$ as expected.

The capacitance is then $C = Q/V_{0} = (1 + \chi_{e})C_{0}$, and the change in the potential energy is:

\begin{equation}
    \Delta U = \frac{Q^{2}d}{2A\epsilon_{r}\epsilon_{0}} - \frac{Q_{0}^{2}d}{2A\epsilon_{0}} = \chi_{e} U_{0}
\end{equation}

\subproblem{c}
Still consider the previous relationship but keep the charge constant. The increase in permittivity causes a decrease in the potential difference $V \to V/\epsilon_{r} < V$.

The capacitance is a property of geometry and material (permittivity) so it is unchanged, i.e., $C = (1 + \chi_{e})C_{0}$. The change in the potential energy is:

\begin{equation}
    \Delta U = \frac{Q_{0}^{2}d}{2A\epsilon_{r}\epsilon_{0}} - \frac{Q_{0}^{2}d}{2A\epsilon_{0}} = -\frac{\chi_{e}}{1 + \chi_{e}} U_{0}
\end{equation}
\qed


\problem{2.2}{Capacitor half-filled with a dielectric}

\subproblem{a}
Label the regions from $1$ to $4$ from left to right as depicted in the figure. Apparently the electric fields and polarisations in region $1$ and $4$ are zero. Applying Gauss' law in region $2$ and $3$ gives:

\begin{equation}
\begin{split}
    D_{2} \alpha &= \sigma \alpha = (\epsilon_{0} E_{2} + P) \alpha \\
    D_{3} \alpha &= \sigma \alpha = \epsilon_{0} E_{3} \alpha
\end{split}
\end{equation}

Solving the equations gives $E_{2} = \sigma/\epsilon_{0} \epsilon_{r}$, $P = (1 - 1/\epsilon_{r}) \sigma$, and $E_{3} = \sigma/\epsilon_{0}$. Thus:

\begin{equation}
\begin{split}
    \mathbf{D}_{2} &= \mathbf{D}_{3} = \sigma \hat{x} \\
    \mathbf{E}_{2} &= \frac{\sigma}{\epsilon_{0} \epsilon_{r}} \hat{x} \\
    \mathbf{E}_{3} &= \frac{\sigma}{\epsilon_{0}} \hat{x} \\
    \mathbf{P} &= \left( 1 - \frac{1}{\epsilon_{r}} \right) \sigma \hat{x}
\end{split}
\end{equation}

where $\hat{x}$ is the unit vector pointing from the positive plate to the negative one.

\subproblem{b}
Taking the negative plate as the reference point, the potential difference between the plates is:

\begin{equation}
\begin{split}
    V &= \int_{0}^{d} E_{2} \, \mathrm{d}x + \int_{d}^{2d} E_{3} \, \mathrm{d}x \\
    &= \left( \frac{1}{\epsilon_{0} \epsilon_{r}} + \frac{1}{\epsilon_{0}} \right) \sigma d \\
\end{split}
\end{equation}

The capacitance is:

\begin{equation}
    C = \frac{\sigma A}{V} = \frac{A/d}{1/\epsilon_{0} \epsilon_{r} + 1/\epsilon_{0}}
\end{equation}

Note that:

\begin{equation}
    \frac{1}{C} = \frac{d}{A} \left( \frac{1}{\epsilon_{0} \epsilon_{r}} + \frac{1}{\epsilon_{0}} \right) = \frac{1}{C_{1}} + \frac{1}{C_{2}}
\end{equation}

where $C_{1}$ is the capacitance of the left half and $C_{2}$ is the capacitance of the right half. 

This is as though the capacitance of two capacitors in series.

\subproblem{c}
As the polarisation is constant, there is no volume bound density since $\nabla \cdot \mathbf{P} = 0$. On the interface between region $1$ and $2$:

\begin{equation}
    \sigma_{b} = -p = -\left( 1 - \frac{1}{\epsilon_{r}} \right) \sigma
\end{equation}

On the interface between region $2$ and $3$:

\begin{equation}
    \sigma_{b} = p = \left( 1 - \frac{1}{\epsilon_{r}} \right) \sigma
\end{equation}

Using all surface charge densities, the electric field in four regions are:

\begin{equation}
\begin{split}
    \mathbf{E}_{1} &= \frac{\sigma}{2\epsilon_{0}} \left[ -\frac{1}{\epsilon_{r}} - \left( 1 - \frac{1}{\epsilon_{r}} \right) + 1 \right] \hat{x} = 0 \\
    \mathbf{E}_{2} &= \frac{\sigma}{2\epsilon_{0}} \left[ \frac{1}{\epsilon_{r}} - \left( 1 - \frac{1}{\epsilon_{r}} \right) + 1 \right] \hat{x} = \frac{\sigma}{\epsilon_{0}\epsilon_{r}} \hat{x} \\
    \mathbf{E}_{3} &= \frac{\sigma}{2\epsilon_{0}} \left[ \frac{1}{\epsilon_{r}} + \left( 1 - \frac{1}{\epsilon_{r}} \right) + 1 \right] \hat{x} = \frac{\sigma}{\epsilon_{0}} \hat{x} \\
    \mathbf{E}_{4} &= \frac{\sigma}{2\epsilon_{0}} \left[ \frac{1}{\epsilon_{r}} + \left( 1 - \frac{1}{\epsilon_{r}} \right) - 1 \right] \hat{x} = 0
\end{split}
\end{equation}

which are the same as the results in part (a).
\qed


\problem{2.3}{Force on a dielectric}





%==========
\pagebreak
\section*{Magnetic Field in Matter}
%==========


\problem{}{}


\end{document}
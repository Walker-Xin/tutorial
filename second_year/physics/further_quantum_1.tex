\documentclass[12pt]{article}
\usepackage{homework}
\pagestyle{fancy}

% assignment information
\def\course{Further Quantum Mechanics}
\def\assignmentno{Problem Set 1}
\def\assignmentname{}
\def\name{Xin, Wenkang}
\def\time{\today}

\begin{document}

\begin{titlepage}
    \begin{center}
        \large
        \textbf{\course}

        \vfill

        \Huge
        \textbf{\assignmentno}

        \vspace{1.5cm}

        \large{\assignmentname}

        \vfill

        \large
        \name

        \time
    \end{center}
\end{titlepage}


%==========
\pagebreak
\section*{}
%==========


\problem{1}{}
The TDSE reads:

\begin{equation}
    \frac{1}{2m} \left( -i\hbar \nabla - q\mathbf{A} \right)^{2} \psi = i\hbar \frac{\partial \psi}{\partial t}
\end{equation}

Consider the right-hand side of the equation, we have:

\begin{equation}
    \begin{split}
        \left( -i\hbar \nabla - q\mathbf{A} \right)^{2} \psi &=
        \left( i\hbar \nabla + q\mathbf{A} \right) \left( i\hbar \nabla \psi + q\mathbf{A}\psi \right) \\
        &= -\hbar^{2} \nabla^{2} \psi + i\hbar q \nabla \cdot \left( \mathbf{A}\psi \right) + i\hbar q \mathbf{A} \cdot \nabla \psi + q^{2} \mathbf{A}^{2} \psi
    \end{split}
\end{equation}

so that the TDSE becomes:

\begin{equation}
    \begin{split}
        \frac{1}{2m} \left[ i\hbar \nabla^{2} \psi + q \psi (\nabla \cdot \mathbf{A}) + 2q \mathbf{A} \cdot \nabla \psi + \frac{q^{2}}{i\hbar} \mathbf{A}^{2} \psi \right] &= \frac{\partial \psi}{\partial t} \\
    \end{split}
\end{equation}

On the other hand, consider the time derivative of the probability density $\rho = \psi^{*}\psi$, we have:

\begin{equation}
    \begin{split}
        \frac{\partial \rho}{\partial t} &= \frac{\partial}{\partial t} \left( \psi^{*}\psi \right) \\
        &= \psi^{*} \frac{\partial \psi}{\partial t} + \psi \frac{\partial \psi^{*}}{\partial t}
    \end{split}
\end{equation}

Combining the two results, we have:

\begin{equation}
    \begin{split}
        \frac{\partial \rho}{\partial t} &= \frac{\psi^{*}}{2m} \left[ i\hbar \nabla^{2} \psi + q \psi (\nabla \cdot \mathbf{A}) + 2q \mathbf{A} \cdot \nabla \psi + \frac{q^{2}}{i\hbar} \mathbf{A}^{2} \psi \right] \\
        &+ \frac{\psi}{2m} \left[ -i\hbar \nabla^{2} \psi^{*} + q \psi^{*} (\nabla \cdot \mathbf{A}) + 2q \mathbf{A} \cdot \nabla \psi^{*} - \frac{q^{2}}{i\hbar} \mathbf{A}^{2} \psi^{*} \right] \\
        &= \frac{1}{2m} \left[ i\hbar \left( \psi^{*} \nabla^{2} \psi + \psi \nabla^{2} \psi^{*} \right) + 2q \left( \psi^{*}\psi \right) (\nabla \cdot \mathbf{A}) + 2q \mathbf{A} \cdot \nabla \left( \psi^{*}\psi \right) \right] \\
        &= \frac{\hbar}{2im} \left( \psi^{*} \nabla^{2} \psi + \psi \nabla^{2} \psi^{*} \right) + \frac{q}{m} \left[ \psi^{*}\psi (\nabla \cdot \mathbf{A}) + \mathbf{A} \cdot \nabla \left( \psi^{*}\psi \right) \right] \\
        &= -\nabla \cdot \mathbf{j}
    \end{split}
\end{equation}

where we identify the probability current density $\mathbf{j}$ as:

\begin{equation}
    \mathbf{j} = \frac{\hbar}{2im} \left( \psi^{*} \nabla \psi - \psi \nabla \psi^{*} \right) - \frac{q}{m} \mathbf{A} \psi^{*}\psi
\end{equation}

Consider the TISE for an energy eigenfunction $\psi$:

\begin{equation}
    \frac{1}{2m} \left( -i\hbar \nabla - q\mathbf{A} \right)^{2} \psi = E\psi
\end{equation}

Given the trial function $\psi(x, y) = \exp [-(x^{2} + y^{2})/4l_{B}^{2}]$, we have:

\begin{equation}
    \begin{split}
        \left( -i\hbar \nabla - q\mathbf{A} \right)^{2} \psi &= -\hbar^{2} \nabla^{2} \psi + i\hbar q \psi \left( \nabla \cdot \mathbf{A} \right) + 2i\hbar q \mathbf{A} \cdot \nabla \psi + q^{2} \mathbf{A}^{2} \psi \\
        &= -\hbar^{2} \frac{x^{2} + y^{2} - 4l_{B}^{2}}{4l_{B}^{4}} \psi + \frac{B^{2}(x^{2} + y^{2})}{4} \psi
    \end{split}
\end{equation}

where the middle two terms vanish.

We can then solve the TISE:

\begin{equation}
    \frac{B^{2}l_{B}^{4}(x^{2} + y^{2}) - \hbar^{2}(x^{2} + y^{2} - 4l_{B}^{2})}{4l_{B}^{4}} \psi = E\psi
\end{equation}

We require that the coefficient for $x$ and $y$ to be zero, so that:

\begin{equation}
    l_{B} = \sqrt{\frac{\hbar}{B}}
\end{equation}

and the energy eigenvalue is:

\begin{equation}
    E = 4\hbar^{2}l_{B}^{2} = 4 \frac{\hbar^{3}}{B}
\end{equation}

Since the eigenfunction is real, only the second term in $\mathbf{j}$ contributes:

\begin{equation}
    \mathbf{j} = -\frac{qB}{2m} \exp \left( -\frac{x^{2} + y^{2}}{2l_{B}^{2}} \right) (-y, x, 0)^{\intercal}
\end{equation}

which is in the same direction as the magnetic field $\mathbf{B}$.

This describes the ground state of a charged particle in a magnetic field, which follows a helix trajectory around $\mathbf{B}$. The direction of $\mathbf{j}$ confirms this.
\qed


\problem{2}{}

Consider the Hamiltonian:

\begin{equation}
    \hat{H} = \hat{H}_{0} + \epsilon \hat{V}
    =
    \begin{pmatrix}
        A_{1} & 0     \\
        0     & A_{2}
    \end{pmatrix}
    +
    \epsilon
    \begin{pmatrix}
        B_{1} & B_{2} \\
        B_{2} & 0
    \end{pmatrix}
\end{equation}

where the first term is a simple Hamiltonian and the second term is a perturbation.

If $A_{1} \ne A_{2}$, the energy eigenvalues of the unperturbed system are non-degenerate. The first-order correction to the energy eigenvalues is:

\begin{equation}
    E_{n}^{(1)} = \mel{n}{\hat{V}}{n}
\end{equation}

where $\ket{n}$ are the energy eigenstates of $\hat{H}_{0}$ given by:

\begin{equation}
    \ket{1} = (1, 0)^{\intercal} \qquad \ket{2} = (0, 1)^{\intercal}
\end{equation}

Then the first-order corrections are:

\begin{equation}
    \begin{split}
        E_{1}^{(1)} &= \mel{1}{\hat{V}}{1} = B_{1} \\
        E_{2}^{(1)} &= \mel{2}{\hat{V}}{2} = 0
    \end{split}
\end{equation}

If $A_{1} = A_{2}$, the energy eigenvalues of the unperturbed system are degenerate. We can diagonalise $\hat{V}$ to find the perturbation matrix in the basis of the degenerate energy eigenstates:

\begin{equation}
    \hat{V} \to
    \frac{1}{2}
    \begin{pmatrix}
        B_{1} - \sqrt{B_{1}^{2} + 4B_{2}^{2}} & 0                                     \\
        0                                     & B_{1} + \sqrt{B_{1}^{2} + 4B_{2}^{2}}
    \end{pmatrix}
\end{equation}

so that the first-order corrections are:

\begin{equation}
    \begin{split}
        E_{1}^{(1)} &= \frac{1}{2} \left( B_{1} - \sqrt{B_{1}^{2} + 4B_{2}^{2}} \right) \\
        E_{2}^{(1)} &= \frac{1}{2} \left( B_{1} + \sqrt{B_{1}^{2} + 4B_{2}^{2}} \right)
    \end{split}
\end{equation}

We can also find the energy eigenstates from the Hamiltonian directly:



\problem{3}{}

\subproblem{a}
We have the perturbed Hamiltonian:

\begin{equation}
    \hat{H} = \hat{H}_{0} + \epsilon \hat{V} = \frac{1}{2} m \omega^{2} \hat{x}^{2} + \epsilon \hat{x}^{2}
\end{equation}

But the perturbed system is still a harmonic oscillator, so the energy eigenstates change exactly to:

\begin{equation}
    E'_{n} = \hbar (\kappa \omega)^{2} \left( n + \frac{1}{2} \right)
\end{equation}

where $\kappa = \sqrt{1 + 2\epsilon/m\omega^{2}}$.

We could expand $\kappa$ in terms of $\epsilon$ to second order:

\begin{equation}
    \kappa \approx 1 + \frac{\epsilon}{m\omega^{2}} - \frac{\epsilon^{2}}{2m^{2}\omega^{4}}
\end{equation}

so that the change in energy eigenvalues in the ground state is:

\begin{equation}
    \Delta E_{0} = E'_{0} - E_{0} = \frac{1}{2} \hbar \omega \left( \frac{\epsilon}{m\omega^{2}} - \frac{\epsilon^{2}}{2m^{2}\omega^{4}} \right)
\end{equation}

\subproblem{b}
Treating this as a perturbation problem, we have the ground state wave function of the unperturbed system:

\begin{equation}
    \psi_{0}(x) = \braket{x}{0} = \left( \frac{m\omega}{\pi\hbar} \right)^{1/4} \exp \left( -\frac{m\omega x^{2}}{2\hbar} \right)
\end{equation}

The first-order correction to the ground state energy is:

\begin{equation}
    \begin{split}
        E_{0}^{(1)} &= \mel{0}{\hat{V}}{0} \\
        &= \int \braket{0}{x} \mel{x}{\hat{V}}{0} \, \mathrm{d}x \\
        &= \int \psi_{0} x^{2} \psi_{0} \, \mathrm{d}x \\
        &= \frac{\hbar}{2m\omega}
    \end{split}
\end{equation}

which indeed agrees with the previous result up to first order.

\subproblem{c}
The first-order correction to the ground state is:

\begin{equation}
    \ket{0^{(1)}} = \sum_{m \ne 0} \frac{\mel{m}{\hat{V}}{0}}{E_{0} - E_{m}} \ket{m}
\end{equation}

Let us focus on the matrix element $\mel{m}{\hat{V}}{0}$:

\begin{equation}
    \begin{split}
        \mel{m}{\hat{V}}{0} &= \int \braket{m}{x} \mel{x}{\hat{V}}{0} \, \mathrm{d}x \\
        &= \int \psi_{m} x^{2} \psi_{0} \, \mathrm{d}x
    \end{split}
\end{equation}

We know that the energy eigenfunctions of a harmonic oscillator are Hermite polynomials of order $n$, so this integral is non-zero only when $m = 2$:

\begin{equation}
    \mel{2}{\hat{V}}{0} = \int \psi_{2} x^{2} \psi_{0} \, \mathrm{d}x = \frac{\hbar}{\sqrt{2}m\omega}
\end{equation}

which gives us the first-order correction to the ground state:

\begin{equation}
    \ket{0^{(1)}} = \frac{\mel{2}{\hat{V}}{0}}{E_{0} - E_{2}} \ket{2} = \frac{\hbar}{\sqrt{2}m\omega} \frac{1}{-2\hbar\omega} \ket{2} = -\frac{l^{2}}{\sqrt{2}\hbar \omega} \ket{2}
\end{equation}

where $l = \sqrt{\hbar/2m\omega}$ is the characteristic length scale of the harmonic oscillator.

\subproblem{d}
The second-order correction to the ground state energy is:

\begin{equation}
    E_{0}^{(2)} = \sum_{m \ne 0} \frac{\mel{0}{\hat{V}}{m} \mel{m}{\hat{V}}{0}}{E_{0} - E_{m}}
\end{equation}

Again, this sum is non-zero only when $m = 2$:

\begin{equation}
    E_{0}^{(2)} = \frac{\mel{0}{\hat{V}}{2} \mel{2}{\hat{V}}{0}}{E_{0} - E_{2}} = -\frac{\hbar}{4m^{2}\omega^{3}}
\end{equation}

which also agrees with the previous result up to second order.
\qed


\problem{4}{}
We have the disturbed Hamiltonian of the form:

\begin{equation}
    \hat{H} = \frac{1}{2m} \hat{p}^{2} + \frac{1}{2} m\omega^{2} \hat{x}^{2} + \epsilon \hat{x}
\end{equation}

Consider the operator $\hat{X} \equiv \hat{x} + \epsilon/m\omega^{2}$, we can rewrite the Hamiltonian as:

\begin{equation}
    \hat{H} = \frac{1}{2m} \hat{p}^{2} + \frac{1}{2} m\omega^{2} \hat{X}^{2} - \frac{\epsilon^{2}}{2m \omega^{2}}
\end{equation}

which is the Hamiltonian of a harmonic oscillator with a constant potential offset that does not affect the energy eigenvalues.

The energy eigenvalues of the perturbed system are thus unchanged:

\begin{equation}
    E_{n} = \hbar \omega \left( n + \frac{1}{2} \right)
\end{equation}

Treating instead the term $\epsilon \hat{x}$ as a perturbation, we have the first-order correction to the energy eigenvalues:

\begin{equation}
    E_{n}^{(1)} = \mel{n}{\hat{V}}{n} = \epsilon \mel{n}{\hat{x}}{n} = 0
\end{equation}

where we have used the fact that the energy eigenstates of a harmonic oscillator has zero position expectation.

This agrees with the previous result that the energy eigenvalues are unchanged up to first order. The component of the perturbed ground state along the unperturbed first excited state is:

\begin{equation}
    \begin{split}
        \braket{1}{0^{(1)}} &= \frac{\mel{1}{\hat{V}}{0}}{E_{0} - E_{1}} \\
        &= -\epsilon \frac{\mel{1}{\hat{x}}{0}}{\hbar \omega} \\
        &= -\epsilon \frac{1}{\sqrt{2} (\hbar m^{3} \omega^{7})^{1/4}}
    \end{split}
\end{equation}

The exact ground state wave function is:

\begin{equation}
    \psi'_{0}(x) = \left( \frac{m\omega}{\pi\hbar} \right)^{1/4} \exp \left[ -\frac{m\omega (x + \epsilon/m\omega^{2})^{2}}{2\hbar} \right]
\end{equation}

From a computer-assisted calculation, we have:

\begin{equation}
    \int \psi'_{0} \psi_{1} \, \mathrm{d}x = -\frac{\epsilon}{\sqrt{2} (\hbar m^{3} \omega^{7})^{1/4}} \exp \left( -\frac{\epsilon^{2}}{4\hbar m \omega^{3}} \right)
\end{equation}

which is identical with results from the perturbation theory up to first order.
\qed


\problem{5}{}
First note the following relation in a harmonic oscillator:

\begin{equation}
    \frac{l}{\sqrt{2}} (\hat{a} + \hat{a}^{\dagger}) = \hat{x}
\end{equation}

where we define $l \equiv \sqrt{\hbar/m\omega}$.

Given the perturbation $\hat{V} = \epsilon \hat{x}^{4}$, we write the first-order correction to the energy eigenvalues as:

\begin{equation}
    \begin{split}
        E_{n}^{(1)} &= \mel{n}{\hat{V}}{n} \\
        &= \epsilon \mel{n}{\hat{x}^{4}}{n} \\
        &= \epsilon \mel{n}{\left[ \frac{l}{\sqrt{2}} (\hat{a} + \hat{a}^{\dagger}) \right]^{4}}{n} \\
        &= \epsilon \frac{l^{4}}{4} \mel{n}{(\hat{a} + \hat{a}^{\dagger})^{4}}{n}
    \end{split}
\end{equation}

Let us write out the operator $(\hat{a} + \hat{a}^{\dagger})^{4}$:

\begin{equation}
    \begin{split}
        (\hat{a} + \hat{a}^{\dagger})^{4} &= \left[ (\hat{a} + \hat{a}^{\dagger})^{2} \right]^{2} \\
        &= \left( \hat{a}^{2} + \hat{a}^{\dagger 2} + \hat{a}\hat{a}^{\dagger} + \hat{a}^{\dagger}\hat{a} \right)^{2}
    \end{split}
\end{equation}

Apparently, the combination $(\hat{a} + \hat{a}^{\dagger})^{2}$ is Hermitian, so we consider the product:

\begin{equation}
    \begin{split}
        (\hat{a} + \hat{a}^{\dagger})^{2}\ket{n} &= \left( \hat{a}^{2} + \hat{a}^{\dagger 2} + \hat{a}\hat{a}^{\dagger} + \hat{a}^{\dagger}\hat{a} \right)\ket{n} \\
        &= \sqrt{n(n-1)} \ket{n-2} + \sqrt{(n+1)(n+2)} \ket{n+2} + (2n+1) \ket{n}
    \end{split}
\end{equation}

This implies:

\begin{equation}
    \begin{split}
        \mel{n}{(\hat{a} + \hat{a}^{\dagger})^{4}}{n} &= \mel{n}{(\hat{a} + \hat{a}^{\dagger})^{2}(\hat{a} + \hat{a}^{\dagger})^{2}}{n} \\
        &= n(n-1) + (n+1)(n+2) + (2n+1)^{2} \\
        &= 6n^{2} + 6n + 3
    \end{split}
\end{equation}

It follows that the first-order correction is:

\begin{equation}
    E_{n}^{(1)} = 3 (2n^{2} + 2n + 1) \epsilon \left( \frac{\hbar}{2m\omega} \right)^{2}
\end{equation}
\qed


\problem{6}{}
The unperturbed Hamiltonian is an infinite square well of width $2a$ centred at the origin. Its even-parity energy eigenstates have the position representation:

\begin{equation}
    \psi_{n}(x) = \sqrt{\frac{1}{a}} \cos{\left( \frac{n\pi x}{2a} \right)}
\end{equation}

where $n$ is an odd integer.

We also have the odd-parity energy eigenstates:

\begin{equation}
    \psi_{n}(x) = \sqrt{\frac{1}{a}} \sin{\left( \frac{n\pi x}{2a} \right)}
\end{equation}

where $n$ is an even integer.

In any case, the energy eigenvalues are:

\begin{equation}
    E_{n} = \frac{\hbar^{2} \pi^{2} n^{2}}{8ma^{2}} = n^{2} E_{1}
\end{equation}

Note that the ground state has even parity with the label $n = 1$:

\begin{equation}
    \psi_{1}(x) = \sqrt{\frac{1}{a}} \cos{\left( \frac{\pi x}{2a} \right)}
\end{equation}

Consider a perturbation of the form $\hat{V} = (\epsilon/a) \hat{x}$. The first-order correction to the energy eigenvalues is:

\begin{equation}
    E_{n}^{(1)} = \mel{n}{\hat{V}}{n} = \frac{\epsilon}{a} \mel{n}{\hat{x}}{n} = 0
\end{equation}

as the position expectation of the energy eigenstates of the infinite square well is zero.

The first-order correction to the ground state is given by:

\begin{equation}
    \begin{split}
        \ket{1^{(1)}} &= \sum_{j \ne 1} \frac{\mel{j}{\hat{V}}{1}}{E_{1} - E_{j}} \ket{j} \\
        &= \frac{\epsilon}{a} \frac{1}{E_{1}} \sum_{j \ne 1} \frac{\mel{j}{\hat{x}}{0}}{1 - j^{2}} \ket{j}
    \end{split}
\end{equation}

Let us consider the matrix element $\mel{j}{\hat{x}}{0}$. For odd $j = 1, 3, 5, \ldots$, representing the even-parity energy eigenstates, we have:

\begin{equation}
    \begin{split}
        \mel{j}{\hat{x}}{0} &= \int_{-a}^{a} \sqrt{\frac{1}{a}} \cos{\left( \frac{j\pi x}{2a} \right)} x \sqrt{\frac{1}{a}} \cos{\left( \frac{\pi x}{2a} \right)} \, \mathrm{d}x \\
        &= \frac{1}{a} \int_{-a}^{a} \cos{\left( \frac{j\pi x}{2a} \right)} x \cos{\left( \frac{\pi x}{2a} \right)} \, \mathrm{d}x \\
        &= \frac{4a}{\pi^{2}} \int_{-\pi/2}^{\pi/2} \cos{(ju)} u \cos{u} \, \mathrm{d}u \\
        &= 0
    \end{split}
\end{equation}

where we use the substitution $u = \pi x/2a$ and the fact that the integrand is odd.

On the other hand, for even $j = 2, 4, 6, \ldots$, representing the odd-parity energy eigenstates, we finally have:

\begin{equation}
    \begin{split}
        \mel{j}{\hat{x}}{0} &= \frac{4a}{\pi^{2}} \int_{-\pi/2}^{\pi/2} \sin{(ju)} u \cos{u} \, \mathrm{d}u \\
    \end{split}
\end{equation}

Denoting the integral as $I_{j}$, we have, after two integrations by parts:

\begin{equation}
    \begin{split}
        I_{j} &= -\int_{-\pi/2}^{\pi/2} \sin{(ju)} \sin{u} \, \mathrm{d}u - j \int_{-\pi/2}^{\pi/2} \cos{(ju)} \cos{u} \, \mathrm{d}u + j^{2} I_{j} \\
        (j^{2} - 1) I_{j} &= \int_{-\pi/2}^{\pi/2} \sin{(ju)} \sin{u} \, \mathrm{d}u + j \int_{-\pi/2}^{\pi/2} \cos{(ju)} \cos{u} \, \mathrm{d}u \\
        (j^{2} - 1) I_{j} &= -\frac{4j}{j^{2} - 1} (-1)^{j/2}
    \end{split}
\end{equation}

Combining the results, we have:

\begin{equation}
    \ket{1^{(1)}} = \frac{16\epsilon}{\pi^{2} E_{1}} \sum_{\text{even } j} (-1)^{j/2} \frac{j}{(j^{2} - 1)^{3}} \ket{j}
\end{equation}

In position representation, the perturbed ground state wave function is therefore:

\begin{equation}
    \psi'_{1}(x) = \sqrt{\frac{1}{a}} \cos{\left( \frac{\pi x}{2a} \right)} + \frac{16\epsilon}{\pi^{2} E_{1}} \sum_{\text{even } j} (-1)^{j/2} \frac{j}{(j^{2} - 1)^{3}} \sqrt{\frac{1}{a}} \sin{\left( \frac{j\pi x}{2a} \right)}
\end{equation}

\qed


\problem{7}{}
From electrodynamics, we know that the field in a uniformly charged sphere is linear:

\begin{equation}
    \mathbf{E} = \frac{1}{4\pi\epsilon} \frac{Ze}{R^{3}} \mathbf{r}
\end{equation}

We can work out the potential energy with infinity taken as the reference point:

\begin{equation}
    \begin{split}
        \Phi(r) &= -\int_{\infty}^{R} \frac{1}{4\pi\epsilon} \frac{Ze}{r^{2}} \, \mathrm{d}r - \int_{R}^{r} \frac{1}{4\pi\epsilon} \frac{Ze}{R^{3}} r \, \mathrm{d}r \\
        &= \frac{Ze}{4\pi\epsilon} \left( \frac{1}{R} - \frac{r^{2}}{2R^{3}} + \frac{1}{2R} \right) \\
        &= \frac{Ze}{4\pi\epsilon} \left( \frac{3}{2R} - \frac{r^{2}}{2R^{3}} \right) \\
        &= \frac{Ze}{4\pi\epsilon r} + \frac{Ze}{4\pi\epsilon} \left( \frac{3}{2R} - \frac{r^{2}}{2R^{3}} - \frac{1}{r} \right)
    \end{split}
\end{equation}

We can then identify the perturbation as:

\begin{equation}
    \hat{V} = \frac{Ze}{4\pi\epsilon} \left( \frac{3}{2a_{p}} - \frac{\hat{r}^{2}}{2a_{p}^{3}} - \frac{1}{\hat{r}} \right)
\end{equation}

which is present in the range $r \in [0, a_{p}]$.

The first order correction to the ground state energy is:

\begin{equation}
    \begin{split}
        E_{0}^{(1)} &= \mel{0}{\hat{V}}{0} \\
        &= \frac{Ze}{4\pi\epsilon} \int_{0}^{a_{p}} \psi_{0}^{2} \left( \frac{3}{2a_{p}} - \frac{r^{2}}{2a_{p}^{3}} - \frac{1}{r} \right) 4\pi r^{2} \, \mathrm{d}r \\
    \end{split}
\end{equation}

where the ground state wave function is:

\begin{equation}
    \psi_{0}(r) = \frac{1}{\sqrt{\pi a_{Z}^{3}}} e^{-r/a_{Z}}
\end{equation}

with $a_{Z} = a_{0}/Z$ is of the order $\qty{e-11}{m}$.

Note that $a_{p} \approx \qty{e-15}{m}$ is much smaller than $a_{Z}$, so that in the range of integration, we can approximate the exponential function as unity and the wave function as a constant, i.e., $\psi_{0}^{2} \approx (\pi a_{Z}^{3})^{-1}$. We simplify the integral:

\begin{equation}
    \begin{split}
        E_{0}^{(1)} &\approx \frac{Ze}{4\pi\epsilon} \int_{0}^{a_{p}} \frac{1}{\pi a_{Z}^{3}} \left( \frac{3}{2a_{p}} - \frac{r^{2}}{2a_{p}^{3}} - \frac{1}{r} \right) 4\pi r^{2} \, \mathrm{d}r \\
        &= \frac{Ze}{4\pi\epsilon} \frac{1}{\pi a_{Z}^{3}} \left( -\frac{2\pi a_{p}^{2}}{5} \right) \\
        &= -\frac{Ze}{4\pi\epsilon a_{Z}} \frac{2a_{p}^{2}}{5a_{Z}^{2}}
    \end{split}
\end{equation}

which is of order $\qty{e-8}{}$ compared to the typical potential energy $Ze/4\pi\epsilon a_{Z}$.

The shift is also linear in $Z$, so the effect is not very significant even for heavier hydrogen-like ions.
\qed


\problem{8}{}
A two dimensional harmonic oscillator has the Hamiltonian:

\begin{equation}
    \hat{H} = \hat{H}_{x} + \hat{H}_{y} = \frac{1}{2m} \left( \hat{p}_{x}^{2} + \hat{p}_{y}^{2} \right) + \frac{1}{2} m\omega^{2} \left( \hat{x}^{2} + \hat{y}^{2} \right)
\end{equation}

The energy eigenvalues are the sums of the energy eigenvalues of two one-dimensional harmonic oscillators:

\begin{equation}
    E_{n_{x}, n_{y}} = \hbar \omega \left( n_{x} + \frac{1}{2} \right) + \hbar \omega \left( n_{y} + \frac{1}{2} \right) = \hbar \omega \left( n_{x} + n_{y} + 1 \right)
\end{equation}

The first two excited states are:

\begin{equation}
    \begin{split}
        \ket{1, 0} &= \ket{1_{x}} \otimes \ket{0_{y}} \\
        \ket{0, 1} &= \ket{0_{x}} \otimes \ket{1_{y}}
    \end{split}
\end{equation}

In general, the nth excited state can be achieved if $n_{x} + n_{y} = n$ where $n_{x}, n_{y}$ both take non-negative integers. The number of such states is $n+1$.

Consider a perturbation of the form:

\begin{equation}
    \hat{V} = \lambda \hat{x}\hat{y} = \lambda \frac{l^{2}}{2} (\hat{a}_{x} + \hat{a}_{x}^{\dagger})(\hat{a}_{y} + \hat{a}_{y}^{\dagger})
\end{equation}

where $l = \sqrt{\hbar/m\omega}$.

The first-order correction to the energy associated with the state $\ket{1, 0}$ is:

\begin{equation}
    \begin{split}
        E_{1, 0}^{(1)} &= \mel{1, 0}{\hat{V}}{1, 0} \\
        &= \lambda \frac{l^{2}}{2} \mel{1, 0}{(\hat{a}_{x} + \hat{a}_{x}^{\dagger})(\hat{a}_{y} + \hat{a}_{y}^{\dagger})}{1, 0} \\
        &= \lambda \frac{l^{2}}{2}  \mel{1, 0}{\hat{a}_{x} \hat{a}_{y} + \hat{a}_{x} \hat{a}_{y}^{\dagger} + \hat{a}_{x}^{\dagger} \hat{a}_{y} + \hat{a}_{x}^{\dagger} \hat{a}_{y}^{\dagger}}{1, 0} \\
    \end{split}
\end{equation}


\problem{9}{}
In the Hamiltonian of a charged particle in a magnetic field, the terms involving $B$ arise from the definition of mechanical momentum operator:

\begin{equation}
    \hat{\Pi} = \hat{p} - q \mathbf{A}
\end{equation}

The Hamiltonian is then:

\begin{equation}
    \begin{split}
        \hat{H} &= \frac{1}{2m} \hat{\Pi}^{2} - \frac{e^{2}}{4\pi \epsilon_{0} \hat{r}} \\
        &= \frac{1}{2m} \hat{p}^{2} - \frac{e^{2}}{4\pi \epsilon_{0} \hat{r}} - \frac{e}{m} \hat{p} \cdot \mathbf{A} + \frac{e^{2}}{2m} \mathbf{A}^{2}
    \end{split}
\end{equation}

The last two terms describe the perturbation due to the magnetic field. Consider the simple case of a uniform magnetic field $\mathbf{B} = B\mathbf{z}$, we can choose the vector potential as $\mathbf{A} = (\mathbf{B} \times \mathbf{r})/2$, so that the perturbation is:

\begin{equation}
    \begin{split}
        \hat{V} &= -\frac{e}{m} \hat{p} \cdot \mathbf{A} + \frac{e^{2}}{2m} \mathbf{A}^{2} \\
        &= -\frac{e}{2m} \hat{p} \cdot (\mathbf{B} \times \hat{r}) + \frac{e^{2}}{8m} B^{2} (\hat{x}^{2} + \hat{y}^{2}) \\
        &= -\frac{e}{2m} \mathbf{B} \cdot (\hat{r} \times \hat{p}) + \frac{e^{2}}{8m} B^{2} (\hat{x}^{2} + \hat{y}^{2}) \\
        &= -\frac{e}{2m} B \hat{L}_{z} + \frac{e^{2}}{8m} B^{2} (\hat{x}^{2} + \hat{y}^{2})
    \end{split}
\end{equation}

where $\hat{L}_{z} = \hat{x}\hat{p}_{y} - \hat{y}\hat{p}_{x}$ is the $z$-component of the angular momentum operator.

We should also add to this the interaction between the spin and the magnetic field:

\begin{equation}
    -\mathbf{\mu} \cdot \mathbf{B} = -\frac{e}{m} B \hat{S}_{z}
\end{equation}

so that the full perturbation is:

\begin{equation}
    \hat{V} = -\frac{e}{2m} B (\hat{L}_{z} + 2\hat{S}_{z}) + \frac{e^{2}}{8m} B^{2} (\hat{x}^{2} + \hat{y}^{2})
\end{equation}

It is usually the case that the first term dominates the second term, so we may disregard the second order term in the perturbation.

The energy eigenstates of the Hamiltonian is labelled by four quantum numbers $n, l, m, s$.
\qed


\end{document}
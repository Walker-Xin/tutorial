\documentclass[12pt]{article}
\usepackage{homework}
\pagestyle{fancy}

% assignment information
\def\course{Optics}
\def\assignmentno{Problem Set III}
\def\assignmentname{}
\def\name{Xin, Wenkang}
\def\time{\today}

\begin{document}

\begin{titlepage}
    \begin{center}
        \large
        \textbf{\course}

        \vfill

        \Huge
        \textbf{\assignmentno}

        \vspace{1.5cm}

        \large{\assignmentname}

        \vfill

        \large
        \name

        \time
    \end{center}
\end{titlepage}


%==========
\pagebreak
\section*{}
%==========


\problem{9}{}
Given a fixed path difference, the intensity pattern of a Michelson interferometer is a series of bright and dark rings if the source is collimated. If the path difference is small enough, there will be few enough rings in the output plane that the fringes are almost straight. The fringes are localised at infinity as the output beams are parallel.
\qed


\problem{10}{}

\subproblem{a}
The order of the interference pattern is related to the path difference by:

\begin{equation}
    2kt\cos{\theta} = p \pi
\end{equation}

At $\theta = 0$, the equation becomes:

\begin{equation}
    2kt = p_{0} \pi
\end{equation}

\subproblem{b}
The condition $\alpha$ must fulfil is the above equation with $\theta = \alpha$.

\subproblem{c}
At angle $\alpha$, the order of interference is given by:

\begin{equation}
    p = \frac{2kt\cos{\alpha}}{\pi} = p_{0} \cos{\alpha} \le p_{0}
\end{equation}

\subproblem{d}
The radius of $p$th order ring is given by:

\begin{equation}
    r_{p} = f \tan{\theta_{p}} \approx f \theta_{p}
\end{equation}

On the other hand, we have:

\begin{equation}
    \frac{p \pi}{2kt} = \cos{\theta_{p}} \approx 1 - \frac{\theta_{p}^{2}}{2}
\end{equation}

which leads to:

\begin{equation}
    r_{p} \approx f \sqrt{2 - \frac{p\pi}{kt}}
\end{equation}

This leads to the difference:

\begin{equation}
    r_{p}^{2} - r_{p+1}^{2} = \frac{\lambda f^{2}}{2t}
\end{equation}

\subproblem{e}
As $t$ decreases, the rings become farther apart and no rings will be seen when $t = 0$. If the path difference continues to decrease, the rings will reappear, repeating the pattern.
\qed


\problem{11}{}

\subproblem{a}
The reflection if the light from the horizontal arm to output 1 experiences a phase change of $\pi$ due to the reflection at a glass-air interface. The reflection from the vertical arm to output 2 does not experience a phase change. This means that there is a phase difference of $\pi$ between the two beams so when output 1 is bright, output 2 is dark.

\subproblem{b}
Suppose that the input amplitude is $u_{0}$. Each traversal through the beamsplitter leads to a factor of $1/\sqrt{2}$. Output 1 has the amplitude:

\begin{equation}
    u_{1} = \frac{1}{2} u_{0} - \frac{1}{2} u_{0} e^{i \delta}
\end{equation}

where the second term, representing contribution from the horizontal arm, has a phase change of $\pi$ due to internal reflection.

Output 2 has the amplitude:

\begin{equation}
    u_{2} = \frac{1}{2} u_{0} + \frac{1}{2} u_{0} e^{i \delta}
\end{equation}

The intensities are:

\begin{equation}
    \begin{split}
        I_{1} = \frac{1}{2} u_{0}^{2} (1 - \cos{\delta}) \\
        I_{2} = \frac{1}{2} u_{0}^{2} (1 + \cos{\delta})
    \end{split}
\end{equation}

so that $I_{1} + I_{2} = u_{0}^{2}$ as required by conservation of energy.
\qed


\problem{12}{}
We have the angle sum formula:

\begin{equation}
    \begin{split}
        3\cos{(K_{1} x)} \cos{(K_{2} x)} - \sin{(K_{1} x)} \sin{(K_{2} x)} &= \frac{3}{2} \left[ \cos{[(K_{1} + K_{2})x]} + \cos{[(K_{1} - K_{2})x]} \right] \\
        &- \frac{1}{2} \left[ \cos{[(K_{1} + K_{2})x]} - \cos{[(K_{1} - K_{2})x]} \right] \\
        &= \cos{[(K_{1} + K_{2})x]} + 2 \cos{[(K_{1} - K_{2})x]}
    \end{split}
\end{equation}

that is:

\begin{equation}
    \begin{split}
        3 + 3\cos{(K_{1} x)} \cos{(K_{2} x)} - \sin{(K_{1} x)} \sin{(K_{2} x)} &= 1 + \cos{[(K_{1} + K_{2})x]} \\
        &+ 2 \left\{ 1 + \cos{[(K_{1} - K_{2})x]} \right\}
    \end{split}
\end{equation}

Therefore, we interpret the intensity pattern as the sum of two patterns, one with intensity $I_{0}$ of wave number $K_{1} + K_{2}$ and the other with intensity $2I_{0}$ of wave number $K_{1} - K_{2}$. The mean wave number is $K_{1}$ and the difference in wave numbers is $K_{2}$.

If the actual pattern is enveloped by a Gaussian of the form $\exp (-K_{3} x^{2})$, this indicates that there may be some Gaussian aperture limiting the pattern (a Gaussian retains form under Fourier transform). This may be due to the design of the detector or the source.
\qed


\problem{13}{}

\subproblem{a}
The output amplitudes in an etalon have the form:

\begin{equation}
    u_{0} (t_{1} t_{2}) (r_{1} r_{2} e^{i \delta})^{n}
\end{equation}

where $t_{i}$ and $r_{i}$ are the transmission and reflection coefficients and $\delta = 2kd \cos{\theta}$.

The total amplitude is then:

\begin{equation}
    u = u_{0} (t_{1} t_{2}) \sum_{n = 0}^{\infty} (r_{1} r_{2} e^{i \delta})^{n} = \frac{u_{0} (t_{1} t_{2})}{1 - r_{1} r_{2} e^{i \delta}}
\end{equation}

The intensity is then:

\begin{equation}
    \begin{split}
        I &= \abs{u}^{2} = \abs{\frac{u_{0} (t_{1} t_{2})}{1 - r_{1} r_{2} e^{i \delta}}}^{2} \\
        &= \frac{u_{0}^{2} (t_{1} t_{2})^{2}}{\abs{1 - r_{1} r_{2} e^{i \delta}}^{2}} \\
        &= \frac{u_{0}^{2} (t_{1} t_{2})^{2}}{1 + r_{1}^{2} r_{2}^{2} - 2 r_{1} r_{2} \cos{\delta}} \\
        &= \frac{u_{0}^{2} (t_{1} t_{2})^{2}}{1 + r_{1}^{2} r_{2}^{2} - 2 r_{1} r_{2} (1 - 2 \sin^{2}{(\delta/2)})} \\
        &\propto \frac{1}{1 + A \sin^{2}{(\delta/2)}}
    \end{split}
\end{equation}

where $A = 4R/(1 - R)^{2}$.

We define the finesse as:

\begin{equation}
    F = \frac{\pi}{2} \sqrt{A} = \frac{\pi \sqrt{R}}{1 - R}
\end{equation}

\subproblem{b}
For the reflection function, the amplitude are similar:

\begin{equation}
    u_{0} (t_{1} t_{2} r_{2}) (r_{1} r_{2} e^{i \delta})^{n}
\end{equation}

but there is an additional first reflection:

\begin{equation}
    u_{0}r_{1}
\end{equation}

In total, the amplitude is:

\begin{equation}
    u = u_{0} r_{1} + u_{0} (t_{1} t_{2} r_{2}) \sum_{n = 0}^{\infty} (r_{1} r_{2} e^{i \delta})^{n} = u_{0} \frac{t_{1} t_{2} r_{2} + r_{1}(1 - r_{1} r_{2} e^{i \delta})}{1 - r_{1} r_{2} e^{i \delta}}
\end{equation}

The sum of the intensities is then:

\begin{equation}
    \begin{split}
        \frac{I}{u_{0}^{2}} &= \abs{\frac{(t_{1} t_{2})}{1 - r_{1} r_{2} e^{i \delta}}}^{2} + \abs{\frac{t_{1} t_{2} r_{2} + r_{1}(1 - r_{1} r_{2} e^{i \delta})}{1 - r_{1} r_{2} e^{i \delta}}}^{2} \\
        &= \frac{(t_{1} t_{2})^{2} + (t_{1} t_{2} r_{2})^{2}}{1 + r_{1}^{2} r_{2}^{2} - 2 r_{1} r_{2} \cos{\delta}} \\
        &= 1
    \end{split}
\end{equation}

which proves that energy is conserved.
\qed


\problem{14}{}
Consider the intensity pattern of a Fabry-Perot etalon:

\begin{equation}
    I = \frac{I_{0}}{1 + A \sin^{2}{(\delta/2)}}
\end{equation}

where $A = 4R/(1 - R)^{2}$ and $\delta = 2kd \cos{\theta}$.

For maxima in $I$, we require $kd \cos{\theta} = p \pi$ or:

\begin{equation}
    1 - \frac{\theta_{p}^{2}}{2} = \frac{p \pi}{kd} = \frac{p}{2\nu d}
\end{equation}

where we have used the small angle approximation.

This leads to the expression for the diameter of the $p$-order ring:

\begin{equation}
    d_{p} = 2\theta_{p} L = 2L \sqrt{2 - \frac{p}{\nu d}}
\end{equation}

Suppose that the two components in the input beam are $\nu_{1} = \bar{\nu} + \Delta \nu/2$ and $\nu_{2} = \bar{\nu} - \Delta \nu/2$. Let us consider the lowest-order ring (largest), which has to belong to $\nu_{1}$ since $\nu_{1} > \nu_{2}$. We have:

\begin{equation}
    2L \sqrt{2 - \frac{p_{min, 1}}{\nu_{1} d}} = \qty{7.15}{mm}
\end{equation}

Now consider the next largest ring of diameter $\qty{6.60}{mm}$. There are two possibilities: it could be the next ring of $\nu_{1}$, in which case:

\begin{equation}
    2L \sqrt{2 - \frac{p_{min, 1} + 1}{\nu_{1} d}} = \qty{6.60}{mm}
\end{equation}

or it could be the first ring of $\nu_{2}$, in which case:

\begin{equation}
    2L \sqrt{2 - \frac{p_{min, 2}}{\nu_{2} d}} = \qty{6.60}{mm}
\end{equation}


\qed


\problem{15}{}
Assuming normal incidence, we have the expression for the phase difference:

\begin{equation}
    \delta = 2kd = 4\pi n d \nu
\end{equation}

For FWHM (full width at half maximum), we require $A \sin^{2}{(\delta/2)} = 1$ or:

\begin{equation}
    \sin^{2}{(\delta/2)} = \frac{1}{A} = \frac{4 \mathcal{F}^{2}}{\pi^{2}}
\end{equation}

In small angle approximation, we have:

\begin{equation}
    \Delta \delta_{\text{FWHM}} = \frac{2\pi}{\mathcal{F}}
\end{equation}

The free spectral range in $\delta$ is just the distance between two adjacent maxima, $2\pi$. Transforming to frequency via $\Delta \delta = 4\pi n d \Delta \nu$:

\begin{equation}
    \Delta \nu_{\text{FSR}} = \frac{1}{2d}
\end{equation}

We then transform the FWHM to frequency to obtain the instrument width:

\begin{equation}
    \Delta \nu_{\text{FWHM}} = \frac{1}{2d \mathcal{F}}
\end{equation}

The resolving power is then:

\begin{equation}
    RP = \frac{\nu}{\Delta \nu_{\text{FWHM}}} = \frac{p/2d}{1/2d \mathcal{F}} = \mathcal{F} p
\end{equation}

Since the resolving power does not depend on $n$, changing the medium will not affect the resolving power.
\qed

\end{document}
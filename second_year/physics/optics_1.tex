\documentclass[12pt]{article}
\usepackage{homework}
\pagestyle{fancy}

% assignment information
\def\course{Optics}
\def\assignmentno{Problem Set I}
\def\assignmentname{}
\def\name{Xin, Wenkang}
\def\time{\today}

\begin{document}

\begin{titlepage}
    \begin{center}
        \large
        \textbf{\course}

        \vfill

        \Huge
        \textbf{\assignmentno}

        \vspace{1.5cm}

        \large{\assignmentname}

        \vfill

        \large
        \name

        \time
    \end{center}
\end{titlepage}


%==========
\pagebreak
\section*{}
%==========


\problem{1}{}

\subproblem{a}
By law of refraction, we have the angle of refraction $\theta_1$ after the first incidence:

\begin{equation}
    \sin{\theta_{1}} = \frac{n_{1}}{n_{2}}\sin{\theta_{i}}
\end{equation}

which is just equal to $\theta_{r}$.

By geometry, we see that the incident angle at point B is just $\theta_{r}$, so that the angle of refraction $\theta_{2}$ after the second incidence is:

\begin{equation}
    \sin{\theta_{2}} = \frac{n_{2}}{n_{1}}\sin{\theta_{r}} = \frac{n_{2}}{n_{1}}\sin{\theta_{1}}  = \sin{\theta_{i}}
\end{equation}

This suggests that the two light rays are parallel to each other.

\subproblem{b}
Consider the length of $OB$:

\begin{equation}
    OB = 2d\tan{\theta_{r}}
\end{equation}

which means that the length of $OA$ is:

\begin{equation}
    OA = OB\cos{(\pi/2 - \theta_{i})} = 2d\tan{\theta_{r}}\sin{\theta_{i}} = 2d\frac{n_{2}}{n_{1}} \tan{\theta_{r}} \sin{\theta_{r}}
\end{equation}

which takes a time:

\begin{equation}
    t_{1} = \frac{OA}{c/n_{1}} = \frac{2d}{c} n_{2} \tan{\theta_{r}} \sin{\theta_{r}}
\end{equation}

The light ray in the second medium travels a distance:

\begin{equation}
    l = \frac{2d}{\cos{\theta_{r}}}
\end{equation}

which takes a time:

\begin{equation}
    t_{2} = \frac{l}{c/n_{2}} = \frac{2d}{c} n_{2} \sec{\theta_{r}}
\end{equation}

\subproblem{c}
The frequency of the light, which does not depend on the medium, is:

\begin{equation}
    f = \frac{c/n_{1}}{\lambda} = k_{0} \frac{c/n_{1}}{2\pi}
\end{equation}

Taking the incident medium as vacuum, we set $n_{1} = 1$. We have the angular frequency:

\begin{equation}
    \omega = 2\pi f = ck_{0}
\end{equation}

The phase difference due to the path difference is:

\begin{equation}
    \omega(t_{2} - t_{1}) = \omega \frac{2d}{c} n_{2} \left( \sec{\theta_{r}} - \tan{\theta_{r}} \sin{\theta_{r}} \right) = 2d k_{0} n_{2} \cos{\theta_{r}}
\end{equation}

The actual phase difference has an additional term $\pi$ due to the reflection at the second interface, so that the phase difference is:

\begin{equation}
    \Delta \phi = 2d k_{0} n_{2} \cos{\theta_{r}} + \pi
\end{equation}

\subproblem{d}
Confining $\theta_{i}$ to the range $[0, \pi/2]$, a larger $\theta_{i}$ corresponds to a larger $\theta_{r}$ and thus a lower $\cos{\theta_{r}}$. This means that the phase difference is smaller for larger $\theta_{i}$.

\subproblem{e}
The superposed wave due to the two light rays has the form:

\begin{equation}
    \begin{split}
        u &= u_{0} \cos{(\omega t + k_{0}x)} + u_{0} \cos{(\omega t + k_{0}x + \Delta \phi)} \\
        &= 2u_{0} \cos{\left( \frac{\Delta \phi}{2} \right)} \cos{\left( \omega t + k_{0}x + \frac{\Delta \phi}{2} \right)} \\
        &= 2u_{0} \cos{\left( \frac{\delta + \pi}{2} \right)} \cos{\left( \omega t + k_{0}x + \frac{\delta + \pi}{2} \right)} \\
        &= 2u_{0} \sin{\left( \frac{\delta}{2} \right)} \sin{\left( \omega t + k_{0}x + \frac{\delta}{2} \right)}
    \end{split}
\end{equation}

The (average) intensity of the superposed wave is:

\begin{equation}
    I = 4u_{0}^{2} \sin^{2}{\left( \frac{\delta}{2} \right)}
\end{equation}
\qed


\problem{2}{}

\subproblem{a}
Following the standard derivation, it is easy to show that in a classical young's double-slit experiment, the path difference between the light rays from the two slits is:

\begin{equation}
    \delta = \delta_{1} - \delta_{2} \approx d\sin{\theta} \approx \frac{dy}{D}
\end{equation}

where $y$ is the distance from the center of the screen to the point of interest.

The superposed wave has the form:

\begin{equation}
    \begin{split}
        u &= u_{0} e^{i(\omega t - k \delta_{1})} + u_{0} e^{i(\omega t - k \delta_{2})} \\
        &= u_{0} e^{i\omega t} \left( e^{-ik\delta_{1}} + e^{-ik\delta_{2}} \right) \\
    \end{split}
\end{equation}

The intensity of the superposed wave is:

\begin{equation}
    \begin{split}
        I &= \frac{1}{2} u^{*}u \\
        &= \frac{1}{2} u_{0}^{2} \left( e^{ik\delta_{1}} + e^{ik\delta_{2}} \right) \left( e^{-ik\delta_{1}} + e^{-ik\delta_{2}} \right) \\
        &= \frac{1}{2} u_{0}^{2} \left( 1 + e^{ik(\delta_{1} - \delta_{2})} + e^{ik(\delta_{2} - \delta_{1})} + 1 \right) \\
        &= u_{0}^{2} \left( 1 + 2\cos{k\delta} \right) \\
        &= 2I_{0} \left[ 1 + 2\cos{\left( \frac{2\pi d}{\lambda} \sin{\theta} \right)} \right]
    \end{split}
\end{equation}

where we have defined the intensity at the center of the screen as $2I_{0}$.

\subproblem{b}
Given the function:

\begin{equation}
    f(x) = \delta(x - d/2) + \delta(x + d/2)
\end{equation}

Its Fourier transform is:

\begin{equation}
    F(k) = \frac{1}{\sqrt{2\pi}} \int_{-\infty}^{\infty} f(x) e^{-ikx} \, \mathrm{d}x = \frac{1}{\sqrt{2\pi}} \left( e^{-ikd/2} + e^{ikd/2} \right) = \sqrt{\frac{2}{\pi}} \cos{\left( \frac{kd}{2} \right)}
\end{equation}

This is functionally equivalent to (varying part of) the intensity of the superposed wave in the previous part, which suggests that Fraunhofer diffraction is equivalent to a Fourier transform of the aperture function.

\subproblem{c}
Given the normalised square wave function:

\begin{equation}
    f(x) =
    \begin{cases}
        1/w & \text{if } \left\lvert x \right\rvert < w/2 \\
        0   & \text{otherwise}
    \end{cases}
\end{equation}

Its Fourier transform is:

\begin{equation}
    F(k) = \frac{1}{\sqrt{2\pi}} \int_{-w/2}^{w/2} \frac{1}{w} e^{-ikx} \, dx = \frac{1}{\sqrt{2\pi}} \frac{1}{ikw} \left( e^{ikw/2} - e^{-ikw/2} \right) = \sqrt{\frac{2}{\pi}} \frac{\sin{(kw/2)}}{kw/2}
\end{equation}

\subproblem{d}
The convolution theorem states that the Fourier transform of the convolution of $f$ and $g$ is the product of their Fourier transforms:

\begin{equation}
    \mathcal{F}(f * g) = \mathcal{F}(f) \cdot \mathcal{F}(g)
\end{equation}

\subproblem{e}
The Fourier transform of the convolution between two delta functions and a normalised square wave function is:

\begin{equation}
    \sqrt{\frac{2}{\pi}} \cos{\left( \frac{kd}{2} \right)} \cdot \sqrt{\frac{2}{\pi}} \frac{\sin{(kw/2)}}{kw/2} = \frac{2}{\pi} \cos{\left( \frac{kd}{2} \right)} \frac{\sin{(kw/2)}}{kw/2}
\end{equation}

This suggests that the intensity pattern due to a pair of finite slits of width $w$ has the form:

\begin{equation}
    I = I_{0} \cos{\left( \frac{kd}{2} \right)} \frac{\sin{(kw/2)}}{kw/2} + C
\end{equation}

where $C$ is some constant.
\qed


\problem{3}{}

\subproblem{a}
A triangular wave function of height $1/a$ and width $2a$ in the range $[0, 2a]$ is:

\begin{equation}
    g(x) =
    \begin{cases}
        x/a^2        & \text{if } 0 \leq x \leq a  \\
        (2a - x)/a^2 & \text{if } a \leq x \leq 2a \\
    \end{cases}
\end{equation}

The convolution of a square wave function of width $a$ with itself is:

\begin{equation}
    (f * f)(y) = \int f(x) f(y - x) \, \mathrm{d}x = \int f(x) f(x - y) \, \mathrm{d}x
\end{equation}

since the square wave function is symmetric about the origin.

This integration is zero if $\left\lvert x \right\rvert > a/2$, so that the convolution is non-zero only in the range $[-a, a]$. In the range $y \in [0, a]$, the convolution is:

\begin{equation}
    (f * f)(y) = \int_{-a/2+y}^{a/2} \frac{1}{a^{2}} \, \mathrm{d}x = \frac{a - y}{a^{2}}
\end{equation}

In the range $y \in [-a, 0]$, the convolution is:

\begin{equation}
    (f * f)(y) = \int_{-a/2}^{a/2+y} \frac{1}{a^{2}} \, \mathrm{d}x = \frac{a + y}{a^{2}}
\end{equation}

Overall, the convolution is:

\begin{equation}
    (f * f)(y) =
    \begin{cases}
        \frac{a - y}{a^{2}} & \text{if } 0 \leq y \leq a  \\
        \frac{a + y}{a^{2}} & \text{if } -a \leq y \leq 0 \\
    \end{cases}
\end{equation}

Consider the change of variable $y \to a - y$:

\begin{equation}
    (f * f)(y) =
    \begin{cases}
        \frac{y}{a^{2}}      & \text{if } 0 \leq y \leq a  \\
        \frac{2a - y}{a^{2}} & \text{if } a \leq y \leq 2a \\
    \end{cases}
\end{equation}

which is the same as the previous triangular wave function.

This suggests that a square wave function of width $2a$ at location $a$ can be convolved with itself to produce a triangular wave function.

\subproblem{b}
Consider the square wave function at location $a$:

\begin{equation}
    f(x) =
    \begin{cases}
        1/2a & \text{if } 0 \leq x \leq 2a \\
        0    & \text{otherwise}
    \end{cases}
\end{equation}

Its Fourier transform is:

\begin{equation}
    F(k) = \frac{1}{\sqrt{2\pi}} \int_{0}^{2a} \frac{1}{2a} e^{-ikx} \, \mathrm{d}x = \frac{1}{\sqrt{2\pi}} \frac{1}{2ika} \left( 1 - e^{-2ika} \right)
\end{equation}

This means that the Fourier transform of the triangular wave function is:

\begin{equation}
    G(k) = \left[ F(k) \right]^{2} = -\frac{1}{2\pi} \frac{1}{4k^{2}a^{2}} \left( 1 - e^{-2ika} \right)^{2}
\end{equation}

which is independent of $a$ up to second order in $ka$.

\subproblem{d}
The Fourier transform of the triangular wave function located at $-a$ can be obtained by considering the square wave function located at $-a$ with its Fourier transform:

\begin{equation}
    F(k) = \frac{1}{\sqrt{2\pi}} \frac{1}{2ika} \left( 1 - e^{2ika} \right)
\end{equation}

The Fourier transform of the left triangular wave is:

\begin{equation}
    G(k) = \left[ F(k) \right]^{2} = -\frac{1}{2\pi} \frac{1}{4k^{2}a^{2}} \left( 1 - e^{2ika} \right)^{2}
\end{equation}

This suggests that the Farunhofer diffraction pattern of a pair of triangular slits has the form:

\begin{equation}
    \begin{split}
        u &= -u_{0}\frac{1}{8\pi^{2} (ka\sin{\theta})^2} \left( 1 - e^{2ika\sin{\theta}} \right)^{2} - u_{0}\frac{1}{8\pi^{2} (ka\sin{\theta})^2} \left( 1 - e^{-2ika\sin{\theta}} \right)^{2} \\
        &= -u_{0}\frac{1}{8\pi^{2} \delta^{2}} \left[ \left( 1 - e^{2i\delta} \right)^{2} + \left( 1 - e^{-2i\delta} \right)^{2} \right] \\
        &= -u_{0}\frac{1}{4\pi^{2} \delta^{2}} \left( 1 - 2\cos{2\delta} + \cos{4\delta} \right)
    \end{split}
\end{equation}

where $\delta = ka\sin{\theta}$.

The intensity pattern is:

\begin{equation}
    I = u^{*}u = u_{0}^{2}\frac{1}{16 \pi^{2} \delta^{4}} \left( 1 - 2\cos{2\delta} + \cos{4\delta} \right)^{2} \approx u_{0}^{2}\frac{1}{16\pi^{2} \delta^{4}} \left( 16\delta^{4} - \frac{224}{3} \delta^{6} + \frac{656}{5} \delta^{8} \right)
\end{equation}

Taking the derivative of the intensity pattern with respect to $\delta$:

\begin{equation}
    \frac{\partial I}{\partial \delta} \propto -\frac{446}{3} \delta + \frac{2624}{5} \delta^{3}
\end{equation}

which suggests that the secondary maximum is at $\delta \approx 0.532$.

\subproblem{e}
For zero intensity, we satisfy the equation:

\begin{equation}
    1 - 2\cos{2\delta} + \cos{4\delta} = 2\cos^{2}{2\delta} - 2\cos{2\delta} = 0
\end{equation}

which suggests that the first minimum is at $\delta = \pi/4$.
\qed


\problem{4}{}

\subproblem{a}
Consider the two-dimensional square aperture:

\begin{equation}
    T(x, y) =
    \begin{cases}
        1/a^{2} & \text{if } \left\lvert x \right\rvert < a/2 \text{ and } \left\lvert y \right\rvert < a/2 \\
        0       & \text{otherwise}
    \end{cases}
\end{equation}

The Fourier transform of the aperture function is:

\begin{equation}
    \begin{split}
        T(k_{x}, k_{y}) &= \frac{1}{2\pi} \int_{-a/2}^{a/2} \int_{-a/2}^{a/2} \frac{1}{a^{2}} e^{-i(k_{x}x + k_{y}y)} \, \mathrm{d}x \, \mathrm{d}y \\
        &= \frac{1}{2\pi} \frac{1}{a^{2}} \int_{-a/2}^{a/2} e^{-ik_{x}x} \, \mathrm{d}x \int_{-a/2}^{a/2} e^{-ik_{y}y} \, \mathrm{d}y \\
        &= \frac{1}{2\pi} \frac{1}{a^{2}} \frac{1}{ik_{x}} \left( e^{ik_{x}a/2} - e^{-ik_{x}a/2} \right) \frac{1}{ik_{y}} \left( e^{ik_{y}a/2} - e^{-ik_{y}a/2} \right) \\
        &= \frac{2}{\pi a^{2} k_{x}k_{y}} \sin{\left( \frac{k_{x}a}{2} \right)} \sin{\left( \frac{k_{y}a}{2} \right)}
    \end{split}
\end{equation}

\subproblem{b}
The intensity pattern has the form:

\begin{equation}
    I = \frac{I_{0}}{\delta_{x}^{2} \delta_{y}^{2}} \sin^{2}{\left( \frac{\delta_{x}}{2} \right)} \sin^{2}{\left( \frac{\delta_{y}}{2} \right)}
\end{equation}

where $\delta_{x} = 2\pi a \sin{\theta_{x}}/\lambda$ and $\delta_{y} = 2\pi a \sin{\theta_{y}}/\lambda$.

The loci of the first minimum satisfy:

\begin{equation}
    \frac{2\pi}{\lambda} a \sin{\theta_{x}} = \frac{2\pi}{\lambda} a \sin{\theta_{y}} = \pi
\end{equation}

which suggests that the first minimum is at $\theta_{x} = \theta_{y} = \sin^{-1}{(\lambda/2a)}$.

The size of the central maximum is:

\begin{equation}
    s_{F} = \pi \left( D\tan{\theta_{x}} \right) \left( D\tan{\theta_{y}} \right) = \pi D^{2} \frac{\lambda^{2}}{4a^{2} - \lambda^{2}}
\end{equation}

\subproblem{c}
Based on geometric optics, the image of a small source due to the aperture is a square of side length:

\begin{equation}
    l = \frac{u + D}{u}a
\end{equation}

leading to a size of the image:

\begin{equation}
    s_{R} = \frac{(u + D)^{2}}{u^{2}} a^{2}
\end{equation}

\subproblem{d}
Assuming that the actual size of the image follows the form:

\begin{equation}
    s^{2} = s_{F}^{2} + s_{R}^{2} = \pi^{2} D^{4} \frac{\lambda^{4}}{\left( 4a^{2} - \lambda^{2} \right)^{2}} + \frac{(u + D)^{4}}{u^{4}} a^{4}
\end{equation}

We can find the minimum of $s^{2}$ by taking the derivative with respect to $a$:

\begin{equation}
    \frac{\partial s^{2}}{\partial a} = -16\pi^{2} D^{4} \frac{\lambda^{4}}{\left( 4a^{2} - \lambda^{2} \right)^{3}} a + 4\frac{(u + D)^{4}}{u^{4}} a^{3} = 0
\end{equation}

This leads to the approximate solution in the limit $D \gg u$ and $a \gg \lambda$:

\begin{equation}
    a \approx \left( \frac{\pi}{4} \right)^{1/4} \sqrt{\lambda u}
\end{equation}
\qed


\end{document}
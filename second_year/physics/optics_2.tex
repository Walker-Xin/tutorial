\documentclass[12pt]{article}
\usepackage{homework}
\pagestyle{fancy}

% assignment information
\def\course{Optics}
\def\assignmentno{Problem Set II}
\def\assignmentname{}
\def\name{Xin, Wenkang}
\def\time{\today}

\begin{document}

\begin{titlepage}
    \begin{center}
        \large
        \textbf{\course}

        \vfill

        \Huge
        \textbf{\assignmentno}

        \vspace{1.5cm}

        \large{\assignmentname}

        \vfill

        \large
        \name

        \time
    \end{center}
\end{titlepage}


%==========
\pagebreak
\section*{}
%==========


\problem{5}{}

\subproblem{a}
Since the magnification of a telescope is given by:

\begin{equation}
    M = -\frac{f_{o}}{f_{s}}
\end{equation}

we choose $f_{s} = \qty{18}{mm}$ for a better magnification.

\subproblem{b}
The magnification obtained is:

\begin{equation}
    M = -\frac{30}{18} = -\frac{5}{3}
\end{equation}

\subproblem{c}

\subproblem{d}
The angular resolution is limited by diffraction via the objective lens:

\begin{equation}
    \theta = 1.22 \frac{\lambda}{D}
\end{equation}

Taking $\lambda = \qty{500}{nm}$ for typical visible light, we have:

\begin{equation}
    \theta = \qty{1.22e-5}{rad}
\end{equation}

The size of the diffraction pattern is thus:

\begin{equation}
    \pi [\theta (f_{o} + f_{s})]^{2} = \pi (\qty{3.66e-6}{m})^{2} = \qty{4.20e-11}{m^{2}}
\end{equation}

Note that $\qty{3.66e-6}{m}$ is around twice of the spacing between two photoreceptors in the human eye, so the diffraction pattern is not resolved by the human eye.

\subproblem{e}
Diffraction by the human pupil has the angular resolution:

\begin{equation}
    \theta = 1.22 \frac{\lambda}{D_{p}} = \qty{6.1e-4}{rad}
\end{equation}

where we take $D_{p} = \qty{1}{mm}$ for the diameter of the pupil.

The size of the diffraction pattern is thus:

\begin{equation}
    \pi [\theta (f_{e})]^{2} = \pi (\qty{1.83e-5}{m})^{2} = \qty{3.35e-10}{m^{2}}
\end{equation}

where we take $f_{e} = \qty{3}{cm}$ for the focal length of the eye.
\qed


\problem{6}{}

\subproblem{a}
From geometry, the path difference between a ray hitting location $x$ and the ray hitting the centre of the glass is:

\begin{equation}
    \delta = x(\sin{\theta_{i}} - \sin{\theta})
\end{equation}

The diffraction pattern is then the Fourier transform of a constant transmission function:

\begin{equation}
    \begin{split}
        u(x) &= \int_{-D/2}^{D/2} u_{0} e^{i k \delta} \, \mathrm{d}x \\
        &= u_{0} \int_{-D/2}^{D/2} e^{i k x (\sin{\theta_{i}} - \sin{\theta})} \, \mathrm{d}x \\
        &= u_{0} \frac{1}{k (\sin{\theta_{i}} - \sin{\theta})} 2 \sin{[kD (\sin{\theta_{i}} - \sin{\theta})/2]} \\
        &= u_{0}D \operatorname{sinc}{\left[ kD (\sin{\theta_{i}} - \sin{\theta})/2 \right]}
    \end{split}
\end{equation}

The intensity of the diffraction pattern is then:

\begin{equation}
    I(x) = I_{0}D^{2} \operatorname{sinc}^{2}{\left[ kD (\sin{\theta_{i}} - \sin{\theta})/2 \right]}
\end{equation}

\subproblem{b}
The sinc function has its maximum when the argument is zero, i.e., $\sin{\theta_{i}} = \sin{\theta}$ which agrees with law of reflection.

\subproblem{c}
The angular displacement of the first minimum satisfies:

\begin{equation}
    \sin{\theta} - \sin{\theta_{i}} = \frac{2\pi}{kD} = \frac{\lambda}{D}
\end{equation}

Consider $\lambda = \qty{500}{nm}$ and $D = \qty{5}{cm}$, which gives:

\begin{equation}
    \Delta \theta \approx \frac{\lambda}{D} = \qty{1e-5}{rad}
\end{equation}

which is too small an angle to be resolved by the human eye.
\qed


\problem{7}{}

\subproblem{a}
The transmission function of the grating is:

\begin{equation}
    u_{0} \sum_{r = 0}^{N/2 - 1} \delta[x - (r + \frac{1}{2})d] + \delta[x + (r + \frac{1}{2})d]
\end{equation}

The diffraction pattern is then:

\begin{equation}
    \begin{split}
        u(x) &= u_{0} \int \sum_{r = 0}^{N/2 - 1} \delta[x - (r + \frac{1}{2})d] + \delta[x + (r + \frac{1}{2})d] e^{i k x\sin{\theta}} \, \mathrm{d}x \\
        &= u_{0} \sum_{r = 0}^{N/2 - 1} \left[ e^{i k (r + \frac{1}{2})d \sin{\theta}} + e^{-i k (r + \frac{1}{2})d \sin{\theta}} \right] \\
        &= 2 u_{0} \sum_{r = 0}^{N/2 - 1} \cos{\left[ (r + \frac{1}{2}) \delta \right]} \\
        &= u_{0} \frac{\sin{(N\delta/2)}}{\sin{(\delta/2)}} \\
    \end{split}
\end{equation}

so that the intensity pattern is:

\begin{equation}
    I(\delta) = \frac{I_{0}}{N^{2}} \frac{\sin^{2}{(N\delta/2)}}{\sin^{2}{(\delta/2)}}
\end{equation}

\subproblem{b}
The maxima satisfy:

\begin{equation}
    \sin{\theta_{p}} = \left( n + \frac{1}{2} \right) \frac{\lambda}{Nd}
\end{equation}

whereas the minima satisfy:

\begin{equation}
    \sin{\theta_{m}} = n \frac{\lambda}{Nd}
\end{equation}

This implies that the distance between adjacent maximum and minimum satisfies:

\begin{equation}
    \sin{(\theta_{p} + \Delta \theta)} - \sin{\theta_{p}} = \frac{\lambda}{Nd}
\end{equation}

Assuming small $\Delta \theta$, we have:

\begin{equation}
    \Delta \theta = \frac{\lambda}{Nd \cos{\theta_{p}}}
\end{equation}

\subproblem{c}
Compare this with the width from the previous problem $\lambda/D$, the extra factor of $1/N$ ensures that the central maximum is much smaller.

\subproblem{d}


\problem{8}{}

\subproblem{a}
For a reflection grating of distance $d$ between lines, the path difference is given by:

\begin{equation}
    d (\cos{\theta_{i}} - \cos{\theta_{r}})
\end{equation}

where $\theta_{i}$ and $\theta_{r}$ are the incident and reflected angles, respectively.

Second order maxima satisfy:

\begin{equation}
    d (\cos{\theta_{i}} - \cos{\theta_{r}}) = \lambda
\end{equation}

so that the reflected angle is:

\begin{equation}
    \cos{\theta_{r}} = \cos{\theta_{i}} - \frac{\lambda}{d}
\end{equation}

\subproblem{b}
If the diffracting elements are inclined at an angle $\theta_{B}$, the reflection angle becomes $\theta_{i} - \theta_{B}$, so that the path difference becomes:

\begin{equation}
    d [\cos{\theta_{i}} - \cos{(\theta_{i} - \theta_{B})}]
\end{equation}

The second order maxima satisfy:

\begin{equation}
    d [\cos{\theta_{i}} - \cos{(\theta_{i} - \theta_{B})}] = \lambda
\end{equation}

\subproblem{c}
The maximum range of $\theta_{i}$ is $\pi/2$, so that the maximum range of the reflected angle is $\pi/2 - \theta_{B}$. The longest wavelength that can be measured in the second order is then:

\begin{equation}
    \lambda_{\text{max}} = -d \cos{(\pi/2 - \theta_{B})} = d \sin{\theta_{B}}
\end{equation}



\end{document}
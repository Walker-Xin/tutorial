\documentclass[12pt]{article}
\usepackage{homework}
\pagestyle{fancy}

% assignment information
\def\course{Statistical Mechanics}
\def\assignmentno{Problem Sheet 4}
\def\assignmentname{Relativistic and Fermi Gases}
\def\name{Xin, Wenkang}
\def\time{\today}

\begin{document}

\begin{titlepage}
    \begin{center}
        \large
        \textbf{\course}

        \vfill

        \Huge
        \textbf{\assignmentno}

        \vspace{1.5cm}

        \large{\assignmentname}

        \vfill

        \large
        \name

        \time
    \end{center}
\end{titlepage}


%==========
\pagebreak
\section*{Relativistic and Fermi gases}
%==========


\problem{4.1}{}

\subproblem{i}
For relativistic gas, the energy of a particle is given by:

\begin{equation}
    E = \sqrt{p^{2} c^{2} + m^{2} c^{4}} = \sqrt{\hbar^{2} k^{2} c^{2} + m^{2} c^{4}}
\end{equation}

where $k_{i} = \pi n_{i} / L$ are quantised wave numbers.

Proceeding as in the ordinary ideal gas case, we have the single particle partition function:

\begin{equation}
    \begin{split}
        Z_{1} &= \sum_{n} e^{-\beta E_{n}} \\
        &\approx \int e^{-\beta \sqrt{\hbar^{2} k^{2} c^{2} + m^{2} c^{4}}} g(k) \, \mathrm{d}k
    \end{split}
\end{equation}

where $g(k) = Vk^{2}/2\pi^{2}$ is the usual density of states.

The overall partition function is still $Z = Z_{1}^{N}/N!$ such that its logarithm is:

\begin{equation}
    \begin{split}
        \ln{Z} &= N \ln{Z_{1}} - \ln{N!} \\
        &\approx N \left( \ln{Z_{1}} - \ln{N} - 1 \right) \\
    \end{split}
\end{equation}

We may compute the pressure from the free energy:

\begin{equation}
    \begin{split}
        P &= -\left( \frac{\partial F}{\partial V} \right)_{T, N} \\
        &= k_{B} T \left( \frac{\partial \ln{Z}}{\partial V} \right)_{T, N} \\
        &= N k_{B} T \left( \frac{\partial \ln{Z_{1}}}{\partial V} \right)_{T, N} \\
        &= \frac{N k_{B} T}{V}
    \end{split}
\end{equation}

where the last equality follows because $Z_{1}$ is proportional to $V$.

We see that the equation of state $PV = Nk_{B}T$ is the same as for the non-relativistic case.

\subproblem{ii}

\qed


\problem{4.2}{}
We have the expression for Fermi energy:

\begin{equation}
    \varepsilon_{F} = \frac{\hbar^{2} k_{F}^{2}}{2m} = \frac{\hbar^{2}}{2m} \left( \frac{6\pi^{2} n}{2s + 1} \right)^{2/3}
\end{equation}

and we define the Fermi temperature as:

\begin{equation}
    T_{F} = \frac{\varepsilon_{F}}{k_{B}}
\end{equation}

\subproblem{a}
For liquid helium, we have $m = 3u$, $s = 1/2$ and $n = \rho/m$. Thus:

\begin{equation}
    \varepsilon_{F} = \frac{\hbar^{2}}{2m} \left( \frac{6\pi^{2} \rho}{2m} \right)^{2/3} = \qty{6.87e-27}{J}
\end{equation}

and:

\begin{equation}
    T_{F} = \frac{\varepsilon_{F}}{k_{B}} = \qty{4.98e-4}{K}
\end{equation}

\subproblem{b}
For electrons in aluminium, we have $m = \qty{9.11e-31}{kg}$, $s = 1/2$ and $n = \rho/m$:

\begin{equation}
    \varepsilon_{F} = \qty{1.66e-19}{J}
\end{equation}
\qed


\problem{4.3}{}

\subproblem{i}
Consider the Fermi-Dirac distribution:

\begin{equation}
    \bar{n}_{i} = \frac{1}{e^{\beta (\varepsilon_{i} - \mu)} + 1}
\end{equation}

At absolute zero, the distribution function is a step function that is zero for $\varepsilon_{i} > \mu$ and unity for $\varepsilon_{i} < \mu$. We define the Fermi energy as $\varepsilon_{F} = \mu(T = 0)$. The total energy is then:

\begin{equation}
    \begin{split}
        U &= \sum_{i} \varepsilon_{i} \bar{n}_{i} \\
        &\approx \int g(\varepsilon) \frac{\varepsilon}{e^{\beta (\varepsilon - \mu)} + 1} \, \mathrm{d}\varepsilon \\
        &= \int_{0}^{\varepsilon_{F}} g(\varepsilon) \varepsilon \, \mathrm{d}\varepsilon
    \end{split}
\end{equation}

The mean energy per particle is then:

\begin{equation}
    \begin{split}
        u &= \frac{U}{N} \\
        &= \frac{\int_{0}^{\varepsilon_{F}} g(\varepsilon) \varepsilon \, \mathrm{d}\varepsilon}{\int_{0}^{\varepsilon_{F}} g(\varepsilon) \, \mathrm{d}\varepsilon} \\
        &= \frac{3}{5} \varepsilon_{F}
    \end{split}
\end{equation}

since $g(\varepsilon) \propto \varepsilon^{1/2}$.

Thus the total energy is just $U = 3N \varepsilon_{F}/5$.

\subproblem{ii}
The partition function of a Fermi gas is:

\begin{equation}
    \ln{\mathcal{Z}} = \sum_{i} \ln{\left[ 1 \pm e^{-\beta (\varepsilon_{i} - \mu)} \right]}
\end{equation}



We could compute the grand potential $\Phi$ from the partition function $\mathcal{Z}$:

\begin{equation}
    \begin{split}
        \Phi &= -k_{B}T \ln{\mathcal{Z}} \\
        &= -\frac{1}{\beta} \sum_{i} \ln{\left[ 1 + e^{-\beta (\varepsilon_{i} - \mu)} \right]} \\
        &= -\frac{1}{\beta} \int_{0}^{\infty} g(\varepsilon) \ln{\left[ 1 + e^{-\beta (\varepsilon - \mu)} \right]} \, \mathrm{d}\varepsilon \\
        &= -\frac{2(2s+1)}{\sqrt{\pi}} \frac{V}{\lambda_{\text{th}}^3} \sqrt{\beta} \int_{0}^{\infty} \sqrt{\varepsilon} \ln{\left[ 1 + e^{-\beta (\varepsilon - \mu)} \right]} \, \mathrm{d}\varepsilon
    \end{split}
\end{equation}

The integral can be solved by parts:

\begin{equation}
    \begin{split}
        \int_{0}^{\infty} \sqrt{x} \ln{\left[ 1 + e^{-x + \mu} \right]} \, \mathrm{d}x &= \left[ \frac{2}{3} x^{3/2} \ln{\left[ 1 + e^{-x + \mu} \right]} \right]_{0}^{\infty} - \int_{0}^{\infty} \frac{2}{3} x^{3/2} \frac{-e^{-x + \mu}}{1 + e^{-x + \mu}} \, \mathrm{d}x \\
        &= \frac{2}{3} \int_{0}^{\infty} x^{3/2} \frac{e^{-x + \mu}}{1 + e^{-x + \mu}} \, \mathrm{d}x \\
        &= \frac{2}{3} \int_{0}^{\infty} x^{3/2} \frac{1}{e^{x - \mu} + 1} \, \mathrm{d}x
    \end{split}
\end{equation}

so that we finally obtain:

\begin{equation}
    \Phi = -\frac{2}{3} \frac{2(2s+1)}{\sqrt{\pi}} \frac{V}{\lambda_{\text{th}}^3} \frac{1}{\beta} \int_{0}^{\infty} \frac{x^{3/2}}{e^{x - \beta \mu} \pm 1} \, \mathrm{d}x
\end{equation}

This turns out to be simply $-2U/3$. Since pressure is grand potential per unit volume, we have:

\begin{equation}
    P = -\frac{\Phi}{V} = \frac{2U}{3V} = \frac{2}{3} \frac{N \varepsilon_{F}}{V}
\end{equation}
\qed


\problem{4.4}{}

\subproblem{i}
For non-relativistic gas, the energy of a particle is given by:

\begin{equation}
    E = \frac{p^{2}}{2m} = \frac{\hbar^{2} k^{2}}{2m}
\end{equation}

where for each energy, the degeneracy is $(2s + 1)$ due to the spin.

The sum over single particle states can be approximated by an integral:

\begin{equation}
    \begin{split}
        \sum_{i} &= (2s + 1) \sum_{k} \\
        &\approx (2s + 1) \frac{V}{(2\pi)^{3}} \int \, \mathrm{d}^{3}k \\
        &= (2s + 1) \frac{V}{(2\pi)^{3}} \int_{0}^{\infty} 4\pi k^{2} \, \mathrm{d}k \\
        &= \int_{0}^{\infty} g(k) \, \mathrm{d}k
    \end{split}
\end{equation}

where we define:

\begin{equation}
    g(k) \equiv (2s + 1) \frac{Vk^{2}}{2\pi^{2}}
\end{equation}

For electrons, we have $s = 1/2$ and $g(k) \, \mathrm{d}k = 2 \times Vk^{2} \, \mathrm{d}k / 2\pi^{2}$.

If we instead consider a two-dimensional gas, the approximation becomes:

\begin{equation}
    \begin{split}
        \sum_{i} &= (2s + 1) \sum_{k} \\
        &\approx (2s + 1) \frac{A}{(2\pi)^{2}} \int \, \mathrm{d}^{2}k \\
        &= (2s + 1) \frac{A}{(2\pi)^{2}} \int k \, \mathrm{d}k \mathrm{d}\theta \\
        &= \int h(k) \, \mathrm{d}k
    \end{split}
\end{equation}

where the new density of states is:

\begin{equation}
    h(k) \equiv (2s + 1) \frac{Ak}{2\pi}
\end{equation}

Consider the change of variable from $k$ to $\epsilon = \hbar^{2} k^{2}/2m$:

\begin{equation}
    h(k) \, \mathrm{d}k = \tilde{h}(k) \, \mathrm{d}\epsilon
\end{equation}

where:

\begin{equation}
    \tilde{h}(\epsilon) = (2s + 1) \frac{Am}{2\pi\hbar^{2}}
\end{equation}

For this two-dimensional gas, we can compute the total particle number:

\begin{equation}
    \begin{split}
        N &= \int \bar{n} \tilde{h}(\epsilon) \, \mathrm{d}\epsilon \\
        &= (2s + 1) \frac{Am}{2\pi\hbar^{2}} \int_{0}^{\epsilon_{F}} \frac{1}{e^{\beta (\epsilon - \mu)} + 1} \, \mathrm{d}\epsilon \\
        &= (2s + 1) \frac{Am}{2\pi\hbar^{2}} \epsilon_{F}
    \end{split}
\end{equation}

which gives the Fermi energy for a two-dimensional electron gas with spin $s = 1/2$:

\begin{equation}
    \epsilon_{F} = \frac{\pi\hbar^{2} N}{m A}
\end{equation}

\subproblem{ii}
With $n = \qty{4e17}{m^{-2}}$ and $m = 0.15 \times \qty{9.11e-31}{kg}$, we have the Fermi energy:

\begin{equation}
    \epsilon_{F} = \qty{1.01e-19}{J}
\end{equation}

\subproblem{iii}
For an one-dimensional gas, the approximation becomes:

\begin{equation}
    \begin{split}
        \sum_{i} &= (2s + 1) \sum_{k} \\
        &\approx (2s + 1) \frac{L}{2\pi} \int \, \mathrm{d}k \\
        &= \int u(k) \, \mathrm{d}k
    \end{split}
\end{equation}

where the density of state is $u(k) = (2s + 1) L / 2\pi$.

Changing variable to $\epsilon = \hbar^{2} k^{2}/2m$, we have:

\begin{equation}
    \tilde{u}(\epsilon) = (2s + 1) \frac{L \sqrt{2m}}{4\pi \hbar} \epsilon^{-1/2}
\end{equation}

so that the total particle number is:

\begin{equation}
    \begin{split}
        N &= \int \bar{n} \tilde{u}(\epsilon) \, \mathrm{d}\epsilon \\
        &= (2s + 1) \frac{L \sqrt{2m}}{4\pi \hbar} \int_{0}^{\epsilon_{F}} \frac{1}{e^{\beta (\epsilon - \mu)} + 1} \epsilon^{-1/2} \, \mathrm{d}\epsilon \\
        &= (2s + 1) \frac{L \sqrt{2m}}{2\pi \hbar} \epsilon_{F}^{1/2}
    \end{split}
\end{equation}

which gives the Fermi energy for a one-dimensional electron gas:

\begin{equation}
    \epsilon_{F} = \frac{\pi^{2} \hbar^{2} }{2m} \left( \frac{N}{L} \right)^{2}
\end{equation}

\subproblem{iv}
For the given long-chain molecule, we have $n = 0.5/\qty{1e-10}{m} = \qty{5e9}{m^{-1}}$. Thus the Fermi energy is:

\begin{equation}
    \epsilon_{F} = \qty{9.95e-18}{J}
\end{equation}
\qed


\problem{4.5}{}

\subproblem{i}
If most of the particles are ultra-relativistic, we change the energy of a particle to:

\begin{equation}
    \epsilon = pc = \hbar k c
\end{equation}

The density of states in $k$-space is still $g(k) = (2s + 1) V k^{2} / 2\pi^{2}$, but the switch to $\epsilon$-space is altered:

\begin{equation}
    \tilde{g}(\epsilon) = (2s + 1) \frac{V}{2\pi^{2}(\hbar c)^{3}} \epsilon^{2}
\end{equation}

The total number of particles is:

\begin{equation}
    \begin{split}
        N &= \int \bar{n} \tilde{g}(\epsilon) \, \mathrm{d}\epsilon \\
        &= (2s + 1) \frac{V}{2\pi^{2}(\hbar c)^{3}} \int_{0}^{\epsilon_{F}} \frac{\epsilon^{2}}{e^{\beta (\epsilon - \mu)} + 1} \, \mathrm{d}\epsilon \\
        &= (2s + 1) \frac{V}{2\pi^{2}(\hbar c)^{3}} \frac{1}{3} \epsilon_{F}^{3}
    \end{split}
\end{equation}

so that the Fermi energy for an ultra-relativistic electron gas is:

\begin{equation}
    \epsilon_{F} = \hbar c (3n \pi^{2})^{1/3} = hc \left( \frac{3n}{8\pi} \right)^{1/3}
\end{equation}

\subproblem{ii}
The total energy is:

\begin{equation}
    \begin{split}
        U &= \int \epsilon \bar{n} \tilde{g}(\epsilon) \, \mathrm{d}\epsilon \\
        &= (2s + 1) \frac{V}{2\pi^{2}(\hbar c)^{3}} \int_{0}^{\epsilon_{F}} \frac{\epsilon^{3}}{e^{\beta (\epsilon - \mu)} + 1} \, \mathrm{d}\epsilon \\
        &= (2s + 1) \frac{V}{2\pi^{2}(\hbar c)^{3}} \frac{1}{4} \epsilon_{F}^{4}
    \end{split}
\end{equation}

which means that the energy per particle is:

\begin{equation}
    u = \frac{U}{N} = \frac{3}{4} \epsilon_{F}
\end{equation}

and the energy density is just $3n \epsilon_{F}/4$.
\qed


\problem{4.6}{}

\subproblem{i}
For a sphere of mass $M$ and radius $R$, the gravitational field inside the sphere is linear due to Gauss' law:

\begin{equation}
    g = -\frac{GM}{R^{3}} r
\end{equation}

whereas outside the sphere, the field is:

\begin{equation}
    g = -\frac{GM}{r^{2}}
\end{equation}

The potential energy of the star is found by integrating the energy density $u = -g^{2}/8\pi G$ over the entire space:

\begin{equation}
    \begin{split}
        U_{\text{grav}} &= \int_{\text{all space}} u \, \mathrm{d}V \\
        &= -\frac{1}{8\pi G} \left[ \int_{0}^{R} \left( \frac{GM}{R^{3}} \right)^{2} r^{2} 4\pi r^{2} \, \mathrm{d}r + \int_{R}^{\infty} \left( \frac{GM}{r^{2}} \right)^{2} 4\pi r^{2} \, \mathrm{d}r \right] \\
        &= -\frac{3}{5} \frac{GM^{2}}{R}
    \end{split}
\end{equation}

\subproblem{ii}
Suppose that in a white dwarf star, there are $N$ electrons, $N$ protons and $N$ neutrons. The total mass of the star is $M = 2Nm_{p}$ as we neglect the mass of the electrons. The total Fermi energy of the electrons is:

\begin{equation}
    \begin{split}
        \frac{3}{5} N \varepsilon_{F} &= \frac{3}{5} N \left[ \frac{\hbar^{2}}{2m_{e}} \left( 3\pi^{2} n \right)^{2/3} \right] \\
        &= \frac{3}{5} N \left[ \frac{\hbar^{2}}{2m_{e}} \left( \frac{9\pi^{2} N}{4\pi R^{3}} \right)^{2/3} \right] \\
        &= 0.0088 \frac{h^{2}M^{5/3}}{m_{e}m_{p}^{5/3}R^{2}}
    \end{split}
\end{equation}

\subproblem{iii}
The total energy of the star is an inverse square function of the radius $R$ minus an inverse function:

\begin{equation}
    U_{\text{tot}} = 0.0088 \frac{h^{2}M^{5/3}}{m_{e}m_{p}^{5/3}R^{2}} - \frac{3}{5} \frac{GM^{2}}{R}
\end{equation}

To minimise the energy, we differentiate with respect to $R$ and set the result to zero:

\begin{equation}
    R_{\text{min}} = 0.0088 \times \frac{10h^{2}}{3m_{e}m_{p}^{5/3}G} M^{-1/3}
\end{equation}

\subproblem{iv}
With $M = \qty{2e30}{kg}$, we have $R = \qty{2.34e3}{km}$ which is of the same order as the radius of Earth $\qty{6.37e3}{km}$.

\subproblem{v}
At $R_{\text{min}}$, the Fermi energy is $\varepsilon_{F} = \qty{1e44}{J}$ which is too large for the non-relativistic approximation to be valid. Rather, the ultra-relativistic approximation would be more appropriate.
\qed


\problem{4.7}{}

\subproblem{i}
In the ultra-relativistic case, the Fermi energy is:

\begin{equation}
    \begin{split}
        \varepsilon_{F} &= h c \left( \frac{3n}{8\pi} \right)^{1/3} \\
        &= hc \left( \frac{9N}{32\pi^{2} R^{3}} \right)^{1/3} \\
    \end{split}
\end{equation}

which scales as $R^{-1}$.

\subproblem{ii}
If the Fermi energy is the same order as the rest energy of an electron, we need:

\begin{equation}
    \begin{split}
        \varepsilon_{F} &= m_{e} c^{2} \\
        hc \left( \frac{9M}{64\pi^{2} m_{p} R^{3}} \right)^{1/3} &= m_{e} c^{2} \\
        M &= \frac{64\pi^{2} m_{p} R^{3}}{9} \left( \frac{m_{e} c}{h} \right)^{3}
    \end{split}
\end{equation}

Taking $R \approx \qty{2000}{km}$, we have $M \approx \qty{e30}{kg}$, which is of the same order as the mass of the Sun.
\qed


\problem{4.8}{}

\subproblem{i}
Consider a neutron star with mass $M$, radius $R$ and $N = M/m_{n}$ neutrons. The gravitational potential energy of the star is:

\begin{equation}
    U_{\text{grav}} = -\frac{3}{5} \frac{GM^{2}}{R}
\end{equation}

The total Fermi energy of the neutrons, assuming non-relativistic behaviour, is:

\begin{equation}
    \begin{split}
        \frac{3}{5} N \varepsilon_{F} &= \frac{3}{5} N \left[ \frac{\hbar^{2}}{2m_{n}} \left( 3\pi^{2} n \right)^{2/3} \right] \\
        &= \frac{3}{5} N \left[ \frac{\hbar^{2}}{2m_{n}} \left( \frac{9\pi^{2} N}{4\pi R^{3}} \right)^{2/3} \right] \\
        &= 0.0088 (2)^{5/3} \frac{h^{2}M^{5/3}}{m_{n}^{8/3}R^{2}}
    \end{split}
\end{equation}

where the only difference from the white dwarf case is the $2^{5/3}$ factor.

We see that the results from the white dwarf star carries over to the neutron star with scaling $M \to 2M$. Thus, the radius-mass relation for a neutron star is:

\begin{equation}
    R_{\text{min}} = 0.0088 \times \frac{10h^{2}}{3m_{n}^{8/3}G} M^{-1/3} \times (2)^{-1/3}
\end{equation}

\subproblem{ii}
With $M = \qty{2e30}{kg}$, we have $R = \qty{1034}{m}$ which is very small for a celestial body.
\qed


\problem{4.9}{}
Fermi energy is the chemical potential of a Fermi gas at zero temperature, that is, it is the energy cost of adding one more Fermion to the system. Consider a three-dimensional harmonic oscillator with energy levels:

\begin{equation}
    \varepsilon_{n} = \hbar \omega \left( n_{x} + n_{y} + n_{z} + \frac{3}{2} \right)
\end{equation}

where each energy level has total degeneracy $(2s + 1)(n + 1)(n + 2)/2$ due to the spin and the three-dimensional nature of the oscillator.

For a Fermi gas of $N$ non-interacting particles, the particles fill up the energy levels from the lowest to the highest, with each level allowing at most $(n + 1)(n + 2)$ particles. Therefore, for $1 \leq N \leq 50$, the Fermi energy is a piecewise function:

\begin{equation}
    \frac{\varepsilon_{F}}{\hbar \omega} = \begin{cases}
        3/2  & \text{if } N = 1             \\
        4/2  & \text{if } 2 \leq N \leq 5   \\
        7/2  & \text{if } 6 \leq N \leq 11  \\
        9/2  & \text{if } 12 \leq N \leq 19 \\
        11/2 & \text{if } 20 \leq N \leq 29 \\
        13/2 & \text{if } 30 \leq N \leq 41 \\
        15/2 & \text{if } 42 \leq N \leq 50 \\
    \end{cases}
\end{equation}

Consider the sum of states denoted by $n \equiv (n_{x} + n_{y} + n_{z})$:

\begin{equation}
    \begin{split}
        \sum_{\text{states}} &= \sum_{n} (2s + 1) \frac{(n + 1)(n + 2)}{2} \\
        &= \sum_{n} (2s + 1) \frac{(n + 1)(n + 2)}{2} \Delta n \\
        &\approx \int (n + 1)(n + 2) \, \mathrm{d}n
    \end{split}
\end{equation}

such that the density of states in the $n$-space is $g(n) = (n + 1)(n + 2)$. We may switch to the energy space by changing variable to $\varepsilon = \hbar \omega (n + 3/2)$:

\begin{equation}
    \begin{split}
        \tilde{g}(\varepsilon) &= \frac{1}{\hbar \omega} \left( \frac{\varepsilon}{\hbar \omega} - \frac{1}{2} \right) \left( \frac{\varepsilon}{\hbar \omega} + \frac{1}{2} \right) \\
        &\approx \frac{\varepsilon^{2}}{(\hbar \omega)^{3}}
    \end{split}
\end{equation}

in the limit $\varepsilon \gg \hbar \omega$.

The total number of particles is:

\begin{equation}
    \begin{split}
        N &= \int \bar{n} \tilde{g}(\varepsilon) \, \mathrm{d}\varepsilon \\
        &= \int_{0}^{\varepsilon_{F}} \frac{\varepsilon^{2}}{(\hbar \omega)^{3}} \, \mathrm{d}\varepsilon \\
        &= \frac{\varepsilon_{F}^{3}}{3(\hbar \omega)^{3}}
    \end{split}
\end{equation}

which gives the Fermi energy:

\begin{equation}
    \varepsilon_{F} = (3N)^{1/3} \hbar \omega = \qty{9.51e-27}{J}
\end{equation}


\end{document}
\documentclass[12pt]{article}
\usepackage{homework}
\pagestyle{fancy}

% assignment information
\def\course{Quantum Mechanics}
\def\assignmentno{Problem Sheet 1}
\def\assignmentname{Dirac Notation, Operators \& Commutators}
\def\name{Xin, Wenkang}
\def\time{\today}

\begin{document}

\begin{titlepage}
    \begin{center}
        \large
        \textbf{\course}

        \vfill

        \Huge
        \textbf{\assignmentno}

        \vspace{1.5cm}

        \large{\assignmentname}

        \vfill

        \large
        \name

        \time
    \end{center}
\end{titlepage}


%==========
\pagebreak
\section*{Dirac Notation}
%==========


\problem{1}{}
The fact that electrons interfere like waves require the use of probability amplitudes that can superpose.
\qed


\problem{2}{}
Given $\ket{\psi} = e^{i\pi/5} \ket{a} + e^{i\pi/4} \ket{b}$, we have:

\begin{equation}
    \bra{\psi} = e^{-i\pi/5} \bra{a} + e^{-i\pi/4} \bra{b}
\end{equation}
\qed


\problem{3}{}

\subproblem{a}
Given $\ket{\psi} = a \ket{A} + b \ket{B}$, we need $\braket{\psi}{\psi} = 1$ or $a^2 + b^2 = 1$. With $a = i/2$, we have $P(A) = 1/4$ and $P(B) = 3/4$.

\subproblem{b}
With $b = e^{i\pi}$, we have $P(B) = 1$ and $P(A) = 0$.

\subproblem{c}
With $b = 1/3 + i/\sqrt{2}$, we have $P(B) = 1/9 + 1/2 = 11/18$ and $P(A) = 7/18$.
\qed


\problem{4}{}
We have the probability of finding the particle in the state $\ket{n}$:

\begin{equation}
    P(n) = \abs{a_n}^2 = \frac{1}{2} \left( \frac{1}{3} \right)^{\abs{n}}
\end{equation}

The probability of $P(n \geq 0)$ is:

\begin{equation}
    P(n \geq 0) = \sum_{n=0}^{\infty} P(n) = \frac{1}{2} \sum_{n=0}^{\infty} \left( \frac{1}{3} \right)^n = \frac{1}{2} \frac{1}{1 - 1/3} = \frac{3}{4}
\end{equation}
\qed


%==========
\pagebreak
\section*{Operators}
%==========


\problem{5}{}

\subproblem{a}
$\braket{\psi}{Q|\psi}$ is the expectation value of the observable represented by the operator $Q$. By expectation value, we mean the average value of the observed value on a large number of identically prepared systems in the state $\ket{\psi}$.

$\abs{\braket{q_{n}}{\psi}}^2$ is the probability of finding the system in the state $\ket{q_n}$ when the system is in the state $\ket{\psi}$.

\subproblem{b}
The operator $\sum_{n} \ket{q_{n}} \bra{q_{n}}$ is the identity operator expressed in the $Q$ representation.

The operator $\sum_{n} q_{n} \ket{q_{n}} \bra{q_{n}}$ can be think of as the operator $Q$ expressed in the $Q$ representation. This is a diagonal matrix with the diagonal elements being the eigenvalues $q_{n}$ of $Q$.
\qed


\problem{6}{}
Given $\hat{A}$ and $\hat{B}$ as Hermitian operators, we have the following:

\begin{equation}
    \begin{split}
        \left( \hat{A} + \hat{B} \right)^{\dagger} &= \left( \hat{A}^{\dagger} + \hat{B}^{\dagger} \right) = \hat{A} + \hat{B} \\
        \left( c \hat{A} \right)^{\dagger} &= c^{*} \hat{A}^{\dagger} \\
        \left( \hat{A} \hat{B} \right)^{\dagger} &= \hat{B}^{\dagger} \hat{A}^{\dagger} = \hat{B} \hat{A} \\
        \left( \hat{A} \hat{B} + \hat{B} \hat{A} \right)^{\dagger} &= \left( \hat{B} \hat{A} + \hat{A} \hat{B} \right)^{\dagger} = \hat{B} \hat{A} + \hat{A} \hat{B}
    \end{split}
\end{equation}

Therefore, $\hat{A} + \hat{B}$ and $\hat{A} \hat{B}$ are Hermitian operators. $c \hat{A}$ is a Hermitian operator if and only if $c$ is real.

Consider the operator $\partial /\partial x$ on function $f$:

\begin{equation}
    \braket{f}{\frac{\partial f}{\partial x}} = \int_{\mathbb{R}} f^{*} \frac{\partial f}{\partial x} \, \mathrm{d}x = \left. f^{*} f \right|_{-\infty}^{\infty} - \int_{\mathbb{R}} \frac{\partial f^{*}}{\partial x} f \, \mathrm{d}x = - \bra{\frac{\partial f}{\partial x}} \ket{f}
\end{equation}

where the boundary term vanishes because $f$ is bounded.

This shows that $\partial /\partial x$ is not Hermitian, but in fact anti-Hermitian.

Consider the operator $-i\hbar \partial /\partial x$ on function $f$:

\begin{equation}
    \braket{f}{-i\hbar \frac{\partial f}{\partial x}} = -i\hbar \int_{\mathbb{R}} f^{*} \frac{\partial f}{\partial x} \, \mathrm{d}x = -i\hbar \left. f^{*} f \right|_{-\infty}^{\infty} + i\hbar \int_{\mathbb{R}} \frac{\partial f^{*}}{\partial x} f \, \mathrm{d}x = \bra{-i\hbar \frac{\partial f}{\partial x}} \ket{f}
\end{equation}

which shows that $-i\hbar \partial /\partial x$ is Hermitian.

$\partial^{2}/\partial x^{2}$ is Hermitian because:

\begin{equation}
    \left( \frac{\partial^{2}}{\partial x^{2}} \right)^{\dagger} = \left( \frac{\partial}{\partial x} \right)^{\dagger} \circ \left( \frac{\partial}{\partial x} \right)^{\dagger} = \left( - \frac{\partial}{\partial x} \right) \circ \left( - \frac{\partial}{\partial x} \right) = \frac{\partial^{2}}{\partial x^{2}}
\end{equation}
\qed


\problem{7}{}
Given $\hat{A}$ and $\hat{B}$ as Hermitian operators, we have:

\begin{equation}
    \left\{ i \left[ \hat{A}, \hat{B} \right] \right\}^{\dagger} = -i \left( \hat{A} \hat{B} - \hat{B} \hat{A} \right)^{\dagger} = -i \left( \hat{B}^{\dagger} \hat{A}^{\dagger} - \hat{A}^{\dagger} \hat{B}^{\dagger} \right) = -i \left( \hat{B} \hat{A} - \hat{A} \hat{B} \right) = i \left[ \hat{A}, \hat{B} \right]
\end{equation}
\qed


\problem{8}{}

\begin{equation}
    \left( \hat{A} \hat{B} \hat{C} \hat{D} \right)^{\dagger} = \hat{D}^{\dagger} \left( \hat{A} \hat{B} \hat{C} \right)^{\dagger} = \hat{D}^{\dagger} \hat{C}^{\dagger} \left( \hat{A} \hat{B} \right)^{\dagger} = \hat{D}^{\dagger} \hat{C}^{\dagger} \hat{B}^{\dagger} \hat{A}^{\dagger}
\end{equation}
\qed


%==========
\pagebreak
\section*{Commutators}
%==========


\problem{9}{}
If $\hat{A}$ and $\hat{B}$ share a complete set of eigenkets $\ket{n}$ such that $\hat{A} \ket{n} = a_{n} \ket{n}$ and $\hat{B} \ket{n} = b_{n} \ket{n}$, we have:

\begin{equation}
    \hat{A} \hat{B} \ket{n} = \hat{A} \left( b_{n} \ket{n} \right) = b_{n} \hat{A} \ket{n} = a_{n} b_{n} \ket{n}
\end{equation}

but also:

\begin{equation}
    \hat{B} \hat{A} \ket{n} = \hat{B} \left( a_{n} \ket{n} \right) = a_{n} \hat{B} \ket{n} = a_{n} b_{n} \ket{n}
\end{equation}

Given non-zero eigenvalues $a_{n}$ and $b_{n}$, we must have $\hat{A} \hat{B} \ket{n} = \hat{B} \hat{A} \ket{n}$ for all $\ket{n}$, which implies $\hat{A} \hat{B} = \hat{B} \hat{A}$ or $[\hat{A}, \hat{B}] = 0$.

The converse is also true but much more difficult to prove. For simplicity let us assume non-degenerate eigenvalues $a_{n}$ such that $\hat{A} \ket{n} = a_{n} \ket{n}$. Since the two operators commute, we have:

\begin{equation}
    \hat{A} \hat{B} \ket{n} = \hat{B} \hat{A} \ket{n} = a_{n} \hat{B} \ket{n}
\end{equation}

This means that $\hat{B} \ket{n}$ must be an eigenket of $\hat{A}$ with eigenvalue $a_{n}$. This means that $\hat{B} \ket{n}$ must be some multiple of $\ket{n}$, i.e. $\hat{B} \ket{n} = b_{n} \ket{n}$ for some $b_{n}$. This shows that $\ket{n}$ is also an eigenket of $\hat{B}$ with eigenvalue $b_{n}$.
\qed


\problem{10}{}
This is true as long as the eigenvalues of $\hat{A}$ are non-degenerate, as argued in the previous problem.
\qed


\problem{11}{}

\subproblem{a}

\begin{equation}
    \left[ \hat{A} \hat{B}, \hat{C} \right] = \hat{A} \hat{B} \hat{C} - \hat{C} \hat{A} \hat{B} = \hat{A} \hat{B} \hat{C} - \hat{A} \hat{C} \hat{B} + \hat{A} \hat{C} \hat{B} - \hat{C} \hat{A} \hat{B} = \hat{A} \left[ \hat{B}, \hat{C} \right] + \left[ \hat{A}, \hat{C} \right] \hat{B}
\end{equation}

\subproblem{b}

\begin{equation}
    \left[ \hat{A} \hat{B} \hat{C}, \hat{D} \right] = \hat{A} \hat{B} \left[ \hat{C}, \hat{D} \right] + \left[ \hat{A} \hat{B}, \hat{D} \right] \hat{C} = \hat{A} \hat{B} \left[ \hat{C}, \hat{D} \right] + \left[ \hat{A}, \hat{D} \right] \hat{B} \hat{C} + \hat{A} \left[ \hat{B}, \hat{D} \right] \hat{C}
\end{equation}

This is similar to a chain rule in calculus, where $\hat{D}$ acts on there other operators `in turn'.

\subproblem{c}

\begin{equation}
    \left[ \hat{x}^{n}, \hat{p} \right] = -i\hbar \left[ \hat{x}^{n}, \frac{\partial}{\partial x} \right]
\end{equation}

Consider $[\hat{x}^{n}, \partial /\partial x]$ acting on a function $f$:

\begin{equation}
    \left[ \hat{x}^{n}, \frac{\partial}{\partial x} \right] f = \hat{x}^{n} \frac{\partial f}{\partial x} - \frac{\partial}{\partial x} \left( \hat{x}^{n} f \right) = -n \hat{x}^{n-1} f
\end{equation}

so that:

\begin{equation}
    \left[ \hat{x}^{n}, \hat{p} \right] = i\hbar n \hat{x}^{n-1}
\end{equation}

\subproblem{d}

\begin{equation}
    \left[ f(\hat{x}), \hat{p} \right] = -i\hbar \left[ \sum_{k} a_{k} \hat{x}^{k}, \frac{\partial}{\partial x} \right] = -i\hbar \sum_{k} a_{k} \left[ \hat{x}^{k}, \frac{\partial}{\partial x} \right] = i\hbar \sum_{k} a_{k} k \hat{x}^{k-1} = i\hbar f'(\hat{x})
\end{equation}

for any well-behaved function $f$ that has a power series expansion.
\qed


\problem{12}{}
This is done in the previous problem.
\qed


\problem{13}{}
By the term `compatible', we mean that the two observables can be measured simultaneously. This implies that for the two associated Hermitian operators $\hat{A}$ and $\hat{B}$, there exists a complete set of eigenkets $\ket{n}$ that are common to both operators such that $\hat{A} \ket{n} = a_{n} \ket{n}$ and $\hat{B} \ket{n} = b_{n} \ket{n}$. Physically, this means that given a system in the state $\ket{n}$, we can measure both $A$ and $B$ and obtain the values $a_{n}$ and $b_{n}$ respectively.

From previous problems, it has already been established that this is true if and only if $[\hat{A}, \hat{B}] = 0$.
\qed


\problem{14}{}
Commuting operators $\hat{A}$ and $\hat{B}$ satisfy $\hat{A} \hat{B} = \hat{B} \hat{A}$, meaning that the order of the operators does not matter. Mutually commuting operators represent observables that can be measured simultaneously.

Take position and momentum as an example. Since the position and momentum operators do not commute, we cannot possibly know both the position and momentum of a particle at the same time.

Given $[P, Q] = U \ne 0$ for some non-commuting operators, we have:

\begin{equation}
    [P, Q] \ket{\psi} = U \ket{\psi}
\end{equation}

which is not necessarily zero as $\ket{\psi}$ may be in the kernel of $U$.
\qed


\problem{15}{}
Note that the delta function $\delta(x)$ is `defined' as:

\begin{equation}
    \delta(x) = \begin{cases}
        \infty & x = 0   \\
        0      & x \ne 0
    \end{cases}
\end{equation}

with the constraint:

\begin{equation}
    \int_{-\infty}^{\infty} \delta(x) \, \mathrm{d}x = 1
\end{equation}

\subproblem{a}

\begin{equation}
    \int_{-\infty}^{\infty} \delta(cx) \, \mathrm{d}x = \frac{1}{\abs{c}} \int_{-\infty}^{\infty} \delta(u) \, \mathrm{d}u = \frac{1}{\abs{c}}
\end{equation}

so that we may write:

\begin{equation}
    \delta(cx) = \frac{1}{\abs{c}} \delta(x)
\end{equation}

This can be thought as a dilation of the delta function by a factor of $1/\abs{c}$.

\subproblem{b}

\begin{equation}
    \int_{-\infty}^{\infty} \delta(x^{2} - c^{2}) \, \mathrm{d}x = \int_{-\infty}^{\infty} \delta(u - c^{2}) \frac{1}{2\sqrt{u}} \, \mathrm{d}u = \frac{1}{2\sqrt{c^{2}}} = \frac{1}{2\abs{c}}
\end{equation}

This function has two peaks at $x = \pm c$ and is symmetric about the origin, so we may write:

\begin{equation}
    \delta(x^{2} - c^{2}) = \frac{1}{2\abs{c}} \left[ \delta(x - c) + \delta(x + c) \right]
\end{equation}

\subproblem{c}
Consider the derivative of the Fourier transform of some function $f(x)$:

\begin{equation}
    \begin{split}
        \mathcal{F}(f') &= \frac{1}{\sqrt{2\pi}} \int_{\mathbb{R}} e^{-i\omega x} \frac{\mathrm{d}f}{\mathrm{d}x} \, \mathrm{d}x \\
        &= \frac{1}{\sqrt{2\pi}} \left. e^{-i\omega x} f(x) \right|_{-\infty}^{\infty} + \frac{i\omega}{\sqrt{2\pi}} \int_{\mathbb{R}} e^{-i\omega x} f(x) \, \mathrm{d}x \\
        &= i\omega \mathcal{F}(f)
    \end{split}
\end{equation}

for $f$ that vanishes at infinity.

The Fourier transform of the translated Heaviside step function is:

\begin{equation}
    \mathcal{F}[\theta(x - c)] = \frac{1}{\sqrt{2\pi}} \int_{\mathbb{R}} e^{-i\omega x} \theta(x - c) \, \mathrm{d}x = \frac{1}{\sqrt{2\pi}} \int_{c}^{\infty} e^{-i\omega x} \, \mathrm{d}x = \frac{1}{\sqrt{2\pi}} \frac{e^{-i\omega c}}{i\omega}
\end{equation}

so that:

\begin{equation}
    \mathcal{F}\left[ \frac{\mathrm{d}}{\mathrm{d}x} \theta(x - c) \right] = \frac{1}{\sqrt{2\pi}} e^{-i\omega c}
\end{equation}

But the Fourier transform of the translated delta function is:

\begin{equation}
    \mathcal{F}[\delta(x - c)] = \frac{1}{\sqrt{2\pi}} \int_{\mathbb{R}} e^{-i\omega x} \delta(x - c) \, \mathrm{d}x = \frac{1}{\sqrt{2\pi}} e^{-i\omega c}
\end{equation}

Since the Fourier transform of a function is unique, we must have:

\begin{equation}
    \frac{\mathrm{d}}{\mathrm{d}x} \theta(x - c) = \delta(x - c)
\end{equation}

\subproblem{d}
The Fourier transform of the delta function is:

\begin{equation}
    \mathcal{F}[\delta(x)] = \frac{1}{\sqrt{2\pi}} \int_{\mathbb{R}} e^{-i\omega x} \delta(x) \, \mathrm{d}x = \frac{1}{\sqrt{2\pi}}
\end{equation}

so that the delta function can be written as the inverse Fourier transform of the constant function $1/\sqrt{2\pi}$:

\begin{equation}
    \delta(x) = \frac{1}{2\pi} \int_{\mathbb{R}} e^{i\omega x} \, \mathrm{d}\omega
\end{equation}

\subproblem{e}

\begin{equation}
    \int_{-\infty}^{\infty} f(x) \delta'(x - x_{0}) \, \mathrm{d}x = \left. f(x) \delta(x - x_{0}) \right|_{-\infty}^{\infty} - \int_{-\infty}^{\infty} f'(x) \delta(x - x_{0}) \, \mathrm{d}x = -f'(x_{0})
\end{equation}

This means that the `derivative' of the delta function has the negative sampling property of the derivative of a function.
\qed


\end{document}
\documentclass[12pt]{article}
\usepackage{homework}
\pagestyle{fancy}

% assignment information
\def\course{Statistical Mechanics}
\def\assignmentno{Problem Set 1}
\def\assignmentname{Probability, Statistics and Fluctuations}
\def\name{Xin, Wenkang}
\def\time{\today}

\begin{document}

\begin{titlepage}
    \begin{center}
        \large
        \textbf{\course}

        \vfill

        \Huge
        \textbf{\assignmentno}

        \vspace{1.5cm}

        \large{\assignmentname}

        \vfill

        \large
        \name

        \time
    \end{center}
\end{titlepage}


%==========
\pagebreak
\section*{Probability, statistics and fluctuations}
%==========


\problem{1.4}{}
We have the probabilities $P(\text{boy}) = P(\text{girl}) = 1/2$. Consider the conditional probability:

\begin{equation}
    P(\text{at least one girl} | \text{at least one boy}) = \frac{P(\text{at least one girl} \cap \text{at least one boy})}{P(\text{at least one boy})}
\end{equation}

This is a combinatorial problem. The number of ways to have at least one boy and one girl is $2^2 - 2 = 2$. The number of ways to have at least one boy out of the two children is $2^2 - 1 = 3$. Therefore, the conditional probability is $2/3$.

On the other hand, if we are informed that \textbf{a particular child} (in this case the taller one) is a boy, then the conditional probability is:

\begin{equation}
    P(\text{at least one girl} | \text{the taller one is a boy}) = \frac{P(\text{at least one girl} \cap \text{the taller one is boy})}{P(\text{the taller one is a boy)}}
\end{equation}

There is only one way to have one girl and one taller boy, and there are two ways to have the taller one to be a boy. Therefore, the conditional probability is $1/2$.
\qed


\problem{1.5}{}
The number of ways to achieve $N/2$ heads and $N/2$ tails out of $N$ coin tosses is given by the binomial coefficient:

\begin{equation}
    \binom{N}{N/2} = \frac{N!}{(N/2)!(N/2)!}
\end{equation}

The number of ways to have $N/2 - m$ heads and $N/2 + m$ tails is given by:

\begin{equation}
    \binom{N}{N/2 - m} = \frac{N!}{(N/2 - m)!(N/2 + m)!}
\end{equation}

For this to be half of the previous number, we must have:

\begin{equation}
    \left( \frac{N}{2} - m \right)! \left( \frac{N}{2} + m \right)! = 2 \left( \frac{N}{2}! \right)^{2}
\end{equation}

Consider the expansion of the denominator:

\begin{equation}
\begin{split}
    \left( \frac{N}{2} - m \right)! \left( \frac{N}{2} + m \right)! &= \left( \frac{N}{2}! \right)^{2} \frac{(N/2 + 1)(N/2 + 2) \cdots (N/2 + m)}{(N/2 - m + 1)(N/2 - m + 2) \cdots (N/2)} \\
\end{split}
\end{equation}
\qed


\problem{1.6}{}
We divide each molar heat capacity by $R$ and obtain:

\begin{table}[h!]
\centering
\begin{tabular}{|c|l|c|l|}
\hline
Al                       & 2.93 & Pb                      & 3.18 \\ \hline
Ar                       & 2.50 & Ne                      & 2.50 \\ \hline
Au                       & 3.06 & N2                      & 3.50 \\ \hline
Cu                       & 2.94 & O2                      & 3.53 \\ \hline
He                       & 2.50 & Ag                      & 3.07 \\ \hline
H2                       & 3.47 & Xe                      & 2.50 \\ \hline
\multicolumn{1}{|l|}{Fe} & 3.02 & \multicolumn{1}{l|}{Zn} & 3.01 \\ \hline
\end{tabular}
\end{table}

There is a trend that gaseous substances have a higher molar heat capacity than solid substances. This is because the heat capacity of a solid is dominated by the lattice vibrations whereas the heat capacity of a gas is dominated by the translational and rotational degrees of freedom.
\qed


\problem{1.7}{}
From direct calculation, we have $\ln{15!} = 27.8993$. Stirling's approximation gives:

\begin{equation}
    \ln{15!} \approx 15\ln{15} - 15 = 25.6208
\end{equation}

which is not a very good approximation.

The leading error made by using Stirling's approximation is:

\begin{equation}
    \frac{\ln{N}}{N}
\end{equation}

For this to be less than $0.02$, we need $N > 282.1$.
\qed


\problem{1.8}{}
There exists only one microstate where all particles have the same energy $\epsilon$. The given macrostate of three zero energy particle, one $\epsilon$ energy particle and one $2\epsilon$ energy particle has the degeneracy:

\begin{equation}
    \binom{5}{1} \times \binom{4}{1} \times \binom{3}{3} = 20
\end{equation}

\begin{table}[h]
\centering
\begin{tabular}{|c|c|c|c|c|c|c|}
\hline
$0$ & $\epsilon$ & $2\epsilon$ & $3\epsilon$ & $4\epsilon$ & $5\epsilon$ & Number \\ \hline
4 &   &   &   &   & 1 & 5      \\ \hline
3 & 1 &   &   & 1 &   & 20     \\ \hline
3 &   & 1 & 1 &   &   & 20     \\ \hline
2 & 2 &   & 1 &   &   & 30     \\ \hline
2 & 1 & 2 &   &   &   & 30     \\ \hline
1 & 3 & 1 &   &   &   & 20     \\ \hline
  & 5 &   &   &   &   & 1      \\ \hline
\end{tabular}
\end{table}
\qed


\problem{1.9}{}
The number of microstates for the given macrostate is:

\begin{equation}
    \binom{16}{1} \times \binom{15}{2} \times \binom{13}{2} \times \binom{11}{4} \times \binom{7}{7} = 43243200
\end{equation}

The entropy is:

\begin{equation}
    S = \ln{\Omega} = 17.58
\end{equation}

The temperature can be approximated by:

\begin{equation}
    T = \frac{E}{k_{B}\ln{\Omega}}
\end{equation}

so that in terms of $\epsilon/k_{B}$, the temperature is:

\begin{equation}
    \frac{18}{\ln{43243200}} = 1.02
\end{equation}
\qed


\problem{Extra}{}
Consider a system of $N$ particles in an isolated system, such that the total energy $U$ is fixed. The entropy of the system is given by (up to a constant):

\begin{equation}
    S = -\sum_{i} P_{i}\ln{P_{i}}
\end{equation}

where $P_{i}$ is the probability of the system being in the $i$-th microstate.

On the other hand, to fulfil the constraint of fixed energy, we require:

\begin{equation}
    \sum_{i} P_{i}E_{i} = U
\end{equation}

where $E_{i}$ is the energy of the $i$-th microstate.

To maximise the entropy, we use the method of Lagrange multipliers. Consider the modified entropy:

\begin{equation}
    S' = -\sum_{i} P_{i}\ln{P_{i}} + \beta\left( \sum_{i} P_{i}E_{i} - U \right)
\end{equation}

The extremum of $S'$ is given by the conditions:

\begin{equation}
\begin{split}
    \frac{\partial S'}{\partial P_{i}} &= -\ln{P_{i}} - 1 - \beta E_{i} = 0 \\
    \frac{\partial S'}{\partial \lambda} &= U - \sum_{i} P_{i}E_{i} = 0
\end{split}
\end{equation}

The first equation gives the Boltzmann distribution:

\begin{equation}
    P_{i} = \frac{1}{Z}e^{-\beta E_{i}}
\end{equation}

where $Z$ is a normalisation constant given by:

\begin{equation}
    Z = \sum_{i} e^{-\beta E_{i}}
\end{equation}

The $\beta$ factor is determined by the equation:

\begin{equation}
    -\frac{\partial \ln{Z}}{\partial \beta} = U
\end{equation}
\qed

\end{document}